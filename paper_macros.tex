%------------------------------------------------------------------------------
% Spacing Options
%------------------------------------------------------------------------------

\newcommand{\thesisForceSingleSpacing}{\singlespacing}
\newcommand{\thesisForceDoubleSpacing}{\doublespacing}

%------------------------------------------------------------------------------
% Portable HREFs
%------------------------------------------------------------------------------

% Common variant
\newcommand{\porthref}[2]{\href{#2}{#1}\printOnlyFootnote{\url{#2}}}

% Custom URLs
\newcommand{\porthreft}[3]{\href{#3}{#1}\printOnlyFootnote{\href{#3}{#2}}}
% Inside of some environments, footnote marks aren't registered properly, so we
% need to manually write the "text" part
\newcommand{\porthreftm}[2]{\href{#2}{#1\printOnlyFootnoteMark}}

%------------------------------------------------------------------------------
% Optional refs in format of (see <ref>)
%------------------------------------------------------------------------------

\newif\ifnotpaper

\newcommand{\formatPaper}[2]{%
    \ifnotpaper
        #1{#2}%
    \else
        \underline{#2}%
    \fi
}

\def\refHelper{\ifnotpaper\else Reference \fi}
\newcommand\multAuthHelper[1]{\ifnotpaper #1\else #1s\fi}

\newcommand\ifblind[2]{\IfEndWith*{\jobname}{_blind}{#1}{#2}}

%------------------------------------------------------------------------------
% Generic "chunks" that get reused
%------------------------------------------------------------------------------

\newenvironment{bigLandscape}{
    \newgeometry{hmargin=1cm, vmargin=2.5cm}
    \begin{landscape}
        }{
    \end{landscape}
    \restoregeometry
    \newpage
}

\DeclareDocumentCommand\seeSrcCode{ m m m g }{%
    (see the \href
    {https://github.com/samm82/TestGen-Thesis/blob/#1/scripts/#2.py\#L#3%
        \IfNoValueF {#4} {-L#4}}
    {relevant source code})%
}

\newcommand{\accelTolTest}{astronauts \citep[p.~11]{MorgunEtAl1999}, aviators
    \citep[pp.~27,~42]{HoweAndJohnson1995}, or catalysts
    \citep[p.~1463]{LiuEtAl2023}}

\newcommand{\procLevel}[1]{``Test level'' can also refer to the scope
of a test process; for example, ``across the whole organization'' or only
``to specific projects'' #1[p.~24]{IEEE2022}}
\newcommand{\phaseDef}{can also refer to the ``period of time in the software
    life cycle'' when testing occurs \citeyearpar[p.~470]{IEEE2017}, usually
    after the implementation phase
    \ifnotpaper
        (\citeyear[pp.~420,~509]{IEEE2017}; \citealp[p.~56]{Perry2006}).
    \else
        \cite[pp.~420,~509]{IEEE2017}, \cite[p.~56]{Perry2006}.
    \fi}

% Define common footnotes about IEEE testing terms for reuse
\newcommand{\distinctIEEE}[1]{distinct from the notion of ``test #1'' described
    in \nameref{tab:ieeeTestTerms}.}
\newcommand{\notDefDistinctIEEE}[1]{\footnote{Not formally defined, but
        \distinctIEEE{#1}}}
\newcommand{\gerrardDistinctIEEE}[1]{\footnote{``Each type of test addresses a
        different risk area'' \citep[p.~12]{Gerrard2000a}, which is
        \distinctIEEE{#1}}}

% Examples of discrepancies
\def\tourDiscrep{There is disagreement on the structure of tours: they can
    either be quite general \citep[p.~34]{IEEE2022} or ``organized around a
    special focus'' \citepISTQB{}.}
\def\alphaDiscrep{Various sources say that alpha testing is performed by
    different people, including ``only by users within the organization
    developing the software'' \citep[p.~17]{IEEE2017}, by ``a small,
    selected group of potential users'' \citep[p.~5-8]{SWEBOK2024}, or
    ``in the developer's test environment by roles outside the
    development organization'' \citepISTQB{}.}
\def\loadDiscrep{Load testing is performed with either loads
    ``between anticipated conditions of low, typical, and peak usage''
    \citep[p.~5]{IEEE2022} or loads that are as large as possible
    \citep[p.~86]{Patton2006}.}

% Capitalization can change
\newcommand{\ftrnote}[1]{#1ault tolerance testing may also be a sub-approach of
    reliability testing \ifnotpaper
        \citetext{\citealp[p.~375]{IEEE2017}; \citealp[p.~7-10]{SWEBOK2024}}%
    \else \cite[p.~375]{IEEE2017}, \cite[p.~7-10]{SWEBOK2024}%
    \fi, which is distinct from robustness testing \citep[p.~53]{Firesmith2015}.}

\def\graphGenDesc{
    To better visualize how test approaches relate to each other, we
    \ifnotpaper\else then \fi developed a tool to automatically
    generate graphs of these relations.
    \ifnotpaper Since parent-child and synonym relations between approaches are
        tracked in \ifblind{our approach glossary}{\href{
                https://github.com/samm82/TestGen-Thesis/blob/main/ApproachGlossary.csv
            }{our approach glossary}} in a consistent format, they can be parsed
        systematically. For example, if the entries in \Cref{tab:exampleGlossary}
        appear in the glossary, then their parent relations are displayed as
        \Cref{fig:exampleGraph} in the generated graph. Relevant citation
        information is also captured in our glossary following the author-year
        citation format, including ``reusing'' information from previous
        citations. For example, the first row of \Cref{tab:exampleGlossary}
        contains the citation ``(Author, 0000; 0001)'', which means that this
        information was present in two documents by ``Author'': one written in
        the year 0000, and one in 0001. The following citation, ``(0000)'',
        contains no author, which means it was written by the same one as the
        previous citationg. These citations are processed according to this
        logic \seeSrcCode{f0b92ad}{csvToGraph}{49}{90} so they can be
        consistently tracked throughout the analysis.

        \begin{table}[hbtp!]
            \centering
            \begin{tabular}{ll} \hline
                Name                        & Parent(s)                        \\ \hline
                A                           & B (Author, 0000; 0001), C (0000) \\
                B                           & C (implied by Author, 0000)      \\
                C                           & D (Author, 0002)                 \\
                D (implied by Author, 0002) &                                  \\ \hline
            \end{tabular}
            \caption{Example glossary entries demonstrating how parent relations are tracked.}
            \label{tab:exampleGlossary}
        \end{table}

        \ExampleGraph{}

    \fi
    All parent-child relations are graphed, as well as synonym relations where either:
    \begin{enumerate}
        \item both synonyms have their own rows in the glossary, or
        \item a term is a synonym to more than one term with its own row in the
              glossary.
    \end{enumerate}
    \ifnotpaper These conditions are also deduced from the information parsed
        from the glossary. For example, if the entries in \Cref{tab:synExampleGlossary}
        appear in the glossary, then they are displayed as \Cref{fig:synExampleGraph}
        in the generated graph (note that X does not appear since it does not
        meet the criteria given above).

        \begin{table}[hbtp!]
            \centering
            \begin{tabular}{ll} \hline
                Name & Synonym(s)                            \\ \hline
                E    & F (Author, 0000; implied by 0001)     \\
                G    & F (Author, 0002), H (implied by 0000) \\
                H    & X                                     \\ \hline
            \end{tabular}
            \caption{Example glossary entries demonstrating how synonym relations are tracked.}
            \label{tab:synExampleGlossary}
        \end{table}

        This allows for automatic detection of some classes of discrepancies. The
        most trivial to automate is ``multi-synonym'' relations, which are already
        found to generate the graph as desired. The list found in \Cref{multiSyns}
        is automatically generated based on glossary entries such as those found
        in \Cref{tab:synExampleGlossary}. The self-referential definitions in
        \Cref{selfPars} were also trivial, found by simply looking for lines
        the generated .tex files starting with \texttt{I -> I} which would
        result in the graph in \Cref{fig:selfExampleGraph}. A similar process
        is used to detect instances where two approaches have a synonym
        \emph{and} a parent-child relation. A dictionary of each term's
        synonyms is built to evaluate which synonym relations are notable
        enough to include in the graph, and these mappings are then checked to
        see if one appears as a parent of the other. For example, if J and K
        are synonyms, a generated .tex file with a parent line starting with
        \texttt{J -> K} would result in these approaches being graphed as shown
        in \Cref{fig:parSynExampleGraph}.

        The visual nature of these graphs makes it possible to represent both
        explicit and implicit relations without double counting them during
        \hyperref[discrep-analysis]{analysis}.
        If a relation is both explicit \emph{and} implicit, the implicit relation
        is only shown in the graph if it is from a more ``trusted''
        \hyperref[sources]{source category}; note that only the explicit
        synonym relation between E and F
        from \Cref{tab:exampleGlossary} is shown in \Cref{fig:synExampleGraph}.
        Implicit approaches and relations are denoted by dashed lines, as shown
        in \Cref{fig:exampleGraph,fig:synExampleGraph}; explicit approaches are
        \emph{always} denoted by solid lines, even if they are also implicit.
        ``Rigid'' versions, which exclude implicit approaches and relations,
        can also be generated for each graph\todo{Should I add an example?}.

    \fi
    Since these graphs tend to be large, it is useful to focus on specific
    subsets of them. \ifnotpaper Graphs limited to approaches from a given
        \hyperref[categories-observ]{approach category} are generated, as well
        as a graph of static approaches; interestingly, static testing seems to
        be considered a separate approach category in \citep[Fig.~2]{IEEE2022}%
        \todo{Better place to include this, if at all?}. Since static testing is
        \hyperref[static-test]{out of scope}, it is also helpful to see how it
        overlaps with the in-scope dynamic testing, so these ``connecting''
        relations are also graphed. Additionally, more specific subsets of
        these graphs \else These \fi can be generated based on a given subset
    of approaches to include, such as those pertaining to
    \hyperref[recov-discrep]{recovery} or \hyperref[scal-discrep]{scability},
    as shown in \Cref{fig:recovery-graph-current,fig:scal-graph-current},
    respectively (albeit with manually created legends). By specifying sets of
    approaches and relations to add or remove, these generated graphs can be
    updated in accordance with our \nameref{recs}, as shown in
    \Cref{fig:recovery-graph-proposed,fig:scal-graph-proposed}, respectively.
    These modifications can also be inherited, as in \Cref{fig:perf-graph},
    which was generated based on the modifications from
    \Cref{fig:recovery-graph-proposed,fig:scal-graph-proposed} and then
    manually tweaked\todo{Ensure these tweaks are on an up-to-date version!}.}

%------------------------------------------------------------------------------
% For populating values from files
%------------------------------------------------------------------------------

\ExplSyntaxOn
\ior_new:N \g_hringriin_file_stream

\NewDocumentCommand{\ReadFile}{mm}
{
    \hringriin_read_file:nn { #1 } { #2 }
    \cs_new:Npn #1 ##1
    {
        \str_if_eq:nnTF { ##1 } { * }
        { \seq_count:c { g_hringriin_file_ \cs_to_str:N #1 _seq } }
        { \seq_item:cn { g_hringriin_file_ \cs_to_str:N #1 _seq } { ##1 } }
    }
}

\cs_new_protected:Nn \hringriin_read_file:nn
{
    \ior_open:Nn \g_hringriin_file_stream { #2 }
    \seq_gclear_new:c { g_hringriin_file_ \cs_to_str:N #1 _seq }
    \ior_map_inline:Nn \g_hringriin_file_stream
    {
        \seq_gput_right:cx
        { g_hringriin_file_ \cs_to_str:N #1 _seq }
        { \tl_trim_spaces:n { ##1 } }
    }
    \ior_close:N \g_hringriin_file_stream
}

\ExplSyntaxOff

% Define/read values for Undefined Terms methodology for reuse and calculation!
\ReadFile{\undefTermCounts}{\miscAssets/undefTermCounts}

\newcount\TotalBefore
\newcount\UndefBefore
\newcount\TotalAfter
\newcount\UndefAfter

\TotalBefore=\undefTermCounts{1}
\UndefBefore=\undefTermCounts{2}
\TotalAfter=\undefTermCounts{3}
\UndefAfter=\undefTermCounts{4}

\def\approachCount{\undefTermCounts{3}}

\ReadFile{\qualityCounts}{build/qualityCount}
\def\qualityCount{\qualityCounts{1}}

\ReadFile{\parSynCounts}{build/parSynCounts}
\def\parSynCount{\parSynCounts{1}}
\def\selfCycleCount{\parSynCounts{2}}

\ReadFile{\stdSources}{build/stdSources}
\ReadFile{\metaSources}{build/metaSources}
\ReadFile{\textSources}{build/textSources}
\ReadFile{\paperSources}{build/paperSources}

\def\srcCount{\the\numexpr\stdSources{3} + \metaSources{3} + \textSources{3} + \paperSources{3}}

\ReadFile{\stdDiscCatBrkdwn}{build/stdDiscCatBrkdwn}
\ReadFile{\metaDiscCatBrkdwn}{build/metaDiscCatBrkdwn}
\ReadFile{\textDiscCatBrkdwn}{build/textDiscCatBrkdwn}
\ReadFile{\paperDiscCatBrkdwn}{build/paperDiscCatBrkdwn}
\ReadFile{\totalDiscCatBrkdwn}{build/totalDiscCatBrkdwn}

\ReadFile{\stdDiscClsBrkdwn}{build/stdDiscClsBrkdwn}
\ReadFile{\metaDiscClsBrkdwn}{build/metaDiscClsBrkdwn}
\ReadFile{\textDiscClsBrkdwn}{build/textDiscClsBrkdwn}
\ReadFile{\paperDiscClsBrkdwn}{build/paperDiscClsBrkdwn}
\ReadFile{\totalDiscClsBrkdwn}{build/totalDiscClsBrkdwn}

\def\stds{\nameref{stds}}
\def\metas{\nameref{metas}}
\def\texts{\nameref{texts}}
\def\papers{\nameref{papers}}

\def\srcCat{\hyperref[sources]{Source Category}}
\def\reduns{\ifnotpaper\nameref{redun}\else Redundancies\footnote{Section omitted for brevity.}\fi}

\def\totalDiscreps{\totalDiscCatBrkdwn{13}}

\def\syns{\hyperref[syns]{Synonyms}}
\def\pars{\hyperref[pars]{Parents}}
\def\cats{\hyperref[cats]{Categories}}
\def\defs{\hyperref[defs]{Definitions}}
\def\terms{\hyperref[terms]{Terminology}}
\def\cites{\hyperref[cites]{Citations}}

%------------------------------------------------------------------------------
% TODOs
%------------------------------------------------------------------------------

% Generic Inlined TODOs
\newcommand{\intodo}[1]{\todo[inline]{#1}}

% Unimportant TODOs for "later" (i.e., finishing touches or changes immediately before submission)
\newcommand{\latertodo}[1]{\todo[backgroundcolor=Cyan]{\textit{Later}: #1}}

% "Important" TODOs
\newcommand{\imptodo}[1]{\todo[inline,backgroundcolor=Red]{\textbf{Important}: #1}}

% "Easy" TODOs
\newcommand{\easytodo}[1]{\todo[inline,backgroundcolor=SeaGreen]{\textit{Easy}: #1}}
\newcommand{\eztodo}[1]{\easytodo{#1}}

% "Tedious" TODOs
\newcommand{\tedioustodo}[1]{\todo[inline,backgroundcolor=PineGreen]{\textit{Needs time}: #1}}

% "Question" TODO Notes
\newcounter{todonoteQuestionsCtr}
\newcommand{\questiontodo}[1]{\stepcounter{todonoteQuestionsCtr}\todo[backgroundcolor=Lavender]{\textbf{Q \#\thetodonoteQuestionsCtr{}}: #1}}
\newcommand{\qtodo}[1]{\questiontodo{#1}}

%------------------------------------------------------------------------------
% Citations
%------------------------------------------------------------------------------

\newcommand{\exhInfCite}{(\citealp[p.~5-5]{SWEBOK2024}; \citealp[p.~4]{IEEE2022};
    \citealp[p.~421]{vanVliet2000}; \citealp[pp.~439, 461]{PetersAndPedrycz2000})}

%------------------------------------------------------------------------------
% Link to Drasil issue
%------------------------------------------------------------------------------

\newcommand{\issueref}[1]{\href{https://github.com/JacquesCarette/Drasil/issues/#1}{\##1}}
\newcommand{\pullref}[1]{\href{https://github.com/JacquesCarette/Drasil/pull/#1}{\##1}}
\newcommand{\thesisissuerefhelper}[1]{\href{https://github.com/samm82/TestGen-Thesis/issues/#1}{\##1}}

\ExplSyntaxOn

% Based on output from ChatGPT
\NewDocumentCommand{\mapthesisissueref}{m}
{
    % Clear temporary sequences to store transformed items
    \seq_clear:N \l_tmpa_seq
    \seq_clear:N \l_tmpb_seq

    \seq_set_split:Nnn \l_tmpa_seq { , } { #1 } % Split the input by commas
    \seq_map_inline:Nn \l_tmpa_seq
    {
        \seq_put_right:Nn \l_tmpb_seq {\thesisissuerefhelper{##1}}
    }
    \seq_use:Nnnn \l_tmpb_seq { ~and~ } { ,~ } { ,~and~ }
}

\ExplSyntaxOff

\newcommand{\thesisissueref}[1]{\todo[backgroundcolor=lightgray]{See \mapthesisissueref{#1}}}

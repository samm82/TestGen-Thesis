%------------------------------------------------------------------------------
% Spacing Options
%------------------------------------------------------------------------------

\newcommand{\thesisForceSingleSpacing}{\singlespacing}
\newcommand{\thesisForceDoubleSpacing}{\doublespacing}

%------------------------------------------------------------------------------
% Portable HREFs
%------------------------------------------------------------------------------

% Common variant
\newcommand{\porthref}[2]{\href{#2}{#1}\printOnlyFootnote{\url{#2}}}

% Custom URLs
\newcommand{\porthreft}[3]{\href{#3}{#1}\printOnlyFootnote{\href{#3}{#2}}}
% Inside of some environments, footnote marks aren't registered properly, so we
% need to manually write the "text" part
\newcommand{\porthreftm}[2]{\href{#2}{#1\printOnlyFootnoteMark}}

%------------------------------------------------------------------------------
% Optional refs in format of (see <ref>)
%------------------------------------------------------------------------------

\newif\ifnotpaper
\newcommand{\seeSection}[1]{%
    \ifnotpaper
        \seeSectionAlways{#1}%
    \fi
}
\newcommand{\seeSectionPar}[2][]{%
    \ifnotpaper
        \seeSectionParAlways{#2}{#1}%
    \fi
}
\newcommand{\seeSectionFoot}[1]{%
    \ifnotpaper
        \seeSectionFootAlways{#1}%
    \fi
}
\NewDocumentCommand{\seeSectionAlways}{ m }{\seeAlways{#1}}
\NewDocumentCommand{\seeSectionParAlways}{ m O{} }{\seeParAlways{#1}[#2]}
\NewDocumentCommand{\seeSectionFootAlways}{ m }{\seeFootAlways{#1}}

\NewDocumentCommand{\seeThesisIssue}{ m }{\ifnotpaper\seeAlways[\thesisissueref]{#1}\fi}
\NewDocumentCommand{\seeThesisIssuePar}{ m o }
{\ifnotpaper\seeParAlways[\thesisissueref]{#1}[#2]\fi}
\NewDocumentCommand{\seeThesisIssueFoot}{ m }
{\ifnotpaper\seeFootAlways[\thesisissueref]{#1}\fi}

\NewDocumentCommand{\seeAlways}{ O{\Cref} m }{{ see #1{#2}}}
\NewDocumentCommand{\seeParAlways}{ O{\Cref} m O{} }{{ (see #1{#2}#3)}}
\NewDocumentCommand{\seeFootAlways}{ O{\Cref} m }{\footnote{See #1{#2}.}}

\newcommand{\formatPaper}[2]{%
    \ifnotpaper
        #1{#2}%
    \else
        \underline{#2}%
    \fi
}

\newcommand\ifblind[2]{\IfEndWith*{\jobname}{BLIND}{#1}{#2}}

%------------------------------------------------------------------------------
% Generic "chunks" that get reused
%------------------------------------------------------------------------------

\newcommand{\accelTolTest}{astronauts \citep[p.~11]{MorgunEtAl1999}, aviators
    \citep[pp.~27,~42]{HoweAndJohnson1995}, or catalysts
    \citep[p.~1463]{LiuEtAl2023}}

\newcommand{\procLevel}[1]{``Test level'' can also refer to the scope
of a test process; for example, ``across the whole organization'' or only
``to specific projects'' #1[p.~24]{IEEE2022}}
\newcommand{\phaseDef}{can also refer to the ``period of time in the software
    life cycle'' when testing occurs \citeyearpar[p.~470]{IEEE2017}, usually
    after the implementation phase
    \ifnotpaper
        (\citeyear[pp.~420,~509]{IEEE2017}; \citealp[p.~56]{Perry2006}).
    \else
        \cite[pp.~420,~509]{IEEE2017}, \cite[p.~56]{Perry2006}.
    \fi}

% Define common footnotes about IEEE testing terms for reuse
\newcommand{\distinctIEEE}[1]{distinct from the notion of ``test #1'' described
    in \nameref{tab:ieeeTestTerms}.}
\newcommand{\notDefDistinctIEEE}[1]{\footnote{Not formally defined, but
        \distinctIEEE{#1}}}
\newcommand{\gerrardDistinctIEEE}[1]{\footnote{``Each type of test addresses a
        different risk area'' \citep[p.~12]{Gerrard2000a}, which is
        \distinctIEEE{#1}}}

\def \ftrnote {Fault tolerance testing may also be a sub-approach of
    reliability testing \ifnotpaper
        \citetext{\citealp[p.~375]{IEEE2017}; \citealp[p.~7-10]{SWEBOK2024}}%
    \else \cite[p.~375]{IEEE2017}, \cite[p.~7-10]{SWEBOK2024}%
    \fi, which is distinct from robustness testing \citep[p.~53]{Firesmith2015}.}

%------------------------------------------------------------------------------
% TODOs
%------------------------------------------------------------------------------

% Generic Inlined TODOs
\newcommand{\intodo}[1]{\todo[inline]{#1}}

% Unimportant TODOs for "later" (i.e., finishing touches or changes immediately before submission)
\newcommand{\latertodo}[1]{\todo[backgroundcolor=Cyan]{\textit{Later}: #1}}

% "Important" TODOs
\newcommand{\imptodo}[1]{\todo[inline,backgroundcolor=Red]{\textbf{Important}: #1}}

% "Easy" TODOs
\newcommand{\easytodo}[1]{\todo[inline,backgroundcolor=SeaGreen]{\textit{Easy}: #1}}
\newcommand{\eztodo}[1]{\easytodo{#1}}

% "Tedious" TODOs
\newcommand{\tedioustodo}[1]{\todo[inline,backgroundcolor=PineGreen]{\textit{Needs time}: #1}}

% "Question" TODO Notes
\newcounter{todonoteQuestionsCtr}
\newcommand{\questiontodo}[1]{\stepcounter{todonoteQuestionsCtr}\todo[backgroundcolor=Lavender]{\textbf{Q \#\thetodonoteQuestionsCtr{}}: #1}}
\newcommand{\qtodo}[1]{\questiontodo{#1}}

%------------------------------------------------------------------------------
% Citations
%------------------------------------------------------------------------------

\newcommand{\exhInfCite}{(\citealp[p.~5-5]{SWEBOK2024}; \citealp[p.~4]{IEEE2022};
    \citealp[p.~421]{vanVliet2000}; \citealp[pp.~439, 461]{PetersAndPedrycz2000})}

%------------------------------------------------------------------------------
% Link to Drasil issue
%------------------------------------------------------------------------------

\newcommand{\issueref}[1]{\href{https://github.com/JacquesCarette/Drasil/issues/#1}{\##1}}
\newcommand{\pullref}[1]{\href{https://github.com/JacquesCarette/Drasil/pull/#1}{\##1}}
\newcommand{\thesisissueref}[1]{\href{https://github.com/samm82/TestGen-Thesis/issues/#1}{\##1}}


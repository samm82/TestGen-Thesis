\documentclass{article}
\usepackage{graphicx}
\usepackage[pdf]{graphviz}
\usepackage{tikz}
\usetikzlibrary{arrows,shapes}

\begin{document}
\digraph{approachGraph}{
rankdir=BT;

// Dummy node to push the legend to the top left
start [style="invis"];

ZeroSwitchTesting [label=<0-Switch<br/>Testing>];
OneSwitchTesting [label=<1-Switch<br/>Testing>];
ABTesting [label=<A/B<br/>Testing>];
AbsoluteCorrectnessTesting [label=<Absolute<br/>Correctness<br/>Testing>,style="dashed"];
AcceptanceTesting [label=<Acceptance<br/>Testing>];
AccessControlTesting [label=<Access<br/>Control<br/>Testing>];
AccessibilityTesting [label=<Accessibility<br/>Testing>];
AcquisitionOrganizationTesting [label=<Acquisition<br/>Organization<br/>Testing>];
AdHocReviews [label=<Ad<br/>Hoc<br/>Reviews>];
AdHocTesting [label=<Ad<br/>Hoc<br/>Testing>];
AdaptiveRandomTesting [label=<Adaptive<br/>Random<br/>Testing>];
AdversarialTesting [label=<Adversarial<br/>Testing>];
AgentbasedTesting [label=<Agent-based<br/>Testing>];
AgileTesting [label=<Agile<br/>Testing>];
AJAXTesting [label=<AJAX<br/>Testing>];
AllCombinationsTesting [label=<All<br/>Combinations<br/>Testing>];
AllRulesTesting [label=<All<br/>Rules<br/>Testing>];
AllTransitionsTesting [label=<All<br/>Transitions<br/>Testing>];
AllCUsesTesting [label=<All-C-Uses<br/>Testing>];
AllCUsesSomePUsesTesting [label=<All-C-Uses/Some-P-Uses<br/>Testing>];
AllDefinitionsTesting [label=<All-Definitions<br/>Testing>];
AllDUPathsTesting [label=<All-DU-Paths<br/>Testing>];
AllInputGUITesting [label=<All-Input-GUI<br/>Testing>];
AllPUsesTesting [label=<All-P-Uses<br/>Testing>];
AllPUsesSomeCUsesTesting [label=<All-P-Uses/Some-C-Uses<br/>Testing>];
AllURLTesting [label=<All-URL<br/>Testing>];
AllUsesTesting [label=<All-Uses<br/>Testing>];
AlphaTesting [label=<Alpha<br/>Testing>];
AntiSpoofingTesting [label=<Anti-Spoofing<br/>Testing>];
AntiTamperTesting [label=<Anti-Tamper<br/>Testing>];
APITesting [label=<API<br/>Testing>];
ApplicationSystemTesting [label=<Application<br/>System<br/>Testing>];
ArchitectTesting [label=<Architect<br/>Testing>];
ArchitecturedrivenTesting [label=<Architecture-driven<br/>Testing>];
AssertionChecking [label=<Assertion<br/>Checking>];
AsynchronousTesting [label=<Asynchronous<br/>Testing>];
Attacks [label=<Attacks>];
AttheBeginningTesting [label=<At-the-Beginning<br/>Testing>,style="dashed"];
AttheEndTesting [label=<At-the-End<br/>Testing>];
AudioTesting [label=<Audio<br/>Testing>];
Audits [label=<Audits>];
AutomatedTesting [label=<Automated<br/>Testing>];
AvailabilityTesting [label=<Availability<br/>Testing>];
BacktoBackTesting [label=<Back-to-Back<br/>Testing>];
BackupandRecoveryTesting [label=<Backup<br/>and<br/>Recovery<br/>Testing>];
BackupTesting [label=<Backup<br/>Testing>];
BackupRecoveryTesting [label=<Backup/Recovery<br/>Testing>];
BackwardsCompatibilityTesting [label=<Backwards<br/>Compatibility<br/>Testing>];
BaseChoiceTesting [label=<Base<br/>Choice<br/>Testing>];
BaselineTesting [label=<Baseline<br/>Testing>];
BasicBlockTesting [label=<Basic<br/>Block<br/>Testing>];
BehaviouralTesting [label=<Behavioural<br/>Testing>];
BehaviourAnalysis [label=<Behaviour<br/>Analysis>,style="dashed"];
BetaTesting [label=<Beta<br/>Testing>];
BigBangTesting [label=<Big-Bang<br/>Testing>];
BlockTesting [label=<Block<br/>Testing>,style="dashed"];
BlueTeamTesting [label=<Blue<br/>Team<br/>Testing>];
BottomUpTesting [label=<Bottom-Up<br/>Testing>];
BoundaryConditionTesting [label=<Boundary<br/>Condition<br/>Testing>];
BoundaryValueAnalysis [label=<Boundary<br/>Value<br/>Analysis>];
BranchConditionCombinationTesting [label=<Branch<br/>Condition<br/>Combination<br/>Testing>];
BranchConditionTesting [label=<Branch<br/>Condition<br/>Testing>];
BranchTesting [label=<Branch<br/>Testing>];
BrowserPageTesting [label=<Browser<br/>Page<br/>Testing>];
BuddyTesting [label=<Buddy<br/>Testing>];
BufferOverrunTesting [label=<Buffer<br/>Overrun<br/>Testing>,style="dashed"];
BugHuntTesting [label=<Bug<br/>Hunt<br/>Testing>];
BuildVerificationTesting [label=<Build<br/>Verification<br/>Testing>];
BuiltInTesting [label=<Built-In<br/>Testing>];
BusinessAcceptanceTesting [label=<Business<br/>Acceptance<br/>Testing>];
BusinessProcessbasedTesting [label=<Business<br/>Process-based<br/>Testing>];
CanaryTesting [label=<Canary<br/>Testing>];
CapacityTesting [label=<Capacity<br/>Testing>];
CaptureReplayDrivenTesting [label=<Capture-Replay<br/>Driven<br/>Testing>];
CauseEffectGraphing [label=<Cause-Effect<br/>Graphing>];
Certification [label=<Certification>];
CGIComponentTesting [label=<CGI<br/>Component<br/>Testing>];
ChangeRelatedTesting [label=<Change-Related<br/>Testing>];
CheckedStatementTesting [label=<Checked<br/>Statement<br/>Testing>];
ChecklistbasedReviews [label=<Checklist-based<br/>Reviews>];
ChecklistbasedTesting [label=<Checklist-based<br/>Testing>];
ClassificationTreeMethod [label=<Classification<br/>Tree<br/>Method>];
ClosedBetaTesting [label=<Closed<br/>Beta<br/>Testing>];
ClosedLoopTesting [label=<Closed<br/>Loop<br/>Testing>];
CloudTesting [label=<Cloud<br/>Testing>];
CodeInjection [label=<Code<br/>Injection>];
CodeReviews [label=<Code<br/>Reviews>];
CoexistenceTesting [label=<Co-existence<br/>Testing>];
CombinatorialTesting [label=<Combinatorial<br/>Testing>];
CommandFormTesting [label=<Command-Form<br/>Testing>];
CommandLineInterfaceTesting [label=<Command-Line<br/>Interface<br/>Testing>];
ComparisonTesting [label=<Comparison<br/>Testing>];
CompatibilityTesting [label=<Compatibility<br/>Testing>];
CompilerTesting [label=<Compiler<br/>Testing>];
CompleteRegressionTesting [label=<Complete<br/>Regression<br/>Testing>];
ComplianceTesting [label=<Compliance<br/>Testing>];
ComponentIntegrationTesting [label=<Component<br/>Integration<br/>Testing>];
ConcreteExecution [label=<Concrete<br/>Execution>];
ConcurrencyTesting [label=<Concurrency<br/>Testing>];
ConfigurationTesting [label=<Configuration<br/>Testing>];
ConformanceTesting [label=<Conformance<br/>Testing>];
ConsistencyTesting [label=<Consistency<br/>Testing>];
ConstructionTesting [label=<Construction<br/>Testing>];
ContentChecking [label=<Content<br/>Checking>];
ContentUsageTesting [label=<Content<br/>Usage<br/>Testing>];
ContinuousTesting [label=<Continuous<br/>Testing>];
ContractualAcceptanceTesting [label=<Contractual<br/>Acceptance<br/>Testing>];
ControlFlowAnalysis [label=<Control<br/>Flow<br/>Analysis>];
ControlFlowTesting [label=<Control<br/>Flow<br/>Testing>];
ControlSystemTesting [label=<Control<br/>System<br/>Testing>,style="dashed"];
ConversionTesting [label=<Conversion<br/>Testing>];
CookieTesting [label=<Cookie<br/>Testing>];
CorrectnessTesting [label=<Correctness<br/>Testing>];
COTSTesting [label=<COTS<br/>Testing>];
COTSVendorTesting [label=<COTS<br/>Vendor<br/>Testing>];
CoveragebasedTesting [label=<Coverage-based<br/>Testing>];
CrossBrowserCompatibilityTesting [label=<Cross-Browser<br/>Compatibility<br/>Testing>];
CrowdTesting [label=<Crowd<br/>Testing>];
CustomerAcceptanceTesting [label=<Customer<br/>Acceptance<br/>Testing>];
DarkLaunches [label=<Dark<br/>Launches>];
DataCenterTesting [label=<Data<br/>Center<br/>Testing>];
DataDependenceTransitionRelationTesting [label=<Data<br/>Dependence<br/>Transition<br/>Relation<br/>Testing>];
DataFlowAnalysis [label=<Data<br/>Flow<br/>Analysis>];
DataFlowTesting [label=<Data<br/>Flow<br/>Testing>];
DataIntegrityTesting [label=<Data<br/>Integrity<br/>Testing>];
DataMigrationTesting [label=<Data<br/>Migration<br/>Testing>];
DataTesting [label=<Data<br/>Testing>];
DatabaseAdminTesting [label=<Database<br/>Admin<br/>Testing>];
DatabaseCoverageTesting [label=<Database<br/>Coverage<br/>Testing>,style="dashed"];
DatabaseIntegrityTesting [label=<Database<br/>Integrity<br/>Testing>];
DatadrivenTesting [label=<Data-driven<br/>Testing>];
DecisionConditionTesting [label=<Decision<br/>Condition<br/>Testing>];
DecisionTableTesting [label=<Decision<br/>Table<br/>Testing>];
DecisionTesting [label=<Decision<br/>Testing>];
DefectbasedTesting [label=<Defect-based<br/>Testing>];
DenialofService [label=<Denial<br/>of<br/>Service>];
DesignbasedTesting [label=<Design-based<br/>Testing>];
DesigndrivenTesting [label=<Design-driven<br/>Testing>];
DeskChecking [label=<Desk<br/>Checking>];
DesktopDevelopmentTesting [label=<Desktop<br/>Development<br/>Testing>];
DeterministicTesting [label=<Deterministic<br/>Testing>,style="dashed"];
DevSecOpsTesting [label=<Dev/Sec/Ops<br/>Testing>,style="dashed"];
DeveloperTesting [label=<Developer<br/>Testing>];
DevelopmentEnvironmentTesting [label=<Development<br/>Environment<br/>Testing>];
DevelopmentOrganizationTesting [label=<Development<br/>Organization<br/>Testing>];
DevelopmentTesting [label=<Development<br/>Testing>];
DevelopmentToolTesting [label=<Development<br/>Tool<br/>Testing>];
DevicebasedTesting [label=<Device-based<br/>Testing>];
DevOpsTesting [label=<DevOps<br/>Testing>];
DifferentialAssertionChecking [label=<Differential<br/>Assertion<br/>Checking>];
DisasterRecoveryTesting [label=<Disaster/Recovery<br/>Testing>];
DistributedTesting [label=<Distributed<br/>Testing>];
DOMTesting [label=<DOM<br/>Testing>];
DomainAnalysis [label=<Domain<br/>Analysis>,style="dashed"];
DomainBasedTesting [label=<Domain-Based<br/>Testing>];
DomainIndependentTesting [label=<Domain-Independent<br/>Testing>];
DomainSpecificTesting [label=<Domain-Specific<br/>Testing>];
DTOrganizationTesting [label=<DT<br/>Organization<br/>Testing>];
DynamicAnalysis [label=<Dynamic<br/>Analysis>];
DynamicTesting [label=<Dynamic<br/>Testing>];
EachChoiceTesting [label=<Each<br/>Choice<br/>Testing>];
EfficiencyTesting [label=<Efficiency<br/>Testing>];
ElasticityTesting [label=<Elasticity<br/>Testing>];
ElementaryComparisonTesting [label=<Elementary<br/>Comparison<br/>Testing>];
EmbeddedTesterTesting [label=<Embedded<br/>Tester<br/>Testing>];
EncryptionTesting [label=<Encryption<br/>Testing>];
EndtoendFunctionalityTesting [label=<End-to-end<br/>Functionality<br/>Testing>];
EndtoendTesting [label=<End-to-end<br/>Testing>];
EnduranceTesting [label=<Endurance<br/>Testing>];
EquivalenceChecking [label=<Equivalence<br/>Checking>];
EquivalencePartitioning [label=<Equivalence<br/>Partitioning>];
ErgonomicsTesting [label=<Ergonomics<br/>Testing>];
ErrorForcing [label=<Error<br/>Forcing>];
ErrorGuessing [label=<Error<br/>Guessing>];
ErrorSeeding [label=<Error<br/>Seeding>];
ErrorToleranceTesting [label=<Error<br/>Tolerance<br/>Testing>];
ErrorbasedTesting [label=<Error-based<br/>Testing>];
EventSpaceTesting [label=<Event<br/>Space<br/>Testing>];
EvidencebasedTesting [label=<Evidence-based<br/>Testing>];
ExhaustiveTesting [label=<Exhaustive<br/>Testing>];
ExperiencebasedTesting [label=<Experience-based<br/>Testing>];
ExpertUsabilityReviews [label=<Expert<br/>Usability<br/>Reviews>];
ExploratoryTesting [label=<Exploratory<br/>Testing>];
ExtendedEntryTableTesting [label=<Extended<br/>Entry<br/>Table<br/>Testing>];
ExternalLinksIntegrationTesting [label=<External<br/>Links<br/>Integration<br/>Testing>,style="dashed"];
ExtremeValueAnalysis [label=<Extreme<br/>Value<br/>Analysis>,style="dashed"];
FactoryAcceptanceTesting [label=<Factory<br/>Acceptance<br/>Testing>];
FailoverTesting [label=<Failover<br/>Testing>];
FailoverRecoveryTesting [label=<Failover/Recovery<br/>Testing>];
FailureToleranceTesting [label=<Failure<br/>Tolerance<br/>Testing>];
FaultInjectionTesting [label=<Fault<br/>Injection<br/>Testing>];
FaultToleranceTesting [label=<Fault<br/>Tolerance<br/>Testing>];
FaultTreeAnalysis [label=<Fault<br/>Tree<br/>Analysis>];
FaultbasedTesting [label=<Fault-based<br/>Testing>];
FeaturebasedTesting [label=<Feature-based<br/>Testing>];
FeaturesTesting [label=<Features<br/>Testing>];
FieldTesting [label=<Field<br/>Testing>];
FlexibilityTesting [label=<Flexibility<br/>Testing>];
FollowonOperationalTesting [label=<Follow-on<br/>Operational<br/>Testing>];
ForcingExceptionTesting [label=<Forcing<br/>Exception<br/>Testing>];
FormalMethods [label=<Formal<br/>Methods>];
FormalModularVerification [label=<Formal<br/>Modular<br/>Verification>];
FormalReviews [label=<Formal<br/>Reviews>];
FormalTesting [label=<Formal<br/>Testing>];
FormativeEvaluations [label=<Formative<br/>Evaluations>];
FullConformanceTesting [label=<Full<br/>Conformance<br/>Testing>,style="dashed"];
FunctionalSuitabilityTesting [label=<Functional<br/>Suitability<br/>Testing>];
FunctionalTesting [label=<Functional<br/>Testing>];
FunctionalityTesting [label=<Functionality<br/>Testing>];
FunctionsTesting [label=<Functions<br/>Testing>,style="dashed"];
FuzzTesting [label=<Fuzz<br/>Testing>];
Galumphing [label=<Galumphing>];
GraphicalUserInterfaceTesting [label=<Graphical<br/>User<br/>Interface<br/>Testing>];
GrayBoxTesting [label=<Gray-Box<br/>Testing>];
GroupTesting [label=<Group<br/>Testing>];
GUITesting [label=<GUI<br/>Testing>];
Heartbeat [label=<Heartbeat>];
HeuristicEvaluations [label=<Heuristic<br/>Evaluations>];
HighFrequencyTesting [label=<High<br/>Frequency<br/>Testing>];
HighLevelTesting [label=<High-Level<br/>Testing>];
HumanFactorsEngineerTesting [label=<Human<br/>Factors<br/>Engineer<br/>Testing>];
HumanintheLoopTesting [label=<Human-in-the-Loop<br/>Testing>];
HyperlinkTesting [label=<Hyperlink<br/>Testing>];
HypothesisTesting [label=<Hypothesis<br/>Testing>];
IncontainerTesting [label=<In-container<br/>Testing>];
IncrementalTesting [label=<Incremental<br/>Testing>];
IndependentTestOrganizationTesting [label=<Independent<br/>Test<br/>Organization<br/>Testing>];
IndependentTesterTesting [label=<Independent<br/>Tester<br/>Testing>];
IndividualTesting [label=<Individual<br/>Testing>];
InductiveAssertionMethods [label=<Inductive<br/>Assertion<br/>Methods>];
IndustrialWebApplicationTesting [label=<Industrial<br/>Web<br/>Application<br/>Testing>];
InformalReviews [label=<Informal<br/>Reviews>];
InformalTesting [label=<Informal<br/>Testing>];
InfrastructureCompatibilityTesting [label=<Infrastructure<br/>Compatibility<br/>Testing>];
InfrastructureTesting [label=<Infrastructure<br/>Testing>];
InitialOperationalTesting [label=<Initial<br/>Operational<br/>Testing>];
InputDataTesting [label=<Input<br/>Data<br/>Testing>];
InputValidationTesting [label=<Input<br/>Validation<br/>Testing>];
InputParameterTesting [label=<Input-Parameter<br/>Testing>];
InsourcedTesting [label=<Insourced<br/>Testing>];
CodeInspections [label=<(Code)<br/>Inspections>];
InstallabilityTesting [label=<Installability<br/>Testing>];
InstallationTesting [label=<Installation<br/>Testing>];
IntakeTesting [label=<Intake<br/>Testing>];
IntegratedSystemTesting [label=<Integrated<br/>System<br/>Testing>];
IntegrationTesting [label=<Integration<br/>Testing>];
IntegrityTesting [label=<Integrity<br/>Testing>,style="dashed"];
InterfaceTesting [label=<Interface<br/>Testing>];
InternationalizationTesting [label=<Internationalization<br/>Testing>];
InteroperabilityTesting [label=<Interoperability<br/>Testing>];
InterruptdrivenBuiltInTesting [label=<Interrupt-driven<br/>Built-In<br/>Testing>];
IsolationTesting [label=<Isolation<br/>Testing>];
KeyworddrivenTesting [label=<Keyword-driven<br/>Testing>];
LargeScaleIntegrationTesting [label=<Large<br/>Scale<br/>Integration<br/>Testing>];
LayerbasedTesting [label=<Layer-based<br/>Testing>];
LegacySystemIntegrationTesting [label=<Legacy<br/>System<br/>Integration<br/>Testing>,style="dashed"];
LegacyTesting [label=<Legacy<br/>Testing>];
LicenseComplianceAudits [label=<License<br/>Compliance<br/>Audits>];
LifecyclebasedTesting [label=<Lifecycle-based<br/>Testing>];
LinearScripting [label=<Linear<br/>Scripting>];
LinkChecking [label=<Link<br/>Checking>];
LinkDependenceTransitionRelationTesting [label=<Link<br/>Dependence<br/>Transition<br/>Relation<br/>Testing>];
LoadBalancingTesting [label=<Load<br/>Balancing<br/>Testing>,style="dashed"];
LoadTesting [label=<Load<br/>Testing>];
LocalTesting [label=<Local<br/>Testing>];
LocalizationTesting [label=<Localization<br/>Testing>];
LoopTesting [label=<Loop<br/>Testing>];
LoopbackTesting [label=<Loopback<br/>Testing>];
LowLevelTesting [label=<Low-Level<br/>Testing>];
MachineLearningAssistedTesting [label=<Machine<br/>Learning-Assisted<br/>Testing>,style="dashed"];
MaintainabilityTesting [label=<Maintainability<br/>Testing>];
MaintenanceTesting [label=<Maintenance<br/>Testing>];
MalwareScanning [label=<Malware<br/>Scanning>];
ManualTesting [label=<Manual<br/>Testing>];
MarkovChainTesting [label=<Markov<br/>Chain<br/>Testing>,style="dashed"];
MathematicalbasedTesting [label=<Mathematical-based<br/>Testing>];
MCDCTesting [label=<MC/DC<br/>Testing>];
MemoryManagementTesting [label=<Memory<br/>Management<br/>Testing>];
MenuItemTesting [label=<Menu<br/>Item<br/>Testing>,style="dashed"];
MetamorphicTesting [label=<Metamorphic<br/>Testing>];
MethodTesting [label=<Method<br/>Testing>,style="dashed"];
MinimizedTesting [label=<Minimized<br/>Testing>,style="dashed"];
MigrationTesting [label=<Migration<br/>Testing>];
MixedEntryTableTesting [label=<Mixed<br/>Entry<br/>Table<br/>Testing>];
MLModelTesting [label=<ML<br/>Model<br/>Testing>];
FlashMobTesting [label=<(Flash)<br/>Mob<br/>Testing>];
MobileTesting [label=<Mobile<br/>Testing>];
ModelVerification [label=<Model<br/>Verification>];
ModelbasedTesting [label=<Model-based<br/>Testing>];
MonkeyTesting [label=<Monkey<br/>Testing>];
MultiplayerTesting [label=<Multiplayer<br/>Testing>];
MultipleHitDecisionTableTesting [label=<Multiple-Hit<br/>Decision<br/>Table<br/>Testing>];
MultiUserTesting [label=<Multi-User<br/>Testing>];
MutationTesting [label=<Mutation<br/>Testing>];
NeedsDrivenTesting [label=<Needs-Driven<br/>Testing>];
NegativeTesting [label=<Negative<br/>Testing>];
NeighborhoodIntegrationTesting [label=<Neighborhood<br/>Integration<br/>Testing>];
NetworkAdminTesting [label=<Network<br/>Admin<br/>Testing>];
NetworkTrafficTesting [label=<Network<br/>Traffic<br/>Testing>];
NeuronCoverageTesting [label=<Neuron<br/>Coverage<br/>Testing>,style="dashed"];
NonfunctionalTesting [label=<Non-functional<br/>Testing>];
NSwitchTesting [label=<N-Switch<br/>Testing>];
ObjectbasedTesting [label=<Object-based<br/>Testing>];
ObjectOrientedTesting [label=<Object-Oriented<br/>Testing>];
OfflineMBT [label=<Offline<br/>MBT>];
OfflineTesting [label=<Offline<br/>Testing>];
OnetoOneTesting [label=<One-to-One<br/>Testing>,style="dashed"];
OngoingBuiltInTesting [label=<Ongoing<br/>Built-In<br/>Testing>];
OnlineMBT [label=<Online<br/>MBT>];
OnlineTesting [label=<Online<br/>Testing>];
OOWebTesting [label=<OO<br/>Web<br/>Testing>];
OpenBetaTesting [label=<Open<br/>Beta<br/>Testing>];
OpenLoopTesting [label=<Open<br/>Loop<br/>Testing>];
OpenSourceTesting [label=<Open<br/>Source<br/>Testing>];
OperationalTesting [label=<Operational<br/>(Acceptance)<br/>Testing>];
OperationalEffectivenessTesting [label=<Operational<br/>Effectiveness<br/>Testing>];
OperationalProfileTesting [label=<Operational<br/>Profile<br/>Testing>];
OperationalSuitabilityTesting [label=<Operational<br/>Suitability<br/>Testing>];
OperationsOrganizationTesting [label=<Operations<br/>Organization<br/>Testing>];
OperatorTesting [label=<Operator<br/>Testing>];
OrganizationbasedTesting [label=<Organization-based<br/>Testing>];
OrthogonalArrayTesting [label=<Orthogonal<br/>Array<br/>Testing>];
OTOrganizationTesting [label=<OT<br/>Organization<br/>Testing>];
OutsideInTesting [label=<Outside-In<br/>Testing>];
OutsourcedTesting [label=<Outsourced<br/>Testing>];
PageTesting [label=<Page<br/>Testing>];
PairTesting [label=<Pair<br/>Testing>];
PairwiseIntegrationTesting [label=<Pairwise<br/>Integration<br/>Testing>];
PairwiseTesting [label=<Pairwise<br/>Testing>];
PartialRegressionTesting [label=<Partial<br/>Regression<br/>Testing>];
PasswordCracking [label=<Password<br/>Cracking>];
PathTesting [label=<Path<br/>Testing>];
PatternsbasedTesting [label=<Patterns-based<br/>Testing>];
PeerReviews [label=<Peer<br/>Reviews>];
PenetrationTesting [label=<Penetration<br/>Testing>];
PerformanceEfficiencyTesting [label=<Performance<br/>Efficiency<br/>Testing>];
PerformanceTesting [label=<Performance<br/>Testing>];
PerformancerelatedTesting [label=<Performance-related<br/>Testing>];
PeriodicBuiltInTesting [label=<Periodic<br/>Built-In<br/>Testing>];
PersonalizationTesting [label=<Personalization<br/>Testing>];
Pharming [label=<Pharming>];
PhysicalConfigurationAudits [label=<Physical<br/>Configuration<br/>Audits>];
PlayerPerspectiveTesting [label=<Player<br/>Perspective<br/>Testing>];
Playtesting [label=<Playtesting>];
PortabilityTesting [label=<Portability<br/>Testing>];
PositiveTesting [label=<Positive<br/>Testing>];
PostDeploymentMonitoring [label=<Post-Deployment<br/>Monitoring>];
PostReleaseTesting [label=<Post-Release<br/>Testing>];
PowerTesting [label=<Power<br/>Testing>,style="dashed"];
PowerUpBuiltInTesting [label=<Power-Up<br/>Built-In<br/>Testing>];
PrimeContractorTesting [label=<Prime<br/>Contractor<br/>Testing>];
PrimePathTesting [label=<Prime<br/>Path<br/>Testing>,style="dashed"];
PrioritizationTesting [label=<Prioritization<br/>Testing>];
PrivacyTesting [label=<Privacy<br/>Testing>];
ProcedureTesting [label=<Procedure<br/>Testing>];
ProcessDrivenScripting [label=<Process-Driven<br/>Scripting>];
ProcessorintheLoopTesting [label=<Processor-in-the-Loop<br/>Testing>];
ProductLinesTesting [label=<Product<br/>Lines<br/>Testing>];
ProductionAcceptanceTesting [label=<Production<br/>Acceptance<br/>Testing>];
ProductionVerificationTesting [label=<Production<br/>Verification<br/>Testing>];
PrognosticsandHealthManagement [label=<Prognostics<br/>and<br/>Health<br/>Management>];
ProgrammerTesting [label=<Programmer<br/>Testing>];
ProofsofCorrectness [label=<Proofs<br/>of<br/>Correctness>];
ProofsofPartialCorrectness [label=<Proofs<br/>of<br/>Partial<br/>Correctness>,style="dashed"];
ProofsofTotalCorrectness [label=<Proofs<br/>of<br/>Total<br/>Correctness>,style="dashed"];
ProtectionSystemTesting [label=<Protection<br/>System<br/>Testing>,style="dashed"];
QualificationOperationalTesting [label=<Qualification<br/>Operational<br/>Testing>];
QualificationTesting [label=<Qualification<br/>Testing>];
QuickTesting [label=<Quick<br/>Testing>];
RandomTesting [label=<Random<br/>Testing>];
RandomWalkTesting [label=<Random<br/>Walk<br/>Testing>];
RapidPrototypingTesting [label=<Rapid<br/>Prototyping<br/>Testing>];
ReactiveTesting [label=<Reactive<br/>Testing>];
RecoverabilityTesting [label=<Recoverability<br/>Testing>];
RecoveryTesting [label=<Recovery<br/>Testing>];
RedTeamTesting [label=<Red<br/>Team<br/>Testing>];
RegressionTesting [label=<Regression<br/>Testing>];
RegulatoryAcceptanceTesting [label=<Regulatory<br/>Acceptance<br/>Testing>];
RelativeCorrectnessTesting [label=<Relative<br/>Correctness<br/>Testing>,style="dashed"];
ReliabilityEnhancementTesting [label=<Reliability<br/>Enhancement<br/>Testing>];
ReliabilityGrowthTesting [label=<Reliability<br/>Growth<br/>Testing>];
ReliabilityMechanismTesting [label=<Reliability<br/>Mechanism<br/>Testing>];
ReliabilityTesting [label=<Reliability<br/>Testing>];
RemoteTesting [label=<Remote<br/>Testing>];
RepetitionTesting [label=<Repetition<br/>Testing>];
RequestTesting [label=<Request<br/>Testing>];
RequirementsbasedTesting [label=<Requirement(s)-based<br/>Testing>];
RequirementsAnimation [label=<Requirements<br/>Animation>];
RequirementsEngineerTesting [label=<Requirements<br/>Engineer<br/>Testing>];
RequirementsdrivenTesting [label=<Requirements-driven<br/>Testing>];
ResourceUtilizationTesting [label=<Resource<br/>Utilization<br/>Testing>];
ResponseTimeTesting [label=<Response-Time<br/>Testing>];
Retesting [label=<Retesting>];
ReuseTesting [label=<Reuse<br/>Testing>];
Reviews [label=<Reviews>];
ReWebTesting [label=<ReWeb<br/>Testing>];
RiskbasedTesting [label=<Risk-based<br/>Testing>];
RobustnessTesting [label=<Robustness<br/>Testing>];
RolebasedReviews [label=<Role-based<br/>Reviews>];
RolebasedTesting [label=<Role-based<br/>Testing>];
RuntimeAssertionChecking [label=<Runtime<br/>Assertion<br/>Checking>];
SafetyDemonstrations [label=<Safety<br/>Demonstrations>];
SafetyEngineerTesting [label=<Safety<br/>Engineer<br/>Testing>];
SafetyTesting [label=<Safety<br/>Testing>];
SandwichTesting [label=<Sandwich<br/>Testing>];
ScalabilityTesting [label=<Scalability<br/>Testing>];
ScenarioTesting [label=<Scenario<br/>Testing>];
ScenarioWalkthroughs [label=<Scenario<br/>Walkthroughs>];
ScenariobasedEvaluations [label=<Scenario-based<br/>Evaluations>];
ScenariobasedTesting [label=<Scenario-based<br/>Testing>];
ScenariobasedReviews [label=<Scenario-based<br/>Reviews>];
ScriptedTesting [label=<Scripted<br/>Testing>];
SecurityAttacks [label=<Security<br/>Attacks>];
SecurityAudits [label=<Security<br/>Audits>];
SecurityEngineerTesting [label=<Security<br/>Engineer<br/>Testing>];
SecurityTesting [label=<Security<br/>Testing>];
SelfTesting [label=<Self-Testing>,style="dashed"];
SessionbasedTesting [label=<Session-based<br/>Testing>];
ShiftLeftTesting [label=<Shift-Left<br/>Testing>];
ShoeTesting [label=<Shoe<br/>Testing>];
ShutdownBuiltInTesting [label=<Shutdown<br/>Built-In<br/>Testing>];
SignChangeTesting [label=<Sign<br/>Change<br/>Testing>];
SignSignTesting [label=<Sign-Sign<br/>Testing>];
SimilaritybasedPrioritizationTesting [label=<Similarity-based<br/>Prioritization<br/>Testing>];
SingleHitDecisionTable [label=<Single-Hit<br/>Decision<br/>Table>];
SiteAcceptanceTesting [label=<Site<br/>Acceptance<br/>Testing>];
Slicing [label=<Slicing>,style="dashed"];
SmokeTesting [label=<Smoke<br/>Testing>];
SOATesting [label=<SOA<br/>Testing>];
SoftwareDesignAudits [label=<Software<br/>Design<br/>Audits>];
SoftwareInteractionTesting [label=<Software<br/>Interaction<br/>Testing>];
SoftwareQualificationTesting [label=<Software<br/>Qualification<br/>Testing>];
SoftwareintheLoopTesting [label=<Software-in-the-Loop<br/>Testing>];
SoSIntegrationTesting [label=<SoS<br/>Integration<br/>Testing>];
SoSTesting [label=<SoS<br/>Testing>];
SpecificationbasedTesting [label=<Specification-based<br/>Testing>];
SpikeTesting [label=<Spike<br/>Testing>];
SpiralTesting [label=<Spiral<br/>Testing>];
SQLInjection [label=<SQL<br/>Injection>];
StabilityTesting [label=<Stability<br/>Testing>];
StateTesting [label=<State<br/>Testing>];
StateTransitionTesting [label=<State<br/>Transition<br/>Testing>];
StatebasedWebBrowserTesting [label=<State-based<br/>Web<br/>Browser<br/>Testing>];
StatementTesting [label=<Statement<br/>Testing>];
StaticAnalysis [label=<Static<br/>Analysis>];
StaticAssertionChecking [label=<Static<br/>Assertion<br/>Checking>];
StaticTesting [label=<Static<br/>Testing>];
StatisticalTesting [label=<Statistical<br/>Testing>];
StepwiseAbstraction [label=<Stepwise<br/>Abstraction>];
StressTesting [label=<Stress<br/>Testing>];
StrongMutationTesting [label=<Strong<br/>Mutation<br/>Testing>];
StructuralAnalysis [label=<Structural<br/>Analysis>];
StructuralTesting [label=<Structural<br/>Testing>];
StructurebasedTesting [label=<Structure-based<br/>Testing>];
StructuredScripting [label=<Structured<br/>Scripting>];
StructuredTesting [label=<Structured<br/>Testing>];
StructuredWalkthroughs [label=<Structured<br/>Walkthroughs>];
StuckKeyTesting [label=<Stuck<br/>Key<br/>Testing>];
SubboundaryConditionTesting [label=<Sub-boundary<br/>Condition<br/>Testing>];
SubcontractorTesting [label=<Subcontractor<br/>Testing>];
SubsystemTesting [label=<Subsystem<br/>Testing>];
SummativeEvaluations [label=<Summative<br/>Evaluation(s)>];
SymbolicExecution [label=<Symbolic<br/>Execution>];
SynchronousTesting [label=<Synchronous<br/>Testing>];
SyntaxTesting [label=<Syntax<br/>Testing>];
SysAdminTesting [label=<Sys<br/>Admin<br/>Testing>];
SystemIntegrationTesting [label=<System<br/>Integration<br/>Testing>];
SystemQualificationTesting [label=<System<br/>Qualification<br/>Testing>];
SystemTesting [label=<System<br/>Testing>];
SystemsIntegrationTesting [label=<Systems<br/>Integration<br/>Testing>];
TailoredConformanceTesting [label=<Tailored<br/>Conformance<br/>Testing>,style="dashed"];
TechnicalReviews [label=<Technical<br/>Reviews>];
TechnicalTesting [label=<Technical<br/>Testing>,style="dashed"];
TemplateVariableTesting [label=<Template<br/>Variable<br/>Testing>];
TestBrowsing [label=<Test<br/>Browsing>];
TestEnvironmentTesting [label=<Test<br/>Environment<br/>Testing>];
TestToolTesting [label=<Test<br/>Tool<br/>Testing>];
TestdrivenDevelopment [label=<Test-driven<br/>Development>];
TesterTesting [label=<Tester<br/>Testing>];
TestUMLTesting [label=<TestUML<br/>Testing>];
TestWebTesting [label=<TestWeb<br/>Testing>];
ThinkAloudUsabilityTesting [label=<Think<br/>Aloud<br/>Usability<br/>Testing>];
ThreeValueBoundaryTesting [label=<Three-Value<br/>Boundary<br/>Testing>];
ThresholdTesting [label=<Threshold<br/>Testing>];
TimingTesting [label=<Timing<br/>Testing>,style="dashed"];
TopDownTesting [label=<Top-Down<br/>Testing>];
Tours [label=<Tours>];
TransactionFlowTesting [label=<Transaction<br/>Flow<br/>Testing>];
TransactionTesting [label=<Transaction<br/>Testing>];
TransactionVerification [label=<Transaction<br/>Verification>];
TranslationValidation [label=<Translation<br/>Validation>];
twiseTesting [label=<t-wise<br/>Testing>];
TwoValueBoundaryTesting [label=<Two-Value<br/>Boundary<br/>Testing>];
UITesting [label=<UI<br/>Testing>];
UMLModelbasedTesting [label=<UML<br/>Model-based<br/>Testing>];
UnitTesting [label=<Unit<br/>Testing>];
UnscriptedTesting [label=<Unscripted<br/>Testing>];
UsabilityTestScripting [label=<Usability<br/>Test<br/>Script(ing)>];
UsabilityTesting [label=<Usability<br/>Testing>];
UsabilityReviews [label=<Usability<br/>Reviews>];
UsabilityWalkthroughs [label=<Usability<br/>Walkthroughs>];
UsagebasedTesting [label=<Usage-based<br/>Testing>];
UseCaseTesting [label=<Use<br/>Case<br/>Testing>];
UserAcceptanceTesting [label=<User<br/>Acceptance<br/>Testing>];
UserasTesterTesting [label=<User<br/>as<br/>Tester<br/>Testing>];
UserInterfaceNavigationTesting [label=<User<br/>Interface<br/>Navigation<br/>Testing>];
UserOrganizationTesting [label=<User<br/>Organization<br/>Testing>];
UserSessionDataTesting [label=<User<br/>Session<br/>Data<br/>Testing>];
UserSessionTesting [label=<User<br/>Session<br/>Testing>];
UserStoryTesting [label=<User<br/>Story<br/>Testing>];
UserSurveys [label=<User<br/>Surveys>];
UserTesting [label=<User<br/>Testing>];
UserAgentBasedTesting [label=<User-Agent<br/>Based<br/>Testing>];
UserbasedEvaluations [label=<User-based<br/>Evaluations>];
UserinitiatedBuiltInTesting [label=<User-initiated<br/>Built-In<br/>Testing>];
UsersessionbasedTesting [label=<User-session-based<br/>Testing>];
ValidationTesting [label=<Validation<br/>Testing>];
VerificationTesting [label=<Verification<br/>Testing>];
VisualBrowserValidation [label=<Visual<br/>Browser<br/>Validation>];
VisualTesting [label=<Visual<br/>Testing>];
VModelTesting [label=<V-Model<br/>Testing>];
VolumeTesting [label=<Volume<br/>Testing>];
VulnerabilityScanning [label=<Vulnerability<br/>Scanning>];
Walkthroughs [label=<Walkthroughs>];
WaterfallTesting [label=<Waterfall<br/>Testing>];
WeakMutationTesting [label=<Weak<br/>Mutation<br/>Testing>];
WebApplicationTesting [label=<Web<br/>Application<br/>Testing>];
WebAppSlicing [label=<WebApp<br/>Slicing>];
WModelTesting [label=<W-Model<br/>Testing>];

AcceptanceTesting -> QualificationTesting[dir=none,color="blue"];
BackupandRecoveryTesting -> BackupRecoveryTesting[dir=none,style="dashed",color="green"];
BehaviouralTesting -> FunctionalTesting[dir=none,style="dashed",color="maroon"];
CorrectnessTesting -> FunctionalTesting[dir=none,style="dashed",color="blue"];
BetaTesting -> UserTesting[dir=none,style="dashed",color="blue"];
BranchConditionCombinationTesting -> DecisionTesting[dir=none,color="blue"];
ConditionTesting [label=<Condition<br/>Testing>,style="dotted"];
BranchConditionCombinationTesting -> ConditionTesting[dir=none,color="maroon"];
BranchConditionTesting -> ConditionTesting[dir=none,color="blue"];
DecisionTesting -> ConditionTesting[dir=none,color="blue"];
BranchConditionCombinationTesting -> ExhaustiveTesting[dir=none,style="dashed",color="maroon"];
PathTesting -> ExhaustiveTesting[dir=none,color="maroon"];
LinkTesting [label=<Link<br/>Testing>,style="dotted"];
BranchTesting -> LinkTesting[dir=none,style="dashed",color="green"];
ComponentIntegrationTesting -> LinkTesting[dir=none,style="dashed"];
IntegrationTesting -> LinkTesting[dir=none,style="dashed"];
BuiltInTesting -> SelfTesting[dir=none,style="dashed",color="blue"];
CapacityTesting -> ScalabilityTesting[dir=none,color="green"];
CompatibilityTesting -> CoexistenceTesting[dir=none,color="green"];
ConfigurationTesting -> PortabilityTesting[dir=none,style="dashed"];
ConformanceTesting -> CorrectnessTesting[dir=none,style="dashed",color="gray"];
ConversionTesting -> MigrationTesting[dir=none,style="dashed"];
FunctionalTesting -> ConformanceTesting[dir=none,color="blue"];
DataTesting -> DatadrivenTesting[dir=none,style="dashed",color="gray"];
DynamicTesting -> DynamicAnalysis[dir=none,style="dashed",color="green"];
SoakTesting [label=<Soak<br/>Testing>,style="dotted"];
EnduranceTesting -> SoakTesting[dir=none,color="green"];
ReliabilityTesting -> SoakTesting[dir=none];
InvalidTesting [label=<Invalid<br/>Testing>,style="dotted"];
ErrorToleranceTesting -> InvalidTesting[dir=none];
NegativeTesting -> InvalidTesting[dir=none,color="blue"];
NegativeTesting -> InvalidTesting[dir=none,style="dashed",color="green"];
ErrorbasedTesting -> FaultbasedTesting[dir=none,style="dashed",color="gray"];
FailoverTesting -> FailoverRecoveryTesting[dir=none,style="dashed",color="blue"];
FaultToleranceTesting -> RobustnessTesting[dir=none,color="blue"];
FieldTesting -> OperationalTesting[dir=none,style="dashed",color="gray"];
QualificationTesting -> OperationalTesting[dir=none,style="dashed",color="gray"];
TestingtoFail [label=<Testing-to-Fail>,style="dotted"];
ForcingExceptionTesting -> TestingtoFail[dir=none,style="dashed",color="maroon"];
NegativeTesting -> TestingtoFail[dir=none,color="maroon"];
ForcingExceptionTesting -> ErrorForcing[dir=none,style="dashed",color="maroon"];
FunctionalTesting -> SpecificationbasedTesting[dir=none,color="green"];
FunctionalityTesting -> FunctionalSuitabilityTesting[dir=none,style="dashed",color="blue"];
HyperlinkTesting -> LinkChecking[dir=none,style="dashed",color="gray"];
IntegratedSystemTesting -> SystemsIntegrationTesting[dir=none,style="dashed",color="gray"];
InternationalizationTesting -> LocalizationTesting[dir=none,style="dashed"];
OperationalTesting -> ProductionAcceptanceTesting[dir=none,color="blue"];
ProductionVerificationTesting -> ProductionAcceptanceTesting[dir=none,style="dashed",color="gray"];
OrganizationbasedTesting -> RolebasedTesting[dir=none,style="dashed",color="gray"];
OrthogonalArrayTesting -> PairwiseTesting[dir=none,color="blue"];
PerformanceTesting -> PerformancerelatedTesting[dir=none];
PortabilityTesting -> FlexibilityTesting[dir=none,color="green"];
ProgrammerTesting -> DeveloperTesting[dir=none,style="dashed",color="gray"];
RecoverabilityTesting -> RecoveryTesting[dir=none];
Reviews -> StructuralAnalysis[dir=none,style="dashed",color="maroon"];
ScenarioTesting -> UseCaseTesting[dir=none,color="blue"];
UserScenarioTesting [label=<User<br/>Scenario<br/>Testing>,style="dotted"];
ScenarioTesting -> UserScenarioTesting[dir=none,color="blue"];
UseCaseTesting -> UserScenarioTesting[dir=none];
UseCaseTesting -> UserScenarioTesting[dir=none,style="dashed",color="green"];
ScenariobasedEvaluations -> ScenariobasedTesting[dir=none,style="dashed",color="gray"];
SmokeTesting -> BuildVerificationTesting[dir=none,color="blue"];
SmokeTesting -> IntakeTesting[dir=none,color="blue"];
StatebasedTesting [label=<State-based<br/>Testing>,style="dotted"];
StateTransitionTesting -> StatebasedTesting[dir=none,style="dashed",color="blue"];
StatebasedWebBrowserTesting -> StatebasedTesting[dir=none,color="blue"];
StateTransitionTesting -> AllTransitionsTesting[dir=none,style="dashed"];
StaticVerification [label=<Static<br/>Verification>,style="dotted"];
StaticAssertionChecking -> StaticVerification[dir=none];
StaticTesting -> StaticVerification[dir=none,style="dashed"];
StructurebasedTesting -> StructuralTesting[dir=none,color="green"];
SystemQualificationTesting -> SystemTesting[dir=none,style="dashed",color="gray"];
SystemsIntegrationTesting -> LargeScaleIntegrationTesting[dir=none];
TransactionFlowTesting -> TransactionTesting[dir=none,style="dashed"];
UserSessionTesting -> UsersessionbasedTesting[dir=none,style="dashed",color="gray"];
Walkthroughs -> StructuredWalkthroughs[dir=none,color="blue"];

ZeroSwitchTesting -> StateTransitionTesting[color="green"];
OneSwitchTesting -> NSwitchTesting[color="green"];
ABTesting -> StatisticalTesting[color="green"];
ABTesting -> UsabilityTesting[color="blue"];
AbsoluteCorrectnessTesting -> CorrectnessTesting[color="gray"];
AcceptanceTesting -> UsabilityTesting[style="dashed",color="maroon"];
AcceptanceTesting -> FormalTesting;
AcceptanceTesting -> VModelTesting;
AcceptanceTesting -> WModelTesting;
AcceptanceTesting -> SystemTesting[style="dashed",color="maroon"];
AcceptanceTesting -> IntegrationTesting[style="dashed"];
AccessControlTesting -> SecurityTesting[color="blue"];
AccessibilityTesting -> UsabilityTesting[color="green"];
AccessibilityTesting -> ModelbasedTesting[color="green"];
AccessibilityTesting -> ConformanceTesting[color="green"];
AccessibilityTesting -> RegressionTesting[style="dashed",color="blue"];
AcquisitionOrganizationTesting -> OrganizationbasedTesting[color="blue"];
AdHocReviews -> Reviews[color="gray"];
AdHocTesting -> ExperiencebasedTesting[color="blue"];
AdHocTesting -> InformalTesting[color="blue"];
AdaptiveRandomTesting -> RandomTesting[color="blue"];
AdversarialTesting -> MLModelTesting[color="gray"];
AgentbasedTesting -> WebApplicationTesting;
AgentbasedTesting -> ObjectOrientedTesting;
AgentbasedTesting -> DataFlowTesting;
AgentbasedTesting -> FunctionalTesting;
AgentbasedTesting -> SpecificationbasedTesting[style="dashed"];
AgileTesting -> ContinuousTesting[color="blue"];
AgileTesting -> TestdrivenDevelopment[color="blue"];
AgileTesting -> LifecyclebasedTesting[color="blue"];
AgileTesting -> IncrementalTesting;
AgileTesting -> DevOpsTesting[style="dashed",color="blue"];
AgileTesting -> AttheBeginningTesting[style="dashed"];
AJAXTesting -> WebApplicationTesting[color="blue"];
AllCombinationsTesting -> CombinatorialTesting[color="green"];
AllRulesTesting -> WebApplicationTesting[color="blue"];
AllRulesTesting -> FormalTesting[color="gray"];
AllRulesTesting -> ControlFlowTesting[style="dashed",color="gray"];
AllTransitionsTesting -> StateTransitionTesting[color="green"];
AllTransitionsTesting -> ModelbasedTesting[color="green"];
AllCUsesTesting -> DataFlowTesting[color="green"];
AllCUsesTesting -> AllUsesTesting[color="green"];
AllCUsesTesting -> ControlFlowTesting[color="green"];
AllCUsesTesting -> AllDUPathsTesting[color="green"];
AllCUsesTesting -> AllCUsesSomePUsesTesting[style="dashed",color="maroon"];
AllCUsesSomePUsesTesting -> DataFlowTesting[color="maroon"];
AllCUsesSomePUsesTesting -> AllUsesTesting[style="dashed",color="maroon"];
AllDefinitionsTesting -> DataFlowTesting[color="green"];
AllDefinitionsTesting -> ControlFlowTesting[color="green"];
AllDefinitionsTesting -> AutomatedTesting[color="green"];
AllDefinitionsTesting -> AllDUPathsTesting[style="dashed",color="blue"];
AllDefinitionsTesting -> AllUsesTesting[style="dashed",color="maroon"];
AllDefinitionsTesting -> AllCUsesSomePUsesTesting[style="dashed",color="maroon"];
AllDefinitionsTesting -> AllPUsesSomeCUsesTesting[style="dashed",color="maroon"];
AllDUPathsTesting -> DataFlowTesting[color="green"];
AllDUPathsTesting -> ControlFlowTesting[color="green"];
AllDUPathsTesting -> StructurebasedTesting[color="green"];
AllDUPathsTesting -> PathTesting[style="dashed",color="green"];
AllInputGUITesting -> WebApplicationTesting[color="gray"];
AllInputGUITesting -> GUITesting[color="gray"];
AllInputGUITesting -> InputParameterTesting[style="dashed",color="gray"];
AllPUsesTesting -> DataFlowTesting[color="green"];
AllPUsesTesting -> AllUsesTesting[color="green"];
AllPUsesTesting -> ControlFlowTesting[color="green"];
AllPUsesTesting -> AllDUPathsTesting[color="green"];
AllPUsesTesting -> StructurebasedTesting[color="green"];
AllPUsesTesting -> AllPUsesSomeCUsesTesting[style="dashed",color="maroon"];
AllPUsesSomeCUsesTesting -> DataFlowTesting[color="maroon"];
AllPUsesSomeCUsesTesting -> AllUsesTesting[style="dashed",color="maroon"];
AllURLTesting -> WebApplicationTesting[color="blue"];
AllURLTesting -> ControlFlowTesting[style="dashed",color="gray"];
AllUsesTesting -> DataFlowTesting[color="green"];
AllUsesTesting -> ControlFlowTesting[color="green"];
AllUsesTesting -> AllDUPathsTesting[style="dashed",color="blue"];
AlphaTesting -> AcceptanceTesting[color="green"];
AlphaTesting -> UnscriptedTesting[color="blue"];
AlphaTesting -> TesterTesting[color="blue"];
AlphaTesting -> SecurityTesting[color="blue"];
AlphaTesting -> OperationalTesting;
AlphaTesting -> UserasTesterTesting[style="dashed",color="blue"];
AntiSpoofingTesting -> SecurityTesting[color="blue"];
AntiTamperTesting -> SecurityTesting[color="blue"];
APITesting -> InterfaceTesting[color="blue"];
ApplicationSystemTesting -> FunctionalTesting;
ApplicationSystemTesting -> FunctionalityTesting;
ApplicationSystemTesting -> SystemTesting;
ApplicationSystemTesting -> DynamicTesting;
ApplicationSystemTesting -> ScenarioTesting[style="dashed"];
ArchitectTesting -> DeveloperTesting[color="blue"];
ArchitectTesting -> ArchitecturedrivenTesting[color="gray"];
AssertionChecking -> OngoingBuiltInTesting[color="blue"];
AssertionChecking -> SelfTesting[style="dashed",color="blue"];
AssertionChecking -> FormalModularVerification[style="dashed"];
AssertionChecking -> SpecificationbasedTesting[style="dashed",color="gray"];
AsynchronousTesting -> TestEnvironmentTesting[style="dashed"];
Attacks -> ExperiencebasedTesting[color="green"];
Attacks -> PrivacyTesting[style="dashed",color="blue"];
Attacks -> DisasterRecoveryTesting[style="dashed"];
Attacks -> SpecificationbasedTesting[style="dashed",color="green"];
AttheBeginningTesting -> StaticTesting[style="dashed"];
AttheBeginningTesting -> WModelTesting[style="dashed"];
AttheBeginningTesting -> IncrementalTesting[style="dashed",color="gray"];
AttheBeginningTesting -> ContinuousTesting[style="dashed",color="gray"];
AttheEndTesting -> WaterfallTesting[color="blue"];
Audits -> StaticAnalysis[style="dashed",color="blue"];
Audits -> StaticTesting[style="dashed",color="green"];
Audits -> ComplianceTesting[style="dashed",color="green"];
Audits -> IndependentTestOrganizationTesting[style="dashed",color="gray"];
AutomatedTesting -> ScriptedTesting[color="green"];
AutomatedTesting -> DeveloperTesting[style="dashed"];
AvailabilityTesting -> ReliabilityTesting[color="green"];
AvailabilityTesting -> SecurityTesting[color="green"];
AvailabilityTesting -> BackupTesting;
AvailabilityTesting -> DisasterRecoveryTesting;
AvailabilityTesting -> PostDeploymentMonitoring;
AvailabilityTesting -> NonfunctionalTesting;
AvailabilityTesting -> AutomatedTesting;
AvailabilityTesting -> DynamicTesting;
BacktoBackTesting -> NonfunctionalTesting[color="blue"];
BackupandRecoveryTesting -> ReliabilityTesting[color="green"];
BackupandRecoveryTesting -> SecurityTesting[style="dashed"];
BackupandRecoveryTesting -> ComplianceTesting[style="dashed"];
BackupandRecoveryTesting -> PerformanceTesting[style="dashed"];
BackupTesting -> BackupandRecoveryTesting;
BackupTesting -> DistributedTesting[style="dashed"];
BackupRecoveryTesting -> DisasterRecoveryTesting[color="green"];
BackwardsCompatibilityTesting -> CompatibilityTesting[color="blue"];
BaseChoiceTesting -> CombinatorialTesting[color="green"];
BaseChoiceTesting -> ScenarioTesting[style="dashed",color="green"];
BaseChoiceTesting -> SmokeTesting[style="dashed",color="gray"];
BasicBlockTesting -> DataFlowTesting[color="maroon"];
BasicBlockTesting -> DecisionTesting[style="dashed",color="maroon"];
BehaviouralTesting -> DynamicTesting[color="maroon"];
BehaviouralTesting -> SpecificationbasedTesting[color="maroon"];
BehaviourAnalysis -> StaticTesting;
BehaviourAnalysis -> WModelTesting;
BehaviourAnalysis -> StaticAnalysis[style="dashed",color="gray"];
BetaTesting -> AcceptanceTesting[color="green"];
BetaTesting -> UnscriptedTesting[color="blue"];
BetaTesting -> UserTesting[style="dashed",color="blue"];
BetaTesting -> SecurityTesting[color="blue"];
BetaTesting -> OperationalTesting;
BetaTesting -> UserasTesterTesting[style="dashed",color="gray"];
BigBangTesting -> IntegrationTesting[color="green"];
BlockTesting -> StructurebasedTesting[style="dashed",color="blue"];
BlockTesting -> ControlFlowTesting[style="dashed",color="gray"];
BlueTeamTesting -> PenetrationTesting[color="blue"];
BottomUpTesting -> IntegrationTesting[color="blue"];
BoundaryConditionTesting -> DataTesting[color="maroon"];
BoundaryConditionTesting -> BoundaryValueAnalysis[style="dashed",color="maroon"];
BoundaryConditionTesting -> EquivalencePartitioning[style="dashed",color="maroon"];
BoundaryValueAnalysis -> SpecificationbasedTesting[color="green"];
BoundaryValueAnalysis -> EquivalencePartitioning[color="green"];
BoundaryValueAnalysis -> ModelbasedTesting[color="green"];
BoundaryValueAnalysis -> NegativeTesting[style="dashed",color="green"];
BoundaryValueAnalysis -> OnetoOneTesting[style="dashed",color="green"];
BoundaryValueAnalysis -> MinimizedTesting[style="dashed",color="green"];
BoundaryValueAnalysis -> GrayBoxTesting[color="blue"];
BoundaryValueAnalysis -> DynamicTesting;
BoundaryValueAnalysis -> WModelTesting;
BoundaryValueAnalysis -> FunctionalTesting[style="dashed",color="green"];
BranchConditionCombinationTesting -> StructurebasedTesting[color="green"];
BranchConditionCombinationTesting -> ControlFlowTesting[color="green"];
BranchConditionCombinationTesting -> ModelbasedTesting[color="green"];
BranchConditionCombinationTesting -> CoveragebasedTesting[color="maroon"];
BranchConditionCombinationTesting -> DecisionTesting[style="dashed",color="blue"];
BranchConditionCombinationTesting -> PathTesting[style="dashed",color="maroon"];
BranchConditionCombinationTesting -> ExhaustiveTesting[style="dashed",color="maroon"];
BranchConditionTesting -> StructurebasedTesting[color="green"];
BranchConditionTesting -> ControlFlowTesting[color="blue"];
BranchConditionTesting -> BranchConditionCombinationTesting[color="green"];
BranchConditionTesting -> ModelbasedTesting[color="green"];
BranchConditionTesting -> MCDCTesting[style="dashed",color="green"];
BranchTesting -> StructurebasedTesting[color="green"];
BranchTesting -> ControlFlowTesting[color="green"];
BranchTesting -> ModelbasedTesting[color="green"];
BranchTesting -> SpecificationbasedTesting[color="green"];
BranchTesting -> GrayBoxTesting[color="green"];
BranchTesting -> PathTesting[color="maroon"];
BranchTesting -> PathTesting[style="dashed",color="green"];
BranchTesting -> BranchConditionCombinationTesting[color="maroon"];
BranchTesting -> UnitTesting[style="dashed",color="maroon"];
BranchTesting -> MCDCTesting[style="dashed",color="green"];
BranchTesting -> AllPUsesTesting[style="dashed",color="green"];
BranchTesting -> WeakMutationTesting[style="dashed",color="maroon"];
BrowserPageTesting -> SmokeTesting;
BrowserPageTesting -> FunctionalTesting;
BrowserPageTesting -> DynamicTesting;
BrowserPageTesting -> DesktopDevelopmentTesting;
BrowserPageTesting -> WebApplicationTesting[color="gray"];
BuddyTesting -> AdHocTesting[color="blue"];
BuddyTesting -> GroupTesting[color="blue"];
BuddyTesting -> EmbeddedTesterTesting[color="blue"];
BufferOverrunTesting -> BoundaryConditionTesting[color="maroon"];
BufferOverrunTesting -> SecurityTesting[color="maroon"];
BufferOverrunTesting -> CodeInjection[style="dashed",color="maroon"];
BugHuntTesting -> ExperiencebasedTesting[color="blue"];
BuildVerificationTesting -> AutomatedTesting[color="blue"];
BuildVerificationTesting -> FunctionalityTesting[color="blue"];
BuildVerificationTesting -> StabilityTesting[color="blue"];
BuildVerificationTesting -> IntegrityTesting[color="blue"];
BusinessAcceptanceTesting -> AcceptanceTesting[color="blue"];
BusinessProcessbasedTesting -> ScenariobasedTesting[color="blue"];
CapacityTesting -> PerformanceEfficiencyTesting[color="green"];
CapacityTesting -> PerformancerelatedTesting[color="green"];
CapacityTesting -> ModelbasedTesting[color="green"];
CapacityTesting -> ConformanceTesting[color="green"];
CapacityTesting -> PerformanceTesting[color="blue"];
CapacityTesting -> BackupTesting[style="dashed"];
CaptureReplayDrivenTesting -> AutomatedTesting[color="green"];
CaptureReplayDrivenTesting -> SpecificationbasedTesting[style="dashed",color="blue"];
CauseEffectGraphing -> SpecificationbasedTesting[color="green"];
CauseEffectGraphing -> ModelbasedTesting[color="green"];
CauseEffectGraphing -> DynamicTesting;
CauseEffectGraphing -> WModelTesting;
CGIComponentTesting -> SmokeTesting;
CGIComponentTesting -> InfrastructureTesting;
CGIComponentTesting -> FunctionalTesting;
CGIComponentTesting -> DynamicTesting;
CGIComponentTesting -> TransactionTesting[style="dashed"];
CGIComponentTesting -> UnitTesting[style="dashed"];
CheckedStatementTesting -> StatementTesting;
CheckedStatementTesting -> StructurebasedTesting;
ChecklistbasedReviews -> ChecklistbasedTesting[color="gray"];
ChecklistbasedReviews -> Reviews[color="gray"];
ChecklistbasedTesting -> ExperiencebasedTesting[color="green"];
ClassificationTreeMethod -> SpecificationbasedTesting[color="green"];
ClassificationTreeMethod -> ModelbasedTesting[color="green"];
ClosedBetaTesting -> BetaTesting[color="blue"];
ClosedLoopTesting -> LoopTesting[color="gray"];
ClosedLoopTesting -> SpecificationbasedTesting;
ClosedLoopTesting -> FormalTesting;
ClosedLoopTesting -> ControlSystemTesting[style="dashed"];
ClosedLoopTesting -> PerformanceTesting[style="dashed"];
ClosedLoopTesting -> CorrectnessTesting[style="dashed"];
ClosedLoopTesting -> FunctionalTesting[style="dashed"];
ClosedLoopTesting -> NonfunctionalTesting[style="dashed"];
ClosedLoopTesting -> OnlineTesting[style="dashed"];
ClosedLoopTesting -> ModelbasedTesting[style="dashed"];
ClosedLoopTesting -> DomainSpecificTesting[style="dashed",color="gray"];
CloudTesting -> BackupandRecoveryTesting[style="dashed"];
CodeInjection -> SecurityAttacks[color="blue"];
CodeInjection -> InterfaceTesting[style="dashed",color="blue"];
CodeReviews -> Reviews[color="blue"];
CodeReviews -> PeerReviews[color="blue"];
CodeReviews -> StaticAnalysis[style="dashed",color="blue"];
CoexistenceTesting -> CompatibilityTesting[color="green"];
CombinatorialTesting -> SpecificationbasedTesting[color="green"];
CombinatorialTesting -> ModelbasedTesting[color="green"];
CommandFormTesting -> WebApplicationTesting[color="blue"];
CommandFormTesting -> DatabaseCoverageTesting[style="dashed",color="blue"];
CommandFormTesting -> IntegrationTesting[style="dashed",color="gray"];
CompatibilityTesting -> NonfunctionalTesting[color="blue"];
CompatibilityTesting -> RegressionTesting[style="dashed",color="blue"];
CompleteRegressionTesting -> RegressionTesting[color="blue"];
ComplianceTesting -> SecurityTesting;
ComponentIntegrationTesting -> IntegrationTesting[color="blue"];
ConcreteExecution -> StaticTesting[style="dashed",color="gray"];
ConcurrencyTesting -> PerformanceTesting;
ConcurrencyTesting -> CompatibilityTesting[style="dashed",color="green"];
ConfigurationTesting -> NonfunctionalTesting;
ConfigurationTesting -> SmokeTesting;
ConfigurationTesting -> DynamicTesting;
ConfigurationTesting -> DataCenterTesting[color="blue"];
ConfigurationTesting -> CombinatorialTesting[style="dashed",color="green"];
ConfigurationTesting -> DeveloperTesting[style="dashed"];
ConformanceTesting -> FormalTesting[style="dashed"];
ContentChecking -> UsabilityTesting;
ContentChecking -> StaticTesting;
ContentChecking -> DesktopDevelopmentTesting;
ContentChecking -> WebApplicationTesting[color="maroon"];
ContentChecking -> GrayBoxTesting[style="dashed",color="maroon"];
ContentUsageTesting -> UsabilityTesting[color="blue"];
ContinuousTesting -> AutomatedTesting[color="green"];
ContinuousTesting -> LifecyclebasedTesting[color="blue"];
ContinuousTesting -> DeveloperTesting[style="dashed"];
ContractualAcceptanceTesting -> AcceptanceTesting[color="blue"];
ControlFlowAnalysis -> StaticAnalysis[color="blue"];
ControlFlowTesting -> ModelbasedTesting[color="green"];
ControlFlowTesting -> StructurebasedTesting[color="blue"];
ControlFlowTesting -> DynamicTesting[color="maroon"];
ControlFlowTesting -> WebApplicationTesting[color="blue"];
ControlFlowTesting -> CoveragebasedTesting[style="dashed",color="maroon"];
ControlSystemTesting -> SystemTesting[color="gray"];
ControlSystemTesting -> DomainSpecificTesting[style="dashed",color="gray"];
ConversionTesting -> ModelbasedTesting[color="green"];
ConversionTesting -> ConformanceTesting[color="green"];
ConversionTesting -> FunctionalSuitabilityTesting[color="green"];
ConversionTesting -> RegressionTesting[style="dashed",color="blue"];
ConversionTesting -> PortabilityTesting[style="dashed",color="gray"];
CookieTesting -> PenetrationTesting;
CookieTesting -> WebApplicationTesting[color="gray"];
CorrectnessTesting -> FunctionalSuitabilityTesting[color="green"];
COTSTesting -> ReuseTesting[color="blue"];
COTSVendorTesting -> DevelopmentOrganizationTesting[color="blue"];
COTSVendorTesting -> COTSTesting[color="gray"];
CrossBrowserCompatibilityTesting -> WebApplicationTesting[color="blue"];
CrossBrowserCompatibilityTesting -> UITesting;
CrossBrowserCompatibilityTesting -> FunctionalityTesting;
CrossBrowserCompatibilityTesting -> AutomatedTesting;
CrossBrowserCompatibilityTesting -> BacktoBackTesting[style="dashed"];
CrossBrowserCompatibilityTesting -> DifferentialAssertionChecking[style="dashed"];
CrossBrowserCompatibilityTesting -> RegressionTesting[style="dashed"];
CrossBrowserCompatibilityTesting -> ManualTesting[style="dashed"];
CrossBrowserCompatibilityTesting -> DeveloperTesting;
CrossBrowserCompatibilityTesting -> FaultbasedTesting[style="dashed",color="blue"];
CrossBrowserCompatibilityTesting -> ConfigurationTesting[style="dashed"];
CrossBrowserCompatibilityTesting -> CompatibilityTesting[color="gray"];
CustomerAcceptanceTesting -> AcceptanceTesting[color="blue"];
DataCenterTesting -> OperationalTesting[color="blue"];
DataDependenceTransitionRelationTesting -> WebApplicationTesting[color="blue"];
DataDependenceTransitionRelationTesting -> DatadrivenTesting[style="dashed",color="gray"];
DataFlowAnalysis -> StaticAnalysis[color="blue"];
DataFlowTesting -> StructurebasedTesting[color="green"];
DataFlowTesting -> ControlFlowTesting[color="green"];
DataFlowTesting -> ModelbasedTesting[color="green"];
DataFlowTesting -> ExhaustiveTesting[color="maroon"];
DataFlowTesting -> WebApplicationTesting[color="blue"];
DataIntegrityTesting -> SecurityTesting;
DataIntegrityTesting -> IntegrityTesting[color="gray"];
DataMigrationTesting -> ConversionTesting[color="green"];
DataMigrationTesting -> CorrectnessTesting[color="blue"];
DataMigrationTesting -> DataIntegrityTesting[style="dashed",color="gray"];
DataMigrationTesting -> DatabaseAdminTesting[style="dashed",color="gray"];
DataTesting -> DynamicTesting[color="maroon"];
DatabaseAdminTesting -> OperatorTesting[color="blue"];
DatabaseCoverageTesting -> WebApplicationTesting[color="blue"];
DatabaseCoverageTesting -> ControlFlowTesting[style="dashed",color="gray"];
DatabaseIntegrityTesting -> WebApplicationTesting[style="dashed"];
DatabaseIntegrityTesting -> BackupTesting[style="dashed"];
DatabaseIntegrityTesting -> IntegrityTesting[color="gray"];
DatabaseIntegrityTesting -> DataIntegrityTesting[style="dashed",color="gray"];
DatadrivenTesting -> AutomatedTesting[color="green"];
DatadrivenTesting -> ScriptedTesting;
DecisionConditionTesting -> SpecificationbasedTesting;
DecisionConditionTesting -> MCDCTesting[style="dashed",color="gray"];
DecisionTableTesting -> SpecificationbasedTesting[color="green"];
DecisionTableTesting -> ModelbasedTesting[color="green"];
DecisionTableTesting -> DataTesting[color="maroon"];
DecisionTesting -> StructurebasedTesting[color="green"];
DecisionTesting -> ControlFlowTesting[color="green"];
DecisionTesting -> ModelbasedTesting[color="green"];
DecisionTesting -> DataFlowTesting[color="maroon"];
DecisionTesting -> AllPUsesTesting[style="dashed",color="green"];
DecisionTesting -> PathTesting[style="dashed",color="green"];
DecisionTesting -> AllCUsesTesting[style="dashed",color="maroon"];
DecisionTesting -> DecisionConditionTesting[style="dashed"];
DefectbasedTesting -> FaultbasedTesting[style="dashed",color="gray"];
DenialofService -> SecurityAttacks[color="blue"];
DenialofService -> StressTesting[style="dashed",color="gray"];
DesktopDevelopmentTesting -> WebApplicationTesting[color="gray"];
DesktopDevelopmentTesting -> DevelopmentTesting[style="dashed",color="gray"];
DevSecOpsTesting -> DevOpsTesting[color="blue"];
DevSecOpsTesting -> SecurityTesting[color="blue"];
DevSecOpsTesting -> AgileTesting[color="blue"];
DevSecOpsTesting -> ShiftLeftTesting[color="blue"];
DevSecOpsTesting -> ContinuousTesting[color="blue"];
DevSecOpsTesting -> AutomatedTesting[color="blue"];
DeveloperTesting -> DevelopmentTesting[color="green"];
DeveloperTesting -> RolebasedTesting[color="blue"];
DevelopmentEnvironmentTesting -> DevelopmentTesting[style="dashed"];
DevelopmentOrganizationTesting -> OrganizationbasedTesting[color="blue"];
DevelopmentOrganizationTesting -> DevelopmentTesting[color="gray"];
DevelopmentToolTesting -> DevelopmentTesting[style="dashed",color="gray"];
DevOpsTesting -> ContinuousTesting[color="blue"];
DevOpsTesting -> AutomatedTesting[style="dashed",color="blue"];
DifferentialAssertionChecking -> AssertionChecking;
DifferentialAssertionChecking -> RegressionTesting;
DifferentialAssertionChecking -> RelativeCorrectnessTesting;
DifferentialAssertionChecking -> BacktoBackTesting[style="dashed"];
DisasterRecoveryTesting -> ModelbasedTesting[color="green"];
DisasterRecoveryTesting -> ConformanceTesting[color="green"];
DisasterRecoveryTesting -> SystemTesting[style="dashed",color="green"];
DisasterRecoveryTesting -> FaultToleranceTesting[color="green"];
DisasterRecoveryTesting -> RecoverabilityTesting[color="green"];
DisasterRecoveryTesting -> RiskbasedTesting[style="dashed"];
DistributedTesting -> AvailabilityTesting[style="dashed"];
DistributedTesting -> ReliabilityTesting[style="dashed"];
DistributedTesting -> LoadBalancingTesting[style="dashed"];
DistributedTesting -> EfficiencyTesting[style="dashed"];
DistributedTesting -> ScalabilityTesting[style="dashed"];
DistributedTesting -> DataIntegrityTesting[style="dashed"];
DistributedTesting -> RecoveryTesting[style="dashed"];
DistributedTesting -> FaultToleranceTesting[style="dashed"];
DOMTesting -> WebApplicationTesting;
DOMTesting -> ModelbasedTesting[style="dashed",color="blue"];
DomainAnalysis -> EquivalencePartitioning[color="green"];
DomainAnalysis -> BoundaryValueAnalysis[color="green"];
DomainIndependentTesting -> DomainBasedTesting[color="blue"];
DomainSpecificTesting -> DomainBasedTesting[color="blue"];
DTOrganizationTesting -> IndependentTestOrganizationTesting[color="blue"];
DTOrganizationTesting -> DevelopmentTesting[style="dashed",color="blue"];
DynamicAnalysis -> DynamicTesting[style="dashed",color="green"];
DynamicTesting -> VModelTesting;
DynamicTesting -> WModelTesting;
EachChoiceTesting -> CombinatorialTesting[color="green"];
EachChoiceTesting -> twiseTesting[color="gray"];
EfficiencyTesting -> BackupTesting[style="dashed"];
ElasticityTesting -> MemoryManagementTesting[color="blue"];
ElasticityTesting -> ResourceUtilizationTesting[color="blue"];
ElasticityTesting -> NonfunctionalTesting[color="blue"];
ElasticityTesting -> StressTesting[color="gray"];
ElementaryComparisonTesting -> SpecificationbasedTesting;
ElementaryComparisonTesting -> ComparisonTesting[style="dashed",color="gray"];
EmbeddedTesterTesting -> TesterTesting[color="blue"];
EncryptionTesting -> SecurityTesting[color="blue"];
EndtoendFunctionalityTesting -> SmokeTesting;
EndtoendFunctionalityTesting -> DynamicTesting;
EndtoendFunctionalityTesting -> SystemsIntegrationTesting;
EndtoendFunctionalityTesting -> AcceptanceTesting;
EndtoendFunctionalityTesting -> ScenariobasedTesting[style="dashed"];
EndtoendFunctionalityTesting -> FunctionalityTesting[color="gray"];
EndtoendFunctionalityTesting -> EndtoendTesting[color="gray"];
EndtoendTesting -> IntegrationTesting;
EndtoendTesting -> SpecificationbasedTesting[color="blue"];
EndtoendTesting -> OperationalTesting[style="dashed",color="gray"];
EnduranceTesting -> PerformancerelatedTesting[color="green"];
EnduranceTesting -> ModelbasedTesting[color="green"];
EnduranceTesting -> ConformanceTesting[color="green"];
EnduranceTesting -> PerformanceEfficiencyTesting[color="green"];
EnduranceTesting -> ReliabilityTesting[color="blue"];
EquivalenceChecking -> IncrementalTesting[style="dashed"];
EquivalenceChecking -> RelativeCorrectnessTesting;
EquivalenceChecking -> RegressionTesting[style="dashed"];
EquivalencePartitioning -> SpecificationbasedTesting[color="green"];
EquivalencePartitioning -> ModelbasedTesting[color="green"];
EquivalencePartitioning -> NegativeTesting[color="green"];
EquivalencePartitioning -> PositiveTesting[color="green"];
EquivalencePartitioning -> OnetoOneTesting[style="dashed",color="green"];
EquivalencePartitioning -> MinimizedTesting[style="dashed",color="green"];
EquivalencePartitioning -> UsabilityTesting[style="dashed",color="green"];
EquivalencePartitioning -> ClassificationTreeMethod[style="dashed",color="green"];
EquivalencePartitioning -> GrayBoxTesting[color="blue"];
EquivalencePartitioning -> DynamicTesting;
EquivalencePartitioning -> WModelTesting;
EquivalencePartitioning -> FunctionalTesting[style="dashed",color="green"];
ErrorForcing -> DataTesting[color="maroon"];
ErrorForcing -> ErrorToleranceTesting[style="dashed",color="maroon"];
ErrorGuessing -> ExperiencebasedTesting[color="green"];
ErrorGuessing -> ModelbasedTesting[color="green"];
ErrorGuessing -> ChecklistbasedTesting[color="green"];
ErrorGuessing -> FaultbasedTesting[color="blue"];
ErrorGuessing -> ErrorbasedTesting[style="dashed",color="gray"];
ErrorSeeding -> ErrorbasedTesting[style="dashed",color="gray"];
ErrorToleranceTesting -> RobustnessTesting[color="blue"];
EventSpaceTesting -> WebApplicationTesting[color="blue"];
EventSpaceTesting -> GUITesting[color="gray"];
EventSpaceTesting -> ControlFlowTesting[style="dashed",color="gray"];
EvidencebasedTesting -> SpecificationbasedTesting[color="blue"];
ExhaustiveTesting -> DynamicTesting[color="green"];
ExperiencebasedTesting -> DynamicTesting[color="green"];
ExperiencebasedTesting -> UnscriptedTesting[style="dashed",color="green"];
ExperiencebasedTesting -> SpecificationbasedTesting[style="dashed",color="maroon"];
ExpertUsabilityReviews -> InformalTesting[color="blue"];
ExpertUsabilityReviews -> UsabilityTesting[color="blue"];
ExpertUsabilityReviews -> Reviews[color="blue"];
ExploratoryTesting -> ExperiencebasedTesting[color="green"];
ExploratoryTesting -> UnscriptedTesting[color="green"];
ExploratoryTesting -> DynamicTesting[color="maroon"];
ExploratoryTesting -> SpecificationbasedTesting[color="maroon"];
ExploratoryTesting -> BehaviouralTesting[color="maroon"];
ExploratoryTesting -> InformalTesting;
ExtendedEntryTableTesting -> DecisionTableTesting[color="green"];
ExtendedEntryTableTesting -> EquivalencePartitioning[color="green"];
ExternalLinksIntegrationTesting -> FunctionalTesting;
ExternalLinksIntegrationTesting -> DynamicTesting;
ExternalLinksIntegrationTesting -> LargeScaleIntegrationTesting;
ExternalLinksIntegrationTesting -> IntegrationTesting[color="gray"];
ExternalLinksIntegrationTesting -> WebApplicationTesting[color="gray"];
ExtremeValueAnalysis -> SpecificationbasedTesting;
ExtremeValueAnalysis -> EquivalencePartitioning[style="dashed"];
FactoryAcceptanceTesting -> AcceptanceTesting[color="green"];
FailoverTesting -> NonfunctionalTesting[color="blue"];
FailoverTesting -> BottomUpTesting;
FailoverTesting -> FailureToleranceTesting[style="dashed",color="blue"];
FailoverTesting -> DataCenterTesting[style="dashed",color="blue"];
FailoverRecoveryTesting -> DisasterRecoveryTesting[color="green"];
FailureToleranceTesting -> RobustnessTesting[color="blue"];
FaultInjectionTesting -> RobustnessTesting[color="green"];
FaultInjectionTesting -> FaultToleranceTesting[color="blue"];
FaultToleranceTesting -> ReliabilityTesting[color="green"];
FaultToleranceTesting -> AvailabilityTesting[color="green"];
FaultToleranceTesting -> RobustnessTesting[color="blue"];
FaultTreeAnalysis -> StaticAnalysis[color="gray"];
FeaturebasedTesting -> UnitTesting[style="dashed",color="blue"];
FeaturesTesting -> DynamicTesting;
FeaturesTesting -> WModelTesting;
FeaturesTesting -> IntegrationTesting[style="dashed"];
FeaturesTesting -> SystemTesting[style="dashed"];
FieldTesting -> BetaTesting[style="dashed"];
FieldTesting -> OperationalTesting[style="dashed",color="gray"];
FollowonOperationalTesting -> OperationalTesting[color="blue"];
ForcingExceptionTesting -> ScenariobasedTesting[style="dashed",color="blue"];
ForcingExceptionTesting -> NegativeTesting[style="dashed",color="gray"];
FormalMethods -> FormalTesting[style="dashed",color="gray"];
FormalModularVerification -> FormalTesting[color="gray"];
FormalReviews -> Reviews[color="blue"];
FormalReviews -> FormalTesting[color="gray"];
FormativeEvaluations -> DevelopmentTesting[style="dashed",color="gray"];
FullConformanceTesting -> ConformanceTesting[color="gray"];
FunctionalSuitabilityTesting -> CompatibilityTesting[style="dashed",color="green"];
FunctionalTesting -> SpecificationbasedTesting[color="green"];
FunctionalTesting -> StructurebasedTesting[color="green"];
FunctionalTesting -> FunctionalSuitabilityTesting[color="green"];
FunctionalTesting -> AutomatedTesting[style="dashed"];
FunctionalityTesting -> BuildVerificationTesting[color="blue"];
FunctionalityTesting -> RegressionTesting[style="dashed",color="green"];
FunctionalityTesting -> SmokeTesting[style="dashed",color="gray"];
FunctionalityTesting -> FunctionalSuitabilityTesting[style="dashed",color="gray"];
FunctionsTesting -> StructurebasedTesting[style="dashed",color="blue"];
FunctionsTesting -> ControlFlowTesting[style="dashed",color="gray"];
FuzzTesting -> MathematicalbasedTesting[color="green"];
FuzzTesting -> SecurityTesting[color="blue"];
FuzzTesting -> RandomTesting[color="blue"];
FuzzTesting -> RandomTesting[style="dashed",color="green"];
Galumphing -> ExploratoryTesting[color="blue"];
GraphicalUserInterfaceTesting -> BrowserPageTesting[style="dashed"];
GraphicalUserInterfaceTesting -> DynamicTesting[color="gray"];
GrayBoxTesting -> StructurebasedTesting[color="green"];
GrayBoxTesting -> SpecificationbasedTesting[color="green"];
Heartbeat -> PeriodicBuiltInTesting[color="blue"];
Heartbeat -> SelfTesting[style="dashed",color="blue"];
HeuristicEvaluations -> UsabilityReviews[color="blue"];
HeuristicEvaluations -> UsabilityTesting;
HighFrequencyTesting -> IntegrationTesting;
HumanFactorsEngineerTesting -> DeveloperTesting[color="blue"];
HumanintheLoopTesting -> SystemTesting[color="blue"];
HyperlinkTesting -> WebApplicationTesting[color="blue"];
HyperlinkTesting -> SmokeTesting;
HyperlinkTesting -> DesktopDevelopmentTesting;
HyperlinkTesting -> GrayBoxTesting[style="dashed",color="maroon"];
HyperlinkTesting -> PageTesting[color="gray"];
HyperlinkTesting -> ControlFlowTesting[style="dashed",color="gray"];
IncontainerTesting -> WebApplicationTesting[color="blue"];
IncrementalTesting -> LifecyclebasedTesting[color="blue"];
IndependentTestOrganizationTesting -> OrganizationbasedTesting[color="blue"];
IndependentTestOrganizationTesting -> IndependentTesterTesting[color="gray"];
IndependentTesterTesting -> TesterTesting[color="blue"];
InductiveAssertionMethods -> ProofsofCorrectness[color="green"];
InductiveAssertionMethods -> MathematicalbasedTesting[color="gray"];
InductiveAssertionMethods -> AssertionChecking[color="gray"];
InductiveAssertionMethods -> StaticTesting[style="dashed",color="gray"];
IndustrialWebApplicationTesting -> WebApplicationTesting;
InformalReviews -> Reviews[color="blue"];
InformalReviews -> InformalTesting[color="gray"];
InfrastructureCompatibilityTesting -> CompatibilityTesting[color="blue"];
InfrastructureCompatibilityTesting -> InfrastructureTesting[color="gray"];
InfrastructureTesting -> SecurityTesting[color="blue"];
InfrastructureTesting -> NonfunctionalTesting[color="blue"];
InitialOperationalTesting -> OperationalTesting[color="blue"];
InputDataTesting -> MLModelTesting[color="gray"];
InputValidationTesting -> DynamicTesting;
InputValidationTesting -> WModelTesting;
InputValidationTesting -> PathTesting[color="gray"];
InputValidationTesting -> DatadrivenTesting[style="dashed",color="gray"];
InputParameterTesting -> WebApplicationTesting[color="blue"];
InputParameterTesting -> SpecificationbasedTesting[color="blue"];
InputParameterTesting -> ModelbasedTesting[color="green"];
InputParameterTesting -> CombinatorialTesting[color="gray"];
CodeInspections -> StaticAnalysis[color="green"];
CodeInspections -> StaticTesting[color="green"];
CodeInspections -> FormalReviews[color="blue"];
CodeInspections -> RolebasedTesting[color="maroon"];
CodeInspections -> Reviews[color="maroon"];
CodeInspections -> WModelTesting;
InstallabilityTesting -> PortabilityTesting[color="green"];
InstallabilityTesting -> CompatibilityTesting[color="green"];
InstallabilityTesting -> ModelbasedTesting[color="green"];
InstallationTesting -> DynamicTesting;
InstallationTesting -> WModelTesting;
InstallationTesting -> OperationalTesting[style="dashed",color="maroon"];
InstallationTesting -> SystemTesting[style="dashed"];
InstallationTesting -> OnlineTesting[style="dashed",color="gray"];
IntegratedSystemTesting -> DataCenterTesting[color="blue"];
IntegrationTesting -> ConstructionTesting[color="blue"];
IntegrationTesting -> VModelTesting;
IntegrationTesting -> WModelTesting;
IntegrationTesting -> StructurebasedTesting;
IntegrityTesting -> ScenarioTesting[color="green"];
IntegrityTesting -> BackupTesting[style="dashed"];
InterfaceTesting -> IntegrationTesting[color="blue"];
InterfaceTesting -> DesignbasedTesting;
InterfaceTesting -> SystemIntegrationTesting[style="dashed"];
InternationalizationTesting -> FlexibilityTesting[color="blue"];
InternationalizationTesting -> UsabilityTesting;
InternationalizationTesting -> FunctionalTesting;
InternationalizationTesting -> DynamicTesting;
InteroperabilityTesting -> CompatibilityTesting[color="green"];
InteroperabilityTesting -> ModelbasedTesting[color="green"];
InterruptdrivenBuiltInTesting -> BuiltInTesting[color="blue"];
IsolationTesting -> UnitTesting[style="dashed",color="gray"];
KeyworddrivenTesting -> ScriptedTesting[color="blue"];
KeyworddrivenTesting -> AutomatedTesting[style="dashed",color="green"];
KeyworddrivenTesting -> ManualTesting[style="dashed",color="green"];
KeyworddrivenTesting -> DatadrivenTesting;
LargeScaleIntegrationTesting -> IntegrationTesting[style="dashed",color="gray"];
LayerbasedTesting -> UnitTesting[style="dashed",color="blue"];
LegacySystemIntegrationTesting -> FunctionalTesting;
LegacySystemIntegrationTesting -> DynamicTesting;
LegacySystemIntegrationTesting -> LargeScaleIntegrationTesting;
LegacySystemIntegrationTesting -> SystemIntegrationTesting[color="gray"];
LegacySystemIntegrationTesting -> LegacyTesting[color="gray"];
LegacyTesting -> ReuseTesting[color="blue"];
LicenseComplianceAudits -> Audits[color="green"];
LinearScripting -> ScriptedTesting[color="blue"];
LinkChecking -> TestBrowsing;
LinkChecking -> AutomatedTesting;
LinkChecking -> SmokeTesting;
LinkChecking -> DynamicTesting;
LinkChecking -> PostDeploymentMonitoring;
LinkChecking -> GrayBoxTesting[style="dashed",color="maroon"];
LinkChecking -> AvailabilityTesting[color="gray"];
LinkChecking -> WebApplicationTesting[color="gray"];
LinkDependenceTransitionRelationTesting -> WebApplicationTesting[color="blue"];
LinkDependenceTransitionRelationTesting -> HyperlinkTesting[style="dashed",color="gray"];
LoadTesting -> PerformancerelatedTesting[color="green"];
LoadTesting -> PerformanceTesting[color="green"];
LoadTesting -> ReliabilityTesting[color="green"];
LoadTesting -> ModelbasedTesting[color="green"];
LoadTesting -> ConformanceTesting[color="green"];
LoadTesting -> PerformanceEfficiencyTesting[color="green"];
LoadTesting -> StabilityTesting[color="blue"];
LoadTesting -> RobustnessTesting[color="blue"];
LoadTesting -> NonfunctionalTesting[color="blue"];
LoadTesting -> NegativeTesting[style="dashed",color="maroon"];
LoadTesting -> CapacityTesting[color="blue"];
LoadTesting -> TimingTesting[style="dashed"];
LocalTesting -> SynchronousTesting[style="dashed"];
LocalizationTesting -> FunctionalSuitabilityTesting[color="green"];
LocalizationTesting -> AccessibilityTesting[color="green"];
LocalizationTesting -> PortabilityTesting[color="green"];
LocalizationTesting -> ManualTesting[style="dashed"];
LocalizationTesting -> WebApplicationTesting[style="dashed"];
LoopTesting -> StructurebasedTesting;
LoopTesting -> DynamicTesting;
LoopTesting -> WModelTesting;
LoopTesting -> ControlFlowTesting[style="dashed"];
LoopTesting -> PerformanceTesting[style="dashed"];
LoopTesting -> UnitTesting[style="dashed"];
MaintainabilityTesting -> ModelbasedTesting[color="green"];
MaintainabilityTesting -> RegressionTesting[style="dashed",color="blue"];
MaintenanceTesting -> ChangeRelatedTesting[style="dashed",color="gray"];
MaintenanceTesting -> OperationalTesting[style="dashed",color="gray"];
MalwareScanning -> StaticAnalysis[color="blue"];
MalwareScanning -> InterfaceTesting[style="dashed",color="blue"];
ManualTesting -> ScriptedTesting[color="green"];
MarkovChainTesting -> WebApplicationTesting;
MarkovChainTesting -> StatisticalTesting;
MarkovChainTesting -> SpecificationbasedTesting[style="dashed"];
MarkovChainTesting -> NonfunctionalTesting;
MarkovChainTesting -> UsagebasedTesting;
MarkovChainTesting -> ReliabilityTesting[style="dashed"];
MarkovChainTesting -> PerformanceTesting[style="dashed"];
MarkovChainTesting -> MaintenanceTesting[style="dashed"];
MarkovChainTesting -> CorrectnessTesting[style="dashed"];
MathematicalbasedTesting -> StructurebasedTesting[style="dashed",color="maroon"];
MCDCTesting -> StructurebasedTesting[color="green"];
MCDCTesting -> ControlFlowTesting[color="green"];
MCDCTesting -> ModelbasedTesting[color="green"];
MCDCTesting -> BranchConditionCombinationTesting[style="dashed",color="green"];
MemoryManagementTesting -> PerformancerelatedTesting[color="green"];
MemoryManagementTesting -> ModelbasedTesting[color="green"];
MemoryManagementTesting -> ConformanceTesting[color="green"];
MemoryManagementTesting -> ResourceUtilizationTesting[color="gray"];
MenuItemTesting -> StructurebasedTesting[color="green"];
MenuItemTesting -> SystemTesting[color="green"];
MenuItemTesting -> GUITesting[color="gray"];
MetamorphicTesting -> SpecificationbasedTesting[color="green"];
MetamorphicTesting -> ModelbasedTesting[color="green"];
MetamorphicTesting -> MutationTesting[color="blue"];
MetamorphicTesting -> ScriptedTesting[style="dashed"];
MetamorphicTesting -> MathematicalbasedTesting[color="gray"];
MethodTesting -> StructurebasedTesting[style="dashed",color="blue"];
MethodTesting -> ControlFlowTesting[style="dashed",color="gray"];
MigrationTesting -> RegressionTesting[style="dashed",color="blue"];
MixedEntryTableTesting -> DecisionTableTesting[color="green"];
MLModelTesting -> DomainSpecificTesting[style="dashed",color="gray"];
FlashMobTesting -> UsabilityTesting[color="blue"];
FlashMobTesting -> GroupTesting[color="blue"];
MobileTesting -> CompatibilityTesting[color="blue"];
ModelVerification -> StaticTesting[color="green"];
ModelbasedTesting -> MathematicalbasedTesting[color="green"];
ModelbasedTesting -> AutomatedTesting[color="blue"];
ModelbasedTesting -> WebApplicationTesting[color="blue"];
MonkeyTesting -> AdHocTesting[color="blue"];
MonkeyTesting -> UnscriptedTesting[color="blue"];
MonkeyTesting -> RandomTesting[color="blue"];
MonkeyTesting -> FuzzTesting[style="dashed",color="gray"];
MultiplayerTesting -> MultiUserTesting[color="gray"];
MultipleHitDecisionTableTesting -> DecisionTableTesting[color="green"];
MultiUserTesting -> DynamicTesting;
MultiUserTesting -> WModelTesting;
MutationTesting -> StructurebasedTesting[color="blue"];
MutationTesting -> FaultbasedTesting[color="blue"];
MutationTesting -> WebApplicationTesting[color="blue"];
NegativeTesting -> ExperiencebasedTesting[style="dashed",color="green"];
NegativeTesting -> SecurityTesting[style="dashed",color="blue"];
NegativeTesting -> ForcingExceptionTesting[style="dashed",color="blue"];
NegativeTesting -> BoundaryValueAnalysis[style="dashed",color="green"];
NegativeTesting -> ScenariobasedTesting[style="dashed",color="green"];
NegativeTesting -> DataTesting[style="dashed",color="maroon"];
NegativeTesting -> StateTesting[style="dashed",color="maroon"];
NegativeTesting -> RobustnessTesting[style="dashed",color="gray"];
NeighborhoodIntegrationTesting -> IntegrationTesting[color="blue"];
NetworkAdminTesting -> OperatorTesting[color="blue"];
NetworkTrafficTesting -> DataCenterTesting[color="blue"];
NetworkTrafficTesting -> NetworkAdminTesting[style="dashed",color="gray"];
NeuronCoverageTesting -> MLModelTesting[style="dashed",color="blue"];
NonfunctionalTesting -> DynamicTesting[style="dashed"];
NonfunctionalTesting -> WModelTesting;
NSwitchTesting -> StateTransitionTesting[color="green"];
ObjectbasedTesting -> WebApplicationTesting;
ObjectbasedTesting -> ObjectOrientedTesting;
ObjectbasedTesting -> StructurebasedTesting;
ObjectbasedTesting -> DataFlowTesting;
ObjectbasedTesting -> FunctionalTesting;
OfflineMBT -> OfflineTesting[color="blue"];
OfflineMBT -> ModelbasedTesting[color="blue"];
OngoingBuiltInTesting -> BuiltInTesting[color="blue"];
OngoingBuiltInTesting -> ContinuousTesting[style="dashed",color="gray"];
OnlineMBT -> OnlineTesting[color="blue"];
OnlineMBT -> ModelbasedTesting[color="blue"];
OOWebTesting -> WebApplicationTesting;
OOWebTesting -> ObjectOrientedTesting;
OOWebTesting -> StructurebasedTesting;
OOWebTesting -> DataFlowTesting;
OOWebTesting -> FunctionalTesting;
OpenBetaTesting -> BetaTesting[color="blue"];
OpenLoopTesting -> LoopTesting[color="gray"];
OpenLoopTesting -> AutomatedTesting[style="dashed"];
OpenLoopTesting -> SafetyTesting[style="dashed"];
OpenLoopTesting -> NonfunctionalTesting[style="dashed"];
OpenLoopTesting -> CorrectnessTesting[style="dashed"];
OpenLoopTesting -> ModelbasedTesting[style="dashed"];
OpenLoopTesting -> DomainSpecificTesting[style="dashed",color="gray"];
OpenSourceTesting -> ReuseTesting[color="blue"];
OperationalTesting -> AcceptanceTesting[color="green"];
OperationalTesting -> NonfunctionalTesting;
OperationalTesting -> ProcedureTesting[style="dashed",color="green"];
OperationalTesting -> ReliabilityTesting[color="blue"];
OperationalEffectivenessTesting -> OperationalTesting[color="blue"];
OperationalProfileTesting -> StatisticalTesting;
OperationalProfileTesting -> ScenarioTesting[style="dashed",color="gray"];
OperationalProfileTesting -> UsagebasedTesting[style="dashed",color="gray"];
OperationalSuitabilityTesting -> OperationalTesting[color="blue"];
OperationsOrganizationTesting -> OrganizationbasedTesting[color="blue"];
OperationsOrganizationTesting -> OperationalTesting[style="dashed",color="gray"];
OperatorTesting -> RolebasedTesting[color="blue"];
OrganizationbasedTesting -> RolebasedTesting[style="dashed",color="gray"];
OrthogonalArrayTesting -> CombinatorialTesting;
OrthogonalArrayTesting -> StatisticalTesting;
OrthogonalArrayTesting -> MathematicalbasedTesting;
OrthogonalArrayTesting -> SoftwareInteractionTesting;
OrthogonalArrayTesting -> EquivalencePartitioning;
OrthogonalArrayTesting -> PairwiseTesting;
OrthogonalArrayTesting -> CompilerTesting[style="dashed"];
OTOrganizationTesting -> IndependentTestOrganizationTesting[color="blue"];
OTOrganizationTesting -> OperationalTesting[style="dashed",color="blue"];
OutsideInTesting -> IntegrationTesting[style="dashed"];
OutsourcedTesting -> IndependentTesterTesting[style="dashed",color="gray"];
PageTesting -> WebApplicationTesting[color="blue"];
PageTesting -> ControlFlowTesting[style="dashed",color="gray"];
PairTesting -> AdHocTesting[color="blue"];
PairTesting -> GroupTesting[color="blue"];
PairTesting -> EmbeddedTesterTesting[color="blue"];
PairTesting -> DeveloperTesting[style="dashed"];
PairwiseIntegrationTesting -> IntegrationTesting[color="blue"];
PairwiseTesting -> CombinatorialTesting[color="green"];
PairwiseTesting -> SpecificationbasedTesting[color="green"];
PairwiseTesting -> ModelbasedTesting[color="green"];
PairwiseTesting -> AllCombinationsTesting[color="green"];
PairwiseTesting -> twiseTesting[color="gray"];
PartialRegressionTesting -> RegressionTesting[color="blue"];
PasswordCracking -> SecurityAttacks[color="blue"];
PasswordCracking -> SecurityTesting[style="dashed",color="blue"];
PasswordCracking -> NetworkAdminTesting[style="dashed",color="gray"];
PathTesting -> ControlFlowTesting[color="blue"];
PathTesting -> StructurebasedTesting[color="green"];
PathTesting -> ExhaustiveTesting[color="maroon"];
PathTesting -> DynamicTesting;
PathTesting -> WModelTesting;
PeerReviews -> Reviews[color="green"];
PeerReviews -> InformalTesting[color="maroon"];
PeerReviews -> StaticAnalysis[style="dashed",color="blue"];
PenetrationTesting -> SecurityTesting[color="green"];
PenetrationTesting -> PrivacyTesting[color="green"];
PenetrationTesting -> WebApplicationTesting[style="dashed",color="blue"];
PenetrationTesting -> Attacks;
PenetrationTesting -> OnlineTesting[style="dashed",color="gray"];
PenetrationTesting -> OperationalTesting[style="dashed",color="gray"];
PerformanceEfficiencyTesting -> ReliabilityTesting[color="green"];
PerformanceEfficiencyTesting -> PerformancerelatedTesting[color="green"];
PerformanceEfficiencyTesting -> PerformanceTesting[style="dashed",color="green"];
PerformanceEfficiencyTesting -> EfficiencyTesting[style="dashed",color="green"];
PerformanceTesting -> PerformancerelatedTesting[color="green"];
PerformanceTesting -> ModelbasedTesting[color="green"];
PerformanceTesting -> ConformanceTesting[color="green"];
PerformanceTesting -> NonfunctionalTesting[color="blue"];
PerformanceTesting -> AutomatedTesting;
PerformanceTesting -> DynamicTesting;
PerformanceTesting -> PerformanceTesting;
PerformanceTesting -> PostDeploymentMonitoring;
PerformanceTesting -> DynamicAnalysis;
PerformanceTesting -> StaticAnalysis;
PerformanceTesting -> RegressionTesting[style="dashed",color="green"];
PerformanceTesting -> WebApplicationTesting[style="dashed"];
PeriodicBuiltInTesting -> BuiltInTesting[color="blue"];
PersonalizationTesting -> FlexibilityTesting[color="blue"];
Pharming -> SecurityAttacks[color="blue"];
PhysicalConfigurationAudits -> Audits[color="green"];
Playtesting -> AdHocTesting[color="blue"];
Playtesting -> UserasTesterTesting[color="gray"];
PortabilityTesting -> ModelbasedTesting[color="green"];
PortabilityTesting -> ProcedureTesting[style="dashed",color="green"];
PostDeploymentMonitoring -> AutomatedTesting;
PostDeploymentMonitoring -> PerformanceTesting[style="dashed"];
PostDeploymentMonitoring -> RegressionTesting[color="gray"];
PostReleaseTesting -> OnlineTesting[style="dashed",color="gray"];
PostReleaseTesting -> OperationalTesting[style="dashed",color="gray"];
PowerUpBuiltInTesting -> BuiltInTesting[color="blue"];
PrimeContractorTesting -> DevelopmentOrganizationTesting[color="blue"];
PrimePathTesting -> ModelbasedTesting[color="blue"];
PrivacyTesting -> SecurityTesting[color="green"];
PrivacyTesting -> ComplianceTesting[style="dashed"];
ProcedureTesting -> FunctionalSuitabilityTesting[color="green"];
ProcedureTesting -> ModelbasedTesting[color="green"];
ProcessDrivenScripting -> ScriptedTesting[color="blue"];
ProcessDrivenScripting -> ScenarioTesting[color="blue"];
ProcessorintheLoopTesting -> SystemTesting[color="blue"];
ProductionAcceptanceTesting -> AcceptanceTesting[color="blue"];
ProductionVerificationTesting -> AcceptanceTesting[color="green"];
ProductionVerificationTesting -> OnlineTesting[style="dashed",color="gray"];
ProductionVerificationTesting -> OperationalTesting[style="dashed",color="gray"];
PrognosticsandHealthManagement -> OngoingBuiltInTesting[color="blue"];
PrognosticsandHealthManagement -> SelfTesting[style="dashed",color="blue"];
ProgrammerTesting -> DeveloperTesting[color="blue"];
ProofsofCorrectness -> ManualTesting[color="maroon"];
ProofsofCorrectness -> StaticAnalysis[color="maroon"];
ProofsofCorrectness -> SpecificationbasedTesting[color="maroon"];
ProofsofCorrectness -> FormalTesting[color="maroon"];
ProofsofCorrectness -> CorrectnessTesting[color="gray"];
ProofsofCorrectness -> MathematicalbasedTesting[color="gray"];
ProofsofCorrectness -> StaticTesting[style="dashed",color="gray"];
ProofsofPartialCorrectness -> ProofsofCorrectness[color="gray"];
ProofsofPartialCorrectness -> FormalTesting[color="gray"];
ProofsofPartialCorrectness -> MathematicalbasedTesting[color="gray"];
ProofsofPartialCorrectness -> StaticTesting[style="dashed",color="gray"];
ProofsofTotalCorrectness -> ProofsofCorrectness[color="gray"];
ProofsofTotalCorrectness -> FormalTesting[color="gray"];
ProofsofTotalCorrectness -> MathematicalbasedTesting[color="gray"];
ProofsofTotalCorrectness -> StaticTesting[style="dashed",color="gray"];
ProtectionSystemTesting -> SystemTesting[color="gray"];
ProtectionSystemTesting -> DomainSpecificTesting[style="dashed",color="gray"];
QualificationOperationalTesting -> OperationalTesting[color="blue"];
QualificationOperationalTesting -> QualificationTesting[style="dashed",color="gray"];
QuickTesting -> AdHocTesting[color="blue"];
RandomTesting -> SpecificationbasedTesting[color="green"];
RandomTesting -> MathematicalbasedTesting[color="green"];
RandomTesting -> ModelbasedTesting[color="green"];
RandomTesting -> OperationalProfileTesting[color="green"];
RandomTesting -> UsagebasedTesting[color="blue"];
RandomWalkTesting -> PathTesting[style="dashed"];
RandomWalkTesting -> WebApplicationTesting[style="dashed"];
RapidPrototypingTesting -> LifecyclebasedTesting[color="gray"];
RapidPrototypingTesting -> AttheBeginningTesting[style="dashed",color="gray"];
ReactiveTesting -> DynamicTesting[style="dashed",color="blue"];
RecoverabilityTesting -> ReliabilityTesting[color="green"];
RecoverabilityTesting -> UsabilityTesting[style="dashed",color="blue"];
RecoveryTesting -> PerformancerelatedTesting[color="green"];
RecoveryTesting -> ReliabilityTesting[color="green"];
RecoveryTesting -> AvailabilityTesting[color="green"];
RecoveryTesting -> FaultToleranceTesting[color="blue"];
RecoveryTesting -> NonfunctionalTesting[color="blue"];
RecoveryTesting -> DynamicTesting[style="dashed"];
RedTeamTesting -> PenetrationTesting[color="blue"];
RegressionTesting -> FunctionalTesting[color="blue"];
RegressionTesting -> NonfunctionalTesting[color="blue"];
RegressionTesting -> ContinuousTesting[color="blue"];
RegressionTesting -> AgileTesting[color="blue"];
RegressionTesting -> DevOpsTesting[color="blue"];
RegressionTesting -> ChangeRelatedTesting[color="blue"];
RegressionTesting -> WebApplicationTesting[color="blue"];
RegressionTesting -> DeveloperTesting[style="dashed"];
RegressionTesting -> AutomatedTesting[style="dashed",color="maroon"];
RegressionTesting -> Retesting[color="green"];
RegressionTesting -> RelativeCorrectnessTesting[color="gray"];
RegulatoryAcceptanceTesting -> AcceptanceTesting[color="blue"];
RelativeCorrectnessTesting -> CorrectnessTesting[color="gray"];
ReliabilityEnhancementTesting -> ReliabilityTesting[color="blue"];
ReliabilityGrowthTesting -> ReliabilityTesting[color="blue"];
ReliabilityMechanismTesting -> ReliabilityTesting[color="blue"];
ReliabilityTesting -> NonfunctionalTesting[color="blue"];
ReliabilityTesting -> ModelbasedTesting[color="green"];
ReliabilityTesting -> OperationalProfileTesting[color="green"];
ReliabilityTesting -> SecurityTesting[style="dashed",color="green"];
ReliabilityTesting -> WebApplicationTesting[color="blue"];
ReliabilityTesting -> PerformanceTesting;
ReliabilityTesting -> AutomatedTesting;
ReliabilityTesting -> SmokeTesting;
ReliabilityTesting -> DynamicTesting;
ReliabilityTesting -> RegressionTesting[style="dashed",color="green"];
RemoteTesting -> AsynchronousTesting[style="dashed"];
RemoteTesting -> ManualTesting[style="dashed"];
RepetitionTesting -> NegativeTesting[color="maroon"];
RequestTesting -> WebApplicationTesting[color="blue"];
RequirementsbasedTesting -> SpecificationbasedTesting[style="dashed",color="green"];
RequirementsbasedTesting -> ModelbasedTesting[color="green"];
RequirementsbasedTesting -> RiskbasedTesting[color="green"];
RequirementsbasedTesting -> PositiveTesting[color="green"];
RequirementsbasedTesting -> UsersessionbasedTesting[color="blue"];
RequirementsAnimation -> StaticTesting;
RequirementsAnimation -> WModelTesting;
RequirementsAnimation -> RequirementsdrivenTesting[color="gray"];
RequirementsEngineerTesting -> DeveloperTesting[color="blue"];
RequirementsEngineerTesting -> RequirementsdrivenTesting[style="dashed",color="gray"];
ResourceUtilizationTesting -> PerformanceEfficiencyTesting[color="green"];
ResourceUtilizationTesting -> PerformanceTesting[color="maroon"];
ResourceUtilizationTesting -> EfficiencyTesting[style="dashed"];
ResponseTimeTesting -> PerformancerelatedTesting[color="green"];
ResponseTimeTesting -> PerformanceTesting[color="blue"];
ResponseTimeTesting -> TimingTesting[style="dashed"];
Retesting -> ChangeRelatedTesting[color="blue"];
Reviews -> StaticTesting[color="green"];
Reviews -> WModelTesting;
Reviews -> StructuralAnalysis[color="maroon"];
Reviews -> StructuredTesting[color="maroon"];
Reviews -> RolebasedTesting[color="maroon"];
Reviews -> StaticAnalysis[style="dashed",color="blue"];
Reviews -> ReliabilityTesting[style="dashed",color="maroon"];
Reviews -> MaintainabilityTesting[style="dashed",color="maroon"];
Reviews -> PortabilityTesting[style="dashed",color="maroon"];
ReWebTesting -> WebApplicationTesting;
ReWebTesting -> StructurebasedTesting;
ReWebTesting -> ControlFlowTesting;
ReWebTesting -> FunctionalTesting;
RiskbasedTesting -> SpecificationbasedTesting[color="blue"];
RiskbasedTesting -> DynamicTesting[style="dashed",color="green"];
RobustnessTesting -> StressTesting[style="dashed"];
RolebasedReviews -> Reviews[color="blue"];
RolebasedReviews -> RolebasedTesting[color="gray"];
RolebasedReviews -> ScenariobasedTesting[style="dashed",color="gray"];
RuntimeAssertionChecking -> AssertionChecking;
RuntimeAssertionChecking -> DynamicAnalysis[style="dashed",color="gray"];
RuntimeAssertionChecking -> SpecificationbasedTesting[style="dashed",color="gray"];
SafetyDemonstrations -> SafetyTesting[color="gray"];
SafetyEngineerTesting -> DeveloperTesting[color="blue"];
SafetyEngineerTesting -> SafetyTesting[color="gray"];
SandwichTesting -> IntegrationTesting[color="blue"];
ScalabilityTesting -> PortabilityTesting[color="green"];
ScalabilityTesting -> NonfunctionalTesting[color="blue"];
ScalabilityTesting -> LoadTesting[color="blue"];
ScalabilityTesting -> VolumeTesting[color="blue"];
ScalabilityTesting -> TransactionFlowTesting[style="dashed",color="blue"];
ScalabilityTesting -> ElasticityTesting[style="dashed",color="blue"];
ScalabilityTesting -> PerformanceTesting;
ScalabilityTesting -> EfficiencyTesting;
ScalabilityTesting -> PerformanceEfficiencyTesting[style="dashed"];
ScalabilityTesting -> MemoryManagementTesting[style="dashed",color="green"];
ScalabilityTesting -> BackupTesting[style="dashed"];
ScenarioTesting -> SpecificationbasedTesting[color="green"];
ScenarioTesting -> ModelbasedTesting[color="green"];
ScenarioTesting -> SystemTesting[color="green"];
ScenarioTesting -> AcceptanceTesting[style="dashed",color="green"];
ScenarioTesting -> StressTesting[style="dashed",color="green"];
ScenarioTesting -> ErrorbasedTesting[style="dashed",color="green"];
ScenarioTesting -> EndtoendTesting[style="dashed",color="green"];
ScenarioTesting -> FunctionalTesting[style="dashed",color="green"];
ScenarioTesting -> ObjectOrientedTesting[style="dashed"];
ScenarioWalkthroughs -> StaticTesting;
ScenarioWalkthroughs -> WModelTesting;
ScenarioWalkthroughs -> Walkthroughs[color="gray"];
ScenarioWalkthroughs -> ScenarioTesting[color="gray"];
ScenariobasedEvaluations -> StructuralAnalysis[style="dashed",color="maroon"];
ScenariobasedEvaluations -> ScenariobasedTesting[style="dashed",color="gray"];
ScenariobasedReviews -> Reviews[color="blue"];
ScenariobasedReviews -> ScenariobasedTesting[color="gray"];
ScriptedTesting -> DynamicTesting[color="green"];
ScriptedTesting -> ManualTesting[style="dashed",color="green"];
SecurityAttacks -> SecurityTesting[color="blue"];
SecurityAttacks -> Attacks[color="blue"];
SecurityAttacks -> IntegrityTesting[color="blue"];
SecurityAudits -> SecurityTesting[color="green"];
SecurityAudits -> Audits[color="green"];
SecurityAudits -> StaticTesting[color="green"];
SecurityAudits -> Reviews[color="green"];
SecurityAudits -> Walkthroughs[color="green"];
SecurityEngineerTesting -> DeveloperTesting[color="blue"];
SecurityEngineerTesting -> SecurityTesting[color="gray"];
SecurityTesting -> ReliabilityTesting[color="green"];
SecurityTesting -> ModelbasedTesting[color="green"];
SecurityTesting -> ConformanceTesting[color="green"];
SecurityTesting -> FunctionalityTesting;
SecurityTesting -> DynamicTesting;
SecurityTesting -> SysAdminTesting;
SecurityTesting -> PostDeploymentMonitoring;
SecurityTesting -> NonfunctionalTesting;
SecurityTesting -> BackupTesting[style="dashed"];
SecurityTesting -> ComplianceTesting[style="dashed"];
SecurityTesting -> SystemTesting[style="dashed"];
SessionbasedTesting -> WebApplicationTesting[color="blue"];
ShiftLeftTesting -> LifecyclebasedTesting[style="dashed",color="gray"];
ShoeTesting -> RandomTesting[color="blue"];
ShutdownBuiltInTesting -> BuiltInTesting[color="blue"];
SignChangeTesting -> MLModelTesting[color="gray"];
SignSignTesting -> MLModelTesting[color="gray"];
SimilaritybasedPrioritizationTesting -> PrioritizationTesting[color="blue"];
SingleHitDecisionTable -> DecisionTableTesting[color="green"];
SiteAcceptanceTesting -> AcceptanceTesting[color="blue"];
SmokeTesting -> AdHocTesting[color="blue"];
SmokeTesting -> QuickTesting[color="blue"];
SmokeTesting -> UnscriptedTesting[style="dashed",color="blue"];
SmokeTesting -> IntegrationTesting;
SmokeTesting -> AutomatedTesting[style="dashed"];
SmokeTesting -> OfflineTesting[style="dashed",color="gray"];
SoftwareDesignAudits -> Audits[color="green"];
SoftwareDesignAudits -> Reviews[color="green"];
SoftwareQualificationTesting -> SafetyTesting[style="dashed"];
SoftwareQualificationTesting -> SystemTesting[style="dashed",color="gray"];
SoftwareQualificationTesting -> DynamicTesting[style="dashed",color="gray"];
SoftwareintheLoopTesting -> SystemTesting[color="blue"];
SoftwareintheLoopTesting -> FormalTesting;
SoftwareintheLoopTesting -> DynamicTesting[color="blue"];
SoftwareintheLoopTesting -> IntegrationTesting[style="dashed"];
SoSIntegrationTesting -> SystemTesting[color="blue"];
SoSIntegrationTesting -> IntegrationTesting[color="gray"];
SoSTesting -> SystemTesting[color="blue"];
SpecificationbasedTesting -> DynamicTesting[color="green"];
SpecificationbasedTesting -> ModelbasedTesting[style="dashed",color="green"];
SpiralTesting -> LifecyclebasedTesting[color="blue"];
SpiralTesting -> RiskbasedTesting[style="dashed",color="blue"];
SpiralTesting -> ContinuousTesting[color="gray"];
SQLInjection -> CodeInjection[color="blue"];
StabilityTesting -> BuildVerificationTesting[color="blue"];
StabilityTesting -> ReliabilityTesting[style="dashed",color="blue"];
StateTesting -> ModelbasedTesting[color="green"];
StateTesting -> StateTransitionTesting[color="green"];
StateTesting -> BehaviouralTesting[color="maroon"];
StateTransitionTesting -> SpecificationbasedTesting[style="dashed",color="green"];
StateTransitionTesting -> ModelbasedTesting[color="green"];
StateTransitionTesting -> WebApplicationTesting[color="blue"];
StatebasedWebBrowserTesting -> WebApplicationTesting;
StatebasedWebBrowserTesting -> FormalMethods;
StatebasedWebBrowserTesting -> SpecificationbasedTesting;
StatebasedWebBrowserTesting -> ControlFlowTesting;
StatebasedWebBrowserTesting -> FunctionalTesting;
StatebasedWebBrowserTesting -> StateTransitionTesting;
StatebasedWebBrowserTesting -> TransactionTesting;
StatebasedWebBrowserTesting -> TransactionVerification;
StatementTesting -> StructurebasedTesting[color="green"];
StatementTesting -> ControlFlowTesting[color="green"];
StatementTesting -> ModelbasedTesting[color="green"];
StatementTesting -> AutomatedTesting[style="dashed",color="green"];
StatementTesting -> BranchTesting[color="maroon"];
StatementTesting -> BranchTesting[style="dashed",color="green"];
StatementTesting -> BranchConditionCombinationTesting[color="maroon"];
StatementTesting -> UnitTesting[style="dashed",color="maroon"];
StatementTesting -> DeveloperTesting[style="dashed",color="green"];
StatementTesting -> DecisionTesting[style="dashed",color="green"];
StaticAnalysis -> StaticTesting[color="green"];
StaticAnalysis -> WModelTesting;
StaticAnalysis -> DeveloperTesting[style="dashed"];
StaticAssertionChecking -> AssertionChecking;
StaticAssertionChecking -> StaticAnalysis;
StaticAssertionChecking -> SpecificationbasedTesting[style="dashed",color="gray"];
StaticTesting -> WModelTesting;
StatisticalTesting -> UsagebasedTesting[color="blue"];
StatisticalTesting -> ModelbasedTesting[color="gray"];
StepwiseAbstraction -> StructuralAnalysis[style="dashed",color="maroon"];
StressTesting -> PerformancerelatedTesting[color="green"];
StressTesting -> PerformanceEfficiencyTesting[color="green"];
StressTesting -> ModelbasedTesting[color="green"];
StressTesting -> ConformanceTesting[color="green"];
StressTesting -> NonfunctionalTesting[color="blue"];
StressTesting -> PerformanceTesting[color="blue"];
StressTesting -> CapacityTesting[color="blue"];
StressTesting -> BoundaryConditionTesting[color="maroon"];
StressTesting -> BoundaryConditionTesting[style="dashed",color="green"];
StressTesting -> NegativeTesting[color="maroon"];
StressTesting -> DynamicTesting;
StressTesting -> WModelTesting;
StrongMutationTesting -> MutationTesting[color="maroon"];
StrongMutationTesting -> FaultbasedTesting[color="maroon"];
StrongMutationTesting -> SystemTesting[style="dashed",color="gray"];
StructuralAnalysis -> StaticTesting[color="maroon"];
StructuralAnalysis -> StructurebasedTesting[color="maroon"];
StructuralAnalysis -> AttheBeginningTesting[style="dashed",color="maroon"];
StructuralTesting -> DynamicTesting[color="maroon"];
StructuralTesting -> StructurebasedTesting[color="maroon"];
StructurebasedTesting -> ModelbasedTesting[color="green"];
StructurebasedTesting -> DynamicTesting[color="green"];
StructurebasedTesting -> ControlFlowTesting[color="green"];
StructurebasedTesting -> AutomatedTesting[style="dashed",color="green"];
StructurebasedTesting -> StaticTesting[style="dashed",color="green"];
StructuredScripting -> ScriptedTesting[color="gray"];
StructuredScripting -> StructuredTesting[color="gray"];
StructuredWalkthroughs -> Walkthroughs[color="gray"];
StructuredWalkthroughs -> StructuredTesting[color="gray"];
StructuredWalkthroughs -> StaticTesting[color="gray"];
StuckKeyTesting -> RandomTesting[color="blue"];
SubboundaryConditionTesting -> DataTesting[color="maroon"];
SubboundaryConditionTesting -> StructurebasedTesting[color="maroon"];
SubboundaryConditionTesting -> BoundaryConditionTesting[style="dashed",color="maroon"];
SubcontractorTesting -> DevelopmentOrganizationTesting[color="blue"];
SubsystemTesting -> SystemTesting[color="blue"];
SubsystemTesting -> UnitTesting[style="dashed",color="blue"];
SummativeEvaluations -> SystemTesting[style="dashed",color="gray"];
SymbolicExecution -> StaticTesting[color="gray"];
SymbolicExecution -> StaticAnalysis[style="dashed",color="gray"];
SymbolicExecution -> CorrectnessTesting[style="dashed",color="gray"];
SyntaxTesting -> SpecificationbasedTesting[color="green"];
SyntaxTesting -> ModelbasedTesting[color="green"];
SyntaxTesting -> FormalTesting[color="green"];
SyntaxTesting -> PositiveTesting[color="green"];
SyntaxTesting -> NegativeTesting[color="green"];
SyntaxTesting -> ExperiencebasedTesting[color="green"];
SyntaxTesting -> OnetoOneTesting[style="dashed",color="green"];
SyntaxTesting -> MinimizedTesting[style="dashed",color="green"];
SyntaxTesting -> DataTesting[color="maroon"];
SyntaxTesting -> GrayBoxTesting[style="dashed",color="maroon"];
SyntaxTesting -> StaticTesting;
SyntaxTesting -> StaticTesting[style="dashed",color="blue"];
SyntaxTesting -> SmokeTesting;
SyntaxTesting -> DesktopDevelopmentTesting;
SyntaxTesting -> FunctionalTesting[style="dashed",color="blue"];
SyntaxTesting -> WebApplicationTesting[style="dashed"];
SysAdminTesting -> OperatorTesting[color="blue"];
SystemIntegrationTesting -> IntegrationTesting[color="blue"];
SystemIntegrationTesting -> SystemTesting[color="blue"];
SystemQualificationTesting -> SystemTesting[style="dashed",color="gray"];
SystemQualificationTesting -> DynamicTesting[style="dashed",color="gray"];
SystemTesting -> SystemTesting[color="blue"];
SystemTesting -> SpecificationbasedTesting[color="maroon"];
SystemTesting -> VModelTesting;
SystemTesting -> WModelTesting;
SystemTesting -> DynamicTesting;
SystemTesting -> RandomTesting[style="dashed",color="maroon"];
SystemTesting -> ReliabilityTesting[style="dashed",color="maroon"];
SystemTesting -> IntegrationTesting[style="dashed"];
SystemsIntegrationTesting -> IntegrationTesting[color="gray"];
TailoredConformanceTesting -> ConformanceTesting[color="gray"];
TechnicalReviews -> FormalReviews[color="blue"];
TechnicalTesting -> NonfunctionalTesting[style="dashed",color="gray"];
TemplateVariableTesting -> WebApplicationTesting[color="blue"];
TemplateVariableTesting -> IntegrationTesting;
TemplateVariableTesting -> GrayBoxTesting;
TestBrowsing -> IntegrationTesting;
TestBrowsing -> WebApplicationTesting[color="gray"];
TestdrivenDevelopment -> DeveloperTesting;
TestdrivenDevelopment -> RobustnessTesting;
TestdrivenDevelopment -> FaultToleranceTesting;
TestdrivenDevelopment -> IncrementalTesting;
TesterTesting -> RolebasedTesting[color="blue"];
TestUMLTesting -> WebApplicationTesting;
TestUMLTesting -> FormalMethods;
TestUMLTesting -> StructurebasedTesting;
TestUMLTesting -> ControlFlowTesting;
TestUMLTesting -> FunctionalTesting;
TestUMLTesting -> RandomWalkTesting[style="dashed"];
TestWebTesting -> WebApplicationTesting;
TestWebTesting -> StructurebasedTesting;
TestWebTesting -> ControlFlowTesting;
TestWebTesting -> FunctionalTesting;
ThinkAloudUsabilityTesting -> UsabilityTesting[color="gray"];
ThreeValueBoundaryTesting -> BoundaryValueAnalysis[color="green"];
ThresholdTesting -> MLModelTesting[color="gray"];
TimingTesting -> NegativeTesting[color="maroon"];
TopDownTesting -> IntegrationTesting[color="blue"];
TopDownTesting -> IncrementalTesting;
Tours -> ExploratoryTesting[color="green"];
Tours -> ExperiencebasedTesting[color="green"];
TransactionFlowTesting -> ScenarioTesting[color="green"];
TransactionTesting -> DynamicTesting[style="dashed"];
TransactionTesting -> FunctionalityTesting;
TransactionTesting -> SystemTesting;
TransactionTesting -> FunctionalTesting;
TransactionTesting -> WebApplicationTesting[style="dashed"];
TransactionTesting -> WModelTesting[style="dashed"];
TransactionTesting -> IntegrationTesting;
TransactionTesting -> EndtoendFunctionalityTesting;
TransactionVerification -> SmokeTesting;
TransactionVerification -> TestBrowsing;
TransactionVerification -> StaticTesting;
TransactionVerification -> WebApplicationTesting[style="dashed"];
TransactionVerification -> IntegrationTesting[style="dashed"];
TranslationValidation -> EquivalenceChecking[style="dashed"];
twiseTesting -> AllCombinationsTesting[color="blue"];
TwoValueBoundaryTesting -> BoundaryValueAnalysis[color="green"];
UITesting -> WebApplicationTesting[color="blue"];
UITesting -> KeyworddrivenTesting[style="dashed",color="green"];
UMLModelbasedTesting -> ModelbasedTesting[color="gray"];
UnitTesting -> ConstructionTesting[color="blue"];
UnitTesting -> FunctionalityTesting[style="dashed",color="green"];
UnitTesting -> SmokeTesting[style="dashed",color="green"];
UnitTesting -> AutomatedTesting[style="dashed",color="blue"];
UnitTesting -> VModelTesting;
UnitTesting -> WModelTesting;
UnitTesting -> DynamicTesting;
UnscriptedTesting -> DynamicTesting[color="green"];
UsabilityTestScripting -> UsabilityTesting[color="blue"];
UsabilityTestScripting -> StructuredTesting[color="blue"];
UsabilityTesting -> ModelbasedTesting[color="green"];
UsabilityTesting -> ConformanceTesting[color="green"];
UsabilityTesting -> NonfunctionalTesting;
UsabilityTesting -> SystemTesting;
UsabilityTesting -> ManualTesting;
UsabilityTesting -> UsabilityTesting;
UsabilityTesting -> RegressionTesting[style="dashed",color="blue"];
UsabilityReviews -> UsabilityTesting[color="blue"];
UsabilityReviews -> Reviews[color="blue"];
UsabilityWalkthroughs -> UsabilityTesting[color="green"];
UsabilityWalkthroughs -> Walkthroughs[color="gray"];
UsagebasedTesting -> OperationalProfileTesting[color="blue"];
UsagebasedTesting -> OnlineTesting[style="dashed",color="gray"];
UseCaseTesting -> ModelbasedTesting[color="green"];
UseCaseTesting -> ScenarioTesting[color="green"];
UseCaseTesting -> SpecificationbasedTesting;
UseCaseTesting -> UserStoryTesting[style="dashed",color="gray"];
UserAcceptanceTesting -> AcceptanceTesting[color="green"];
UserAcceptanceTesting -> UserTesting[color="gray"];
UserasTesterTesting -> UserTesting[color="blue"];
UserInterfaceNavigationTesting -> SpecificationbasedTesting[color="blue"];
UserInterfaceNavigationTesting -> UITesting[color="gray"];
UserInterfaceNavigationTesting -> UserTesting[color="gray"];
UserInterfaceNavigationTesting -> ProcedureTesting[style="dashed",color="gray"];
UserOrganizationTesting -> OrganizationbasedTesting[color="blue"];
UserOrganizationTesting -> UserTesting[color="gray"];
UserSessionDataTesting -> WebApplicationTesting;
UserSessionDataTesting -> UserSessionTesting;
UserSessionDataTesting -> SpecificationbasedTesting[style="dashed"];
UserSessionDataTesting -> ControlFlowTesting;
UserSessionDataTesting -> FunctionalTesting;
UserSessionTesting -> UserTesting[color="gray"];
UserStoryTesting -> SpecificationbasedTesting[color="blue"];
UserStoryTesting -> UserTesting[color="gray"];
UserSurveys -> UsabilityTesting[color="gray"];
UserSurveys -> UserTesting[color="gray"];
UserTesting -> RolebasedTesting[color="blue"];
UserbasedEvaluations -> UsabilityTesting[color="green"];
UserbasedEvaluations -> UserTesting[color="gray"];
UserbasedEvaluations -> AcceptanceTesting[style="dashed",color="gray"];
UserinitiatedBuiltInTesting -> BuiltInTesting[color="blue"];
UsersessionbasedTesting -> SessionbasedTesting[color="gray"];
UsersessionbasedTesting -> UserasTesterTesting[style="dashed",color="gray"];
UsersessionbasedTesting -> UserStoryTesting[style="dashed",color="gray"];
VisualBrowserValidation -> UsabilityTesting;
VisualBrowserValidation -> ManualTesting;
VisualBrowserValidation -> StaticTesting[style="dashed"];
VisualBrowserValidation -> DynamicTesting[style="dashed"];
VisualBrowserValidation -> CrossBrowserCompatibilityTesting[style="dashed"];
VisualBrowserValidation -> WebApplicationTesting[color="gray"];
VisualBrowserValidation -> VisualTesting[color="gray"];
VisualTesting -> GUITesting;
VisualTesting -> WebApplicationTesting;
VModelTesting -> WaterfallTesting[color="blue"];
VModelTesting -> LifecyclebasedTesting[color="blue"];
VModelTesting -> WModelTesting[style="dashed"];
VolumeTesting -> PerformancerelatedTesting[color="green"];
VolumeTesting -> ModelbasedTesting[color="green"];
VolumeTesting -> ConformanceTesting[color="green"];
VolumeTesting -> PerformanceEfficiencyTesting[color="green"];
VolumeTesting -> NonfunctionalTesting[color="blue"];
VolumeTesting -> CapacityTesting[color="blue"];
VolumeTesting -> CapacityTesting[style="dashed",color="green"];
VolumeTesting -> DynamicTesting;
VolumeTesting -> WModelTesting;
VulnerabilityScanning -> SecurityTesting[color="green"];
VulnerabilityScanning -> AutomatedTesting[color="green"];
Walkthroughs -> StaticAnalysis[color="green"];
Walkthroughs -> FormalReviews[color="blue"];
Walkthroughs -> Reviews[color="blue"];
Walkthroughs -> RolebasedTesting[style="dashed",color="maroon"];
WaterfallTesting -> LifecyclebasedTesting[color="blue"];
WeakMutationTesting -> MutationTesting[color="maroon"];
WeakMutationTesting -> FaultbasedTesting[color="maroon"];
WeakMutationTesting -> StrongMutationTesting[style="dashed",color="maroon"];
WeakMutationTesting -> UnitTesting[color="gray"];
WebApplicationTesting -> DomainSpecificTesting[style="dashed",color="gray"];
WebAppSlicing -> WebApplicationTesting;
WebAppSlicing -> Slicing;
WebAppSlicing -> StructurebasedTesting;
WebAppSlicing -> DataFlowTesting;
WebAppSlicing -> FunctionalTesting;
WModelTesting -> IncrementalTesting[style="dashed",color="gray"];
WModelTesting -> ContinuousTesting[style="dashed",color="gray"];

subgraph cluster_legend {
    label="Legend";
    labelloc="b";
    fontsize="48pt"
    {
        rank=same
        chd [label="Child"];
        par [label="Parent"];
        chd -> par [label="                "];
        syn1 [label="Synonym"];
        syn2 [label="Synonym"];
        syn1 -> syn2 [dir=none label="                "];
    }
    {
        rank=same
        imp1 [label="Child"];
        imp2 [label=<Implied<br/>Parent>];
        imp1 -> imp2 [style="dashed" label="                "]
        imp3 [label=<Implied<br/>Synonym>];
        imp4 [label=<Implied<br/>Synonym>];
        imp3 -> imp4 [style="dashed" dir=none label="                "]
    }
    {
        rank=same
        imp5 [label=<Implied<br/>Term> style="dashed"]
        syn3 [label=<Term>]
        syn4 [label=<Synonym<br/>to Both> style="dotted"]
        syn5 [label=<Term>]
        syn3 -> syn4 -> syn5 [dir=none]
    }
{
rank=same
src1 [style=invis];
src2 [style=invis];
src1 -> src2 [color=green, label=<From Established<br/>Standards>]
src3 [style=invis];
src4 [style=invis];
src3 -> src4 [color=blue, label=<From ``Meta-level''<br/>Collections>]
}
{
rank=same
src5 [style=invis];
src6 [style=invis];
src5 -> src6 [color=maroon, label=<From Textbooks>]
src7 [style=invis];
src8 [style=invis];
src7 -> src8 [color=gray, label=<From Inferences>]
}
    edge [style="invis"]
    imp1 -> chd
    imp2 -> par
    imp3 -> syn1
    imp4 -> syn2
imp5 -> imp1
syn3 -> imp2
syn4 -> imp3
syn5 -> imp4
src1 -> imp5
src2 -> syn3
src3 -> syn4
src4 -> syn5
src5 -> src1
src6 -> src2
src7 -> src3
src8 -> src4
}

// Connect the dummy node to the first node of the legend
start -> chd [style="invis"];
}
\end{document}

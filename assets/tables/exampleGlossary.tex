% makecell with new lines so VS Code doesn't freak out
\def\app{\makecell{Approach\\Category}}

\begin{table}[hbtp!]
    \centering
    \caption{Selected entries from glossary of test approaches with ``Notes'' column excluded for brevity.}
    \label{tab:approachGlossaryExcerpt}
    \begin{tabularx}{\linewidth}{|m{1.7cm}|m{4.2cm}|X|m{7.7cm}|m{2.8cm}|}
        \hline
        \thead{Name}               & \thead{\app}                                                                                                                   & \thead{Definition}                                                                                                                                   & \thead{Parent(s)}                                                                                                                                                                                                                                                                                                                                                                                                      & \thead{Synonym(s)}                            \\
        \hline
        A/B Testing                & Practice \citep[p.~22]{IEEE2022}, Type (implied by \citealp[p.~58]{Firesmith2015})                                             & Testing ``that allows testers to determine which of two systems or components performs better'' \citep[p.~1]{IEEE2022}                               & Statistical Testing \citep[pp.~1,~35]{IEEE2022}, Usability Testing \citep[p.~58]{Firesmith2015}                                                                                                                                                                                                                                                                                                                        & Split-Run Testing \citep[pp.~1,~35]{IEEE2022} \\[1cm]
        All Combinations Testing   & Technique (\citealp[p.~22]{IEEE2022}; \citeyear[pp.~2,~16]{IEEE2021}; \citealp[p.~5-11]{SWEBOK2024})                           & Testing that covers ``all unique combinations of P-V pairs'' \citep[p.~16]{IEEE2021}                                                                 & Combinatorial Testing \citetext{\citealp[p.~22]{IEEE2022}; \citeyear[pp.~2,~16,~Fig.~2]{IEEE2021}; \citealp[p.~5-11]{SWEBOK2024}}                                                                                                                                                                                                                                                                                      & ---                                           \\[1cm]
        Big-Bang Testing           & Level (inferred from integration testing)                                                                                      & ``Testing in which \dots{} [components of a system] are combined all at once into an overall system, rather than in stages'' \citep[p.~45]{IEEE2017} & Integration Testing \citetext{\citealp[p.~45]{IEEE2017}; \citealp[p.~5-7]{SWEBOK2024}; \citealp[p.~603]{SharmaEtAl2021}; \citealp[p.~42]{Kam2008}}                                                                                                                                                                                                                                                                     & ---                                           \\[1cm]
        Classification Tree Method & Technique (\citealp[p.~22]{IEEE2022}; \citeyear[pp.~2,~12,~Fig.~2]{IEEE2021}; \citealpISTQB{})                                 & Testing ``based on exercising classes in a classification tree'' \citep[p.~22]{IEEE2021}                                                             & Specification-based Testing (\citealp[p.~22]{IEEE2022}; \citeyear[pp.~2,~12,~Fig.~2]{IEEE2021}; \citealpISTQB{}; \citealp[p.~47]{Firesmith2015}), Model-based Testing (\citealp[p.~13]{IEEE2022}; \citeyear[pp.~6,~12]{IEEE2021})                                                                                                                                                                                      & Classification Tree Technique \citepISTQB{}   \\[1.5cm] % Don't know why extra is needed here but it makes it look better
        Data Flow Testing          & Technique (\citealp[p.~22]{IEEE2022}; \citeyear[pp.~3,~27]{IEEE2021}; \citealp[p.~5-13]{SWEBOK2024}; \citealp[p.~43]{Kam2008}) & A ``class of \dots{} techniques based on exercising definition-use pairs'' \citep[p.~3; similar on p.~27]{IEEE2021}                                  & Structure-based Testing (\citealp[p.~22]{IEEE2022}; \citeyear[pp.~3,~27,~Fig.~2]{IEEE2021}; \citealp[p.~43]{Kam2008}), Control Flow Testing (\citeyear[p.~27]{IEEE2021}; implied by \citealp[p.~5-13]{SWEBOK2024}; \citealp[p.~101]{IEEE2017}), Model-based Testing (\citeyear[p.~27]{IEEE2021}; implied by \citealp[p.~179]{DoğanEtAl2014}), Web Application Testing (p. 179; can be in \citealp[pp.~16-17]{Kam2008}) & ---                                           \\
        \hline
    \end{tabularx}
\end{table}
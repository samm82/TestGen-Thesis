% Defined here so VS Code doesn't freak out
\def\ieeeEquiv{\makecell{IEEE\\Equivalent}}
\def\swebokLevel{{Level\\(objective-\\based)\TblrNote{a}}}

\begin{longtblr}[
    note{a} = {See \flawref{stage-level-syns}.},
    note{b} = {Testing methods and guidances are omitted from this table
            since \citet{BarbosaEtAl2006} do not define or give examples of them.},
    note{c} = {Synonyms for these examples are used by
            \citet[p.~3; OG Mathur, 2012]{SouzaEtAl2017} and
            \citet[p.~3]{BarbosaEtAl2006}.},
    caption={Categories of testing given by other sources.},
    label={tab:otherCats}
    ]{
    colspec={|X[0.08,c,m]|X[0.43,m]|X[0.34,m]|Q[c,m]|},
    width = \linewidth, rowhead = 1
    }
    \hline
    \thead{Term}                           & \thead{Definition}           & \thead{Examples} & \thead{\ieeeEquiv{}} \\
    \hline
    % Guidance                               & none given
    % \citep[p.~3]{BarbosaEtAl2006}          & none given         & Technique?                              \\
    \swebokLevel{}                         & Test levels based on the
    purpose of testing \citep[p.~5\=/6]{SWEBOK2024} that ``determine
    how the test suite is identified \dots\ regarding its consistency
    \dots\ and its composition''
    \citetext{p.~5\=/2}                    & conformance testing,
    installation testing, regression testing, performance testing,
    security testing % reliability testing,
    \citep[pp.~5\=/7 to 5\=/9]{SWEBOK2024} & Type?                                                                  \\
    % Method                                 & none given
    % \citep[p.~3]{BarbosaEtAl2006}          & none given         & Practice?                               \\
    Phase                                  & none given
    %(\citealp[p.~221]{Perry2006}; \citealp[p.~3]{BarbosaEtAl2006})  
                                           & unit testing,
    integration testing, system testing, regression testing (\citealp[p.~221]{Perry2006};
    \citealp[p.~3]{BarbosaEtAl2006})       & Level                                                                  \\
    Procedure                              & The basis for how
    testing is performed that guides the process; ``categorized in[to] testing methods,
    testing guidances\TblrNote{b} and testing techniques''
    \citep[p.~3]{BarbosaEtAl2006}          & none given
    generally; see ``Technique''           & Approach                                                               \\
    Process                                & ``A sequence of
    testing steps'' \citep[p.~2]{BarbosaEtAl2006} ``based on a development technology and \dots\
    paradigm, as well as on a testing procedure''
    \citetext{p.~3}                        & none given                   & Practice                                \\
    Stage                                  & An
    alternative to the ``traditional \dots\ test stages'' %\footnote{See ``Level'' in \Cref{tab:ieeeCats}.}
    based on ``clear technical groupings''
    \citep[p.~13]{Gerrard2000a}            & desktop development testing,
    infrastructure testing,
    % system testing, large scale integration, and
    post-deployment monitoring
    \citep[p.~13]{Gerrard2000a}            & Level                                                                  \\
    Technique                              & ``Systematic
    procedures and approaches for generating or selecting the most suitable test suites''
    \citep[p.~5\=/10]{SWEBOK2024}          & specification-based testing,
    % ``on a sound theoretical basis'' \citep[p.~3]{BarbosaEtAl2006}
    structure-based testing, fault-based testing\TblrNote{c}
    % , experience-based testing, usage-based testing
    (\citealp[pp.~5\=/10, 5\=/13 to 5\=/15]{SWEBOK2024})
    % black-box, white-box, defect/fault-based, model-based testing
    % \citetext{\citealp[p.~3]{SouzaEtAl2017}; OG Mathur, 2012};
    % functional, structural, error-based, state-based testing \citep[p.~3]{BarbosaEtAl2006}
                                           & Technique                                                              \\
    \hline
\end{longtblr}

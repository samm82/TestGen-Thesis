% so VS Code doesn't freak out
\newcommand{\techniqueCell}{\makecell{(Design)\\Technique}}
\newcommand{\procLevel}{``Level'' can also refer to the ``level''
    of a test process \citep[p.~24]{IEEE2022}.}

\begin{table}[hbtp!]
    \centering
    \caption{IEEE Testing Terminology}
    \label{tab:ieeeTestTerms}
    % m{0.145\linewidth}
    \begin{tabularx}{\linewidth}{|c|X|m{0.275\linewidth}|}
        \hline
        \rowcolor{McMasterMediumGrey}
        \thead{Term}                    & \thead{Definition}               & \thead{Examples} \\
        \hline
        Approach                        & A ``high-level test
        implementation choice, typically made as part of the test strategy
        design activity'' that includes ``test level, test type, test technique,
        test practice and the form of static testing to be used''
        \citep[p.~10]{IEEE2022}         & any of the examples given below:
        equivalence partitioning, unit testing, scripted testing,
        security testing                                                                      \\
        \techniqueCell                  & A ``procedure used to
        create or select a test model, identify test
        coverage items, and derive corresponding test cases''
        \citep[p.~11]{IEEE2022}; ``a variety \dots is typically
        required to suitably cover any system'' \citep[p.~33]{IEEE2022} and is
        ``often selected based on team skills and familiarity,
        on the format of the test basis'', and on expectations
        \citep[p.~23]{IEEE2022}         & equivalence partitioning,
        boundary value analysis, branch testing \citep[p.~11]{IEEE2022}                       \\
        Level\tablefootnote{\procLevel} & A stage of testing
        ``typically associated with the achievement of particular objectives
        and used to treat particular risks'' \citep[p.~12]{IEEE2022}; more
        generally, ``designat[es] \dots the coverage and detail''
        \citep[p.~249]{IEEE2017}        & unit/component testing,
        integration testing, system testing \citep[p.~12]{IEEE2022}                           \\
        Practice                        & A ``conceptual framework
        that can be applied to \dots [a] test process to facilitate testing''
        \citep[p.~14]{IEEE2022}; more generally, a ``specific type of activity
        that contributes to the execution of a process''
        \citep[p.~331]{IEEE2017}        & scripted testing,
        exploratory testing, automated testing \citep[p.~20]{IEEE2022}                        \\
        Type                            & ``Testing that is focused
        on specific quality characteristics''
        \citep[p.~15]{IEEE2022}         & security testing,
        usability testing, performance testing \citep[p.~15]{IEEE2022}                        \\
        \hline
    \end{tabularx}
\end{table}

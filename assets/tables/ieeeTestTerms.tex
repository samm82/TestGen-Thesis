% makecell with new lines so VS Code doesn't freak out
\newcommand{\techniqueCell}{\makecell{(Design)\footnote{``Design technique''
            is sometimes abbreviated to ``technique''
            \ifnotpaper
                (\citealp[p.~11]{IEEE2022}; \citealpISTQB{})\else
                \cite{ISTQB}, \cite[p.~11]{IEEE2022}\fi.
        }\\Technique}}
\newcommand{\levelCell}{\makecell{Level\footnote{\procLevel{\citep}.}\\
        (sometimes\\``Phase''\footnote{``Test phase'' can be a synonym for
            ``test level''
            \ifnotpaper
                (\citealp[p.~469]{IEEE2017}; \citeyear[p.~9]{IEEE2013})
            \else
                \cite[p.~469]{IEEE2017}, \cite[p.~9]{IEEE2013}
            \fi
            but \phaseDef{}}\\
        or ``Stage''\footnote{Used by \ifnotpaper
                (\citealp[pp.~5-6 to 5-7]{SWEBOK2024}; \citealpISTQB{};
                \citealp[pp.~438-440]{vanVliet2000};
                \citealp[p.~9]{Gerrard2000a})%
            \else
                \cite{ISTQB}, \cite[pp.~5-6 to 5-7]{SWEBOK2024},
                \cite[pp.~438-440]{vanVliet2000},
                \cite[p.~9]{Gerrard2000a}%
            \fi.
        })}}

\begin{paperTable}
    \centering
    \caption{IEEE Testing Terminology}
    \label{tab:ieeeTestTerms}
    % m{0.145\linewidth}
    \begin{minipage}{\linewidth}
        \begin{tabular}{|>{\centering}m{0.08\linewidth}m{0.6\linewidth}m{0.27\linewidth}|}
            \hline
            \thead{Term}                   & \thead{Definition}               & \thead{Examples} \\
            \hline
            Approach                       & A ``high-level test
            implementation choice, typically made as part of the test strategy
            design activity'' that includes ``test level, test type, test technique,
            test practice and the form of static testing to be used''
            \citep[p.~10]{IEEE2022}; described by a \emph{test strategy}
            \citeyearpar[p.~472]{IEEE2017} and is also used to ``pick the particular test case
            values'' \citeyearpar[p.~465]{IEEE2017}
                                           &
            black or white box, minimum and maximum
            boundary value testing \citep[p.~465]{IEEE2017}                                      \\
            \hline
            \techniqueCell{}               & A ``defined'' and ``systematic''
            \citep[p.~464]{IEEE2017} ``procedure used to create or select a test model,
            identify test coverage items, and derive corresponding test cases''
            \ifnotpaper
            (\citeyear[p.~11]{IEEE2022}; similar in \citeyear[p.~467]{IEEE2017})
            \else
            \cite[p.~11]{IEEE2022} (similar in \cite[p.~467]{IEEE2017})
            \fi ``that
            \dots\ generate evidence that test item requirements have been met or that
            defects are present in a test item'' \citeyearpar[p.~vii]{IEEE2021};
            ``a variety \dots\ is typically
            required to suitably cover any system'' \citeyearpar[p.~33]{IEEE2022} and is
            ``often selected based on team skills and familiarity,
            on the format of the test basis'', and on expectations
            \citeyearpar[p.~23]{IEEE2022}  &
            equivalence partitioning,
            boundary value analysis, branch testing \citep[p.~11]{IEEE2022}                      \\
            \hline
            \levelCell{}                   & A stage of testing
            ``typically associated with the achievement of particular objectives
            and used to treat particular risks'', each performed in sequence
            \ifnotpaper
            (\citealp[p.~12]{IEEE2022}; \citeyear[p.~6]{IEEE2021})
            \else
            \cite[p.~12]{IEEE2022}, \cite[p.~6]{IEEE2021}
            \fi
            with their
            ``own documentation and resources'' \citeyearpar[p.~469]{IEEE2017};
            more generally, ``designat[es] \dots\ the coverage and detail''
            \citeyearpar[p.~249]{IEEE2017} &
            unit/component testing, integration testing, system testing, acceptance testing
            \ifnotpaper
            (\citealp[p.~12]{IEEE2022}; \citeyear[p.~6]{IEEE2021};
            \citeyear[p.~467]{IEEE2017})
            \else
            \cite[p.~467]{IEEE2017}, \cite[p.~12]{IEEE2022}, \cite[p.~6]{IEEE2021}
            \fi                                                                                  \\
            \hline
            Practice                       & A ``conceptual framework
            that can be applied to \dots{} [a] test process to facilitate testing''
            \ifnotpaper
            (\citealp[p.~14]{IEEE2022}; \citeyear[p.~471]{IEEE2017}; OG IEEE 2013)\else
            \cite[p.~471]{IEEE2017}, \cite[p.~14]{IEEE2022}\fi;
            more generally, a ``specific type of activity
            that contributes to the execution of a process''
            \citeyearpar[p.~331]{IEEE2017} & scripted testing,
            exploratory testing, automated testing \citep[p.~20]{IEEE2022}                       \\
            \hline
            Type                           & ``Testing that is focused
            on specific quality characteristics''
            \ifnotpaper
            (\citealp[p.~15]{IEEE2022}; \citeyear[p.~7]{IEEE2021};
            \citeyear[p.~473]{IEEE2017}; OG IEEE 2013)
            \else
            \cite[p.~473]{IEEE2017}, \cite[p.~15]{IEEE2022}, \cite[p.~7]{IEEE2021}
            \fi                            &
            security testing, usability testing, performance testing
            \ifnotpaper
            (\citealp[p.~15]{IEEE2022}; \citeyear[p.~473]{IEEE2017})
            \else
            \cite[p.~473]{IEEE2017}, \cite[p.~15]{IEEE2022}
            \fi                                                                                  \\
            \hline
        \end{tabular}
    \end{minipage}
\end{paperTable}

% Makes footnotes in minipages use numbers instead of the default letters 
\renewcommand{\thempfootnote}{\arabic{mpfootnote}}

% makecell with new lines so VS Code doesn't freak out
\newcommand{\techniqueCell}{\makecell{(Design\footnote{While ``design technique''
            seems to be the most common term, this is sometimes abbreviated to
            just ``technique'' (\citealp[p.~11]{IEEE2022};
            \citetalias{ISTQB}).})\\Technique}}
\newcommand{\levelCell}{\makecell{Level\footnote{\procLevel{\citep}.}\\
        (sometimes\\``Phase''\footnote{``Test phase'' can be a synonym for
            ``test level'' (\citealp[p.~469]{IEEE2017};
            \citeyear[p.~9]{IEEE2013}) but \phaseDef{}})}}

\begin{table}[hbtp!]
    \centering
    \caption{IEEE Testing Terminology}
    \label{tab:ieeeTestTerms}
    % m{0.145\linewidth}
    \begin{minipage}{\linewidth}
        \begin{tabular}{|>{\centering}m{0.08\linewidth}m{0.6\linewidth}m{0.27\linewidth}|}
            \hline
            \rowcolor{McMasterMediumGrey}
            \thead{Term}                            & \thead{Definition}                      & \thead{Examples} \\
            \hline
            Approach                                & A ``high-level test
            implementation choice, typically made as part of the test strategy
            design activity'' that includes ``test level, test type, test technique,
            test practice and the form of static testing to be used''
            \citep[p.~10]{IEEE2022}; described by a \emph{test strategy}
            \citeyearpar[p.~472]{IEEE2017} and is also used to ``pick the particular test case
            values'' \citeyearpar[p.~465]{IEEE2017} & black or white box, minimum and maximum
            boundary value testing \citep[p.~465]{IEEE2017}                                                      \\
            \hline
            \techniqueCell{}                        & A ``defined'' and ``systematic''
            \citep[p.~464]{IEEE2017} ``procedure used to
            create or select a test model, identify test
            coverage items, and derive corresponding test cases''
            (\citeyear[p.~11]{IEEE2022}; similar in \citeyear[p.~467]{IEEE2017});
            ``a variety \dots is typically
            required to suitably cover any system'' \citeyearpar[p.~33]{IEEE2022} and is
            ``often selected based on team skills and familiarity,
            on the format of the test basis'', and on expectations
            \citeyearpar[p.~23]{IEEE2022}           & equivalence partitioning,
            boundary value analysis, branch testing \citep[p.~11]{IEEE2022}                                      \\
            \hline
            \levelCell{}                            & A stage of testing
            ``typically associated with the achievement of particular objectives
            and used to treat particular risks'' \citep[p.~12]{IEEE2022} with
            ``its own documentation and resources'' \citeyearpar[p.~469]{IEEE2017}; more
            generally, ``designat[es] \dots the coverage and detail''
            \citeyearpar[p.~249]{IEEE2017}          & unit/component testing,
            integration testing, system testing (\citealp[p.~12]{IEEE2022};
            \citeyear[p.~467]{IEEE2017})                                                                         \\
            \hline
            Practice                                & A ``conceptual framework
            that can be applied to \dots [a] test process to facilitate testing''
            (\citealp[p.~14]{IEEE2022}; \citeyear[p.~471]{IEEE2017}; OG IEEE 2013);
            more generally, a ``specific type of activity
            that contributes to the execution of a process''
            \citeyearpar[p.~331]{IEEE2017}          & scripted testing,
            exploratory testing, automated testing \citep[p.~20]{IEEE2022}                                       \\
            \hline
            Type                                    & ``Testing that is focused
            on specific quality characteristics''
            (\citealp[p.~15]{IEEE2022}; \citeyear[p.~473]{IEEE2017};
            OG IEEE 2013)                           & security testing, usability testing,
            performance testing (\citealp[p.~15]{IEEE2022};
            \citeyear[p.~473]{IEEE2017})                                                                         \\
            \hline
        \end{tabular}
    \end{minipage}
\end{table}

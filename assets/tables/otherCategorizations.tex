\def\selecExs{Deterministic Testing\\ Random Testing}
\def\covCritExs{Input Space Partitioning\\ Graph Coverage\\ Logic Coverage\\ Syntax-based Testing}
\def\execExs{Static Testing\\ Dynamic Testing}
\def\goalExs{Verification Testing\\ Validation Testing}
\def\propExs{Functional Testing\\ Non-functional Testing}
\def\humInvExs{Manual Testing\\ Automated Testing}
\def\humInvCats{Practice \citep[p.~22]{IEEE2022}\\ Technique \citep[implied by][p.~35; see \Cref{tab:multiCats}]{IEEE2022}}
\def\strExs{Scripted Testing\\ Exploratory Testing}
\def\covReqExs{Data Flow Testing\\ Control Flow Testing}
\def\factExs{Correctness Testing\\ Response-Time Testing\\ Access Control Testing\\
    Compliance Testing\\ Reliability Testing\\ Maintainability Testing\\ Portability Testing\\
    Performance Testing\TblrNote{d}}
\def\adqCritExs{Coverage-based Testing\\ Fault-based Testing\\ Error-based Testing}
\def\typeCatExs{Static Testing\\ Test Browsing\\ Functional Testing\\
    Non-functional Testing\\ Large Scale Integration (Testing)}
\def\priorExs{Smoke Testing\\ Usability Testing\\ Performance Testing\\ Functionality Testing}
\def\purpExs{Correctness Testing\\ Performance Testing\\ Reliability Testing\\ Security Testing}

\begin{longtblr}[
    note{a} = {See \Cref{par-chd-rels}.},
    note{b} = {We also consider this categorization meaningful (see \Cref{static-test}).},
    note{c} = {Exploratory testing may instead be a ``technique'' (see \Cref{tab:multiCats}).},
    note{d} = {Other test factors are given that do not unambiguously map to corresponding
            test approaches: file integrity, authorization, audit trail, continuity of processing,
            service levels, ease of use, coupling (e.g., with other applications in a given environment),
            and ease of operation (e.g., documentation, training) \citep[pp.~40--41]{Perry2006}.},
    note{e} = \gerrardDistinctIEEE*{type},
    note{f} = {With the exception of smoke testing, which is categorized as a technique
            (\citealp[p.~5\=/14]{SWEBOK2024}; \citealp[pp.~601, 603, 605--606]{SharmaEtAl2021});
            performance testing is also sometimes categorized as a technique \citep[p.~38]{IEEE2021}.},
    caption = {Alternate categorizations given by the literature.},
    label = {tab:otherCategorizations}
    ]{
    colspec = {|X[0.35,c,m]X[0.3,c,m]X[0.35,c,m]|},
    width = \linewidth, rowhead = 1
    }
    \hline
    \thead{Test Basis}                                            & \thead{Example Approaches} & \thead{Parent\TblrNote{a} IEEE Category}                                                                                               \\
    \hline
    Selection Process \citep[p.~5-16]{SWEBOK2024}                 & \selecExs{}                & Technique \citep[pp.~5-12, 5-16]{SWEBOK2024}                                                                                           \\
    \hline
    Coverage Criteria \citep[pp.~18--19]{AmmannAndOffutt2017}     & \covCritExs{}              & Technique (\citealp[p.~22]{IEEE2022}; \citeyear[Fig.~2]{IEEE2021}; \citealp[p.~5-11]{SWEBOK2024}; \citealp[pp.~47--48]{Firesmith2015}) \\
    \hline
    Execution of Code\TblrNote{b} (\citealp[p.~214]{KuļešovsEtAl2013}; \citealp[p.~12]{Gerrard2000a};
    \citealp[p.~53]{Patton2006})                                  & \execExs{}                 & Approach                                                                                                                               \\
    \hline
    Goal of Testing (\citealp[p.~214]{KuļešovsEtAl2013};
    \citealp[pp.~69--70]{Perry2006})                              & \goalExs{}                 & Approach                                                                                                                               \\
    \hline
    Property of Code \citep[p.~213]{KuļešovsEtAl2013}
    or Test Target \citep[pp.~4--5]{Kam2008}                      & \propExs{}                 & Ambiguous (see \Cref{func-test-discrep})                                                                                               \\
    \hline
    Human Involvement \citep[p.~214]{KuļešovsEtAl2013}            & \humInvExs{}               & \humInvCats{}                                                                                                                          \\
    \hline
    Structuredness \citep[p.~214]{KuļešovsEtAl2013}               & \strExs{}                  & Practice\TblrNote{c} \citep[pp.~20, 22]{IEEE2022}                                                                                      \\
    \hline
    Coverage Requirement \citep[pp.~4--5]{Kam2008}                & \covReqExs{}               & Technique \citep[p.~5\=/13]{SWEBOK2024}                                                                                                \\
    \hline
    Test Factor (also called Quality Factor or Quality Attribute)
    \citep[pp.~40--41]{Perry2006}                                 & \factExs{}                 & Type (\citealp[p.~22]{IEEE2022}; and/or implied by its quality and/or \citealp{Firesmith2015})                                         \\
    \hline
    Adequacy Criterion \citep[pp.~398--399]{vanVliet2000}         & \adqCritExs{}              & Technique \citep[pp.~398--399]{vanVliet2000}                                                                                           \\
    \hline
    Category of Test Type\TblrNote{e} \citep[p.~12]{Gerrard2000a} & \typeCatExs{}              & Ambiguous                                                                                                                              \\
    \hline
    Priority (in the context of testing e-business projects)
    \citep[p.~13]{Gerrard2000a}                                   & \priorExs{}                & Type\TblrNote{f} (\citealp[p.~22]{IEEE2022}; \citealp[Tab.~A.1]{IEEE2021}; and/or implied by \citealp[p.~53]{Firesmith2015})           \\
    \hline
    Purpose \citep{Pan1999}                                       & \purpExs{}                 & Type (\citealp[p.~22]{IEEE2022}; and/or implied by \citealp[p.~53]{Firesmith2015})                                                     \\
    \hline
    % print("\\\\ ".join([x[0].upper() + x[1:] + " Testing" for x in "Coverage-based fault-based error-based".split(" ")]))
\end{longtblr}

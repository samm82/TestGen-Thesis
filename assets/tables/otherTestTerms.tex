\begin{table}[hbtp!]
    \centering
    \caption{Other Testing Terminology}
    \label{tab:otherTestTerms}
    \begin{tabularx}{\linewidth}{|c|X|m{0.33\linewidth}|m{0.1\linewidth}|}
        \hline
        \rowcolor{McMasterMediumGrey}
        \thead{Term}                 & \thead{Definition} & \thead{Examples}   & \thead{IEEE Equiv.} \\
        \hline
        Guidance                     & none given
        \cite[p.~3]{BarbosaEtAl2006} & none given         & Metric? Technique?                       \\
        Level                        & ``distinguished
        based on the object of testing, which is called the \emph{target},
        or on the purpose, which is called the \emph{objective}''
        \cite[p.~86]{SWEBOK2014}     & Target: unit,
        integration, system testing (\citealp[p.~3]{SouzaEtAl2017}; \citealp[p.~86]{SWEBOK2014}) \newline
        Objective: acceptance, installation, reliability, regression, performance, security
        \cite[pp.~86-88]{SWEBOK2014} & Target: Level
        \newline Obj.: Various                                                                       \\
        Method                       & none given
        \cite[p.~3]{BarbosaEtAl2006} & none given         & Practice?                                \\
        Phase                        & none given
        \cite[p.~3]{BarbosaEtAl2006} & unit, integration,
        system, regression testing
        \cite[p.~3]{BarbosaEtAl2006} & Level                                                         \\
        Procedure                    & The basis for how
        testing is performed that guides the process \cite[p.~3]{BarbosaEtAl2006};
        categorized in[to] testing methods, testing guidances and testing techniques
        \cite[p.~3]{BarbosaEtAl2006} & none given
        generally; see examples of
        ``Technique''                & Approach                                                      \\
        Process                      & ``A sequence of
        testing steps'' \cite[p.~2]{BarbosaEtAl2006} that
        is ``based on a development technology and \dots
        paradigm, as well as on a testing procedure''
        \cite[p.~3]{BarbosaEtAl2006} & none given         & Practice                                 \\
        Technique                    & ``How tests are
        generated'' \cite[p.~ 88]{SWEBOK2014} ``in a systematic way and on a sound theoretical basis''
        \cite[p.~3]{BarbosaEtAl2006} & functional,
        structural, error-based, state-based testing \cite[p.~3]{BarbosaEtAl2006};
        black-box, white-box, defect/fault-based, model-based
        (\citealp[p.~3]{SouzaEtAl2017}; \citealp[pp.~88, 90-91]{SWEBOK2014})
                                     & Technique                                                     \\
        \hline
    \end{tabularx}
\end{table}

\todo{find original source for SouzaEtAl2017 technique examples: Mathur (2012)}
% Defined here so VS Code doesn't freak out
\def\ieeeEquiv{\makecell{IEEE\\Equivalent}}
\def\swebokLevel{\makecell{Level\\(objective-\\based)\footnote{
            See \discrepref{stage-level-syns}.}}}

\begin{paperTable}
    \centering
    \caption{Other Testing Terminology}
    \label{tab:otherTestTerms}
    \begin{minipage}{\linewidth}
        % Converted from tabularx with help from ChatGPT
        \begin{tabular}{|>{\centering}m{0.08\linewidth}|m{0.43\linewidth}|m{0.34\linewidth}|c|}
            \hline
            \thead{Term}                           & \thead{Definition}           & \thead{Examples} & \thead{\ieeeEquiv{}} \\
            \hline
            % Guidance                               & none given
            % \citep[p.~3]{BarbosaEtAl2006}          & none given         & Technique?                              \\
            \swebokLevel{}                         & Test levels based on the
            purpose of testing \citep[p.~5\=/6]{SWEBOK2024} that ``determine how the test suite is
            identified \dots\ regarding its consistency \dots\ and its composition''
            \citetext{p.~5\=/2}                    &
            conformance testing, installation testing,
            % reliability testing,
            regression testing, performance testing, security testing
            \citep[pp.~5\=/7 to 5\=/9]{SWEBOK2024} & Type?                                                                  \\
            % Method                                 & none given
            % \citep[p.~3]{BarbosaEtAl2006}          & none given         & Practice?                               \\
            Phase                                  & none given
            %(\citealp[p.~221]{Perry2006}; \citealp[p.~3]{BarbosaEtAl2006})  
                                                   & unit testing,
            integration testing, system testing, regression testing (\citealp[p.~221]{Perry2006};
            \citealp[p.~3]{BarbosaEtAl2006})       & Level                                                                  \\
            Procedure                              & The basis for how
            testing is performed that guides the process; ``categorized in[to] testing methods,
            testing guidances\footnote{Testing methods and guidances are omitted from this table
                since \citet{BarbosaEtAl2006} do not define or give examples of them.} and testing techniques''
            \citep[p.~3]{BarbosaEtAl2006}          & none given
            generally; see ``Technique''           & Approach                                                               \\
            Process                                & ``A sequence of
            testing steps'' \citep[p.~2]{BarbosaEtAl2006} ``based on a development technology and \dots\
            paradigm, as well as on a testing procedure''
            \citetext{p.~3}                        & none given                   & Practice                                \\
            Stage                                  & An
            alternative to the ``traditional \dots\ test stages'' %\footnote{See ``Level'' in \Cref{tab:ieeeTestTerms}.}
            based on ``clear technical groupings''
            \citep[p.~13]{Gerrard2000a}            & desktop development testing,
            infrastructure testing,
            % system testing, large scale integration, and
            post-deployment monitoring
            \citep[p.~13]{Gerrard2000a}            & Level                                                                  \\
            Technique                              & ``Systematic
            procedures and approaches for generating or selecting the most suitable test
            suites'' \citep[p.~5\=/10]{SWEBOK2024} & specification-based testing,
            % ``on a sound theoretical basis'' \citep[p.~3]{BarbosaEtAl2006}
            structure-based testing, fault-based testing\footnote{Synonyms for
                these examples are used by \citet[p.~3; OG Mathur, 2012]{SouzaEtAl2017}
                and \citet[p.~3]{BarbosaEtAl2006}.}
            % , experience-based testing, usage-based testing
            (\citealp[pp.~5\=/10, 5\=/13 to 5\=/15]{SWEBOK2024})
            % black-box, white-box, defect/fault-based, model-based testing
            % \citetext{\citealp[p.~3]{SouzaEtAl2017}; OG Mathur, 2012};
            % functional, structural, error-based, state-based testing \citep[p.~3]{BarbosaEtAl2006}
                                                   & Technique                                                              \\
            \hline
        \end{tabular}
    \end{minipage}
\end{paperTable}
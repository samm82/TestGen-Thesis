\begin{table}[hbtp!]
    \centering
    \caption{Other Testing Terminology}
    \label{tab:otherTestTerms}
    \begin{tabularx}{\linewidth}{|c|X|m{0.37\linewidth}|m{0.1\linewidth}|}
        \hline
        \rowcolor{McMasterMediumGrey}
        \thead{Term}                      & \thead{Definition} & \thead{Examples} & \thead{IEEE Equiv.} \\
        \hline
        Guidance                          & none given
        \cite[p.~3]{BarbosaEtAl2006}      & none given         & Technique?                             \\
        Level                             & ``distinguished
        based on the object of testing, the \emph{target},
        or on the purpose or \emph{objective}''
        \cite[p.~5-6]{SWEBOK2024}; these are ``orthogonal''
        and ``determine how the test suite is identified \dots regarding its consistency
        \dots and its composition''
        \cite[p.~5-2]{SWEBOK2024}         & Target: unit,
        integration, system (\citealp[pp.~5-6 to 5-7]{SWEBOK2024}; \citealp[p.~3]{SouzaEtAl2017}),
        acceptance testing \citep[p.~5-7]{SWEBOK2024} \newline
        Objective: conformance, installation, regression, performance, reliability, security
        \cite[pp.~5-7 to 5-9]{SWEBOK2024} & Target: Level
        \newline Obj.: Mainly type                                                                      \\
        Method                            & none given
        \cite[p.~3]{BarbosaEtAl2006}      & none given         & Practice?                              \\
        Phase                             & none given
        \cite[p.~3]{BarbosaEtAl2006}      & unit, integration,
        system, regression testing
        \cite[p.~3]{BarbosaEtAl2006}      & Level                                                       \\
        Procedure                         & The basis for how
        testing is performed that guides the process \cite[p.~3]{BarbosaEtAl2006};
        categorized in[to] testing methods, testing guidances and testing techniques
        \cite[p.~3]{BarbosaEtAl2006}      & none given
        generally; see examples of
        ``Technique''                     & Approach                                                    \\
        Process                           & ``A sequence of
        testing steps'' \cite[p.~2]{BarbosaEtAl2006} that
        is ``based on a development technology and \dots
        paradigm, as well as on a testing procedure''
        \cite[p.~3]{BarbosaEtAl2006}      & none given         & Practice                               \\
        Technique                         & ``systematic
        procedures and approaches for generating or selecting the most suitable test
        suites'' \cite[p.~5-10]{SWEBOK2024} ``on a sound theoretical basis''
        \cite[p.~3]{BarbosaEtAl2006}      & specification-,
        structure-, experience-, fault-, usage-based testing \citep[pp.~5-10, 5-13 to 5-15]{SWEBOK2024};
        black-box, white-box, defect/fault-based, model-based testing \citep[p.~3]{SouzaEtAl2017};
        functional, structural, error-based, state-based testing
        \cite[p.~3]{BarbosaEtAl2006}
                                          & Technique                                                   \\
        \hline
    \end{tabularx}
\end{table}

\todo{find original source for SouzaEtAl2017 technique examples: Mathur (2012)}
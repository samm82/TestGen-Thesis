\begin{table}[hbtp!]
    \centering
    \caption{Other Testing Terminology}
    \label{tab:otherTestTerms}
    \begin{tabularx}{\linewidth}{|c|X|m{0.22\linewidth}|m{0.12\linewidth}|}
        \hline
        \rowcolor{McMasterMediumGrey}
        \thead{Term}                 & \thead{Definition} & \thead{Examples}  & \thead{IEEE Equiv.} \\
        \hline
        Guidance                     & none given
        \cite[p.~3]{BarbosaEtAl2006} & none given         & Metric/Technique?                       \\
        Method                       & none given
        \cite[p.~3]{BarbosaEtAl2006} & none given         & Practice?                               \\
        Phase                        & none given
        \cite[p.~3]{BarbosaEtAl2006} & unit, integration,
        system, regression testing
        \cite[p.~3]{BarbosaEtAl2006} & Level                                                        \\
        Procedure                    & The basis for how
        testing is performed that guides the process \cite[p.~3]{BarbosaEtAl2006};
        categorized in[to] testing methods, testing guidances and testing techniques
        \cite[p.~3]{BarbosaEtAl2006} & none given
        generally; see examples of
        ``Technique''                & Approach                                                     \\
        Process                      & ``A sequence of
        testing steps'' \cite[p.~2]{BarbosaEtAl2006} that
        is ``based on a development technology and \dots
        paradigm, as well as on a testing procedure''
        \cite[p.~3]{BarbosaEtAl2006} & none given         & Practice                                \\
        Technique                    & The basis for how
        ``to perform the tests in a systematic way and on a sound theoretical basis''
        \cite[p.~3]{BarbosaEtAl2006} & functional,
        structural, error-based, state-based testing
        \cite[p.~3]{BarbosaEtAl2006} & Def: Technique
        Exs. Various                                                                                \\
        \hline
    \end{tabularx}
\end{table}
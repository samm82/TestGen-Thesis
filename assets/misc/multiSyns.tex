\item \textbf{Invalid Testing:}
\begin{itemize}
    \item Error Tolerance Testing \cite[p.~45]{Kam2008}
    \item Negative Testing \cite{ISTQB} (implied by \cite[p.~10]{IEEE2021})
\end{itemize}
\item \textbf{Soak Testing:}
\begin{itemize}
    \item Endurance Testing \cite[p.~39]{IEEE2021}
    \item Reliability Testing\footnote{Endurance testing is given as a kind of
              reliability testing by \cite[p.~55]{Firesmith2015}, although the
              terms are not synonyms.} \cite[Tab.~1,~p.~26]{Gerrard2000b},
          \cite[Tab.~2]{Gerrard2000a}
\end{itemize}
\item \textbf{User Scenario Testing:}
\begin{itemize}
    \item Scenario Testing \cite{ISTQB}
    \item Use Case Testing\footnote{``Scenario testing'' and ``use case testing''
              are given as synonyms by \cite{ISTQB} and \cite[pp.~47-49]{Kam2008}
              but listed separately by \cite[p.~22]{IEEE2022}, \ifnotpaper who
                  also give \else which also gives \fi ``use case testing'' as a
              ``common form of scenario testing'' \cite[p.~20]{IEEE2021}.
              This implies that ``use case testing'' may instead be a child of
              ``user scenario testing'' (see \Cref{tab:parSyns}).}
          \cite[p.~48]{Kam2008} (although ``an actor can be a user or another
          system'' \cite[p.~20]{IEEE2021})
\end{itemize}
\item \textbf{Link Testing:}
\begin{itemize}
    \item Branch Testing (implied by \cite[p.~24]{IEEE2021})
    \item Component Integration Testing \cite[p.~45]{Kam2008}
    \item Integration Testing (implied by \cite[p.~13]{Gerrard2000a})
\end{itemize}
\item \textbf{Invalid Testing:}
\begin{itemize}
	\item Error Tolerance Testing \citep[p.~45]{Kam2008}
	\item Negative Testing \cite{ISTQB}, implied by \cite[p.~10]{IEEE2021}
\end{itemize}
\item \textbf{Soak Testing:}
\begin{itemize}
	\item Endurance Testing \citep[p.~39]{IEEE2021}
	\item Reliability Testing \cite[Tab.~1,~p.~26]{Gerrard2000b}, \cite[Tab.~2]{Gerrard2000a}
\end{itemize}
Endurance testing is given as a kind of reliability testing by
\cite[p.~55]{Firesmith2015}, but the two are not described as synonyms.
\item \textbf{User Scenario Testing:}
\begin{itemize}
	\item Scenario Testing \cite{ISTQB}
	\item Use Case Testing \cite[p.~48]{Kam2008}
\end{itemize}
``Scenario testing'' and ``use case testing'' are given as synonyms
by \cite{ISTQB} and \cite[pp.~47-49]{Kam2008}, but listed
separately by \cite[p.~22]{IEEE2022} and \cite[p.~20]{IEEE2021}; the
latter gives use case testing as a ``common form of scenario testing''.
Since the actor in a use case ``can be a user or another system''
\citep[p.~20]{IEEE2021}, ``use case testing'' may instead be a child of
``user scenario testing''\seeParAlways[\Cref]{tab:parSyns}.
\item \textbf{\emph{Link Testing}:}
\begin{itemize}
	\item Branch Testing (implied by \cite[p.~24]{IEEE2021})
	\item Component Integration Testing \cite[p.~45]{Kam2008}
	\item Integration Testing (implied by \cite[p.~13]{Gerrard2000a})
\end{itemize}
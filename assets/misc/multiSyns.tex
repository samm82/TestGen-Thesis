\item \textbf{Invalid Testing:}
\begin{itemize}
    \item Error Tolerance Testing \citep[p.~45]{Kam2008}
    \item Negative Testing \ifnotpaper
              (\citealpISTQB{}; implied by \citealp[p.~10]{IEEE2021}) \else
              \citep{ISTQB} (implied by \citep[p.~10]{IEEE2021}) \fi
\end{itemize}
\item \textbf{Soak Testing:}
\begin{itemize}
    \item Endurance Testing \citep[p.~39]{IEEE2021}
    \item Reliability Testing\ifnotpaper\
              (\citealp[Tab.~2]{Gerrard2000a}; \citeyear[Tab.~1,~p.~26]{Gerrard2000b})
          \else\footnote{Endurance testing is given as a kind of reliability
                  testing by \citet[p.~55]{Firesmith2015}, although the terms
                  are not synonyms.} \citep[Tab.~1,~p.~26]{Gerrard2000b},
              \citep[Tab.~2]{Gerrard2000a}\fi
\end{itemize}
\item \textbf{User Scenario Testing:}
\begin{itemize}
    \item Scenario Testing \citepISTQB{}
    \item Use Case Testing\ifnotpaper\ \else\footnote{``Scenario testing'' and
                  ``use case testing'' are given as synonyms by \citepISTQB{}
                  and \citep[pp.~47-49]{Kam2008}
                  but listed separately by \citep[p.~22]{IEEE2022}, \ifnotpaper who
                      also give \else which also gives \fi ``use case testing'' as a
                  ``common form of scenario testing'' \citep[p.~20]{IEEE2021}.
                  This implies that ``use case testing'' may instead be a child of
                  ``user scenario testing'' (see \Cref{tab:parSyns}).}\fi
          \citep[p.~48]{Kam2008} (although ``an actor can be a user or another
          system'' \citep[p.~20]{IEEE2021})
\end{itemize}
\item \textbf{Link Testing:}
\begin{itemize}
    \item Branch Testing (implied by \citealp[p.~24]{IEEE2021})
    \item Component Integration Testing \citep[p.~45]{Kam2008}
    \item Integration Testing (implied by \citealp[p.~13]{Gerrard2000a})
\end{itemize}
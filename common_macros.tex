% Extra functionality for command parsing
\usepackage{xparse}

\newif\ifnotpaper

%------------------------------------------------------------------------------
% Reused in seminar slides
%------------------------------------------------------------------------------

\def\rqatext{What testing approaches do the literature describe?}
\def\rqbtext{Are these descriptions consistent?}
\def\rqctext{Can we systematically resolve any of these inconsistencies?}

\def\expBasedCatMain{\citeauthor{IEEE2022} categorize experience-based testing
    as both a test design technique and a test practice on the same
    page---twice \citeyearpar[Fig.~2, p.~34]{IEEE2022}!}

\NewDocumentCommand{\perfAsFamily}{s}{%
    \IfBooleanTF#1{\citealp}{\citep}[p.~1187]{Moghadam2019}\footnote{
        The original source describes ``performance testing \dots\ as a family
        of performance-related testing techniques'', but it makes more sense to
        consider ``performance-related testing'' as the ``family'' with
        ``performance testing'' being one of the variabilities
        (see \Cref{perf-test-rec}).}%
}

%------------------------------------------------------------------------------
% Spacing Options
%------------------------------------------------------------------------------

\newcommand{\thesisForceSingleSpacing}{\singlespacing}
\newcommand{\thesisForceDoubleSpacing}{\doublespacing}

%------------------------------------------------------------------------------
% Portable HREFs
%------------------------------------------------------------------------------

% Common variant
\newcommand{\porthref}[2]{\href{#2}{#1}\printOnlyFootnote{\url{#2}}}

% Custom URLs
\newcommand{\porthreft}[3]{\href{#3}{#1}\printOnlyFootnote{\href{#3}{#2}}}
% Inside of some environments, footnote marks aren't registered properly, so we
% need to manually write the "text" part
\newcommand{\porthreftm}[2]{\href{#2}{#1\printOnlyFootnoteMark}}

\newcommand{\formatPaper}[2]{%
    \ifnotpaper
        #1{#2}%
    \else
        \underline{#2}%
    \fi
}

\def\refHelper{\ifnotpaper\else Reference \fi}
\newcommand\multiAuthHelper[1]{\ifnotpaper #1\else #1s\fi}

\newcommand\discrepref[1]{%
    \ifnotpaper
        \labelcref{#1-discrep}%
    \else
        \Cref{#1-discrep}%
    \fi}

\newcommand\ifblind[2]{\IfEndWith*{\jobname}{_blind}{#1}{#2}}

%------------------------------------------------------------------------------
% Generic "chunks" that get reused
%------------------------------------------------------------------------------

\newenvironment{bigLandscape}{
    \newgeometry{hmargin=1cm, vmargin=2.5cm}
    \begin{landscape}
        }{
    \end{landscape}
    \restoregeometry
    \newpage
}

\DeclareDocumentCommand\seeSrcCode{ m m m g }{%
    (see the \href
    {https://github.com/samm82/TestGen-Thesis/blob/#1/scripts/#2.py\#L#3%
        \IfNoValueF {#4} {-L#4}}
    {relevant source code})%
}

\newcommand{\accelTolTest}{astronauts \citep[p.~11]{MorgunEtAl1999}, aviators
    \citep[pp.~27,~42]{HoweAndJohnson1995}, or catalysts
    \citep[p.~1463]{LiuEtAl2023}}

\newcommand{\procLevel}[1]{``Test level'' can also refer to the scope
of a test process; for example, ``across the whole organization'' or only
``to specific projects'' #1[p.~24]{IEEE2022}}
\newcommand{\phaseDef}{can also refer to the ``period of time in the software
    life cycle'' when testing occurs \citeyearpar[p.~470]{IEEE2017}, usually
    after the implementation phase
    \ifnotpaper
        (\citeyear[pp.~420,~509]{IEEE2017}; \citealp[p.~56]{Perry2006}).
    \else
        \cite[pp.~420,~509]{IEEE2017}, \cite[p.~56]{Perry2006}.
    \fi}

\def\recFigs{\Cref{fig:recoveryGraphs,fig:scalGraphs,fig:perf-graph}}

% Define common footnotes about IEEE testing terms for reuse
\newcommand{\distinctIEEE}[1]{distinct from the notion of ``test #1'' described
    in \Cref{tab:ieeeTestTerms}.}
\newcommand{\notDefDistinctIEEE}[1]{\footnote{Not formally defined, but
        \distinctIEEE{#1}}}
\newcommand{\gerrardDistinctIEEE}[1]{\footnote{``Each type of test addresses a
        different risk area'' \citep[p.~12]{Gerrard2000a}, which is
        \distinctIEEE{#1}}}

% Examples of discrepancies
\NewDocumentCommand\tourDiscrep{s}{%
    \IfBooleanTF#1{t}{T}he structure of tours can be defined as either quite
    general \citep[p.~34]{IEEE2022} or ``organized around a special focus''
    \citepISTQB{}\IfBooleanTF#1{}{.}}
\def\alphaDiscrep{Alpha testing is performed by ``users within the organization
    developing the software'' \citep[p.~17]{IEEE2017}, ``a small, selected
    group of potential users'' \citep[p.~5-8]{SWEBOK2024}, or ``roles outside
    the development organization'' conducted ``in the developer's test
    environment'' \citepISTQB{}.}
\def\loadDiscrep{Load testing is performed with loads ``between anticipated
    conditions of low, typical, and peak usage'' \citep[p.~5]{IEEE2022} or
    loads that are as large as possible \citep[p.~86]{Patton2006}.}

\def\suggSrcs{\href
    {https://github.com/samm82/TestGen-Thesis/issues/14\#issuecomment-1839922715}
    {suggested by Dr.~Carette}}

% Used in parSyns tables
\def\ftrnote{Fault tolerance testing may also be a sub-approach of
    reliability testing \ifnotpaper
        \citetext{\citealp[p.~375]{IEEE2017}; \citealp[p.~7-10]{SWEBOK2024}}%
    \else \cite[p.~375]{IEEE2017}, \cite[p.~7-10]{SWEBOK2024}%
    \fi, which is distinct from robustness testing \citep[p.~53]{Firesmith2015}.}
\def\specfn{See \Cref{spec-func-test}.}
\def\ucstn{See \discrepref{use-case-scenario}.}

%------------------------------------------------------------------------------
% For populating values from files
%------------------------------------------------------------------------------

\ExplSyntaxOn
\ior_new:N \g_hringriin_file_stream

\NewDocumentCommand{\ReadFile}{mm}
{
    \hringriin_read_file:nn { #1 } { #2 }
    \cs_new:Npn #1 ##1
    {
        \str_if_eq:nnTF { ##1 } { * }
        { \seq_count:c { g_hringriin_file_ \cs_to_str:N #1 _seq } }
        { \seq_item:cn { g_hringriin_file_ \cs_to_str:N #1 _seq } { ##1 } }
    }
}

\cs_new_protected:Nn \hringriin_read_file:nn
{
    \ior_open:Nn \g_hringriin_file_stream { #2 }
    \seq_gclear_new:c { g_hringriin_file_ \cs_to_str:N #1 _seq }
    \ior_map_inline:Nn \g_hringriin_file_stream
    {
        \seq_gput_right:cx
        { g_hringriin_file_ \cs_to_str:N #1 _seq }
        { \tl_trim_spaces:n { ##1 } }
    }
    \ior_close:N \g_hringriin_file_stream
}

\ExplSyntaxOff

% Define/read values for Undefined Terms methodology for reuse and calculation!
\ReadFile{\undefTermCounts}{assets/misc/undefTermCounts}

\newcount\TotalBefore
\newcount\UndefBefore
\newcount\TotalAfter
\newcount\UndefAfter

\TotalBefore=\undefTermCounts{1}
\UndefBefore=\undefTermCounts{2}
\TotalAfter=\undefTermCounts{3}
\UndefAfter=\undefTermCounts{4}

\def\approachCount{\undefTermCounts{3}}

\ReadFile{\qualityCounts}{build/qualityCount}
\def\qualityCount{\qualityCounts{1}}

\ReadFile{\parSynCounts}{build/parSynCounts}
\def\parSynCount{\parSynCounts{1}}
\def\selfCycleCount{\parSynCounts{2}}

\ReadFile{\stdSources}{build/stdSources}
\ReadFile{\metaSources}{build/metaSources}
\ReadFile{\textSources}{build/textSources}
\ReadFile{\paperSources}{build/paperSources}

\def\srcCount{\the\numexpr\stdSources{3} + \metaSources{3} + \textSources{3} + \paperSources{3}}

\ReadFile{\stdSmntcDiscBrkdwn}{build/stdSmntcDiscBrkdwn}
\ReadFile{\metaSmntcDiscBrkdwn}{build/metaSmntcDiscBrkdwn}
\ReadFile{\textSmntcDiscBrkdwn}{build/textSmntcDiscBrkdwn}
\ReadFile{\paperSmntcDiscBrkdwn}{build/paperSmntcDiscBrkdwn}
\ReadFile{\totalSmntcDiscBrkdwn}{build/totalSmntcDiscBrkdwn}

\ReadFile{\stdSntxDiscBrkdwn}{build/stdSntxDiscBrkdwn}
\ReadFile{\metaSntxDiscBrkdwn}{build/metaSntxDiscBrkdwn}
\ReadFile{\textSntxDiscBrkdwn}{build/textSntxDiscBrkdwn}
\ReadFile{\paperSntxDiscBrkdwn}{build/paperSntxDiscBrkdwn}
\ReadFile{\totalSntxDiscBrkdwn}{build/totalSntxDiscBrkdwn}

\def\stds{\nameref{stds}}
\def\metas{\nameref{metas}}
\def\texts{\nameref{texts}}
\def\papers{\nameref{papers}}

\def\srcCat{\hyperref[sources]{Source Tier}}
\def\reduns{\ifnotpaper\nameref{redun}\else Redundancies\footnote{Section omitted for brevity.}\fi}

\def\totalDiscreps{\totalSmntcDiscBrkdwn{13}}

\def\cats{\hyperref[cats]{Categories}}
\def\syns{\hyperref[syns]{Synonyms}}
\def\pars{\hyperref[pars]{Parents}}
\def\defs{\hyperref[defs]{Definitions}}
\def\terms{\hyperref[terms]{Terminology}}
\def\cites{\hyperref[cites]{Citations}}

%------------------------------------------------------------------------------
% TODOs
%------------------------------------------------------------------------------

% Generic Inlined TODOs
\newcommand{\intodo}[1]{\todo[inline]{#1}}

% Unimportant TODOs for "later" (i.e., finishing touches or changes immediately before submission)
\newcommand{\latertodo}[1]{\todo[backgroundcolor=Cyan]{\textit{Later}: #1}}

% "Important" TODOs
\newcommand{\imptodo}[1]{\todo[inline,backgroundcolor=Red]{\textbf{Important}: #1}}

% "Easy" TODOs
\newcommand{\easytodo}[1]{\todo[inline,backgroundcolor=SeaGreen]{\textit{Easy}: #1}}
\newcommand{\eztodo}[1]{\easytodo{#1}}

% "Tedious" TODOs
\newcommand{\tedioustodo}[1]{\todo[inline,backgroundcolor=PineGreen]{\textit{Needs time}: #1}}

% "Question" TODO Notes
\newcounter{todonoteQuestionsCtr}
\newcommand{\questiontodo}[1]{\stepcounter{todonoteQuestionsCtr}\todo[backgroundcolor=Lavender]{\textbf{Q \#\thetodonoteQuestionsCtr{}}: #1}}
\newcommand{\qtodo}[1]{\questiontodo{#1}}

% Specific categories of TODOs
\def\ptq{\todo{Present tense?}}

%------------------------------------------------------------------------------
% Citations
%------------------------------------------------------------------------------

\newcommand{\exhInfCite}{(\citealp[p.~5-5]{SWEBOK2024}; \citealp[p.~4]{IEEE2022};
    \citealp[p.~421]{vanVliet2000}; \citealp[pp.~439, 461]{PetersAndPedrycz2000})}

%------------------------------------------------------------------------------
% Link to Drasil issue
%------------------------------------------------------------------------------

\newcommand{\issueref}[1]{\href{https://github.com/JacquesCarette/Drasil/issues/#1}{\##1}}
\newcommand{\pullref}[1]{\href{https://github.com/JacquesCarette/Drasil/pull/#1}{\##1}}
\newcommand{\thesisissuerefhelper}[1]{\href{https://github.com/samm82/TestGen-Thesis/issues/#1}{\##1}}

\ExplSyntaxOn

% Based on output from ChatGPT
\NewDocumentCommand{\mapthesisissueref}{m}
{
    % Clear temporary sequences to store transformed items
    \seq_clear:N \l_tmpa_seq
    \seq_clear:N \l_tmpb_seq

    \seq_set_split:Nnn \l_tmpa_seq { , } { #1 } % Split the input by commas
    \seq_map_inline:Nn \l_tmpa_seq
    {
        \seq_put_right:Nn \l_tmpb_seq {\thesisissuerefhelper{##1}}
    }
    \seq_use:Nnnn \l_tmpb_seq { ~and~ } { ,~ } { ,~and~ }
}

\ExplSyntaxOff

\newcommand{\thesisissueref}[1]{\todo[backgroundcolor=lightgray]{See \mapthesisissueref{#1}}}

\section{Observations}
\label{observ}

\subsection{Categories of Testing Approaches}
\label{categories-observ}

\ifnotpaper
      \newgeometry{margin=1.5cm, top=2.5cm}
      \begin{landscape}
            \ieeeTestTermsTable{}
      \end{landscape}

      \begin{landscape}
            \otherTestTermsTable{}
      \end{landscape}
      \restoregeometry
\else% Moved earlier to display nicely in paper
\fi

Different sources categorize software testing approaches in different ways%
\ifnotpaper
      ; while it is useful to record and think about these
      categorizations\seeSectionPar{testing-categories}, following one (or more)
      during the research
      stage could lead to bias and a prescriptive categorization, instead of letting
      one emerge descriptively during the analysis stage. Since these categorizations
      are not mutually exclusive, it also means that more than one could be useful
      (both in general and to this specific project).\newline \else.\fi\
\ifnotpaper \citet{IEEE2022} \else ISO/IEC and IEEE \cite{IEEE2022} \fi provide
a classification for different kinds of tests (see \refIEEETestTerms{}).
\ifnotpaper A deeper rationale for a proposed classification will be given
      during the analysis stage. \else Since
      this seems to be widely used (``test level'' and ``test type'' in particular)
      and is useful when focusing on a particular subset of testing, this terminology
      is used for now. \fi

\ifnotpaper
      However, other sources \citep{BarbosaEtAl2006, SouzaEtAl2017}
      % \cite{BarbosaEtAl2006, SouzaEtAl2017}
      provide alternate categories
      (see \refOtherTestTerms{}) which may be beneficial to investigate to
      determine if this categorization is sufficient.
      % \fi 
      % \ifnotpaper
      A ``metric'' categorization was considered at one point, but was decided
      to be out of the scope of this project
      \seeSectionPar{scope}{, \thesisissueref{21}, and
            \thesisissueref{22}}.
      Related testing approaches may be grouped into a ``class'' or ``family'' to
      group those with ``commonalities and well-identified variabilities that can be
      instantiated'', where ``the commonalities are large and the variabilities
      smaller''\seeThesisIssuePar{64}. Examples of these are the classes of
      combinatorial \citep[p.~15]{IEEE2021} and data flow testing \citetext{p.~3} and the
      family of performance-related testing \cite[p.~1187]{Moghadam2019}\footnote{The
            original source describes ``performance testing \dots\ as a family of
            performance-related testing techniques'', but it makes more sense to
            consider ``performance-related testing'' as the ``family'' with
            ``performance testing'' being one of the
            variabilities\seeSectionPar{perf-test-rec}.}, and may also be
      implied for security testing, a test type that consists of ``a number of
      techniques\footnote{This may or may not be \distinctIEEE{technique}}''
      \cite[p.~40]{IEEE2021}.
\fi

It also seems that these categories are orthogonal. For example, ``a test type
can be performed at a single test level or across several test levels''
\ifnotpaper
      (\citealp[p.~15]{IEEE2022}; \citeyear[p.~7]{IEEE2021})%
\else
      \cite[p.~15]{IEEE2022}, \cite[p.~7]{IEEE2021}%
\fi. Due to this, a specific
test approach can be derived by combining test approaches from different
categories\ifnotpaper;\seeSection{orthogonal-tests} for some examples
of this\fi.

% Moved here to display nicely in paper
\ifnotpaper\else\discrepsTable{}\fi
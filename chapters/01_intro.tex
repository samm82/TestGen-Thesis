\section{Introduction}

% TODO: tighten up, add sources

Testing software is complicated, expensive, and often overlooked. The
productivity of testing and testing research would benefit from a standard
language for communication. For example, \citet[p.~7]{KanerEtAl2011}
\multAuthHelper{give} the example of complete testing, which could require the
tester to discover ``every bug in the product'', exhaust the time allocated to
the testing phase, or simply implement every test previously agreed upon.
% They go on to say that
Having a clear  definition for ``complete testing'' reduces the chance for
miscommunication and, ultimately, the tester getting ``blamed for not
doing \dots{} [their] job'' \citep[p.~7]{KanerEtAl2011}. These benefits can be
extrapolated to software testing terminology as a whole. Unfortunately, a
search for a systematic, rigorous, and ``complete'' taxonomy for software
testing revealed that the existing ones are inadequate:

\begin{itemize}
    \item \ifnotpaper\else Tebes et al. \fi\citet{TebesEtAl2020a} focus on
          \emph{parts} of the testing process (e.g., test goal, testable entity),
    \item \ifnotpaper\else Souza et al. \fi\citet{SouzaEtAl2017} prioritize
          organizing testing approaches over defining them, and
    \item \ifnotpaper\else Unterkalmsteiner et al. \fi\citet{UnterkalmsteinerEtAl2014}
          focus on the ``information linkage or transfer'' \citetext{p.~A:6}
          between requirements engineering and software testing.
\end{itemize}

\ifnotpaper
    Some existing collections of software testing terminology were found, but
    in addition to being incomplete, they also contained many oversights. For
    example, \citep{IEEE2017} provides the following term-definition pairs:
    \begin{enumerate}
        \item \textbf{Event Sequence Analysis:} ``per'' \citetext{p.~170}
        \item \textbf{Operable:} ``state of'' \citetext{p.~301}
              % This may be a bad example, since the cf. provides some more context
              % \item \textbf{Software Element:} a ``system element that is software''
              %       \citetext{p.~421}
    \end{enumerate}
    These definitions are nonsensical and do not define the terms they claim
    to! To be sure, they \emph{cannot} correspond to the terms given since the
    parts of speech are mismatched: the first defines a noun phrase as a
    preposition, and the second an adjective as a noun phrase fragment.
    \citet{IEEE2017} also \multAuthHelper{define} ``device'' as a ``mechanism
    or piece of equipment designed to serve a purpose or perform a function''
    \citetext{p.~136}, but do\ifnotpaper\else{es}\fi\ not define ``equipment''
    and only \multAuthHelper{define} ``mechanism'' in the software sense as
    ``the means used by a function to transform input into output''
    \citetext{p.~270\todo{OG IEEE 1998}}. This is an example of an incomplete
    definition; while the definition of ``device'' seems logical at first
    glance, it actually leaves much undefined.
\fi

Thus we set about closing this gap in the literature. We first define the scope
of what kinds of
``software testing'' are of interest (\Cref{scope}) and examine the existing
literature (\Cref{methodology})\ifnotpaper, partially through the use of tools
created for analysis (\Cref{tools})\fi. Despite the amount of well understood
and organized knowledge, there are still many discrepancies and
ambiguities in the literature, either within the same source or between various
sources (\Cref{discrep}). This reinforces the need for a proper taxonomy! We
also provide some potential solutions covering some of these discrepancies
(\Cref{recs}).
\ifnotpaper
    The following three research questions guide this process:
    \begin{enumerate}
        \item What testing approaches are described in the literature?
        \item What discrepancies exist between descriptions of these testing
              approaches?
        \item Can any of these discrepancies be resolved/reduced systematically?
    \end{enumerate}
    This research and analysis (including the development of tools for analysis
    outlined in \Cref{tools}) was performed by Samuel Crawford under the
    guidance, supervision, and recommendations of Drs.~Spencer Smith and
    Jacques Carette. The resulting contributions are three glossaries of test
    approaches, software qualities (which may imply \nameref{qual-test}), and
    supplementary terms, as well as tools for automated analysis and
    visualization of these data. These are all available online at
    \url{https://github.com/samm82/TestGen-Thesis} for independent analysis and,
    ideally, extension as more test approaches are discovered and documented.

    In our own project Drasil~\citep{Drasil}, which is aimed at ``generating
    all of the software artifacts for (well understood) research software'', we
    wanted to automate the generation of tests. We did not want to do this
    in an ad hoc manner, so we sought to fully understand the target domain:
    testing. The goal was to uncover the various approaches towards testing, as
    well as which prerequisites (e.g., input files, oracles) are needed for
    each. Then we could evaluate which prerequisites are already ``known'' by
    Drasil, as well as which were possible to ``teach'', and generate test
    cases from them. The lack of a consistent, comprehensive source of this
    information caused the focus of this project to shift drastically.
\fi
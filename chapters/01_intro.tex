\section{Introduction}
\label{intro}

As with all fields of science and technology, software development should
be approached systematically and rigorously. \refHelper \ifnotpaper
    \citeauthor{PetersAndPedrycz2000} \else \cite{PetersAndPedrycz2000} \fi
\multiAuthHelper{claim} that ``to be successful, development of software
systems requires an engineering approach'' that is ``characterized by a
practical, orderly, and measured development of software'' \ifnotpaper
    \citeyearpar[p.~3]{PetersAndPedrycz2000}\else \citetext{p.~3}\fi. When a
NATO study group decided to hold a conference to discuss ``the problems of
software'' in 1968, they chose the
phrase ``software engineering'' to ``imply[] the need for
software manufacture to be based on the types of theoretical foundations and
practical disciplines, [sic] that are traditional in the established branches
of engineering'' \citep[p.~13]{NATO1969}. ``The term was not in general use at
that time'', but conferences such as this ``played a major role in gaining
general acceptance \dots{} for the term'' \citep{McClure2001}. While one of the
goals of the conference was to ``discuss possible techniques, methods and
developments which might lead to the[] solution'' to these problems
\citep[p.~14]{NATO1969}, the format of the conference itself was difficult to
document. Two competing classifications of the report emerged: ``one following
from the normal sequence of steps in the development of a software product''
and ``the other related to aspects like communication, documentation,
management, [etc.]'' \citetext{p.~10}. Perhaps more surprisingly, ``to retain
the spirit and liveliness of the conference, \dots{} points of major
disagreement have been left wide open, and \dots{} no attempt \dots{} [was]
made to arrive at a consensus or majority view'' \citetext{p.~11}!

% Testing software is complicated, expensive, and often overlooked. 

% The productivity of testing and testing research would benefit from a standard
% language for communication. 

% These benefits permeate software testing terminology. 

Perhaps unsurprisingly, there are still concepts in software engineering without
consensus, and many of them can be found in the subdomain of software testing.
\refHelper \ifnotpaper \citeauthor{KanerEtAl2011} \else \cite{KanerEtAl2011}
\fi \multiAuthHelper{give} the example of complete testing, which may require
the tester to discover ``every bug in the product'', exhaust the time allocated
to the testing phase, or simply implement every test previously agreed upon
\ifnotpaper \citeyearpar[p.~7]{KanerEtAl2011}\else \citetext{p.~7}\fi.
% They go on to say that
Having a clear definition of ``complete testing'' would reduce the chance for
miscommunication and, ultimately, the tester getting ``blamed for not
doing \dots{} [their] job'' \citetext{p.~7}. Because software testing uses ``a
subtantial percentage of a software development budget (in the range of 30 to
50\%)'', which is increasingly true ``with the growing complexity of software
systems'' \citep[p.~438]{PetersAndPedrycz2000}, this is crucial to the
efficiency of software development. Even more foundationally, if software
engineering holds code to high standards of clarity, consistency, and
robustness, the same should apply to its supporting literature! \ifnotpaper
    \par We noticed this lack of a standard language for software testing while
    working on our own software framework, Drasil \citep{Drasil}, with the
    goal of ``generating all of the software artifacts for (well understood)
    research software''. Currently, these include \acfp{srs}, READMEs,
    Makefiles, and code in up to six languages, depending on the specific case
    study \citep{HuntEtAl2021}. To improve the quality, functionality, and
    maintainability of Drasil, we want to add test cases to this list
    \citep{PotentialProjects,HuntEtAl2021}. This process would be a part of
    our ``continuous integration system, [sic] so that the generated code for
    the case studies is automatically tested with each build'' and would be a
    big step forward, since we currently only ``test that the generated code
    compiles'' \citep{PotentialProjects}. However, before we can include test
    cases as a generated artifact, the underlying domain---software testing---%
    needs to be ``well understood'', which requires a ``stable knowledge
    base'' \citep{HuntEtAl2021}. \fi

Unfortunately, a search for a systematic,
rigorous, and complete taxonomy for software testing revealed that the existing
ones are inadequate\todo{ProWritingAid suggests that including ``and
    incomplete'' is redundant} and mostly focus on the high-level testing
process rather than the testing approaches themselves: % and incomplete

\begin{itemize}
    \item \citeauthor{TebesEtAl2020a} \citeyearpar{TebesEtAl2020a} focus on
          \emph{parts} of the testing process (e.g., test goal, test plan,
          testing role, testable entity) and how they relate to one another,
    \item \citeauthor{SouzaEtAl2017} \citeyearpar{SouzaEtAl2017} prioritize
          organizing test approaches over defining them,
    \item \citeauthor{Firesmith2015} \citeyearpar{Firesmith2015} similarly
          defines relations between test approaches but not the approaches
          themselves, and
    \item \citeauthor{UnterkalmsteinerEtAl2014}
          \citeyearpar{UnterkalmsteinerEtAl2014}
          focus on the ``information linkage or transfer'' \citetext{p.~A:6}
          between requirements engineering and software testing and ``do[]
          not aim at providing a systematic and exhaustive state-of-the-art
          survey of [either domain]'' \citetext{p.~A:2}.
\end{itemize}
In addition to these taxonomies, many standards documents (see \Cref{stds})
and terminology collections (see \Cref{metas}) define testing terminology,
albeit with their own issues.

% Some existing collections of software testing terminology were found, but
% in addition to being incomplete, they also contained many oversights.

% For example, \citet{IEEE2017} \multiAuthHelper{provide} the following
% term-definition pairs:
% \begin{itemize}
%     \item \textbf{Event Sequence Analysis:} ``per'' \citetext{p.~170}
%     \item \textbf{Operable:} ``state of'' \citetext{p.~301}
%           % This may be a bad example, since the cf. provides some more context
%           % \item \textbf{Software Element:} a ``system element that is software''
%           %       \citetext{p.~421}
% \end{itemize}
% These definitions are nonsensical and do not define the terms they claim
% to! To be sure, they \emph{cannot} correspond to the terms given since the
% parts of speech are mismatched: the first defines a noun phrase as a
% preposition, and the second an adjective as a noun phrase fragment.
% \citet{IEEE2017} also \multiAuthHelper{define} ``device'' as a ``mechanism
% or piece of equipment designed to serve a purpose or perform a function''
% \citetext{p.~136}, but do\ifnotpaper\else{es}\fi\ not define ``equipment''
% and only \multiAuthHelper{define} ``mechanism'' in the software sense as
% ``the means used by a function to transform input into output''
% \citetext{p.~270\todo{OG IEEE 1998}}. This is an example of an incomplete
% definition; while the definition of ``device'' seems logical at first
% glance, it actually leaves much undefined.

For example, a common point of discussion in the field of software is the
distinction between terms for when software does not work correctly. We find
the following four to be most prevalent:
\begin{itemize}
    \item \textbf{Error:} ``a human action that produces an incorrect result''
          \ifnotpaper (\citealp[p.~128]{IEEE2010};
              \citealp[p.~399]{vanVliet2000})\else
              \cite[p.~128]{IEEE2010}, \cite[p.~399]{vanVliet2000}\fi.
          %   ``the difference between a computed, observed, or measured value or
          %   condition and the true, specified, or theoretically correct value or
          %   condition'' (IEEE, 2010, p. 128;
          %   similar in Washizaki, 2024, pp. 17-18 to 17-19, 18-7 to 18-8)
    \item \textbf{Fault:} ``an incorrect step, process, or data definition in a
          computer program'' \citep[p.~140]{IEEE2010} inserted when a developer
          makes an error \ifnotpaper (pp.~128, 140; \citealp[p.~12\=/3]{SWEBOK2024};
              \citealp[pp.~399--400]{vanVliet2000})\else
              \cite[pp.~128, 140]{IEEE2010}, \cite[pp.~399--400]{vanVliet2000},
              \cite[p.~12\=/3]{SWEBOK2024}\fi.
    \item \textbf{Failure:} the inability of a system ``to perform a required
          function or \dots{} within previously specified limits'' \ifnotpaper
              (\citealp[p.~7]{IEEE2019}; \citeyear[p.~139]{IEEE2010}\todo{OG
                  ISO/IEC, 2005}; similar in \citealp[p.~400]{vanVliet2000})
          \else \cite[p.~139]{IEEE2010}, \cite[p.~7]{IEEE2019} \fi that is
          ``externally visible'' \ifnotpaper(\fi\citealp[p.~7]{IEEE2019}%
          \ifnotpaper; similar in \citealp[p.~400]{vanVliet2000})\fi\
          and caused by a fault \ifnotpaper (\citealp[p.~12\=/3]{SWEBOK2024};
              \citealp[p.~400]{vanVliet2000})\else \cite[p.~400]{vanVliet2000},
              \cite[p.~12\=/3]{SWEBOK2024}\fi.
    \item \textbf{Defect:} ``an imperfection or deficiency in a project
          component where that component does not meet its requirements or
          specifications and needs to be either repaired or replaced''
          \citep[p.~96]{IEEE2010}.
\end{itemize}
This distinction is sometimes important, but not always
\citep[p.~4\=/3]{SWEBOK2014}\todo{OG IEEE, 1996}. The term ``fault'' is
``overloaded with too many meanings, as engineers and others use the word to
refer to all different types of anomalies'' \citep[p.~12\=/3]{SWEBOK2024}, and
``defect'' may be used as a ``generic term that can refer to either a fault
(cause) or a failure (effect)'' \ifnotpaper (\citealp[p.~124]{IEEE2017};
    \citeyear[p.~96]{IEEE2010}\todo{OG 2005})\else \cite[p.~96]{IEEE2010},
    \cite[p.~124]{IEEE2017}\fi. Software testers may even choose to ignore
these nuances completely! \bugPattonDiscrep{}

\ifnotpaper
    These decisions are not inherently wrong, since they may be useful in
    certain contexts or for certain teams (cf. \Cref{syn-rels}). Problems start
    to arise when teams need to make these decisions in the first place.
    \citet[p.~14]{Patton2006} notes that ``a well-known computer company spent
    weeks in discussion with its engineers before deciding to rename \acfp{par}
    to \acfp{pir}'', a process that required ``countless dollars'' and updating
    ``all the paperwork, software, forms, and so on''. While consistency and
    clear terminology may have been valuable to the company, ``it's unknown if
        [this decision] made any difference to the programmer's or tester's
    productivity'' \citetext{p.~14}. A potential way to avoid similar resource
    sinks would be to prescribe a standard terminology. Perhaps multiple sets
    of terms could be designed with varying levels of specificity so a company
    would only have to determine which one best suits their needs.
\fi

But why are minor differences between terms like these even important? Our
previous list of terms ``error'', ``fault'', ``failure'', and ``defect''
are used to describe many test approaches, including:%
\qtodo{Should I include a list of sources that describe each, or
    would that make this too cluttered?}
\begin{enumerate}
    \item Defect-based testing
    \item Error forcing
    \item Error guessing
    \item Error tolerance testing
    \item Error-based testing
    \item Error-oriented testing
    \item Failure tolerance testing
    \item Fault injection testing
    \item Fault seeding
    \item Fault sensitivity testing
    \item Fault tolerance testing
    \item Fault tree analysis
    \item Fault-based testing
\end{enumerate}
When considering which approaches to use or when actually using them, the
meanings of these four terms inform what their related approaches accomplish
and how to they are performed. For example, the tester needs to know what a
``fault'' is to perform fault injection testing; otherwise, what would they
inject? Information such as this is critical to the testing team, and should
therefore be standardized.

Discrepancies such as these that can lead to miscommunications---such as that
previously mentioned by \citet[p.~7]{KanerEtAl2011}---are prominent in the
literature. \expBasedCatMain{} \tourDiscrep{} \loadDiscrep{} \alphaDiscrep{}
It is clear that there is a notable gap in the literature, one which we
attempt to describe and fill. While the creation of a complete taxonomy is
unreasonable, especially considering the pace at which the field of software
changes, we can make progress towards this goal that others can extend and
update as new test approaches emerge.

\ifnotpaper
    The following three research questions guide this process:
    \begin{enumerate}
        \item \rqatext{}
        \item \rqbtext{}
        \item \rqctext{}
    \end{enumerate}
    \input{build/methodOverviewIntro}
    An excerpt of this recorded information (excluding other notes for brevity),
    is given in \Cref{tab:approachGlossaryExcerpt}.
\fi

This document describes this process, as well as its results, in more detail.
We first define the scope of
what kinds of software testing are of interest (\Cref{scope}) and examine the
existing literature (\Cref{methodology})\ifnotpaper, partially through the use
of tools created for analysis (\Cref{tools})\fi. Despite the amount of well
understood and organized knowledge, there are still many discrepancies in the
literature, either within the same source or between various
sources (\Cref{discreps}). This reinforces the need for a proper taxonomy! We
provide some potential solutions covering some of these discrepancies
(\Cref{recs}).

\ifnotpaper
    \begin{landscape}
        % makecell with new lines so VS Code doesn't freak out
\def\app{\makecell{Approach\\Category}}

\begin{table}[hbtp!]
    \centering
    \caption{Selected entries from glossary of test approaches with ``Notes'' column excluded for brevity.}
    \label{tab:approachGlossaryExcerpt}
    \begin{tabularx}{\linewidth}{|m{1.7cm}|m{4.2cm}|X|m{7.7cm}|m{2.8cm}|}
        \hline
        \thead{Name}               & \thead{\app}                                                                                   & \thead{Definition}                                                                                                                                   & \thead{Parent(s)}                                                                                                                                                                                                                 & \thead{Synonym(s)}                            \\
        \hline
        A/B Testing                & Practice \citep[p.~22]{IEEE2022}, Type (implied by \citealp[p.~58]{Firesmith2015})             & Testing ``that allows testers to determine which of two systems or components performs better'' \citep[p.~1]{IEEE2022}                               & Statistical Testing \citep[pp.~1,~35]{IEEE2022}, Usability Testing \citep[p.~58]{Firesmith2015}                                                                                                                                   & Split-Run Testing \citep[pp.~1,~35]{IEEE2022} \\[1cm]
        % All Combinations Testing   & Technique (\citealp[p.~22]{IEEE2022}; \citeyear[pp.~2,~16]{IEEE2021}; \citealp[p.~5-11]{SWEBOK2024})                           & Testing that covers ``all unique combinations of P-V pairs'' \citep[p.~16]{IEEE2021}                                                                 & Combinatorial Testing \citetext{\citealp[p.~22]{IEEE2022}; \citeyear[pp.~2,~16,~Fig.~2]{IEEE2021}; \citealp[p.~5-11]{SWEBOK2024}}                                                                                                                                                                                                                                                                                      & ---                                           \\[1cm]
        Big-Bang Testing           & Level (inferred from integration testing)                                                      & ``Testing in which \dots{} [components of a system] are combined all at once into an overall system, rather than in stages'' \citep[p.~45]{IEEE2017} & Integration Testing \citetext{\citealp[p.~45]{IEEE2017}; \citealp[p.~5-7]{SWEBOK2024}; \citealp[p.~603]{SharmaEtAl2021}; \citealp[p.~42]{Kam2008}}                                                                                & ---                                           \\[1cm]
        Classification Tree Method & Technique (\citealp[p.~22]{IEEE2022}; \citeyear[pp.~2,~12,~Fig.~2]{IEEE2021}; \citealpISTQB{}) & Testing ``based on exercising classes in a classification tree'' \citep[p.~22]{IEEE2021}                                                             & Specification-based Testing (\citealp[p.~22]{IEEE2022}; \citeyear[pp.~2,~12,~Fig.~2]{IEEE2021}; \citealpISTQB{}; \citealp[p.~47]{Firesmith2015}), Model-based Testing (\citealp[p.~13]{IEEE2022}; \citeyear[pp.~6,~12]{IEEE2021}) & Classification Tree Technique \citepISTQB{}   \\[1.5cm] % Don't know why extra is needed here but it makes it look better
        % Data Flow Testing          & Technique (\citealp[p.~22]{IEEE2022}; \citeyear[pp.~3,~27]{IEEE2021}; \citealp[p.~5-13]{SWEBOK2024}; \citealp[p.~43]{Kam2008}) & A ``class of \dots{} techniques based on exercising definition-use pairs'' \citep[p.~3; similar on p.~27]{IEEE2021}                                  & Structure-based Testing (\citealp[p.~22]{IEEE2022}; \citeyear[pp.~3,~27,~Fig.~2]{IEEE2021}; \citealp[p.~43]{Kam2008}), Control Flow Testing (\citeyear[p.~27]{IEEE2021}; implied by \citealp[p.~5-13]{SWEBOK2024}; \citealp[p.~101]{IEEE2017}), Model-based Testing (\citeyear[p.~27]{IEEE2021}; implied by \citealp[p.~179]{DoğanEtAl2014}), Web Application Testing (p. 179; can be in \citealp[pp.~16-17]{Kam2008}) & ---                                           \\
        \hline
    \end{tabularx}
\end{table}
    \end{landscape}
\fi
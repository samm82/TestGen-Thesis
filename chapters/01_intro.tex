\section{Introduction}

% TODO: tighten up, add sources

Testing software is complicated, expensive, and often overlooked.
Improving the productivity of testing and testing research requires a standard
language for communication. For example, ``complete testing'' could mean that
the tester has ``completed the discovery of every bug in the product\dots[,] the
agreed-upon tests\dots[, or] the time period assigned to testing'', which can
lead to miscommunication and, ultimately, the tester getting ``blamed for not
doing \dots{} [their] job'' \citep[p.~7]{KanerEtAl2011}. Unfortunately, a search
for a systematic, rigorous, and ``complete'' taxonomy for software testing
revealed that the existing ones are inadequate:

\begin{itemize}
    \item \ifnotpaper\else Tebes et al. \fi\citet{TebesEtAl2020a} focus on
          \emph{parts} of the testing process (e.g., test goal, testable entity),
    \item \ifnotpaper\else Souza et al. \fi\citet{SouzaEtAl2017} prioritize
          organizing testing approaches over defining them, and
    \item \ifnotpaper\else Unterkalmsteiner et al. \fi\citet{UnterkalmsteinerEtAl2014}
          provide a foundation for classification but not how it applies to software
          testing terminology.
\end{itemize}

Thus we set about closing this gap. We first define the scope of what kinds of
``software testing'' are of interest (\Cref{scope}) and examine the existing
literature (\Cref{methodology})\ifnotpaper, partially through the use of tools
created for analysis (\Cref{methods})\fi. Despite the amount of well understood
and organized knowledge (\Cref{observ}), there are still many discrepancies and
ambiguities in the literature, either within the same source or between various
sources (\Cref{discrep}). This reinforces the need for a proper taxonomy! We
also provide some potential solutions covering some of these discrepancies
(\Cref{recs}).

\ifnotpaper
    In our own project Drasil~\citep{Drasil}, which is aimed at ``generating
    all of the software artifacts for (well understood) research software'', we
    wanted to automate the generation of tests. We did not want to do this
    in an ad hoc manner, so we sought to fully understand the target domain:
    testing. The goal was to uncover the various approaches towards testing, as
    well as which prerequisites (e.g., input files, oracles) are needed for
    each. Then we could evaluate which prerequisites are already ``known'' by
    Drasil, as well as which were possible to ``teach'', and generate test
    cases from them. The lack of a consistent, comprehensive source of this
    information caused the focus of this project to shift drastically.
\fi
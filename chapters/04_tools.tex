\ifnotpaper
    \section{Tools}
\else
    \subsection{Tools}
\fi
\label{tools}

\ifnotpaper
    To better understand our findings, we build tools to more intuitively
    visualize relations between test approaches (\Cref{graph-gen}) and
    automatically track their flaws (\Cref{flaw-analysis}). Doing
    this manually would be error-prone due to the amount of data involved (for
    example, we identify \approachCount{} test approaches) and the number of
    situations where the underlying data would change, including more detailed
    analysis, error corrections, and the addition of data. These all require
    tedious updates to the corresponding graphs that may be overlooked or done
    incorrectly. Besides being more systematic, automating these processes also
    allows us to observe the impacts of smaller changes, such as unexpected
    flaws that arise from a new relation between two approaches. It
    also helps verify the tools themselves; for example, tracking a flaw
    manually should affect relevant flaw counts, and we can
    double-check this. We also define macros to help achieve our goals of
    maintainability, traceability, and reproducibility (\Cref{macros}).

    \subsection{Approach Graph Generation}
    \label{graph-gen}
\fi

To better visualize how test approaches relate to each other, we
develop a tool to automatically generate graphs of these relations.
\ifnotpaper Since synonym (see \Cref{syn-rels}) and parent-child relations
    (see \Cref{par-chd-rels}) between approaches are tracked in
    \ourApproachGlossary{} in a consistent format, they can be parsed
    systematically. For example, if the entries in \Cref{tab:exampleGlossary}
    appear in the glossary, then their parent relations are displayed as
    \Cref{fig:exampleGraph} in the generated graph. Relevant citation
    information is also captured in our glossary following the author-year
    citation format, including ``reusing'' information from previous
    citations. For example, the first row of \Cref{tab:exampleGlossary}
    contains the citation ``(Author, 0000; 0001)'', which means that this
    information was present in two documents by ``Author'': one written in
    the year 0000, and one in 0001. The following citation, ``(0000)'',
    contains no author, which means it was written by the same one as the
    previous citation. These citations are processed according to this
    logic \seeSrcCode{897d048}{csvToGraph}{54}{95} so they can be
    consistently tracked throughout the analysis.

    \def\exampleTableNote{``Name'' can refer to the name of a test approach,
        software quality, or other testing-related term, but we only generated
        graphs for test approaches.}
    \newcommand\exampleTableCapHelper[2]{Example glossary entries\
        demonstrating how we track #1 relations (see \Cref{#2}).}

    \begin{table*}[hbtp!]
        \centering
        \begin{talltblr}[
                note{a}={\exampleTableNote},
                caption={\exampleTableCapHelper{parent-child}{par-chd-rels}},
                label={tab:exampleGlossary}
            ]{colspec={|ll|}, rowhead = 1}
            \hline
            \thead{Name\TblrNote{a}}    & \thead{Parent(s)}                \\ \hline
            A                           & B (Author, 0000; 0001), C (0000) \\
            B                           & C (implied by Author, 0000)      \\
            C                           & D (Author, 0002)                 \\
            D (implied by Author, 0002) &                                  \\ \hline
        \end{talltblr}
    \end{table*}

    \ExampleGraph{}

    \newpage\fi
All parent-child relations are graphed, since they are guaranteed to be
visually meaningful. Synonym relations, however, are either excluded from or
included in graphs as follows.\todo{Is this correct grammar?} For each synonym
pair, at least one term will have its own row (or else it would not appear in
the glossary at all), so the following cases are possible:
\begin{description}
    \item[1. (Excluded)] \ifnotpaper\else \hfill\break \fi
          \textbf{Only one synonym has its own row.}
          This is a ``typical'' synonym relation (see \Cref{syn-rels}) where
          the terms are interchangeable. The synonym \emph{could} be included
          as an alternate name inside the node of its partner, but this would
          unnecessarily clutter the graphs.

          % Reference for singular 'row': https://ell.stackexchange.com/q/105868/169502
    \item[2. (Included)] \ifnotpaper\else \hfill\break \fi
          \textbf{Both synonyms have their own row in the glossary.}
          This may indicate that the synonym relation is
          incorrect, since separate rows in the glossary define separate
          approaches (with their own definitions, nuances, etc.).

    \item[3. (Included)] \ifnotpaper\phantomsection{}\label{case-three}
          \else\hfill\break\fi
          \textbf{Two synonym pairs share a synonym without its own row.}
          This is a transitive extension to the previous case. If
          two distinct approaches share a synonym, that implies that they are
          synonyms themselves, resulting in the same possibility of the
          relation being incorrect.
\end{description}
\ifnotpaper
    \todo{Is this multicolumn formatting OK?}
    \begin{minipage}{0.55\textwidth}
        \begin{talltblr}[
                note{a}={\exampleTableNote},
                caption={\exampleTableCapHelper{synonym}{syn-rels}},
                label={tab:synExampleGlossary}
            ]{colspec={|ll|}, rowhead = 1}
            \hline
            \thead{Name\TblrNote{a}} & \thead{Synonym(s)}                \\  \hline
            E                        & F (Author, 0000; implied by 0001) \\
            G                        & {F (Author, 0002),                \\ \displayNL{} H (implied by 0000)} \\
            H                        & X                                 \\ \hline
        \end{talltblr}
    \end{minipage} \hfill
    \begin{minipage}{0.35\textwidth}
        These conditions are deduced from the information parsed
        from the glossary. For example, if the entries in \Cref{tab:synExampleGlossary}
        appear in the glossary, then they are displayed as \Cref{fig:synExampleGraph}
        in the generated graph (note that X does not appear since it does not
        meet the criteria given above).
    \end{minipage} \newpage

    This allows for automatic detection of some classes of flaws. The
    most trivial to automate is ``multi-synonym'' relations, given in
    % TODO: pretty hacky
    \hyperref[case-three]{Case 3} above, since these are already found in order
    to generate the graph as desired. The list found in \Cref{multiSyns}
    is automatically generated based on glossary entries such as those found
    in \Cref{tab:synExampleGlossary}. The self-referential definitions in
    \Cref{selfPars} were also trivial, found by simply looking for lines in
    the generated .tex files with prefixes of the form \texttt{I -> I} (where
    \texttt{I} is the label used for a test approach node in these graphs).
    This process results in output similar to \Cref{fig:selfExampleGraph}. A
    similar process is used to detect instances where two approaches have a
    synonym \emph{and} a parent-child relation. A dictionary of each term's
    synonyms is built to evaluate which synonym relations are notable
    enough to include in the graph, and these mappings are then checked to
    see if one appears as a parent of the other. For example, if \texttt{J}
    and \texttt{K} are synonyms, a generated .tex file with a parent line
    starting with \texttt{J -> K} would result in these approaches being
    graphed as shown in \Cref{fig:parSynExampleGraph}.

    The visual nature of these graphs makes it possible to represent both
    explicit and implicit relations without double counting them during the
    analysis in \Cref{flaw-analysis}. If a relation is both explicit
    \emph{and} implicit, the implicit relation is only shown in the graph
    if it is from a more ``trusted'' source tier (see \Cref{sources}).
    For example, note that only the explicit synonym relation between E and F
    from \Cref{tab:exampleGlossary} is shown in \Cref{fig:synExampleGraph}.
    Implicit approaches and relations are denoted by dashed lines, as shown
    in \Cref{fig:exampleGraph,fig:synExampleGraph}; explicit approaches are
    \emph{always} denoted by solid lines, even if they are also implicit.
    ``Rigid'' versions of these graphs that exclude implicit approaches and
    relations can also be generated; the rigid version of
    \Cref{fig:exampleGraph} is given in \Cref{fig:rigidExampleGraph}.

\fi
Since these graphs tend to be large, it is useful to focus on specific
subsets of them. \ifnotpaper To do this, we generated graphs limited to
    approaches in a selected approach category (see \Cref{categories-observ})
    as well as a graph of static approaches. The latter is done because
    \citet[Fig.~2]{IEEE2022} \multiAuthHelper{consider} static testing to be a
    separate approach category and because static testing is quite different
    from dynamic testing (see \Cref{static-test}). Generated graphs focused
    on static testing include any relations with dynamic approaches (since
    they are our primary focus) and these dynamic approach nodes are
    colored grey, as shown in \Cref{fig:staticExampleGraph}.

    Additionally, more specific subsets of these graphs \else These \fi
can be generated from a given subset of approaches, such as
\ifnotpaper\else those in a selected approach category (see
    \Cref{categories-observ}) or \fi those pertaining to recovery or
scability\ifnotpaper. These areas are of particular note, with their own
sections for discussing flaws (\Cref{recov-discrep,scal-discrep},
respectively). Graphs of just these subsets help visualize relations
between relevant test approaches, so these are generated and given
\else; the latter are shown \fi in \Cref{fig:recovery-graph-current,%
    fig:scal-graph-current}, respectively. By specifying sets of approaches
and relations to add or remove,
these generated graphs can then be updated in accordance with our
recommendations; applying those given in \Cref{rec-test-rec,,scal-test-rec,,%
    \ifnotpaper\else perf-test-rec\fi} results in the updated graphs in
\Cref{fig:recovery-graph-proposed,,fig:scal-graph-proposed,,\ifnotpaper\else
        fig:perf-graph\fi}, respectively. Any added approaches or relations are
colored \textcolor{orange}{orange}.
\ifnotpaper
    Recommendations can also be inherited; for example,
    \Cref{fig:perf-graph} was generated based on the modifications from
    \Cref{fig:recovery-graph-proposed,fig:scal-graph-proposed} along with
    other changes mentioned in \Cref{perf-test-rec}.
\fi

\ifnotpaper
    \subsection{Flaw Analysis}
\label{flaw-analysis}

In addition to analyzing specific flaws, it is also useful to examine them at
a higher level. We automate subsets of this task where applicable
(\Cref{auto-flaw-analysis}) and augment the remaining manual portion with
automated tools (\Cref{aug-flaw-analysis}). This gives us an overview of:
\begin{itemize}
    \item how many flaws there are,
    \item how responsible each source tier (see \Cref{sources}) is for these flaws,
    \item how obvious (or ``rigid''; see \Cref{rigidity}) these flaws are,
    \item how these flaws present themselves (see \Cref{sntxFlaws}), and
    \item in which knowledge domains these flaws occur (see \Cref{smntcFlaws}).
\end{itemize}

To understand where flaws exist in the literature, we group them based on the
source tier(s) responsible for them. Each flaw is then counted \emph{once} per
source tier if it appears within it \emph{and/or} between it and a more
``trusted'' tier. This avoids counting the same flaw more than once for a given
source tier\thesisissueref{83}, which would give the number of \emph{occurrences}
of all flaws instead of the more useful number of flaws \emph{themselves}. The
exception to this is \Cref{fig:flawSources}, which counts the following sources
of flaws separately:
\begin{enumerate}
    \item those that appear once in (or consistently throughout) a document
          (i.e., are ``self-contained'')\thesisissueref{137,138},
    \item those between two parts of a single document
          (i.e., internal conflicts)\thesisissueref{137,138},
    \item those between documents by the same author(s) or standards
          organization(s), and
    \item those within a source tier.
\end{enumerate}
As before, these are not double counted, meaning that the maximum number of
counted flaws possible within a \emph{single} source tier in
\Cref{fig:flawSources} is four (one for each type). This only occurs if
there is an example of each flaw source that is \emph{not} ignored to
avoid double counting; for example, while a single flaw within a single
document would technically fulfill all four criteria, it would only be counted
once. Note that while the different versions of the \acfp{swebok} have
different editors \citep{SWEBOK2024,SWEBOK2014}, we consider them to be written
by the same organization: the IEEE Computer Society (\citealp{AboutSWEBOK}; see
\Cref{metas}).

\phantomsection{}
\label{flaw-analysis-example}
As an example of this process, consider a flaw \emph{within} an IEEE
document (e.g., two different definitions are given for a term within the same
IEEE document) \emph{and} between another IEEE document, the \acs{istqb}
glossary \emph{and} two papers. This would add one to the following rows of
\Cref{tab:sntxFlaws,tab:smntcFlaws} in the relevant column:

\begin{itemize}
    \item \textbf{\stds{}}: this flaw occurs:
          \begin{enumerate}
              \item within one standard and
              \item between two standards.
          \end{enumerate}
          This increments the count by just one to avoid double counting and
          would do so even if only one of the above conditions was true. A more
          nuanced breakdown of flaws that identifies those within a
          singular document and those between documents by the same author is
          given in \Cref{fig:flawSources} and explained in more detail in
          \Cref{aug-flaw-analysis}.
    \item \textbf{\metas{}}: this flaw occurs between a source in this tier and
          a ``more trusted'' one (the IEEE standards).
    \item \textbf{\papers{}}: this flaw occurs between a source in this tier
          and a ``more trusted'' one. Even though there are two sources in this
          tier \emph{and} two ``more trusted'' tier involved, this increments
          the count by just one to avoid double counting.
\end{itemize}

\subsubsection{Automated Flaw Analysis}
\label{auto-flaw-analysis}

As outlined in \Cref{graph-gen}, we can detect some classes of flaws
automatically. While just counting the total number of flaws is trivial,
tracking the source(s) of these flaws is more useful yet more involved. Since
we consistently track the appropriate citations for each piece of information
we record (see \Cref{tab:exampleGlossary,tab:synExampleGlossary} for examples
of how these citations are formatted in the glossaries), we can use them to
identify the offending source tier(s). This comes with the added benefit that
we can format these citations to use with \LaTeX{}'s citation commands in this
\docType{}.

We compare the authors and years of each source involved with a given flaw
to determine if it manifests within a single document and/or between documents
by the same author(s). Then, we group these sources into their tiers
\seeSrcCode{82167b7}{scripts/flawCounter.py}{63}{80}
% done by the function in \Cref{lst:getSrcCat}, since each source tier
% outlined in \Cref{sources} is comprised of a small number of authors (with the
% exception of papers and other documents; see \Cref{papers}).
to determine the row(s) of \Cref{tab:sntxFlaws,tab:smntcFlaws} and the graph(s)
and slice(s) in \Cref{fig:flawSources} that the given flaw should count toward.
We then distill these lists of sources down to sets of tiers and compare them
against each other to determine how many times a given flaw manifests between
source tiers. Examples of this process are described in more detail in
\Cref{aug-flaw-analysis}.

\phantomsection{}
\label{auto-flaw-analysis-rigidity}
Alongside this citation information, we include keywords so we can assess how
``rigid'' a piece of information is (see \Cref{rigidity}). This is useful when
counting flaws, since they can be both explicit and implicit but should not be
double counted as both\thesisissueref{83}! When counting flaws in
\Cref{tab:sntxFlaws,tab:smntcFlaws}, each one is
counted only for its most ``rigid'' manifestation (i.e., it will only increment
a value in the ``Implicit'' column if it is \emph{not} also explicit),
similarly to how we generate graphs (see \Cref{expAndImp}).

\subsubsection{Augmented Flaw Analysis}
\label{aug-flaw-analysis}
While we can detect some subsets of flaws automatically by analyzing
\ourApproachGlossary, most are too complex and need to be tracked manually. We
record these more detailed flaws as \LaTeX{} enumeration items along with
comments that we can parse automatically, allowing us to analyze them more
broadly. We also add these comments to flaws we detect automatically before
generating the corresponding \LaTeX{} file to ensure these flaws also get
analyzed. These comments have the following format:
\begin{displayquote}
    \texttt{\% Flaw count (SEM, SYN): \{A1\} \{A2\} \dots{} | \{B1\} \{B2\}
        \dots{} | \{C1\} \dots}
\end{displayquote}
\texttt{SEM} is a placeholder for the flaw's semantics (see \Cref{smntcFlaws})
identifier and \texttt{SYN} is a placeholder for its syntax (see \Cref{sntxFlaws})
identifier. We omit these designations from the examples of these comments
throughout this chapter for brevity. \texttt{A1}, \texttt{A2}, \texttt{B1},
\texttt{B2}, and \texttt{C1} are each placeholders for a source involved in
this example flaw; in general, there can be arbitrarily many. Each ``group'' of
sources is separated with a pipe symbol (\texttt{|}) to be compared with all
other groups; in general, a flaw can have any number of groups of sources.

We make a distinction between ``self-contained''
flaws and ``internal'' flaws. Self-contained flaws are those that manifest by
comparing a document to a source of ground truth. Sometimes, these do not
require an explicit comparison; for example, \nameref{miss} often fall in this
category, since the lack of information is contained within a single source and
does not need to be cross-checked against a source of ground truth. If only one
group of sources is present in a flaw's comment, such as the first line
below, it is considered to be a self-contained flaw. On the other hand,
internal flaws arise when a document disagrees with itself by containing two
conflicting pieces of information and include many \nameref{contra} and
\nameref{over}. These can even occur on the same page, such as when an acronym
is given to two distinct terms (see \flawref{cat-acro,hil-acro})! If a
source appears in multiple groups in a flaw's comment, it is considered to
be an internal flaw. The second line is a standard example of this, while the
third is more complex; in this case, source Y agrees with only one of the
conflicting sources of information in X.
\begin{displayquote}
    \texttt{\% Flaw count: \{X\}\\\% Flaw count: \{X\} | \{X\}\\
        \% Flaw count: \{X\} | \{X\} \{Y\}}
\end{displayquote}
Discrepancies between groups are not double counted; this means the following
line adds discrepancies between X and Z \emph{and} between Y and Z, without
counting the discrepancy between X and Z twice.
\begin{displayquote}
    \texttt{\% Flaw count: \{X\} | \{X\} \{Y\} | \{Z\}}
\end{displayquote}

Each source is given using its BibTeX key wrapped in curly braces to mimic
\LaTeX{}'s citation commands for ease of parsing, with the exception of the
\acs{istqb} glossary, due to its use of custom commands via
\texttt{\textbackslash citealias}. For example, the line
\begin{displayquote}
    \texttt{\% Flaw count: \{IEEE2022\} | \{IEEE2022\} \{IEEE2017\}
        \displayNL ISTQB \{Kam2008\} \{Bas2024\}}
\end{displayquote}
would be parsed as the example given in \Cref{flaw-analysis-example}. Since
the IEEE documents are written by the same standards organizations
(\begin{NoHyper}\citeauthor{IEEE2022}\end{NoHyper}), they are counted as a
discrepancy between documents by the same author(s) in \Cref{fig:flawSources}.

The rigidity (see \Cref{rigidity}) of flaws can also be manually
specified by inserting the phrase ``implied by'' after the sources of explicit
information and before those of implicit information. Parsing this information
follows the same rules as the automatic flaw analysis
(see \Cref{auto-flaw-analysis-rigidity}).
\begin{displayquote}
    \texttt{\% Flaw count: \{IEEE2022\} implied by \{Kam2008\} |
        \displayNL \{IEEE2017\} implied by \{IEEE2022\}}
\end{displayquote}
For example, the above line indicates that the flaws given below are
present. The second flaw only affects \Cref{fig:flawSources} since it
is less ``rigid'' than the first flaw within standards (see \Cref{stds}).
The rest increment their corresponding count in
\Cref{fig:flawSources,tab:sntxFlaws,tab:smntcFlaws} by only one:
\begin{itemize}
    \item an explicit discrepancy between documents by
          \begin{NoHyper}\citeauthor{IEEE2022}\end{NoHyper},
    \item an implicit discrepancy within a single document, and
    \item an implicit discrepancy between a paper and a standard.
\end{itemize}

Occasionally, a source from a lower tier is used as the ``ground truth'' for a
flaw. For example, \tolTestFlaw*{} This flaw is supported
by additional papers found via a miniature literature review (described in
\Cref{undef-terms}) from a lower source tier than \citep{Firesmith2015}
(which is a terminology collection; see \Cref{sources}). However, it
is really based in \citep{Firesmith2015} and not in these
additional papers, but if these sources where included as detailed above,
this is the way it would be counted. Therefore, we document these ``ground
truth'' sources separately to track them for traceability without incorrectly
counting flaws. This is done for this specific example (and similarly
for other cases) as follows:
\begin{displayquote}
    \texttt{\% Flaw count (TERMS, WRONG): \{Firesmith2015\}\\
        \% Ground truth: \{LiuEtAl2023\} \{MorgunEtAl1999\} \{HolleyEtAl1996\}
        \displayNL \{HoweAndJohnson1995\}}
\end{displayquote}

\subsection[LaTeX Commands]{\LaTeX{} Commands}\label{macros}
To improve maintainability, traceability, and reproducibility, we define
helper commands (also called ``macros'') for content that is prone to change
or used in multiple places. For example, many values are calculated by a Python
script and saved to a file. We then assign these values to a corresponding
\LaTeX{} macro, which we can use instead of manually replacing the value
throughout our documents every time it changes. \Cref{tab:macrosCalc} lists
these macros, along with descriptions of what they define and their current
values.

\begin{longtblr}[
    note{a} = {Calculated in \LaTeX{} from source tier lists; see \Cref{text-macros}.},
    note{b} = {Alias for \texttt{\textbackslash totalSmntcFlawBrkdwn\{13\}}; see \Cref{flawCounts}.},
    note{c} = {These macros are defined as counters to allow them to be used in
            calculations within \LaTeX{} (such as in \Cref{undef-terms,fig:undefPies}).},
    caption = {\LaTeX{} macros for calculated values.},
    label = {tab:macrosCalc}
    ]{
    colspec={|X[0.3,l,m]X[0.5,c,m]X[0.2,c,m]|},
    width = \linewidth, rowhead = 1
    }
    \hline
    \thead{Macro}                                   & \thead{What it Counts}        & \thead{Value}    \\
    \hline
    \macro{approachCount}                           & Identified test approaches    & \approachCount{} \\
    \macro{qualityCount}                            & Identified software qualities & \qualityCount{}  \\
    \macro{srcCount}\TblrNote{a}                    & Sources used in glossaries    & \srcCount{}      \\
    \macro{flawCount}\TblrNote{b}                   & Identified flaws              & \flawCount{}     \\
    \hline
    \macro{TotalBefore}\TblrNote{c}                 & Test approaches identified
    before process in \Cref{undef-terms}            & \the\TotalBefore{}                               \\
    \macro{UndefBefore}\TblrNote{c}                 & Undefined test approaches
    identified before process in \Cref{undef-terms} & \the\UndefBefore{}                               \\
    \macro{TotalAfter}\TblrNote{c}                  & Test approaches identified
    after process in \Cref{undef-terms}             & \the\TotalAfter{}                                \\
    \macro{UndefAfter}\TblrNote{c}                  & Undefined test approaches
    identified after process in \Cref{undef-terms}  & \the\UndefAfter{}                                \\
    \hline
    \macro{parSynCount}                             & Pairs of test approaches
    with a child-parent \emph{and} synonym relation & \parSynCount{}                                   \\
    \macro{selfCycleCount}                          & Test approaches that are
    a parent of themselves                          & \selfCycleCount{}                                \\
    \hline
\end{longtblr}


\phantomsection{}\label{flawCounts}
Additionally, we count flaws based on their rigidity, source tier,
and whether they are syntactic or semantic (see
\Cref{rigidity,sources,auto-flaw-analysis,aug-flaw-analysis,flaws}).
We save these counts to files, a syntactic and semantic version for each
source tier%
% ; for example, syntactic flaws in standards are saved to
% \texttt{build/stdSntxFlawBrkdwn.tex} and semantic flaws in standards to
% \texttt{build/stdSmntcFlawBrkdwn.tex}. These data
, then read them in to macros
% (such as \macro{stdSmntcFlawBrkdwn})
to populate \Cref{tab:sntxFlaws,tab:smntcFlaws}. For example,
\macro[1]{stdSntxFlawBrkdwn} corresponds to the number of explicit mistakes in
standards documents, and \macro[2]{stdSntxFlawBrkdwn} to the number of implicit
ones. These macros also include \macro[13]{totalSntxFlawBrkdwn} and \newline
\macro[13]{totalSntxFlawBrkdwn}, which are identical and
track the total number of identified flaws.

\phantomsection{}\label{text-macros}
Just as with calculated values, it is important that repeated text is updated
consistently, which we accomplish by defining more macros. Some of these are
generated by Python scripts in a similar fashion to calculated values, such as
the lists of sources in \Cref{sources}. These are built by extracting all
sources cited in our three glossaries, categorizing, sorting, and formatting
them (including handling edge cases), and saving them to a file. These are then
defined as \macro{stdSources}, \macro{metaSources}, \macro{textSources}, and
\macro{paperSources}.\todo{Should I explicitly display these lists in a table
    here? That feels redundant and might make things unnecessarily cluttered.}
The numbers of sources in each tier are also saved to build \Cref{fig:sourceSummary}
and calcuate \macro{srcCount} (see \Cref{tab:macrosCalc}). However, most of the
macros for reused text are created manually when the reuse is first noticed.
Some of these macros account for context-specific formatting, such as
capitalization, depending on how they are used; we omit these details here for
brevity. We create macros to reuse many types of information throughout our
documents as shown in \Cref{tab:macrosText}.

\begin{longtblr}[
        caption = {\LaTeX{} macros for reused text.},
        label = {tab:macrosText}
    ]{
        colspec={|ll|},
        width = \linewidth, rowhead = 1
    }
    \hline
    \thead{Macro}             & \thead{Used in}                                           \\
    \hline
    \macro{bugPattonDiscrep}  & \Cref{intro} and \discrepref{bug-patton}                  \\
    \macro{alphaDiscrep}      & \Cref{intro} and \discrepref{alpha-def}                   \\
    \macro{loadDiscrep}       & \Cref{intro} and \discrepref{load-def}                    \\
    \macro{expBasedCatMain}   & \Cref{intro,multiCats}                                    \\
    \macro{tourDiscrep}       & \Cref{intro,discreps} and \discrepref{tour-def}           \\
    \macro{perfAsFamily}      & \Cref{method-family,classFamilyDiscrep}                   \\
    \macro{tolTestingDiscrep} & \Cref{aug-discrep-analysis} and \discrepref{ground-truth} \\
    \hline
\end{longtblr}


\phantomsection{}\label{paper-macros}
In addition to this thesis, we also prepare a conference paper based on our
research. While we can reuse most content without modifying it, there are some
formatting differences between the two document types. For example, our thesis
uses the \texttt{natbib} package for citations while the IEEE guidelines for
paper submissions suggest the use of \texttt{cite} \citep[p.~8]{Shell2015};
we define aliases so that we can reuse text that includes citations
\seeSrcCode{82167b7}{paper_preamble.tex}{19}{60}.

In general, we use the command\todo{Is this a ``command''?}
\texttt{\textbackslash ifnotpaper} to allow for manual distinctions between the
two documents' formats, such as how they handle citations, using this basic format:
\begin{displayquote}
    \texttt{\textbackslash ifnotpaper <thesis code> \textbackslash else <paper code> \textbackslash fi}
\end{displayquote}
For example, in \Cref{nonIEEE-sources}, we provide a list of non-IEEE sources
that support a claim made by the IEEE. Since we sort sources based on
trustworthiness (see \Cref{sources}), publication year, and number of authors,
the relevant thesis code is:
\begin{displayquote}
    \texttt{(\textbackslash citealp[pp.\textasciitilde 5\textbackslash =/6 to 5\textbackslash =/7]\{SWEBOK2024\};
        \displayNL \textbackslash citealpISTQB\{\};
        \textbackslash citealp[pp.\textasciitilde 807\textbackslash ==808]\{Perry2006\};
        \displayNL \textbackslash citealp[pp.\textasciitilde 443\textbackslash ==445]\{PetersAndPedrycz2000\};
        \displayNL \textbackslash citealp[p.\textasciitilde 218]\{KuļešovsEtAl2013\}\textbackslash todo\{OG Black, 2009\};
        \displayNL \textbackslash citealp[pp.\textasciitilde 9, 13]\{Gerrard2000a\})}
\end{displayquote}
Meanwhile, IEEE guidelines prefer that sources are kept in separate brackets,
sorted in order of their first appearance in the document. Therefore, paper
code for this list of sources is:
\begin{displayquote}
    \texttt{\textbackslash cite[pp.\textasciitilde 443\textbackslash ==445]\{PetersAndPedrycz2000\},
        \displayNL \textbackslash cite[pp.\textasciitilde 5\textbackslash =/6 to 5\textbackslash =/7]\{SWEBOK2024\},
        \textbackslash cite\{ISTQB\},
        \displayNL \textbackslash cite[pp.\textasciitilde 807\textbackslash ==808]\{Perry2006\},
        \displayNL \textbackslash cite[pp.\textasciitilde 9, 13]\{Gerrard2000a\},
        \displayNL \textbackslash cite[p.\textasciitilde 218]\{KuļešovsEtAl2013\}}
\end{displayquote}\latertodo{Ensure this code matches the final version!}
In particular, note the usage of the \macro{cite} command, the \emph{lack} of
use of the custom alias for citing the \acs{istqb} glossary, the different
order, and the lack of the \macro{todo}, since these are only rendered for
reference in the thesis. For other edge cases with different formatting styles
for these document types, we define the macros given in \Cref{tab:macrosPaper}
via an example usage of the macro(s) and how the code is rendered in our thesis
\emph{and} paper\footnote{Since this document is the thesis version, some of
    the paper renderings are hardcoded.}\todo{Does this footnote make sense?}:

\def\refHelperEx{\refHelper \ifnotpaper \citet{IEEE2022}
    \else \defcitealias{IEEE2022}{1}[\citetalias{IEEE2022}]
    \fi \multiAuthHelper{form} the basis of this \docType{}.}

\begin{longtblr}[
    note{a} = {Section omitted for brevity.},
    caption = {\LaTeX{} macros for handling formatting differences between thesis and paper.},
    label = {tab:macrosPaper}
    ]{
    colspec={|Q[c,m]|X[l,m]|},
    column{1} = {font=\bfseries},
    width = \textwidth, rowhead = 1
    }
    \hline
    \thead{Context} & \thead{Displayed as}                                \\
    \hline
    Code            & {\texttt{\macro{refHelper} \macro[IEEE2022]{citet}\
    \macro[form]{multiAuthHelper}}                                        \\
    \displayNL\texttt{the basis of this \macro{docType}.}}                \\*
    Thesis          & \refHelperEx{}                                      \\*
    Paper           & {\notpaperfalse \refHelperEx{}}                     \\
    \hline
    Code            & \macro[cat-acro]{flawref}                           \\*
    Thesis          & \flawref{cat-acro}                                  \\*
    Paper           & Section \hyperref[cat-acro]{III-B2}                 \\
    \hline
    Code            & \macro{reduns}                                      \\*
    Thesis          & \nameref{redun}                                     \\*
    Paper           & Redundancies\TblrNote{a}                            \\
    \hline
\end{longtblr}


\else
    % Moved here to display nicely in paper
    \sntxDiscrepsTable{}
    \smntcDiscrepsTable{}
\fi

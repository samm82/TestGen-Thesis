\section{Observations}

\subsection{Categories of Testing Approaches}

For classifying different kinds of tests,
\ifnotpaper
    \citet{IEEE2022} provide
\else
    \cite{IEEE2022} provides
\fi
some terminology (see \refIEEETestTerms{}).
\ifnotpaper
    However, other sources \citep{BarbosaEtAl2006, SouzaEtAl2017}
    % \cite{BarbosaEtAl2006, SouzaEtAl2017}
    provide alternate categories
    (see \refOtherTestTerms{}) which may be beneficial to investigate to
    determine if this categorization is sufficient.
    % \fi 
    % \ifnotpaper
    A ``metric'' categorization was considered at one point, but was decided
    to be out of the scope of this project
    \seeSectionPar{chap:testing:sec:scope}{, \thesisissueref{21}, and
        \thesisissueref{22}}.
\fi
Related testing approaches may be grouped into a ``class'' or ``family'' to
group those with ``commonalities and well-identified variabilities that can be
instantiated'', where ``the commonalities are large and the variabilities
smaller'' (see \thesisissueref{64}). Examples of these are the classes of
combinatorial \citep[p.~15]{IEEE2021} and data flow testing \citetext{p.~3} and the
family of performance-related testing \cite[p.~1187]{Moghadam2019}\footnote{The
    original source describes ``performance testing \dots\ as a family of
    performance-related testing techniques'', but it makes more sense to
    consider ``performance-related testing'' as the ``family'' with
    ``performance testing'' being one of the
    variabilities\seeSectionPar{perf-test-ambiguity}.}, and may also be
implied for security testing, a test type that consists of ``a number of
techniques\footnote{This may or may not be \distinctIEEE{technique}}''
\cite[p.~40]{IEEE2021}.

It also seems that these categories are orthogonal. For example, ``a test type
can be performed at a single test level or across several test levels''
\ifnotpaper
    (\citealp[p.~15]{IEEE2022}; \citeyear[p.~7]{IEEE2021})%
\else
    \cite[p.~15]{IEEE2022}, \cite[p.~7]{IEEE2021}%
\fi. Due to this, a specific
test approach can be derived by combining test approaches from different
categories;\seeSection{chap:testing:sec:orthogonal-tests} for some examples
of this. However, the boundaries between items within a category may be unclear:
``although each technique is defined independently of all others, in practice
    [sic] some can be used in combination with other techniques''
\citep[p.~8]{IEEE2021}. For example, ``the test coverage items derived by
applying equivalence partitioning can be used to identify the input parameters
of test cases derived for scenario testing'' \citetext{p.~8}. Even the categories
themselves are not consistently defined, as some approaches are categorized
differently by different sources; these differences will be tracked noted so
that they can be analyzed more systematically (see \thesisissueref{21}).
There are also several instances of inconsistencies between parent and child
test approach categorizations (which may indicate they aren't necessarily the
same, or that more thought must be given to classification/organization).
Examples of discrepancies in test-approach categorization:

\begin{enumerate}
    \item Experience-based testing is categorized as both a test design
          technique and a test practice on the same page
          \citep[pp.~22, 34]{IEEE2022}!
          \begin{itemize}
              \item These authors previously say ``experience-based testing
                    practices like exploratory testing \dots\ are not
                    \dots\ techniques for designing test cases'', although
                    they ``can use \dots\ test techniques''
                    \citeyearpar[p.~viii]{IEEE2021}. This implies that
                    ``experience-based test design techniques'' are
                    techniques used by the \emph{practice} of experience-based
                    testing, not that experience-based testing is
                    \emph{itself} a test technique. If this is the case, it
                    is not always clearly articulated
                    \ifnotpaper
                        (\citealp[pp.~4,~22]{IEEE2022}; \citeyear[p.~4]{IEEE2021};
                        \citealp[p.~5-13]{SWEBOK2024}; \citealpISTQB{})
                    \else
                        \cite[pp.~4,~22]{IEEE2022}, \cite[p.~4]{IEEE2021},
                        \cite[p.~5-13]{SWEBOK2024}, \cite{ISTQB}
                    \fi
                    and is
                    sometimes contradicted \citep[p.~46]{Firesmith2015}.
                    However, this conflates the distinction between
                    ``practice'' and ``technique'', making these terms less
                    useful, so this may just be a mistake
                    (see \thesisissueref{64}).

                    % Furthermore, if a ``class of \dots\ techniques''
                    % is a practice, then other ``techniques'', such as combinatorial testing
                    % (\citealp[pp.~3,~22]{IEEE2022}; \citeyear[p.~2]{IEEE2021};
                    % \citealp[p.~5-11]{SWEBOK2024}; \citealpISTQB{}), data flow testing
                    % (\citealp[p.~22]{IEEE2022}; \citeyear[p.~3]{IEEE2021};
                    % \citealp[p.~5-13]{SWEBOK2024}; \citealp[p.~43]{Kam2008}), performance(-related)
                    % testing (\citealp[p.~38]{IEEE2021}; \citealp[p.~1187]{Moghadam2019}), and
                    % security testing \citep[p.~40]{IEEE2021} may \emph{also}
                    % actually be practices, since they are also described as classes or families of
                    % techniques. The same could be said of the more
                    % general specification- and structure-based testing, especially since these,
                    % plus experience-based testing, are described as ``complementary'' \citetext{p.~8,~Fig.~2}.
                    % % \citeyearpar[p.~8, Fig.~2]{IEEE2021}

              \item This also causes confusion about its children, such as
                    error guessing and exploratory testing; again, on the
                    same page,
                    \ifnotpaper
                        \citeauthor{IEEE2022} say
                    \else
                        \cite[p.~34]{IEEE2022} says
                    \fi error guessing is
                    an ``experience-based test design technique'' and
                    ``experience-based test practices include \dots\
                    exploratory testing, tours, attacks, and
                    checklist-based testing''%
                    \ifnotpaper{ \citeyearpar[p.~34]{IEEE2022}.}
                    \else{.}
                    \fi
                    Other sources also do not agree whether error guessing
                    is a technique
                    \ifnotpaper
                        (pp.~20,~22; \citeyear[p.~viii]{IEEE2021})
                    \else
                        \cite[pp.~20,~22]{IEEE2022}, \cite[p.~viii]{IEEE2021}
                    \fi
                    or a practice \citep[p.~5-14]{SWEBOK2024}.
          \end{itemize}
    \item The following are test approaches that are categorized as test
          techniques in \citep[p.~38]{IEEE2021}, followed by sources that
          categorize them as test types:
          \begin{enumerate}
              \item Capacity testing
                    \ifnotpaper
                        (\citealp[p.~22]{IEEE2022};
                        \citeyear[p.~2]{IEEE2013}; implied by its quality
                        (\citealp{ISO_IEC2023a}; \citealp[Tab.~A.1]{IEEE2021})
                        and by \citep[p.~53]{Firesmith2015})
                    \else
                        \cite[p.~22]{IEEE2022}, \cite[p.~2]{IEEE2013} (implied by
                        \cite[p.~53]{Firesmith2015} and by its quality
                        \cite{ISO_IEC2023a}, \cite[Tab.~A.1]{IEEE2021})
                    \fi
              \item Endurance testing
                    \ifnotpaper
                        (\citealp[p.~2]{IEEE2013};
                        implied by \citep[p.~55]{Firesmith2015})
                    \else
                        \cite[p.~2]{IEEE2013}
                        (implied by \cite[p.~55]{Firesmith2015})
                    \fi
              \item Load testing
                    \ifnotpaper
                        (\citealp[pp.~5,~20,~22]{IEEE2022};
                        \citeyear[p.~253]{IEEE2017}\todo{OG IEEE 2013};
                        \citealpISTQB{}; implied by \citep[p.~54]{Firesmith2015})
                    \else
                        \cite[pp.~5,~20,~22]{IEEE2022},
                        \cite[p.~253]{IEEE2017}\todo{OG IEEE 2013},
                        \cite{ISTQB} (implied by \cite[p.~54]{Firesmith2015})
                    \fi
              \item Performance testing
                    \ifnotpaper
                        (\citealp[pp.~7,~22,~26-27]{IEEE2022};
                        \citeyear[p.~7]{IEEE2021}; implied by
                        \citep[p.~53]{Firesmith2015})
                    \else
                        \cite[pp.~7,~22,~26-27]{IEEE2022}, \cite[p.~7]{IEEE2021}
                        (implied by \cite[p.~53]{Firesmith2015})
                    \fi
              \item Stress testing
                    \ifnotpaper
                        (\citealp[pp.~9,~22]{IEEE2022};
                        \citeyear[p.~442]{IEEE2017}; implied by
                        \citep[p.~54]{Firesmith2015})
                    \else
                        \cite[pp.~9,~22]{IEEE2022}, \cite[p.~442]{IEEE2017}
                        (implied by \cite[p.~54]{Firesmith2015})
                    \fi
          \end{enumerate}
    \item Model-based testing is categorized as both a test practice
          \ifnotpaper
              (\citealp[p.~22]{IEEE2022}; \citeyear[p.~viii]{IEEE2021})
          \else
              \cite[p.~22]{IEEE2022}, \cite[p.~viii]{IEEE2021}
          \fi
          and
          a test technique
          \ifnotpaper
              (\citealp[p.~4]{Kam2008}; implied by
              \citealp[p.~7]{IEEE2021}; \citeyear[p.~469]{IEEE2017}).
          \else
              \cite[p.~4]{Kam2008} (implied by
              \cite[p.~7]{IEEE2021}, \cite[p.~469]{IEEE2017}).
          \fi
    \item Data-driven testing is categorized as both a test practice
          \citep[p.~22]{IEEE2022} and a test technique
          \citep[p.~43]{Kam2008}\todo{OG Fewster and Graham}.
\end{enumerate}

\ifnotpaper
    \newgeometry{margin=1.5cm, top=2.5cm}
    \begin{landscape}
        \ieeeTestTermsTable{}
    \end{landscape}

    \begin{landscape}
        \otherTestTermsTable{}
    \end{landscape}
    \restoregeometry
\else
    \ieeeTestTermsTable{}
\fi

\ifnotpaper
    \subsubsection{Techniques and Coverage}
\else
    \phantomsection
\fi
\label{tech-cov}

Test techniques are able to ``identify test coverage items \dots\ and
derive corresponding test cases''
\ifnotpaper
    (\citealp[p.~11]{IEEE2022}; similar in \citeyear[p.~467]{IEEE2017})
\else
    \cite[p.~11]{IEEE2022} (similar in \cite[p.~467]{IEEE2017})
\fi
in a ``systematic'' way
\citeyearpar[p.~464]{IEEE2017}.
\ifnotpaper
    This allows for ``the coverage achieved by a specific test
    design technique'' to be calculated as ``the number of test coverage items
    covered by executed test cases'' divided by ``the total number of test
    coverage items identified'' \citeyearpar[p.~30]{IEEE2021}.
\fi
``Coverage levels can range
from 0\% to 100\%'' and may or may not include ``infeasible'' test coverage
items, which are ``not \dots\ executable or [are] impossible to be covered by a
test case'' \citetext{p.~30}. The further implication is that different
coverage metrics imply test approaches aimed to maximize them; for example,
``path testing'' is testing that ``aims to execute all entry-to-exit
control flow paths in a SUT's control flow graph'' \citep[p.~5013]{SWEBOK2024},
thus maximizing the path coverage (see also \thesisissueref{63},
\citep[Fig.~1]{SharmaEtAl2021}).

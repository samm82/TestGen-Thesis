\section{Scope}
\label{scope}

Since our motivation is restricted to testing code, only this component of
\acf{vnv} is considered\thesisissueref{22}. However, some test approaches
are used for testing things other than code, and some approaches can be used
for both! In these cases, only the subsections of these approaches focused on
code are considered. For example, reliability testing and maintainability
testing can start \emph{without} code by ``measur[ing] structural attributes
of representations of the software'' \citep[p.~18]{FentonAndPfleeger1997}, but
only reliability and maintainability testing performed on code \emph{itself} is
in scope of this research. Therefore, some practices are excluded from
consideration either in part or in full; hardware testing (\Cref{hard-test})
and the \acs{vnv} of other artifacts (\Cref{other-vnv})
are completely out of scope, as well as relevant areas of other testing
approaches that are otherwise in scope. Static testing can be performed
on code, so while it isn't relevant to the original motivation of this work, it
is a useful component of software testing and is therefore included at this level
of analysis (\Cref{static-test}). \ifnotpaper Finally, some test approaches can
    be derived from other categories of testing-related terminology
    (\Cref{derived-tests}); of these, approaches derived from programming
    languages (\Cref{lang-test}) or other orthogonal test approaches
    (\Cref{orth-test}) are out of scope. \fi

\subsection{Hardware Testing}
\label{hard-test}
While testing the software run \emph{on} or in control \emph{of} hardware is in
scope, testing performed on the hardware itself is out of scope. The following
are some examples of hardware testing approaches:

\begin{itemize}
    \item Ergonomics testing and proximity-based testing (see \citealpISTQB{})
          are out of scope, since they are used for testing hardware.
    \item Similarly, \acf{emsec} testing \ifnotpaper
              (\citealp{ISO2021}; \citealp[p.~95]{ZhouEtAl2012})\else
              \cite{ISO2021}, \cite[p.~95]{ZhouEtAl2012}\fi, which deals with
          the ``security risk'' of ``information leakage via electromagnetic
          emanation'' \citep[p.~95]{ZhouEtAl2012}, is also out of scope.
          \ifnotpaper
    \item All the examples of domain-specific testing given by
          \citet[p.~26]{Firesmith2015} are focused on hardware, so these
          examples are out of scope. However, this might not be representative
          of all types (e.g., \acf{ml} model testing seems domain-specific), so
          some subset of domain-specific testing may be in scope.
    \item Conversely, the examples of environmental tolerance testing given by
          \citet[p.~56]{Firesmith2015} do \emph{not} seem to apply to software.
          For example, radiation tolerance testing seems to focus on hardware,
          such as motors \citep{MukhinEtAl2022}, robots \citep{ZhangEtAl2020},
          or ``nanolayered carbide and nitride materials''
          \citep[p.~1]{TunesEtAl2022}. Acceleration tolerance testing seems to
          focus on \accelTolTest{} and acoustic tolerance testing on rats
          \citep{HolleyEtAl1996}, which are even less related! Since these all
          focus on environment-specific factors that would not impact the code,
          this category of testing is also out of scope.
    \item Similarly, \citet{SPICE2022} uses the terms ``software
          qualification testing'' and ``system qualification testing'' in the
          context of the automotive industry. While these may be in scope, the
          more general idea of ``qualification testing'' seems to refer to the
          process of making a hardware component, such as an electronic
          component \citep{AhsanEtAl2020}, gas generator \citep{ParateEtAl2021}
          or photovoltaic device, ``into a reliable and marketable product''
          \citep[p.~1]{SuhirEtAl2013}. Therefore, it is currently unclear if
          this is in scope\todo{Investigate further}.
          \fi
    \item \acf{orthat} can be used when testing software \citep{Mandl1985} (in
          scope) but can also be used for hardware \citep[pp.~471-472]{Valcheva2013},
          such as ``processors \dots{} made from pre-built and pre-tested
          hardware components'' \citetext{p.~471} (out of scope). A subset of
          \acs{orthat} called ``\acf{toat}'' is used for ``experimental design
          problems in manufacturing'' \citep[p.~1573]{YuEtAl2011} or ``product
          and manufacturing process design'' \citep[p.~44]{Tsui2007} and is thus
          also out of scope.
          \ifnotpaper
    \item Since control systems often have a software \emph{and} hardware
          component \citep{ISO2015, PreußeEtAl2012,ForsythEtAl2004},
          only the software component is in scope. In some cases, it is
          unclear whether the ``loops''\footnote{Humorously, the testing of
              loops in chemical systems \citep{Dominguez-PumarEtAl2020} and
              copper loops \citep{Goralski1999} are out of scope.} being
          tested are implemented by software or hardware, such as those in
          wide-area damping controllers \citep{PierreEtAl2017, TrudnowskiEtAl2017}.
          \begin{itemize}
              \item A related note: ``path coverage'' or ``path testing''
                    seems to be able to refer to either paths through code
                    (as a subset of control-flow testing)
                    \citep[p.~5-13]{SWEBOK2024} or through a model, such as
                    a finite-state machine (as a subset of model-based
                    testing) \citep[p.~184]{DoğanEtAl2014}.
          \end{itemize}
          \fi
\end{itemize}

\subsection[V\&V of Other Artifacts]{\acs{vnv} of Other Artifacts}
\label{other-vnv}
The only testing of a software artifact produced by the software life cycle
that is in scope is testing of the software itself, as demonstrated by the
following examples:

\begin{itemize}
    \item Design reviews and documentation reviews are out of scope, as they
          focus on the \acs{vnv} of design \citep[pp.~132]{IEEE2017} and
          documentation \ifnotpaper \citetext{p.~144}\else
              \cite[pp.~144]{IEEE2017}\fi, respectively.
          \ifnotpaper
    \item Security audits can focus on ``an organization's \dots{} processes
          and infrastructure'' \citepISTQB{} (out of scope) or
          ``aim to ensure that all of the products installed on a site are
          secure when checked against the known vulnerabilities for those
          products'' \citep[p.~28]{Gerrard2000b} (in scope).
          \fi
    \item Error seeding is the ``process of intentionally adding
          known faults\footnote{
          \label{add-fault}While error seeding and fault injection testing
          both introduce faults as part of testing, they do so with different
          goals: to ``estimat[e] the number of faults remaining''
          \citep[p.~165]{IEEE2017} and ``test the robustness of the system''
          \citeyearpar[p.~42]{IEEE2022}, respectively. Therefore, these
          approaches are not considered synonyms, and the lack of this relation
          in the literature is not included in \Cref{syns} as a synonym
          discrepancy.} to those already in a computer program'',
          done to both ``monitor[] the rate of detection and removal'',
          which is a part of \acs{vnv} of the \acs{vnv} itself (out of scope),
          ``and estimat[e] the number of faults remaining''
          \citep[p.~165]{IEEE2017}, which helps verify the actual code (in scope).
    \item Fault injection testing, where ``faults are artificially
          introduced\textsuperscript{\ref{add-fault}} into the \acs{sut}
          [System Under Test]'', can be used to evaluate the effectiveness of a
          test suite \citep[p.~5-18]{SWEBOK2024}, which is a part of \acs{vnv}
          of the \acs{vnv} itself (out of scope), or ``to test
          the robustness of the system in the event of internal and
          external failures'' \citep[p.~42]{IEEE2022}, which helps verify
          the actual code (in scope).
    \item ``Mutation [t]esting was originally conceived as a
          technique to evaluate test suites in which a mutant is a slightly
          modified version of the \acs{sut}'' \citep[p.~5-15]{SWEBOK2024},
          which is in the realm of \acs{vnv} of the \acs{vnv} itself (out of
          scope). However, it ``can also be categorized as a structure-based
          technique'' and can be used to assist fuzz and metamorphic testing
          \citep[p.~5-15]{SWEBOK2024} (in scope).
\end{itemize}


\subsection{Static Testing}
\label{static-test}
Sometimes, static testing is excluded from software testing
\ifnotpaper
    (\citealp[p.~222]{AmmannAndOffutt2017}; \citealp[p.~13]{Firesmith2015};
    \citealp[p.~439]{PetersAndPedrycz2000})%
\else
    \cite[p.~439]{PetersAndPedrycz2000}, \cite[p.~222]{AmmannAndOffutt2017},
    \cite[p.~13]{Firesmith2015}%
\fi, restricting ``testing'' to mean dynamic validation \citep[p.~5-1]{SWEBOK2024}
or verification ``in which a system or component is
executed'' \citep[p.~427]{IEEE2017}. However, ``terminology is not uniform
among different communities, and some use the term `testing' to refer to
static techniques\notDefDistinctIEEE{technique} as well''
\citep[p.~5-2]{SWEBOK2024}. This is done by \citet[pp.~8-9]{Gerrard2000a} and
\citet[p.~17]{IEEE2022}; the latter even explicitly exclude static testing in
another document \citeyearpar[p.~440]{IEEE2017}%
\todo{Could this be added as a scope discrepancy?}!

\ifnotpaper

    % TODO: can this be formalized at all and included in a glossary or section?
    % \citeauthor{Patton2006} introduces ``specification testing''
    % \citeyearpar[pp.~56-62]{Patton2006}, which is static black-box testing.
    % Most of this section is irrelevant to generating test cases, due to
    % requiring human involvement and verifying the specification, not the code.
    % However, it provides a ``Specification Terminology Checklist''
    % \citep[p.~61]{Patton2006} that includes some keywords that, if found, could
    % trigger an applicable warning to the user (similar to the idea behind the
    % correctness/consistency checks project), indicating that a requirement is
    % ambiguous or incomplete \citep[see][p.~1-8]{SWEBOK2024}:

    % \begin{itemize}
    %     \item \textbf{Potentially unrealistic:} always, every, all, none, every,
    %           certainly, therefore, clearly, obviously, evidently
    %     \item \textbf{Potentially vague:} some, sometimes, often, usually,
    %           ordinarily, customarily, most, mostly, good, high-quality, fast,
    %           quickly, cheap, inexpensive, efficient, small, stable
    %     \item \textbf{Potentially incomplete:} etc., and so forth, and so on,
    %           such as, handled, processed, rejected, skipped, eliminated,
    %           if \dots{} then \dots{} (without ``else'' or ``otherwise''),
    %           to be determined \citep[p.~408]{vanVliet2000}
    % \end{itemize}

    % \citeauthor{Patton2006} also provides a ``Generic Code Review Checklist''
    % \citeyearpar[pp.~99-103]{Patton2006} in the context of static testing.
    % Some of the following issues \emph{may} be able to be detected
    % automatically (e.g., by linters), but this checklist is primarily used
    % by human reviewers:

    % \begin{itemize}
    %     \item \phantomsection
    %           \label{data-ref-errors}
    %           Data reference errors: ``bugs caused by using a variable,
    %           constant, \dots{} [etc.] that hasn't been properly declared
    %           or initialized'' for its context \citetext{p.~99}
    %     \item Data declaration errors: bugs ``caused by improperly
    %           declaring or using variables or constants'' \citetext{p.~100}
    %     \item Computation errors: ``essentially bad math''; e.g., type
    %           mismatches, over/underflow, zero division, out of meaningful
    %           range \citetext{p.~101}
    %           \label{comp-errors}
    %     \item Comparison errors: ``very susceptible to boundary condition
    %           problems''; e.g., correct inclusion, floating point
    %           comparisons \citetext{p.~101}
    %     \item Control flow errors: bugs caused by ``loops and other control
    %           constructs in the language not behaving as expected''
    %           \citetext{p.~102}
    %     \item Subroutine parameter errors: bugs ``due to incorrect passing
    %           of data to and from software subroutines'' \citetext{p.~102}
    %           (could also be called ``interface errors''
    %           \citep[p.~416]{vanVliet2000})
    %     \item Input/output errors: e.g., how are errors handled?
    %           \citetext{pp.~102-103}
    %     \item ASCII character handling, portability, compilation warnings
    %           \citetext{p.~103}
    % \end{itemize}

    Static testing generally seems more ad hoc and less relevant for our
    original goal (the automatic generation of tests). In particular, many
    techniques require human intervention, either by design (such as code
    inspections) or to identify and resolve false positives (such as
    intentional exceptions to linting rules). Nevertheless, understanding
    the breadth of testing approaches requires a ``complete'' picture of how
    software can be tested and how the various approaches relate to one another.
    Parts of these static approaches may even be generated in the
    future! For these reasons, we keep static testing in scope for this stage
    of our work, even though static testing will likely be removed at a later
    step of analysis\thesisissueref{41,44} based on our original motivation.

    \subsection{Derived Test Approaches}
    \label{derived-tests}

    Since the field of software is ever-evolving, being able to adapt to,
    talk about, and understand new developments in software testing is
    crucial. In addition to methods of categorizing test approaches, the
    literature also suggests that other categories of testing-related
    terminology can be used to define new test approaches. These include
    coverage metrics (\Cref{cov-test}), software qualities (\Cref{qual-test}),
    requirements (\Cref{req-test}), and attacks (\Cref{attacks}), which are
    meaningful enough to merit analysis and are therefore in scope. Some
    approaches given in the literature are derived from programming languages
    (\Cref{lang-test}) or other orthogonal test approaches (\Cref{orth-test}),
    but these are out of scope as the information they provide is captured by
    other approaches.

    \subsubsection{Coverage-driven Techniques}
    \label{cov-test}

    Test techniques are able to ``identify test coverage items \dots{} and
    derive corresponding test cases''
    \ifnotpaper
        (\citealp[p.~11]{IEEE2022}; similar in \citeyear[p.~467]{IEEE2017})
    \else
        \cite[p.~11]{IEEE2022} (similar in \cite[p.~467]{IEEE2017})
    \fi
    in a ``systematic'' way
    \citeyearpar[p.~464]{IEEE2017}.
    \ifnotpaper
        This allows for ``the coverage achieved by a specific test design
        technique'' to be calculated as a percentage of ``the number of test
        coverage items covered by executed test cases'' \citeyearpar[p.~30]{IEEE2021}.
        %     ``Coverage levels can range
        %     from 0\% to 100\%'' and may or may not include ``infeasible'' test coverage
        %     items, which are ``not \dots{} executable or [are] impossible to be covered by a
        %     test case'' \citetext{p.~30}. Perhaps more interestingly, the further
        %     implication is
        % \else
        %     This means
    \fi % that
    Therefore, a given coverage metric implies a test approach aimed to
    maximize it. For example, path testing ``aims to execute all entry-to-exit
    control flow paths in a \acs{sut}'s control flow graph'' \citep[p.~5-13]{SWEBOK2024},
    thus maximizing the path coverage
    \ifnotpaper
        \citep[see][Fig.~1\thesisissueref{63}]{SharmaEtAl2021}\else
        (see \cite[Fig.~1]{SharmaEtAl2021}\thesisissueref{63})\fi.

    \subsubsection{Quality-driven Types}
    \label{qual-test}

    Since test types are ``focused on specific quality characteristics''
    \ifnotpaper
        (\citealp[p.~15]{IEEE2022}; \citeyear[p.~7]{IEEE2021};
        \citeyear[p.~473]{IEEE2017}\todo{OG IEEE 2013})%
    \else
        \cite[p.~15]{IEEE2022}, \cite[p.~7]{IEEE2021}, \cite[p.~473]{IEEE2017}%
    \fi, they can be derived from software qualities: ``capabilit[ies] of
    software product[s] to satisfy stated and implied needs when used under
    specified conditions'' \citep[p.~424]{IEEE2017}\todo{OG ISO/IEC 2014}. This
    is supported by reliability and performance testing, which are both examples of
    test types \citep{IEEE2022, IEEE2021} that are based on their underlying
    qualities \citep[p.~18]{FentonAndPfleeger1997}.
    % \ifnotpaper
    %     For quantifying quality-driven testing, measurements should include
    %     an entity to be measured, a specific attribute to measure, and the actual
    %     measure (i.e., units, starting state, ending state, what to include)
    %     \citetext{p.~36} where attributes must be
    %     defined before they can be measured \citetext{p.~38}.
    %
    % \fi
    Given the importance of software qualities to defining test types, the
    definitions of \qualityCount{} software qualities are also tracked in this
    current work\thesisissueref{21,23,27}. This was done by capturing their
    definitions, any precedent for the existence of an associated test type,
    and any additional notes in a glossary. Over time, software qualities were
    ``upgraded'' to test types when mentioned (or implied) by a source.
    Examples of this include conformance testing
    \ifnotpaper
        (\citealp[p.~5-7]{SWEBOK2024}; \citealp[p.~25]{JardEtAl1999}; implied
        by \citealp[p.~93]{IEEE2017})
    \else
        \cite[p.~5-7]{SWEBOK2024}, \cite[p.~25]{JardEtAl1999}
    \fi and efficiency testing \citep[p.~44]{Kam2008}.

    \subsubsection{Requirements-driven Approaches}
    \label{req-test}
    While not as universally applicable, some types of requirements have associated
    types of testing (e.g., functional, non-functional, security). This may mean
    that categories of requirements \emph{also} imply related testing approaches
    (such as ``technical testing''). \ifnotpaper Even assuming this is the case, some types of
        requirements do not apply to the code itself, and as such are out of scope%
        \thesisissueref{43}, such as:
        \begin{itemize}
            \item \textbf{Nontechnical Requirement:} a ``requirement affecting product
                  and service acquisition or development that is not a property of
                  the product or service'' \citep[p.~293]{IEEE2017}
            \item \textbf{Physical Requirement:} a ``requirement that specifies a
                  physical characteristic that a system or system component must
                  possess'' \citep[p.~322]{IEEE2017}
        \end{itemize}
    \fi

    \subsubsection{Attacks}
    \label{attacks}
    While attacks can be ``malicious'' \citep[p.~7]{IEEE2017}, they are also
    given as a test practice (\citeyear[p.~34]{IEEE2022}; see \Cref{tab:multiCats}).
    This means that software attacks, such as code injection and password
    cracking \citepISTQB{}, can also be used for testing software, and only
    this kind of software attack is in scope. This is supported by the fact
    that penetration testing is also called ``ethical hacking testing''
    \citep[p.~13-4]{SWEBOK2024} or just ``ethical hacking''
    \citep[p.~28]{Gerrard2000b}; while hacking in general is not a test
    approach, doing so systematically to test and improve the software is.

    \subsubsection{Language-specific Approaches}
    \label{lang-test}
    Specific programming languages are sometimes used to define test approaches.
    If the reliance on a specific programming language is intentional, then
    this really implies an underlying test approach that may be generalized to
    other languages. These are therefore considered out-of-scope\thesisissueref{63},
    including the following examples:

    \begin{itemize}
        \item ``They implemented an approach \dots{} for JavaScript testing
              (referred to as Randomized)'' \citep[p.~192]{DoğanEtAl2014};
              this really refers to random testing used within JavaScript
        \item ``SQL statement coverage'' is really just statement coverage
              used specifically for SQL statements \citep[Tab.~13]{DoğanEtAl2014}
              \todo{OG Alalfi et al., 2010}
        \item ``Faults specific to PHP'' is just a subcategory of fault-based
              testing, since ``execution failures \dots{} caused by missing an
              included file, wrong MySQL quer[ies] and uncaught exceptions''
              are not exclusive to PHP \citep[Tab.~27]{DoğanEtAl2014}
              \todo{OG Artzi et al., 2008}
        \item While ``HTML testing'' is listed or implied by
              \citeauthor{Gerrard2000a} (\citeyear[Tab.~2]{Gerrard2000a};
              \citeyear[Tab.~1, p.~3]{Gerrard2000b}) and
              \citet[p.~220]{Patton2006}, it seems to be a combination of syntax
              testing, functionality testing, hyperlink testing/link checking,
              cross-browser compatibility testing, performance testing, and
              content checking \citep[p.~3]{Gerrard2000b}
    \end{itemize}

    \subsubsection{Orthogonally Derived Approaches}
    \label{orth-test}
    Some test approaches appear to be combinations of other (seemingly
    orthogonal) approaches. While the use of a combination term can sometimes
    make sense, such as when writing a paper or performing testing that focuses
    on the intersection between two test approaches, they are sometimes given
    the same ``weight'' as their atomic counterparts. For example, \citetISTQB{}
    \multiAuthHelper{include} ``formal reviews'' and ``informal reviews'' in
    \ifnotpaper their \else its \fi glossary as separate terms, despite their
    definitions essentially boiling down to ``reviews that follow (or do not
    follow) a formal process'', which do not provide any new information.
    Regardless, these are considered out of scope since their details are
    captured as the in-scope test approaches that comprise them. For the
    following examples, we indicate the combination for the first item but
    omit the rest for brevity as the name makes it clear what the combination is:
    \begin{enumerate}
        \item Black box conformance testing \citep[p.~25]{JardEtAl1999}
              (combining black box and conformance testing)
        \item Black-box integration testing \citep[p.~345-346]{SakamotoEtAl2013}
        \item Checklist-based reviews \citepISTQB{}
        \item Closed-loop HiL verification \citep[p.~6]{PreußeEtAl2012}
        \item Closed-loop protection system testing \citep[p.~331]{ForsythEtAl2004}
        \item Endurance stability testing \citep[p.~55]{Firesmith2015}
        \item End-to-end functionality testing (\citealp[p.~20]{IEEE2021};
              \citealp[Tab.~2]{Gerrard2000a})
        \item Formal reviews \citepISTQB{}
        \item Gray-box integration testing \citep[p.~344]{SakamotoEtAl2013}
        \item Incremental integration testing \citep[p.~601, 603, 605-606]{SharmaEtAl2021}
              \todo{OG [19]}
        \item Informal reviews \citepISTQB{}
        \item Infrastructure compatibility testing \citep[p.~53]{Firesmith2015}
        \item Invariant-based automatic testing
              \citep[pp.~184-185,~Tab.~21]{DoğanEtAl2014}, including for
              ``AJAX user interfaces'' \citetext{p.~191}
        \item Legacy system integration (testing) \citep[Tab.~2]{Gerrard2000a}
        \item Manual procedure testing \citep[p.~47]{Firesmith2015}
        \item Manual security audits \citep[p.~28]{Gerrard2000b}
        \item Model-based GUI testing (\citealp[Tab.~1]{DoğanEtAl2014}; implied
              by \citealp[p.~356]{SakamotoEtAl2013})
        \item Model-based web application testing (implied by
              \citealp[p.~356]{SakamotoEtAl2013})
        \item Non-functional search-based testing \citep[Tab.~1]{DoğanEtAl2014}
        \item Offline MBT \citepISTQB{}
        \item Online MBT \citepISTQB{}
        \item Role-based reviews \citepISTQB{}
        \item Scenario walkthroughs \citep[Fig.~4]{Gerrard2000a}
        \item Scenario-based reviews \citepISTQB{}
        \item Security attacks \citepISTQB{}
        \item Statistical web testing \citep[p.~185]{DoğanEtAl2014}
        \item Usability test script(ing) \citepISTQB{}
        \item Web application regression testing \cite[Tab.~21]{DoğanEtAl2014}
        \item White-box unit testing \citep[pp.~345-346]{SakamotoEtAl2013}
    \end{enumerate}

    The existence of orthogonal combinations could allow for other test
    approaches to be extrapolated from them. For example, \citet{Moghadam2019}
    uses the phrase ``machine learning-assisted performance testing''; since
    performance testing is a known test approach, this may imply the existence
    of the test approach ``machine learning-assisted testing''. Likewise,
    \citet{JardEtAl1999} \multiAuthHelper{use} the phrases ``local synchronous
    testing'' and ``remote asynchronous testing''. While these can be
    decomposed, for example, into local testing and synchronous testing,
    the two resulting approaches may not be orthogonal, potentially even having
    a parent-child relation (defined in \Cref{par-chd-rels}).

\fi
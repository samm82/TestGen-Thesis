\section{Scope}
\label{scope}

Since our motivation is restricted to testing of code, only the ``testing''
component of \acf{vnv} is considered\seeThesisIssuePar{22}.
For example, design reviews and documentation reviews
\ifnotpaper \citep[see][pp.~132, 144, respectively]{IEEE2017}
\else (see \cite[pp.~132, 144]{IEEE2017}, respectively) \fi
are out of scope, as they focus on the \acs{vnv} of design and documentation,
respectively. Likewise, ergonomics testing
and proximity-based testing (see \citealpISTQB{}) are out of scope as they
fundamentally involve hardware.
\ifnotpaper
      Similarly removed is \acf{emsec} testing
      (\citealp{ISO2021}; \citealp[p.~95]{ZhouEtAl2012}), which deals with the
      ``security risk'' of ``information leakage via electromagnetic emanation''
      \citep[p.~95]{ZhouEtAl2012}.
\fi Security audits that focus on ``an organization's
\dots\ processes and infrastructure'' \citepISTQB{}, are also out of scope,
but security audits that ``aim to ensure that all of the products installed on
a site are secure when checked against the known vulnerabilities for those
products'' \citep[p.~28]{Gerrard2000b} are not.
\ifnotpaper While \acf{oat}
      can be used when testing software \citep{Mandl1985}, it can also be used for
      hardware \citep[pp.~471-472]{Valcheva2013}, such as ``processors \dots\ made
      from pre-built and pre-tested hardware components'' (p.~471). A subset of
      \acs{oat} called ``\acf{toat}'' is used for ``experimental design problems in
      manufacturing'' \citep[p.~1573]{YuEtAl2011} or ``product and manufacturing
      process design'' \cite[p.~44]{Tsui2007} and is thus out of scope.

      \citeauthor{Patton2006} introduces ``specification testing''
      \citeyearpar[pp.~56-62]{Patton2006}, which is static black-box testing.
      Most of this section is irrelevant to generating test cases, due to
      requiring human involvement and verifying the specification, not the code.
      However, it provides a ``Specification Terminology Checklist''
      \citep[p.~61]{Patton2006} that includes some keywords that, if found, could
      trigger an applicable warning to the user (similar to the idea behind the
      correctness/consistency checks project), indicating that a requirement is
      ambiguous or incomplete \citep[see][p.~1-8]{SWEBOK2024}:

      \begin{itemize}
            \item \textbf{Potentially unrealistic:} always, every, all, none, every,
                  certainly, therefore, clearly, obviously, evidently
            \item \textbf{Potentially vague:} some, sometimes, often, usually,
                  ordinarily, customarily, most, mostly, good, high-quality, fast,
                  quickly, cheap, inexpensive, efficient, small, stable
            \item \textbf{Potentially incomplete:} etc., and so forth, and so on,
                  such as, handled, processed, rejected, skipped, eliminated,
                  if \dots\ then \dots\ (without ``else'' or ``otherwise''),
                  to be determined \citep[p.~408]{vanVliet2000}
      \end{itemize}

      Sometimes, wider decisions must be made on whether a whole category of
      testing is in scope or not. For example, while all the examples of domain-specific
      testing given by \citet[p.~26]{Firesmith2015} are focused on hardware, this
      might not be representative of all types (e.g., \acf{ml} model testing seems
      domain-specific). Conversely, the examples of environmental tolerance testing
      (p.~56) do not seem to apply to software. For example, radiation tolerance
      testing seems to focus on hardware, such as motors \citep{MukhinEtAl2022},
      robots \citep{ZhangEtAl2020}, or ``nanolayered carbide and nitride materials''
      \citep[p.~1]{TunesEtAl2022}. Acceleration tolerance testing seems to focus on
      \accelTolTest{} and acoustic tolerance testing on rats \citep{HolleyEtAl1996},
      which are even less related! Since these all seem to focus on
      environment-specific factors that would not impact the code, this category of
      testing is also out of scope.

      It is also interesting to note that different test approaches seem to be more
      specific to certain domains. For example, the terms ``software qualification
      testing'' and ``system qualification testing'' show up throughout
      \citep{SPICE2022}, which was written for the automotive industry, and the more
      general idea of ``qualification testing'' seems to refer to the process of
      making a hardware component, such as an electronic component
      \citep{AhsanEtAl2020}, gas generator \citep{ParateEtAl2021} or photovoltaic
      device, ``into a reliable and marketable product'' \citep[p.~1]{SuhirEtAl2013}.
\fi

Furthermore, only some aspects of some testing approaches are relevant. This
mainly manifests as a testing approach that applies to both the \acs{vnv} itself
and to the code. For example:

\begin{enumerate}
      \item \emph{Error seeding} is the ``process of intentionally adding
            known faults to those already in a computer program'',
            done to both ``monitor[] the rate of detection and removal'',
            which is a part of \acs{vnv} of the \acs{vnv} itself (out of scope),
            ``and estimat[e] the number of faults remaining''
            \citep[p.~165]{IEEE2017}, which helps verify the actual code (in scope).
      \item \emph{Fault injection testing}, where ``faults are artificially
            introduced into the \acs{sut} [System Under Test]'', can be used to
            evaluate the effectiveness of a test suite \citep[p.~5-18]{SWEBOK2024},
            which is a part of \acs{vnv} of the \acs{vnv} itself (out of scope),
            or ``to test
            the robustness of the system in the event of internal and
            external failures'' \citep[p.~42]{IEEE2022}, which helps verify
            the actual code (in scope).
      \item ``\emph{Mutation [t]esting} was originally conceived as a
            technique to evaluate test suites in which a mutant is a slightly
            modified version of the \acs{sut}'' \citep[p.~5-15]{SWEBOK2024},
            which is in the realm of \acs{vnv} of the \acs{vnv} itself (out of
            scope). However, it ``can also be categorized as a structure-based
            technique'' and can be used to assist fuzz and metamorphic testing
            \citep[p.~5-15]{SWEBOK2024} (in scope).
            \ifnotpaper
      \item Even though \emph{reliability testing} and \emph{maintainability
                  testing} can start \emph{without} code by ``measur[ing]
            structural attributes of representations of the software''
            \citep[p.~18]{FentonAndPfleeger1997}, only reliability and
            maintainability testing done \emph{on} code is in scope.
      \item Since control systems often have a software \emph{and} hardware
            component \citep{ISO2015, PreußeEtAl2012,ForsythEtAl2004},
            only the software component is in scope. In some cases, it is
            unclear whether the ``loops''\footnote{Humorously, the testing of
                  loops in chemical systems \citep{Dominguez-PumarEtAl2020} and
                  copper loops \citep{Goralski1999} are out of scope.} being
            tested are implemented by software or hardware, such as those in
            wide-area damping controllers \citep{PierreEtAl2017, TrudnowskiEtAl2017}.
            \begin{itemize}
                  \item A related note: ``path coverage'' or ``path testing''
                        seems to be able to refer to either paths through code
                        (as a subset of control-flow testing)
                        \citep[p.~5-13]{SWEBOK2024} or through a model, such as
                        a finite-state machine (as a subset of model-based
                        testing) \citep[p.~184]{DoğanEtAl2014}.
            \end{itemize}
            \fi
\end{enumerate}

\ifnotpaper
      Specific programming languages are sometimes used to define ``kinds'' of
      testing. These will not be included\seeThesisIssuePar{63}; if the reliance
      on a specific programming language is intentional, then this really implies an
      underlying test approach that may be generalized to other languages. Some
      examples:

      \begin{itemize}
            \item ``They implemented an approach \dots\ for JavaScript testing
                  (referred to as Randomized)'' \citep[p.~192]{DoğanEtAl2014} -
                  this really refers to random testing used within JavaScript
            \item ``SQL statement coverage'' is really just statement coverage
                  used specifically for SQL statements \citep[Tab.~13]{DoğanEtAl2014}
                  \todo{OG Alalfi et al., 2010}
            \item ``Faults specific to PHP'' is just a subcategory of fault-based
                  testing, since ``execution failures \dots\ caused by missing an
                  included file, wrong MySQL quer[ies] and uncaught exceptions''
                  are not exclusive to PHP \citep[Tab.~27]{DoğanEtAl2014}
                  \todo{OG Artzi et al., 2008}
            \item While ``HTML testing'' is listed or implied by
                  \citeauthor{Gerrard2000a} (\citeyear[Tab.~2]{Gerrard2000a};
                  \citeyear[Tab.~1, p.~3]{Gerrard2000b}) and
                  \citet[p.~220]{Patton2006}, it seems to be a combination of syntax
                  testing, functionality testing, hyperlink testing/link checking,
                  cross-browser compatibility testing, performance testing, and
                  content checking \citep[p.~3]{Gerrard2000b}
      \end{itemize}

      \subsection{Static Testing}
      \label{static-test}
      Sometimes, the term ``testing'' excludes static testing
      \ifnotpaper
            (\citealp[p.~222]{AmmannAndOffutt2017}; \citealp[p.~13]{Firesmith2015})%
      \else
            \cite[p.~222]{AmmannAndOffutt2017}, \cite[p.~13]{Firesmith2015}%
      \fi, restricting it to ``dynamic validation'' \citep[p.~5-1]{SWEBOK2024} or
      ``dynamic verification'' ``in which a system or component is
      executed'' \citep[p.~427]{IEEE2017}. Since ``terminology is not uniform
      among different communities, and some use the term \emph{testing} to refer to
      static techniques\notDefDistinctIEEE{technique} as well''
      \citep[p.~5-2]{SWEBOK2024}, the scope of ``testing'' for the purpose of this
      project will include both ``static testing'' and ``dynamic testing'', as
      done by \citet[p.~17]{IEEE2022}, \citet[pp.~8-9]{Gerrard2000a}, and even a
      source that explicitly excluded static testing \citep[p.~440]{IEEE2017}!

      Static testing seems to be somewhat more ad hoc which makes it
      less relevant for own goals (automatic generation of tests). In particular,
      some techniques generate false positives which require human intervention.
      Nevertheless, understanding the breadth of testing approaches provides a more
      complete picture of how software can be tested, how the various approaches are
      related to one another, and potentially how even parts of these ``out-of-scope''
      approaches may be generated in the future! For the current purpose of this
      work, we keep these in-scope at this stage of the analysis.
\fi
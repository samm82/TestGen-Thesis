\section{Scope}
\label{scope}

Since our motivation is restricted to testing code, only this component of
\acf{vnv} is considered\thesisissueref{22}. However, some test approaches
are used for testing things other than code, and some approaches can be used
for both! In these cases, only the subsections of these approaches focused on
code are considered. For example, reliability testing and maintainability
testing can start \emph{without} code by ``measur[ing] structural attributes
of representations of the software'' \citep[p.~18]{FentonAndPfleeger1997}, but
only reliability and maintainability testing performed on code \emph{itself} is
in scope of this research.
\ifnotpaper
    In this section, we provide a high-level overview of what is in scope. We
    exclude some practices either in full, such as hardware testing
    (\Cref{hard-test}), or in part, such as parts of error seeding, fault
    injection testing, and mutation testing that do not directly test code
    (\Cref{other-vnv}).
    % More specific refs are used for now, since the following label points to
    %   Section 9.4 instead of Appendix A for some reason
    % (see \Cref{app-scope} for more detailed discussion on what we exclude).
    Additionally, static testing is a useful component of software testing and
    is therefore included at this level of analysis, despite it not being
    relevant to our original motivation (\Cref{static-test}). Finally, some
    test approaches can be derived from other categories of testing-related
    terminology (\Cref{derived-tests}); of these, approaches derived from
    programming languages (\Cref{lang-test}) or other orthogonal test
    approaches (\Cref{orth-test}) are out of scope.
\fi
\subsection{Flaw Analysis}
\label{flaw-analysis}

In addition to analyzing specific flaws, an overview of their amounts,
sources, rigidities (see \Cref{rigidity}), classes, and categories is
also useful. Subsets of this task can be automated (\Cref{auto-flaw-analysis})
and the remaining manual portion can be augmented with automated
tools (\Cref{aug-flaw-analysis}).

To understand where flaws exist in the literature, they are
grouped based on the source tiers (as described in \Cref{sources})
responsible for them. Each flaw is then counted \emph{once} per source
category if it appears within it \emph{and/or} between it and a more
``trusted'' category. This avoids counting the same flaw more than once
for a given category\thesisissueref{83}, which would result in the number of
\emph{occurrences} of all flaws, instead of the number of flaws
\emph{themselves}, which is more useful. The exception to this is
\Cref{fig:discrepSources}, which counts the following sources
of flaws separately:
\begin{enumerate}
    \item those that appear once in (or consistently throughout) a document
          (i.e., are ``self-contained'')\thesisissueref{137,138},
    \item those between two parts of a single document
          (i.e., internal conflicts)\thesisissueref{137,138},
    \item those between documents by the same author(s) or standards
          organization(s), and
    \item those within a source tier.
\end{enumerate}
As before, these are not double counted, meaning that the maximum number of
counted flaws possible within a \emph{single} source tier in
\Cref{fig:discrepSources} is four (one for each type). This only occurs if
there is an example of each flaw source that is \emph{not} ignored to
avoid double counting; for example, while a single flaw within a single
document would technically fulfill all four criteria, it would only be counted
once. Note that while the different versions of the \acfp{swebok} have
different editors \citep{SWEBOK2024,SWEBOK2014}, we consider them to be written
by the same organization: the IEEE Computer Society (\citealp{AboutSWEBOK}; see
\Cref{metas}).

\phantomsection{}
\label{flaw-analysis-example}
As an example of this process, consider a flaw \emph{within} an IEEE
document (e.g., two different definitions are given for a term within the same
IEEE document) \emph{and} between another IEEE document, the \acs{istqb}
glossary \emph{and} two papers. This would add one to the following rows of
\Cref{tab:sntxDiscreps,tab:smntcDiscreps} in the relevant column:

\begin{itemize}
    \item \textbf{\stds{}}: this flaw occurs:
          \begin{enumerate}
              \item within one standard and
              \item between two standards.
          \end{enumerate}
          This increments the count by just one to avoid double counting and
          would do so even if only one of the above conditions was true. A more
          nuanced breakdown of flaws that identifies those within a
          singular document and those between documents by the same author is
          given in \Cref{fig:discrepSources} and explained in more detail in
          \Cref{aug-flaw-analysis}.
    \item \textbf{\metas{}}: this flaw occurs between a
          source in this category and a ``more trusted'' one
          (the IEEE standards).
    \item \textbf{\papers{}}: this flaw occurs between a
          source in this category and a ``more trusted'' one. Even though there
          are two sources in this category \emph{and} two ``more trusted''
          categories involved, this increments the count by just one to avoid
          double counting.
\end{itemize}

\subsubsection{Automated Flaw Analysis}
\label{auto-flaw-analysis}

As outlined in \Cref{graph-gen}, some types of flaws can be detected
automatically. While just counting the total number of these types of
flaws is trivial, tracking the source(s) of these flaws is more
involved. Since the appropriate citations for each piece of information is
tracked (see \Cref{tab:exampleGlossary,tab:synExampleGlossary} for examples of
how these citations are formatted in the glossaries), they can be used to find
the offending source tiers. This comes with the added benefit of these
citations being available to be formatted for use with \LaTeX{}'s citation
commands for inclusion in this document.

Comparing the authors and years of each source related to a given flaw
can determine if it manifests within a single document and/or between documents
by the same author(s) when creating \Cref{fig:discrepSources}. Then, the
relevant sources can be sorted into their categories based on their citations
\seeSrcCode{e709acf}{discrepCounter}{39}{56}.
% done by the function in \Cref{lst:getSrcCat}, since each source tier
% outlined in \Cref{sources} is comprised of a small number of authors (with the
% exception of papers and other documents; see \Cref{papers}).
This determines
the appropriate row of \Cref{tab:sntxDiscreps,tab:smntcDiscreps} and the appropriate
graph and slice in \Cref{fig:discrepSources}. These lists of sources can then
be distilled down to sets of categories which are compared against
each other to determine how many times a given flaw manifests between
source tiers. Examples of this process are described in more detail in
\Cref{aug-flaw-analysis}.

\phantomsection{}
\label{auto-flaw-analysis-rigidity}
Alongside this citation information are the keywords relevant for assessing a
piece of information's rigidity (see \Cref{rigidity}). This is useful when
counting flaws, since they can be both explicit and implicit,
but should not be double counted as both\thesisissueref{83}! When counting
flaws in \Cref{tab:sntxDiscreps,tab:smntcDiscreps}, each one is
counted only for its most ``rigid'' manifestation (i.e., it will only increment
a value in the ``Implicit'' column if it is \emph{not} also explicit).

\subsubsection{Augmented Flaw Analysis}
\label{aug-flaw-analysis}
While some subsets of flaws can be deduced automatically from analyzing
the testing approach glossary, most need to be tracked
manually. This is done by adding
comments to the relevant \LaTeX{} files (generated or not) of the form
\begin{displayquote}
    \texttt{\% Discrep count (CAT, CLS): \{A1\} \{A2\} \dots{} | \{B1\}
        %\{B2\}
        \dots{} | \{C1\} %\{C2\}
        \dots}
\end{displayquote}
which can then be parsed to detemine where flaws occur. \texttt{CAT} is
a placeholder for the flaw's category (see \Cref{smntcDiscreps})
identifier and \texttt{CLS} is a placeholder for its class (see
\Cref{sntxDiscreps}) identifier. These designations are ommitted
from the following examples of these comments.

Each group of
sources is separated with a pipe symbol to be compared with the others, so any
number of groups are permitted. We make a distinction between ``self-contained''
flaws and ``internal'' flaws. Self-contained flaws are those that manifest by
comparing a document to a source of ground truth. Sometimes, these do not
require an explicit comparison; for example, \nameref{miss} often fall in this
category, since the lack of information is contained within a single source and
does not need to be cross-checked against a source of ground truth. If only one
group of sources is present in a flaw's comment, such as the first line
below, it is considered to be a self-contained flaw. On the other hand,
internal flaws arise when a document disagrees with itself by containing two
conflicting pieces of information and include many \nameref{contra} and
\nameref{over}. These can even occur on the same page, such as when an acronym
is given to two distinct terms (see \discrepref{cat-acro,hil-acro})! If a
source appears in multiple groups in a flaw's comment, it is considered to
be an internal flaw. The second line is a standard example of this, while the
third is more complex; in this case, source Y agrees with only one of the
conflicting sources of information in X.
\begin{displayquote}
    \texttt{\% Discrep count: \{X\}\\\% Discrep count: \{X\} | \{X\}\\
        \% Discrep count: \{X\} | \{X\} \{Y\}}
\end{displayquote}
Discrepancies between groups are not double counted; this means the following
line adds discrepancies between X and Z \emph{and} between Y and Z, without
counting the discrepancy between X and Z twice.
\begin{displayquote}
    \texttt{\% Discrep count: \{X\} | \{X\} \{Y\} | \{Z\}}
\end{displayquote}

Each source is given using its BibTeX key wrapped in curly braces to mimic
\LaTeX{}'s citation commands for ease of parsing, with the exception of the
\acs{istqb} glossary, due to its use of custom commands via
\texttt{\textbackslash citealias}. For example, the line
\begin{displayquote}
    \texttt{\% Discrep count: \{IEEE2022\} | \{IEEE2022\} \{IEEE2017\}\\
        \displayNL ISTQB \{Kam2008\} \{Bas2024\}}
\end{displayquote}
would be parsed as the example given in \Cref{flaw-analysis-example}. Since
the IEEE documents are written by the same standards organizations
(\begin{NoHyper}\citeauthor{IEEE2022}\end{NoHyper}), they are counted as a
discrepancy between documents by the same author(s) in \Cref{fig:discrepSources}.

The rigidity (see \Cref{rigidity}) of flaws can also be manually
specified by inserting the phrase ``implied by'' after the sources of explicit
information and before those of implicit information. Parsing this information
follows the same rules as the automatic flaw analysis
(see \Cref{auto-flaw-analysis-rigidity}).
\begin{displayquote}
    \texttt{\% Discrep count: \{IEEE2022\} implied by \{Kam2008\} |\\
        \displayNL \{IEEE2017\} implied by \{IEEE2022\}}
\end{displayquote}
For example, the above line indicates that the flaws given below are
present. The second flaw only affects \Cref{fig:discrepSources} since it
is less ``rigid'' than the first flaw within standards (see \Cref{stds}).
The rest increment their corresponding count in
\Cref{fig:discrepSources,tab:sntxDiscreps,tab:smntcDiscreps} by only one:
\begin{itemize}
    \item an explicit discrepancy between documents by
          \begin{NoHyper}\citeauthor{IEEE2022}\end{NoHyper},
    \item an implicit discrepancy within a single document, and
    \item an implicit discrepancy between a paper and a standard.
\end{itemize}

Occasionally, a source from a lower tier is used as the ``ground truth'' for a
flaw. For example, \tolTestingDiscrep*{} This flaw is supported
by additional papers found via a miniature literature review (described in
\Cref{undef-terms}) from a lower source tier than \citep{Firesmith2015}
(which is a terminology collection; see \Cref{sources}). However, it
is really based in \citep{Firesmith2015} and not in these
additional papers, but if these sources where included as detailed above,
this is the way it would be counted. Therefore, we document these ``ground
truth'' sources separately to track them for traceability without incorrectly
counting flaws. This is done for this specific example (and similarly
for other cases) as follows:
\begin{displayquote}
    \texttt{\% Discrep count (TERMS, WRONG): \{Firesmith2015\}\\
        \% Ground truth: \{LiuEtAl2023\} \{MorgunEtAl1999\} \{HolleyEtAl1996\}
        \displayNL \{HoweAndJohnson1995\}}
\end{displayquote}

\subsection[LaTeX Commands]{\LaTeX{} Commands}\label{macros}
To improve maintainability, traceability, and reproducibility, we define
helper commands (also called ``macros'') for content that is prone to change
or used in multiple places. For example, many values are calculated by a Python
script and saved to a file. We then assign these values to a corresponding
\LaTeX{} macro, which we can use instead of manually replacing the value
throughout our documents every time it changes. \Cref{tab:macrosCalc} lists
these macros, along with descriptions of what they define and their current
values.

\begin{longtblr}[
    note{a} = {Calculated in \LaTeX{} from source tier lists; see \Cref{text-macros}.},
    note{b} = {Alias for \texttt{\textbackslash totalSmntcFlawBrkdwn\{13\}}; see \Cref{flawCounts}.},
    note{c} = {These macros are defined as counters to allow them to be used in
            calculations within \LaTeX{} (such as in \Cref{undef-terms,fig:undefPies}).},
    caption = {\LaTeX{} macros for calculated values.},
    label = {tab:macrosCalc}
    ]{
    colspec={|X[0.3,l,m]X[0.5,c,m]X[0.2,c,m]|},
    width = \linewidth, rowhead = 1
    }
    \hline
    \thead{Macro}                                   & \thead{What it Counts}        & \thead{Value}    \\
    \hline
    \macro{approachCount}                           & Identified test approaches    & \approachCount{} \\
    \macro{qualityCount}                            & Identified software qualities & \qualityCount{}  \\
    \macro{srcCount}\TblrNote{a}                    & Sources used in glossaries    & \srcCount{}      \\
    \macro{flawCount}\TblrNote{b}                   & Identified flaws              & \flawCount{}     \\
    \hline
    \macro{TotalBefore}\TblrNote{c}                 & Test approaches identified
    before process in \Cref{undef-terms}            & \the\TotalBefore{}                               \\
    \macro{UndefBefore}\TblrNote{c}                 & Undefined test approaches
    identified before process in \Cref{undef-terms} & \the\UndefBefore{}                               \\
    \macro{TotalAfter}\TblrNote{c}                  & Test approaches identified
    after process in \Cref{undef-terms}             & \the\TotalAfter{}                                \\
    \macro{UndefAfter}\TblrNote{c}                  & Undefined test approaches
    identified after process in \Cref{undef-terms}  & \the\UndefAfter{}                                \\
    \hline
    \macro{parSynCount}                             & Pairs of test approaches
    with a child-parent \emph{and} synonym relation & \parSynCount{}                                   \\
    \macro{selfCycleCount}                          & Test approaches that are
    a parent of themselves                          & \selfCycleCount{}                                \\
    \hline
\end{longtblr}


Additionally, flaws are counted based on their rigidity, source tier,
and whether they are syntactic or semantic (see
\Cref{rigidity,sources,auto-flaw-analysis,aug-flaw-analysis,discreps}).
These counts are saved to files, a syntactic and semantic version for each
source tier; for example, syntactic flaws in standards are saved to
\texttt{build/stdSntxDiscBrkdwn.tex} and semantic flaws in standards to
\texttt{build/stdSmntcDiscBrkdwn.tex}. These data are then read in to macros
(such as \macro{stdSmntcDiscBrkdwn}) that are then used to populate
\Cref{tab:sntxDiscreps,tab:smntcDiscreps}. For example,
\macro[1]{stdSntxDiscBrkdwn} and \macro[2]{stdSntxDiscBrkdwn} correspond to
explicit and implicit mistakes in standards documents, respectively.

\phantomsection{}\label{text-macros}
Just as with calculated values, it is important that repeated text is updated
consistently, which we accomplish by defining more macros. Some of these are
generated by Python scripts in a similar fashion to calculated values, such as
the lists of sources in \Cref{sources}. These are built by extracting all
sources cited in our three glossaries, categorizing, sorting, and formatting
them (including handling edge cases), and saving them to a file. These are then
defined as \macro{stdSources}, \macro{metaSources}, \macro{textSources}, and
\macro{paperSources}.\todo{Should I explicitly display these lists in a table
    here? That feels redundant and might make things unnecessarily cluttered.}
The numbers of sources in each tier are also saved to build \Cref{fig:sourceSummary}
and calcuate \macro{srcCount} (see \Cref{tab:macrosCalc}). However, most of the
macros for reused text are created manually when the reuse is first noticed.
Some of these macros account for context-specific formatting, such as
capitalization, depending on how they are used; we omit these details here for
brevity. We create macros to reuse many types of information throughout our
documents as shown in \Cref{tab:macrosText}.

\begin{longtblr}[
        caption = {\LaTeX{} macros for reused text.},
        label = {tab:macrosText}
    ]{
        colspec={|ll|},
        width = \linewidth, rowhead = 1
    }
    \hline
    \thead{Macro}             & \thead{Used in}                                           \\
    \hline
    \macro{bugPattonDiscrep}  & \Cref{intro} and \discrepref{bug-patton}                  \\
    \macro{alphaDiscrep}      & \Cref{intro} and \discrepref{alpha-def}                   \\
    \macro{loadDiscrep}       & \Cref{intro} and \discrepref{load-def}                    \\
    \macro{expBasedCatMain}   & \Cref{intro,multiCats}                                    \\
    \macro{tourDiscrep}       & \Cref{intro,discreps} and \discrepref{tour-def}           \\
    \macro{perfAsFamily}      & \Cref{method-family,classFamilyDiscrep}                   \\
    \macro{tolTestingDiscrep} & \Cref{aug-discrep-analysis} and \discrepref{ground-truth} \\
    \hline
\end{longtblr}


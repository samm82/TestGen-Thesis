\section{Methods}
\label{methods}

To better understand our findings, we built tools to visualize the
relations between test approaches more intuitively (\Cref{graph-gen}) and track
discrepancies surrounding them automatically (\Cref{discrep-count}).

\subsection{Approach Graph Generation}
\label{graph-gen}

\graphGenDesc{}

\subsection{Discrepancy Counting}
\label{discrep-count}

In addition to analyzing specific discrepancies (or classes of discrepancies),
an overview of the amount, severity, and source of these discrepancies is also
useful. Subsets of this task can be automated, and even parts that need to be
done manually (such as finding and categorizing the discrepancies from
\Cref{other-discrep}) can be augmented with automated tools.

As outlined in \Cref{graph-gen}, some types of discrepancies can be detected
automatically. While just counting the total number of these types of
discrepancies is trivial, tracking the source(s) of these discrepancies is more
difficult. Since the appropriate citations for each piece of information is
tracked (see \Cref{tab:exampleGlossary,tab:synExampleGlossary} for examples of
how these citations are formatted
in the glossaries), they can be taken into consideration when performing
these counts. This is how sources are provided in the lists of discrepancies in
\Cref{syns,par-rels}, including \refDiscrepsReqsTable{} (after being formatted
to use \LaTeX{}'s citation commands).

\Cref{tab:discreps} also makes a distinction between explicit and implicit%
\todo{Elaborate on this distinction!}
discrepancies to provide a fuller picture with some degree of nuance.
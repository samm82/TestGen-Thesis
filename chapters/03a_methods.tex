\section{Methods}
\label{methods}

To better understand our findings, we built tools to visualize the
relations between test approaches more intuitively (\Cref{graph-gen}) and track
discrepancies surrounding them automatically (\Cref{discrep-count}).

\subsection{Approach Graph Generation}
\label{graph-gen}

\graphGenDesc{}

\subsection{Discrepancy Counting}
\label{discrep-count}

In addition to analyzing specific discrepancies (or classes of discrepancies),
an overview of the amount, severity, and source of these discrepancies is also
useful. Subsets of this task can be automated, and even parts that need to be
done manually (such as finding and categorizing the discrepancies from
\Cref{other-discrep}) can be augmented with automated tools.

As outlined in \Cref{graph-gen}, some types of discrepancies can be detected
automatically. While just counting the total number of these types of
discrepancies is trivial, tracking the source(s) of these discrepancies is more
difficult. Since the appropriate citations for each piece of information is
tracked (see \Cref{tab:exampleGlossary,tab:synExampleGlossary} for examples of
how these citations are formatted
in the glossaries), they can be taken into consideration when performing
these counts. This is how sources are provided in the lists of discrepancies in
\Cref{syns,par-rels}, including \refDiscrepsTable{}, after being formatted
to use \LaTeX{}'s citation commands. These discrepancies are then categorized
based on the source categories involved (see \Cref{sources}) to identify how
many of a given discrepancy type a source category is responsible for. These
are then counted \emph{once} if this discrepancy exists in a source category at
least as ``trusted'' as the current source category under consideration, which
avoids counting the same discrepancy twice for a given category (see
\thesisissueref{83}). Not doing this would result in the number of
\emph{occurrences} of all discrepancies, instead of the number of discrepancies
\emph{themselves}, which is more useful.

As an example of this process, consider a discrepancy \emph{within} an IEEE
document (e.g., two different definitions are given for a term within the same
IEEE document) \emph{and} between the \acs{swebok} \emph{and} two papers%
\todo{Is this clear?}. This would add one to the following rows of
\refDiscrepsTable{} in the relevant column:

\begin{itemize}
    \item \textbf{\stdDiscBrkdwn{1}}, since this discrepancy occurs between
          sources within this category, even though these ``sources'' are the
          same document\footnote{A more nuanced breakdown that identifies
              discrepancies within a singular document is given in
              \Cref{fig:discrepSources}.},
    \item \textbf{\metaDiscBrkdwn{1}}, since this discrepancy occurs between a
          source in this category and a ``more trusted'' one
          (the IEEE standard), and
    \item \textbf{\otherDiscBrkdwn{1}}, since this discrepancy occurs between a
          source in this category and a ``more trusted'' one; even though there
          are two sources in this category \emph{and} two ``more trusted''
          categories involved, this discrepancy is only counted once to avoid
          any double counting.
\end{itemize}

\Cref{tab:discreps} also makes a distinction between explicit and implicit%
\todo{Elaborate on this distinction!}
discrepancies to provide a fuller picture with some degree of nuance.
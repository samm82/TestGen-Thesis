\section{Discrepancies and Ambiguities}
\label{discrep}

After gathering all this data, we found many discrepancies and
ambiguities. We first report on the more major issues, classified under
the rubrics of Synonyms, Parent Relations, Functional Testing, Operational
(Acceptance) Testing, Recovery Testing, and Scalability Testing. We then
close with some \nameref{minor-discrep}.

\subsection{Synonyms}
\label{syns}
The same approach often has many names. For example,
\emph{specification-based testing} is also called\todo{more in Umar2000}:
\begin{enumerate}
      \item Black-Box Testing
            \ifnotpaper
                  (\citealp[p.~9]{IEEE2022}; \citeyear[p.~8]{IEEE2021};
                  \citeyear[p.~431]{IEEE2017}; \citealp[p.~5-10]{SWEBOK2024};
                  \citealpISTQB{}; \citealp[p.~46]{Firesmith2015} (without hyphen);
                  \citealp[p.~344]{SakamotoEtAl2013}; \citealp[p.~399]{vanVliet2000})
            \else
                  \cite[p.~9]{IEEE2022}, \cite[p.~8]{IEEE2021},
                  \citep[p.~431]{IEEE2017}, \cite[p.~5-10]{SWEBOK2024},
                  \citepISTQB{}, \cite[p.~46]{Firesmith2015} (without hyphen),
                  \cite[p.~344]{SakamotoEtAl2013}, \cite[p.~399]{vanVliet2000}
            \fi
      \item Closed-Box Testing
            \ifnotpaper
                  (\citealp[p.~9]{IEEE2022}; \citeyear[p.~431]{IEEE2017})
            \else
                  \cite[p.~9]{IEEE2022}, \cite[p.~431]{IEEE2017}
            \fi
      \item Functional Testing
            \ifnotpaper
                  (\citeyear[p.~196]{IEEE2017}; \citealp[p.~44]{Kam2008};
                  \citealp[p.~399]{vanVliet2000}; implied by \citeyear[p.~129]{IEEE2021};
                  \citeyear[p.~431]{IEEE2017})
            \else
                  \cite[p.~196]{IEEE2017}, \cite[p.~44]{Kam2008},
                  \cite[p.~399]{vanVliet2000} (implied by \citep[p.~129]{IEEE2021},
                  \cite[p.~431]{IEEE2017})
            \fi
      \item Domain Testing \citep[p.~5-10]{SWEBOK2024}
      \item Input Domain-Based Testing (implied by \citep[p.~4-8]{SWEBOK2014})
\end{enumerate}

While some of these synonyms may express mild variations, their core meaning
is nevertheless the same. Here we use the terms ``specification-based'' and
``structure-based testing'' as they articulate the source of the information
for designing test cases, but a team or project also using gray-box testing may
prefer the terms ``black-box'' and ``white-box testing'' for consistency.
Thus, synonyms do not inherently signify a discrepancy.

However, there are cases in which a term is given a synonym to two (or more)
disjoint, unrelated terms, which would be a source of ambiguity to teams using
these terms. The following are \ifnotpaper \else four out of ten \fi
examples that have arisen\ifnotpaper\ (synonyms in \emph{italics} have at least
      one of their synonyms implied)\fi:

\begin{enumerate}
      \ifnotpaper \item \textbf{Invalid Testing:}
\begin{itemize}
    \item Error Tolerance Testing \citep[p.~45]{Kam2008}
    \item Negative Testing \ifnotpaper
              (\citealpISTQB{}; implied by \citealp[p.~10]{IEEE2021}) \else
              \citep{ISTQB} (implied by \citep[p.~10]{IEEE2021}) \fi
\end{itemize}
\item \textbf{Soak Testing:}
\begin{itemize}
    \item Endurance Testing \citep[p.~39]{IEEE2021}
    \item Reliability Testing\ifnotpaper\
              (\citealp[Tab.~2]{Gerrard2000a}; \citeyear[Tab.~1,~p.~26]{Gerrard2000b})
          \else\footnote{Endurance testing is given as a kind of reliability
                  testing by \citet[p.~55]{Firesmith2015}, although the terms
                  are not synonyms.} \citep[Tab.~1,~p.~26]{Gerrard2000b},
              \citep[Tab.~2]{Gerrard2000a}\fi
\end{itemize}
\item \textbf{User Scenario Testing:}
\begin{itemize}
    \item Scenario Testing \citepISTQB{}
    \item Use Case Testing\ifnotpaper\ \else\footnote{``Scenario testing'' and
                  ``use case testing'' are given as synonyms by \citepISTQB{}
                  and \citep[pp.~47-49]{Kam2008}
                  but listed separately by \citep[p.~22]{IEEE2022}, \ifnotpaper who
                      also give \else which also gives \fi ``use case testing'' as a
                  ``common form of scenario testing'' \citep[p.~20]{IEEE2021}.
                  This implies that ``use case testing'' may instead be a child of
                  ``user scenario testing'' (see \Cref{tab:parSyns}).}\fi
          \citep[p.~48]{Kam2008} (although ``an actor can be a user or another
          system'' \citep[p.~20]{IEEE2021})
\end{itemize}
\item \textbf{Link Testing:}
\begin{itemize}
    \item Branch Testing (implied by \citealp[p.~24]{IEEE2021})
    \item Component Integration Testing \citep[p.~45]{Kam2008}
    \item Integration Testing (implied by \citealp[p.~13]{Gerrard2000a})
\end{itemize} \else \item \textbf{Invalid Testing:}
\begin{itemize}
    \item Error Tolerance Testing \citep[p.~45]{Kam2008}
    \item Negative Testing \ifnotpaper
              (\citealpISTQB{}; implied by \citealp[p.~10]{IEEE2021}) \else
              \citep{ISTQB} (implied by \citep[p.~10]{IEEE2021}) \fi
\end{itemize}
\item \textbf{Soak Testing:}
\begin{itemize}
    \item Endurance Testing \citep[p.~39]{IEEE2021}
    \item Reliability Testing\ifnotpaper\
              (\citealp[Tab.~2]{Gerrard2000a}; \citeyear[Tab.~1,~p.~26]{Gerrard2000b})
          \else\footnote{Endurance testing is given as a kind of reliability
                  testing by \citet[p.~55]{Firesmith2015}, although the terms
                  are not synonyms.} \citep[Tab.~1,~p.~26]{Gerrard2000b},
              \citep[Tab.~2]{Gerrard2000a}\fi
\end{itemize}
\item \textbf{User Scenario Testing:}
\begin{itemize}
    \item Scenario Testing \citepISTQB{}
    \item Use Case Testing\ifnotpaper\ \else\footnote{``Scenario testing'' and
                  ``use case testing'' are given as synonyms by \citepISTQB{}
                  and \citep[pp.~47-49]{Kam2008}
                  but listed separately by \citep[p.~22]{IEEE2022}, \ifnotpaper who
                      also give \else which also gives \fi ``use case testing'' as a
                  ``common form of scenario testing'' \citep[p.~20]{IEEE2021}.
                  This implies that ``use case testing'' may instead be a child of
                  ``user scenario testing'' (see \Cref{tab:parSyns}).}\fi
          \citep[p.~48]{Kam2008} (although ``an actor can be a user or another
          system'' \citep[p.~20]{IEEE2021})
\end{itemize}
\item \textbf{Link Testing:}
\begin{itemize}
    \item Branch Testing (implied by \citealp[p.~24]{IEEE2021})
    \item Component Integration Testing \citep[p.~45]{Kam2008}
    \item Integration Testing (implied by \citealp[p.~13]{Gerrard2000a})
\end{itemize} \fi
\end{enumerate}

\ifnotpaper
      Some interesting notes about these synonyms:
      \begin{itemize}
            \item The terms are not synonyms, although endurance testing is given
                  as a kind of reliability testing by \citet[p.~55]{Firesmith2015}.
            \item ``Scenario testing'' and ``use case testing'' are given as synonyms
                  by \citetISTQB{} and \citet[pp.~47-49]{Kam2008}, but listed
                  separately by \citet[p.~22]{IEEE2022},
                  \ifnotpaper who also give \else which also gives \fi ``use case
                  testing'' as a ``common form of scenario testing''
                  \citeyearpar[p.~20]{IEEE2021}.
            \item \ifnotpaper
                        \citeauthor{ChalinEtAl2006}~list \acf{rac} and \acf{sv} as ``two
                        complementary forms of assertion checking''
                        \citeyearpar[p.~343]{ChalinEtAl2006}%
                  \else
                        \cite[p.~343]{ChalinEtAl2006} lists Runtime Assertion
                        Checking \acf{rac} and Software Verification \acf{sv} as
                        ``two complementary forms of assertion checking''%
                  \fi; based on how the term ``static
                  assertion checking'' is used by \citet[p.~345]{LahiriEtAl2013}, it
                  seems like this should be the complement to \acs{rac} instead.
            \item ``Operational'' and ``production acceptance testing'' are treated
                  as synonyms by \citetISTQB{}, but listed separately by
                  \citet[p.~30]{Firesmith2015}.
            \item ``Production acceptance testing'' \citep[p.~30]{Firesmith2015}
                  seems to be the same as ``production verification testing''
                  \citep[p.~22]{IEEE2022}, but neither are defined.
      \end{itemize}
\fi

There are pairs of synonyms where one is described as a
sub-approach of the other, abusing the meaning of ``synonym'' and
causing confusion. In the next list, lack of source implies the relationship
in inferrable from the definitions.

% I personally thing that this next information would be better conveyed by
% a table, with symbols and color denoting things like synonym / sub and
% other such relationships. Not sure if there is time to do so.
\def\specfn{\footnote{See \Cref{spec-func-test}.}}

\begin{paperTable}
    \centering
    \caption{Pairs of test approaches with both parent-child and synonym relations.}
    \label{tab:parSyns}
    \begin{minipage}{\linewidth}
        \centering
        \begin{tabular}{|rcl|l|l|}
            \hline
            \thead{``Child''}        & \thead{$\to$} & \thead{``Parent''}                       & \thead{Parent-Child Source(s)}                                        & \thead{Synonym Source(s)}                                                   \\
            \hline
            All Transitions Testing  & $\to$         & State Transition Testing                 & \citep[p.~19]{IEEE2021}                                               & \citep[p.~15]{Kam2008}                                                      \\
            Co-existence Testing     & $\to$         & Compatibility Testing                    & \cite[p.~3]{IEEE2022}, \cite{ISO_IEC2023a}, \cite[Tab.~A.1]{IEEE2021} & \citep[p.~37]{IEEE2021}                                                     \\
            Fault Tolerance Testing  & $\to$         & Robustness Testing\footnote{\ftrnote{F}} & \citep[p.~56]{Firesmith2015}                                          & \citepISTQB{}                                                               \\
            Functional Testing       & $\to$         & Specification-based Testing\specfn       & \citep[p.~38]{IEEE2021}                                               & \cite[p.~196]{IEEE2017}, \cite[p.~399]{vanVliet2000}, \cite[p.~44]{Kam2008} \\
            Orthogonal Array Testing & $\to$         & Pairwise Testing                         & \citep[p.~1055]{Mandl1985}                                            & \cite[p.~5-11]{SWEBOK2024}, \cite[p.~473]{Valcheva2013}                     \\
            Performance Testing      & $\to$         & Performance-related Testing              & \cite[p.~22]{IEEE2022}, \cite[p.~38]{IEEE2021}                        & \citep[p.~1187]{Moghadam2019}                                               \\
            Use Case Testing         & $\to$         & Scenario Testing                         & \cite[p.~20]{IEEE2021}\todo{OG Hass, 2008}                            & \cite{ISTQB}, \cite[pp.~47-49]{Kam2008}                                     \\
            \hline
        \end{tabular}
    \end{minipage}
\end{paperTable}


Finally, it is worth pointing out that
\begin{itemize}
      \item Fault tolerance testing may also be a
            subtype\notDefDistinctIEEE{type} of reliability testing
            \citetext{\citealp[p.~375]{IEEE2017}; \citealp[p.~7-10]{SWEBOK2024}},
            which is distinct from reliability testing \citep[p.~53]{Firesmith2015}.
      \item The distinction between organization- and role-based testing in
            \citep[pp.~17,~37,~39]{Firesmith2015}
            seems arbitrary, but further investigation may prove it to be
            meaningful\seeThesisIssuePar{59}.
\end{itemize}

\subsection{Parent Relations}
\label{par-rels}

Parent relations are not immune to difficulties, including self-referential
definitions.

\input{build/selfCycles}

\ifnotpaper \citeauthor{Gerrard2000a} does
\else  \cite[Tab.~2]{Gerrard2000a} and \cite[Tab.~1]{Gerrard2000b} do
\fi
\emph{not} describe performance testing as a sub-category of usability testing%
\ifnotpaper%
      \ (\citeyear[Tab.~2]{Gerrard2000a}; \citeyear[Tab.~1]{Gerrard2000b})%
\fi, which would have been more meaningful.

\subsection{Functional Testing}

``Functional testing'' seems to be described in many
ways, alongside other, likely related, terms:

% Rather than using 'itemize', which steals which a lot of space (by
% indenting), use \paragraph{} instead, i.e.
% \paragraph{Specification-based Testing} is defined as ...
\begin{itemize}
      \item \textbf{Specification-based Testing} is defined as ``testing
            in which the principal test basis is the external inputs and
            outputs of the test item'' \citep[p.~9]{IEEE2022}, which agrees
            with a definition of ``functional testing'': ``testing that
            \dots\ focuses solely on the outputs generated in response to
            selected inputs and execution conditions'' \citep[p.~196]{IEEE2017}.
            \todo{\citet[p.~399]{vanVliet2000} may list these as synonyms;
                  investigate}
            Notably, \citet{IEEE2017} lists both as synonyms of
            ``black-box testing'' (pp. 431, 196, respectively). But
            they are sometimes defined as separate terms: ``specification-based
            testing'' as ``testing based on an analysis of the specification
            of the component or system'' (including ``black-box testing'' as a
            synonym) and ``functional testing'' as ``testing performed to
            evaluate if a component or system satisfies functional
            requirements'' (specifying no synonyms) \citepISTQB{}; the latter
            references \citet[p.~196]{IEEE2017}
            (``testing conducted to evaluate the compliance of a system or
            component with specified functional requirements'') which
            \emph{has} ``black-box testing'' as a synonym, and mirrors
            \citet[p.~21]{IEEE2022} (testing ``used to check the implementation
            of functional requirements''). Overall, specification-based testing
            \citep[pp.~2-4,~6-9,~22]{IEEE2022} and black-box testing
            (\citealp[p.~5-10]{SWEBOK2024}; \citealp[p.~3]{SouzaEtAl2017})
            are test design techniques used to ``derive corresponding test cases''
            \citep[p.~11]{IEEE2022} (from given ``selected inputs and execution
            conditions'' \citep[p.~196]{IEEE2017}).

      \item \textbf{Correctness Testing} \citet[p.~5-7]{SWEBOK2024} says
            ``test cases can be designed to check that the functional
            specifications are correctly implemented, which is variously
            referred to in the literature as conformance testing, correctness
            testing or functional testing''; this mirrors previous definitions
            of ``functional testing'' (\citealp[p.~21]{IEEE2022};
            \citeyear[p.~196]{IEEE2017}) but groups it with ``correctness
            testing''. Since ``correctness'' is a software quality
            (\citealp[p.~104]{IEEE2017}; \citealp[p.~3-13]{SWEBOK2024}) which is
            what defines a ``test type'' \citep[p.~15]{IEEE2022}, it seems
            consistent to label ``functional testing'' as a ``test type''
            \citep[pp.~15,~20,~22]{IEEE2022}. This is listed separately from
            ``functionality testing'' by \citet[p.~53]{Firesmith2015}.

      \item \textbf{Conformance Testing}
            \citet[p.~5-7]{SWEBOK2024} insures ``that the functional
            specifications are correctly implemented'', and can be called
            ``conformance testing'' or ``functional testing''.
            ``Conformance testing'' is later defined as used ``to
            verify that the \acs{sut} conforms to standards, rules,
            specifications, requirements, design, processes, or practices''
            \citep[p.~5-7]{SWEBOK2024}. This definition seems to be a superset
            of testing methods mentioned earlier as the latter includes ``standards'',
            ``rules'', ``requirements'', ``design'', ``processes'', and
            ``practices'' as well as ``specifications''!

            A complicating factor is that ``compliance testing'' is also
            (plausibly!) given as a synonym of ``conformance testing''
            \citep[p.~43]{Kam2008}. However, ``conformance
            testing'' can also be defined as testing that evaluates the degree
            to which ``results \dots\ fall within the limits that define
            acceptable variation for a quality requirement''
            \citep[p.~93]{IEEE2017}\todo{OG PMBOK 5th ed.}, which seems to
            describe something different.

            % TODO: pull out into Recommendations
            % Perhaps this second definition of
            % ``conformance testing'' should be used, and the previous definition
            % of ``compliance testing'' should be used for describing compliance with
            % external standards, rules, etc.~to keep them distinct.

      \item \textbf{Functional Suitability Testing:} Procedure testing is
            called a ``type of functional suitability testing''
            \citep[p.~7]{IEEE2022}, but no definition of that term is given.
            ``Functional suitability'' is the
            ``capability of a product to provide functions that meet stated and
            implied needs of intended users when it is used under specified
            conditions'', including meeting ``the functional specification''
            \citep{ISO_IEC2023a}. This seems to align with the definition of
            ``functional testing'' as related to ``black-box/%
            specification-based testing''. ``Functional suitability'' has
            three child terms: ``functional completeness'' (the ``capability of
            a product to provide a set of functions that covers all the
            specified tasks and intended users' objectives''), ``functional
            correctness'' (the ``capability of a product to provide accurate
            results when used by intended users''), and ``functional
            appropriateness'' (the ``capability of a product to provide
            functions that facilitate the accomplishment of specified tasks and
            objectives'') \citep{ISO_IEC2023a}. Notably, ``functional
            correctness'', which includes precision and accuracy
            (\citealp{ISO_IEC2023a}; \citealpISTQB{}), seems to align with
            the quality/ies that would be tested by ``correctness'' testing.

      \item \textbf{Functionality Testing}, ``Functionality'' is defined as the
            ``capabilities of the various \dots\ features provided by a product''
            \citep[p.~196]{IEEE2017} and is said to be a synonym of
            ``functional suitability'' \citepISTQB{}, although it seems
            like it should really be its ``parent''. Its associated test type
            is implied to be a sub-approach of build verification testing
            \citepISTQB{} and made distinct from ``functional testing'';
            interestingly, security is described as a sub-approach of both
            non-functional and functionality testing \citep[Tab.~2]{Gerrard2000a}.
            This is listed separately from ``correctness testing'' by
            \citet[p.~53]{Firesmith2015}.
\end{itemize}

\subsection{Operational (Acceptance) Testing}
Some sources refer to ``operational acceptance testing'' (\citealp[p.~22]{IEEE2022};
\citealpISTQB{}) while some refer to ``operational testing''
(\citealp[p.~6-9,~in the context of software engineering operations]{SWEBOK2024};
\citealp{ISO_IEC2018}; \citealp[p.~303]{IEEE2017};
\citealp[pp.~4-6,~4-9]{SWEBOK2014}). A distinction is sometimes made
\citep[p.~30]{Firesmith2015} but without accompanying definitions, it is hard
to evaluate its merit. Since this terminology is not standardized, I
propose that the two terms are treated as synonyms (as done by other sources
\citep{LambdaTest2024, BocchinoAndHamilton1996}) as a type of
acceptance testing (\citealp[p.~22]{IEEE2022}; \citealpISTQB{}) that focuses on
``non-functional'' attributes of the system \citep{LambdaTest2024}%
\todo{find more academic sources}.
%% Recommendations in the above: should be split out

%% The following 'summary' appears out of place? I'm not quite understanding
% the point this is trying to make.
A summary of definitions of ``operational (acceptance) testing'' is that
it is ``test[ing] to determine the correct
installation, configuration and operation of a module and that it operates
securely in the operational environment'' \citep{ISO_IEC2018} or ``evaluate a
system or component in its operational environment'' \citep[p.~303]{IEEE2017},
particularly ``to determine if operations and/or systems administration staff
can accept [it]'' \citepISTQB{}.

\subsection{Recovery Testing}
\label{recov-discrep}

``Recovery testing'' is ``testing \dots\ aimed at verifying
software restart capabilities after a system crash or other disaster''
\citep[p.~5-9]{SWEBOK2024} including ``recover[ing] the data directly affected
and re-establish[ing] the desired state of the system''
\ifnotpaper
      (\citealp{ISO_IEC2023a}; similar in \citealp[p.~7-10]{SWEBOK2024})
\else
      \cite{ISO_IEC2023a} (similar in \cite[p.~7-10]{SWEBOK2024})
\fi
so that the system ``can perform required functions'' \citep[p.~370]{IEEE2017}.
It is also called ``recoverability testing'' \cite[p.~47]{Kam2008} and
potentially ``restart \& recovery (testing)'' \cite[Fig.~5]{Gerrard2000a}. The
following terms, along with ``recovery testing'' itself \citep[p.~22]{IEEE2022}
are all classified as test types, and the relations between them can be found
in Figure~\ref{fig:recovery-graph-current}.

%% again, maybe convert to \paragraph ?
\begin{itemize}
      \item \textbf{Recoverability Testing:} Testing ``how well a system or
            software can recover data during an interruption or failure''
            \ifnotpaper
                  (\citealp[p.~7-10]{SWEBOK2024}; similar in \citealp{ISO_IEC2023a})
            \else
                  \cite[p.~7-10]{SWEBOK2024} (similar in \cite{ISO_IEC2023a})
            \fi
            and ``re-establish the desired state of the system'' \citep{ISO_IEC2023a}.
            Synonym for ``recovery testing'' in \citet[p.~47]{Kam2008}.
      \item \textbf{Disaster/Recovery Testing} serves to evaluate if a system
            can ``return to normal operation after a hardware
            or software failure'' \citep[p.~140]{IEEE2017} or if ``operation of
            the test item can be transferred to a different operating site and
            \dots\ be transferred back again once the failure has been
            resolved'' \citeyearpar[p.~37]{IEEE2021}. These two definitions seem to
            describe different aspects of the system, where the first is
            intrinsic to the hardware/software and the second might not be.
      \item \textbf{Backup and Recovery Testing} ``measures the
            degree to which system state can be restored from backup within
            specified parameters of time, cost, completeness, and accuracy in
            the event of failure'' \citep[p.~2]{IEEE2013}. This may be what is
            meant by ``recovery testing'' in the context of performance-related
            testing and seems to correspond to the definition of
            ``disaster/recovery testing'' in \citeyearpar[p.~140]{IEEE2017}.
      \item \textbf{Backup/Recovery Testing:} Testing that determines the
            ability ``to restor[e] from back-up memory in the event of failure,
            without transfer[ing] to a different operating site or back-up
            system'' \citep[p.~37]{IEEE2021}. This seems to correspond to the
            definition of ``disaster/recovery testing'' in
            \citeyearpar[p.~37]{IEEE2021}. It is also given as a sub-type of
            ``disaster/recovery testing'', even though that tests if ``operation
            of the test item can be transferred to a different operating site''
            \citetext{p.~37}.
      \item \textbf{Failover/Recovery Testing:} Testing that determines the
            ability ``to mov[e] to a back-up system in the event of failure,
            without transfer[ing] to a different operating site''
            \citep[p.~37]{IEEE2021}. This is given as a sub-type of
            ``disaster/recovery testing'', even though that tests if ``operation
            of the test item can be transferred to a different operating site''
            \citetext{p.~37}.
      \item \textbf{Failover Testing:} Testing that ``validates the SUT's
            ability to manage heavy loads or unexpected failure to continue
            typical operations'' \citep[p.~5-9]{SWEBOK2024} by entering a
            ``backup operational mode in which [these responsibilities] \dots\
            are assumed by a secondary system'' \citepISTQB{}. While not
            \emph{explicitly} related to recovery, ``failover/recovery testing''
            also describes the idea of ``failover'', and \ifnotpaper
                  \citeauthor{Firesmith2015}\else\cite[p.~56]{Firesmith2015}\fi\
            uses the term ``failover and recovery testing''\ifnotpaper
                  \ \citeyearpar[p.~56]{Firesmith2015}\fi, which could be a synonym of
            both of these terms.
\end{itemize}

\subsection{Scalability Testing}
\label{scal-discrep}

There were three ambiguities around the term ``scalability testing''.
The relations between these test approaches (and other relevant ones)
are shown in Figure~\ref{fig:scal-graph-current}.

\begin{enumerate}
      \item \ifnotpaper \citeauthor{IEEE2021} give
            \else Reference \cite[p.~39]{IEEE2021} gives \fi ``scalability
            testing'' as a synonym of ``capacity testing'' while other sources
            differentiate between the two
            \ifnotpaper \citetext{\citealp[p.~53]{Firesmith2015};
                        \citealp[pp.~22-23]{Bas2024}}
            \else \citep[p.~53]{Firesmith2015}, \citep[pp.~22-23]{Bas2024}
            \fi
      \item \ifnotpaper \citeauthor{IEEE2021} include
            \else Reference \cite[p.~39]{IEEE2021} includes \fi the external
            modification
            of the system as part of ``scalability'', while
            \ifnotpaper \citeauthor{ISO_IEC2023a}
            \else \cite{ISO_IEC2023a} \fi implies that it is limited to the
            system itself \ifnotpaper \citeyearpar{ISO_IEC2023a} \fi
      \item SWEBOK V4's definition of ``scalability testing''
            \citep[p.~5-9]{SWEBOK2024} is really a definition of usability
            testing!
\end{enumerate}

% \subsection{Performance Testing}
% \label{perf-test-ambiguity}

% Similarly, ``performance'' and ``performance efficiency'' are both given as
% software qualities by \ifnotpaper\citeauthor{IEEE2017}\else
%       \cite[p.~319]{IEEE2017}\fi, with the latter defined as the ``performance
% relative to the amount of resources used under stated conditions''
% \ifnotpaper\citeyearpar[p.~319]{IEEE2017} \fi or the ``capability of a product
% to perform its functions within specified time and throughput parameters and be
% efficient in the use of resources under specified conditions'' \citep{ISO_IEC2023a}.
% Initially, there didn't seem to be any meaningful distinction between the two,
% although the term ``performance testing'' is defined
% \ifnotpaper\citeyearpar[p.~320]{IEEE2017}\else\citetext{p.~320}\fi\
% and used by \ifnotpaper\citeauthor{IEEE2017}\else\cite{IEEE2017}\fi\ and the term
% ``performance efficiency testing'' is \emph{also} used by
% \ifnotpaper\citeauthor{IEEE2017}\else\cite{IEEE2017}\fi\ (but not defined
% explicitly). \ifnotpaper Further discussion\seeThesisIssuePar{43} brought us to
%       the conclusion \else It can then be concluded \fi that ``performance
% efficiency testing'' is a subset of ``performance testing'', and the
% difference of ``relative to the amount of resources used'' or ``be efficient in
% the use of resources'' between the two is meaningful.

\subsection{Minor Discrepancies}
\label{minor-discrep}

We now outline ``minor'' discrepancies/ambiguities found
in the literature, grouped by the ``categories'' of sources outlined in
\nameref{method}. These discrepancies can then be grouped into degrees of
severity:

\begin{enumerate}
      \item High: Semantic differences between test approaches
      \item Medium: Differences in related information about test approaches
            (such as classifications or supporting information)
      \item Low: Typos, redundant information, or referencing issues
\end{enumerate}

The numbers of these groups of discrepancies are shown in
\refMinorDiscrepsReqsTable{}, where a given row corresponds to the number of
discrepancies either within/between one or more sources within that category
and/or between a source of that category and one of a ``more trusted'' source
(i.e., a source from a category higher up in the table;\seeAlways{source-order}).

\minorDiscrepsTable{}

\subsubsection{In Standards \citep{IEEE2022, IEEE2021, IEEE2017,
            IEEE2013, IEEE2012, ISO_IEC2023a, ISO_IEC2023b, ISO_IEC2018,
            ISO2021, ISO2015}} % TODO: how much did I *really* use these last two?

%% \paragraph?? Also: \subsubsection uses 1. and enumerate does too, and at
% the exact same level of indent, which is confusing
% I would probably not use \subsubsection for "Minor Discrepancies" at all,
% but paragraph at that level and then maybe use enumerate below? In any
% case, you need to experiment so this displays nicely.
\begin{enumerate}
      \item ``Compatibility testing'' is defined as ``testing that measures the
            degree to which a test item can function satisfactorily alongside
            other independent products in a shared environment (co-existence),
            and where necessary, exchanges information with other systems or
            components (interoperability)'' \citep[p.~3]{IEEE2022}. This
            definition is nonatomic as it combines the ideas of ``co-existence''
            and ``interoperability''. The term ``interoperability testing'' is
            not defined, but is used three times \citep[pp.~22,~43]{IEEE2022}
            (although the third usage seems like it should be ``portability
            testing''). This implies that ``co-existence testing'' and
            ``interoperability testing'' should be defined as their own terms,
            which is supported by definitions of ``co-existence'' and
            ``interoperability'' often being separate (\citealpISTQB{};
            \citealp[pp.~73,~237]{IEEE2017}), the definition of
            ``interoperability testing'' from \citet[p.~238]{IEEE2017},
            and the decomposition of ``compatibility'' into ``co-existence''
            and ``interoperability'' by \citet{ISO_IEC2023a}!
            \begin{itemize}
                  \item The ``interoperability'' element of ``compatibility
                        testing'' is explicitly excluded by
                        \citet[p.~37]{IEEE2021}, (incorrectly) implying that
                        ``compatibility testing'' and ``co-existence testing''
                        are synonyms.
                  \item The definition of ``compatibility testing'' in
                        \citep[p.~43]{Kam2008} unhelpfully says ``See
                        \emph{interoperability testing}'', adding another
                        layer of confusion to the direction of their
                        relationship.
            \end{itemize}
      \item ``Fuzz testing'' is ``tagged'' (?) as ``artificial intelligence''
            \citep[p.~5]{IEEE2022}.
      \item Integration, system, and system integration testing are all listed
            as ``common test levels'' (\citealp[p.~12]{IEEE2022};
            \citeyear[p.~6]{IEEE2021}), but no
            definitions are given for the latter two, making it unclear what
            ``system integration testing'' is; it is a combination of the two?
            somewhere on the spectrum between them? It is listed as a child
            of integration testing by \citetISTQB{} and of system testing
            by \citet[p.~23]{Firesmith2015}.
      \item Similarly, component, integration, and component integration
            testing are all listed in \citep{IEEE2017}, but ``component
            integration testing'' is only defined as ``testing of groups of
            related components'' \citep[p.~82]{IEEE2017}; it is a combination of
            the two? somewhere on the spectrum between them? Likewise, it is
            listed as a child of integration testing by \citetISTQB{}.
      \item Retesting and regression testing seem to be separated from the rest
            of the testing approaches \citep[p.~23]{IEEE2022}, but it is not
            clearly detailed why; \citet[p.~3]{BarbosaEtAl2006} consider
            regression testing to be a test level.
      \item A component is an ``entity with discrete structure \dots\ within a
            system considered at a particular level of analysis''
            \citep{ISO_IEC2023b} and ``the terms module, component, and unit
                  [sic] are often used interchangeably or defined to be subelements
            of one another in different ways depending upon the context'' with
            no standardized relationship \citep[p.~82]{IEEE2017}. This means
            unit/component/module testing can refer to the testing of both a
            module and a specific function in a module\seeThesisIssuePar{14}.
            However, ``component'' is sometimes defined differently than
            ``module'': ``components differ from classical modules for being
            re-used in different contexts independently of their development''
            \citep[p.~107]{BaresiAndPezzè2006}, so distinguishing the two
            may be necessary.
      \item A typo in \citep[Fig.~2]{IEEE2021} means that ``specification-based
            techniques'' is listed twice, when the latter should be
            ``structure-based techniques''.
      \item \citeauthor{IEEE2021} define an ``extended entry (decision) table''
            both as a decision table where the ``conditions consist of multiple
            values rather than simple Booleans'' \citeyearpar[p.~18]{IEEE2021}
            and one where ``the conditions and actions are generally described
            but are incomplete'' \citeyearpar[p.~175]{IEEE2017}\todo{OG ISO1984}.
      \item \citeauthor*{IEEE2017} provide a definition for ``inspections and
            audits'' \citeyearpar[p.~228]{IEEE2017}, despite also giving
            definitions for ``inspection'' (p.~227) and ``audit'' (p.~36);
            while the first term \emph{could} be considered a superset of the
            latter two, this distinction doesn't seem useful.
      \item \citeauthor*{IEEE2017} say that ``test level'' and ``test phase''
            are synonyms, both meaning a ``specific instantiation of [a] test
            sub-process'' (\citeyear[pp.~469,~470]{IEEE2017};
            \citeyear[p.~9]{IEEE2013}), but there are also alternative
            definitions for them. \procLevel{\citeyearpar}, while
            ``test phase'' \phaseDef{}
      \item \citeauthor{IEEE2017} use the same definition for ``partial correctness''
            \citeyearpar[p.~314]{IEEE2017} and ``total correctness'' (p.~480).
\end{enumerate}

\ifnotpaper
      Also of note: \citep{IEEE2022} and \citeyearpar{IEEE2021}, from the
      ISO/IEC/IEEE 29119 family of standards, mention the following 23 test
      approaches without defining them. This means that out of the 114 test
      approaches they mention, about 20\% have no associated definition!

      However, the previous version of this standard, \citeyearpar{IEEE2013},
      generally explained two, provided references for two, and explicitly defined
      one of these terms, for a total of five definitions that could (should) have
      been included in \citeyearpar{IEEE2022}! These terms have been
      \underline{underlined}\ifnotpaper%
            , \emph{italicized}, and \textbf{bolded}, respectively%
      \fi. Additionally, entries marked with an asterisk* were defined (at least
      partially) in \citeyearpar{IEEE2017}, which would have been available when
      creating this family of standards. These terms bring the total count of terms
      that could (should) have been defined to nine; almost 40\% of undefined test
      approaches could have been defined!

      \begin{itemize}
            \item \underline{Acceptance Testing*}
            \item Alpha Testing*
            \item Beta Testing*
            \item Capture-Replay Driven Testing
            \item Data-driven Testing
            \item Error-based Testing
            \item Factory Acceptance Testing
            \item Fault Injection Testing
            \item Functional Suitability Testing (also mentioned but not defined in
                  \citep{IEEE2017})
            \item \underline{Integration Testing}*
            \item Model Verification
            \item Operational Acceptance Testing
            \item Orthogonal Array Testing
            \item Production Verification Testing
            \item Recovery Testing* (Failover/Recovery Testing, Back-up/Recovery
                  Testing, \formatPaper{\textbf}{Backup and Recovery Testing*},
                  Recovery*;\seeSectionAlways{recov-discrep})
            \item Response-Time Testing
            \item \formatPaper{\emph}{Reviews} (ISO/IEC 20246) (Code Reviews*)
            \item Scalability Testing (defined as a synonym of ``capacity
                  testing'';\seeSectionAlways{scal-discrep})
            \item Statistical Testing
            \item System Integration Testing (System Integration*)
            \item System Testing* (also mentioned but not defined in \citep{IEEE2013})
            \item \formatPaper{\emph}{Unit Testing*}
                  (IEEE Std 1008-1987, IEEE Standard for
                  Software Unit Testing implicitly listed in the bibliography!)
            \item User Acceptance Testing
      \end{itemize}
\fi

\subsubsection{In ``Meta-Level'' Sources (SWEBOK \citep{SWEBOK2024,
            SWEBOK2014}, ISTQB \citepISTQB{}, or \citep{Firesmith2015})}

%% You may wish to elide some of these for length
\begin{enumerate}[resume]
      \item SWEBOK V4 defines ``privacy testing'' as testing that ``assess[es]
            the security and privacy of users' personal data to prevent local
            attacks'' \citep[p.~5-10]{SWEBOK2024}; this seems to overlap with
            \citeauthor{IEEE2022}'s definition of ``security testing'', which is
            ``conducted to evaluate the degree to which a test item, and
            associated data and information, are protected so that'' only
            ``authorized persons or systems'' can use them as intended
            \citeyearpar[p.~9]{IEEE2022}, both in scope and name.
      \item Various sources say that alpha testing is performed by different
            people, including ``only by users within the organization
            developing the software'' \citep[p.~17]{IEEE2017}, by ``a small,
            selected group of potential users'' \citep[p.~5-8]{SWEBOK2024}, or
            ``in the developer's test environment by roles outside the
            development organization'' \citepISTQB{}.
      \item While correct, ISTQB's definition of ``specification-based testing''
            is not helpful: ``testing based on an analysis of the specification
            of the component or system'' \citepISTQB{}.
      \item ``ML model testing'' and ``ML functional performance'' are defined
            in terms of ``ML functional performance criteria'', which is defined
            in terms of ``ML functional performance metrics'', which is defined
            as ``a set of measures that relate to the functional correctness of
            an ML system'' \citepISTQB{}. The use of ``performance'' (or
            ``correctness'') in these definitions is at best ambiguous and at
            worst incorrect.
      \item While ergonomics testing is out of scope (as it tests hardware, not
            software), its definition of ``testing to determine whether a
            component or system and its input devices are being used properly
            with correct posture'' \citepISTQB{} seems to focus on how the
            system is \emph{used} as opposed to the system \emph{itself}.
      \item The definition of ``math testing'' given by \citetISTQB{} is
            too specific to be useful, likely taken from an example instead of
            a general definition: ``testing to determine the correctness of the
            pay table implementation, the random number generator results, and
            the return to player computations''.
      \item A similar issue exists with multiplayer testing, where its
            definition specifies ``the casino game world'' \citepISTQB{}.
      \item Thirdly, ``par sheet testing'' from \citepISTQB{} seems to
            refer to this specific example and does not seem more widely
            applicable, since a ``PAR sheet'' is ``a list of all the symbols
            on each reel of a slot machine'' \citep{Bluejay2024}.
      \item \citetISTQB{} describe the term ``software in the loop'' as a
            kind of testing, while the source it references seems to describe
            ``Software-in-the-Loop-Simulation'' as a ``simulation environment''
            that may support software integration testing
            \citep[p.~153]{SPICE2022}; is this a testing approach or a tool
            that supports testing?
      \item The source cited for the definition of ``test type'' from
            \citepISTQB{} does not seem to provide a definition itself.
      \item The same is true for ``visual testing''.
      \item The same is true for ``security attack''.
      \item There is disagreement on the structure of tours; they can either be
            quite general \citep[p.~34]{IEEE2022} or ``organized around a
            special focus'' \citepISTQB{}.
      \item While model testing is said to test the object under test,
            it seems to describe testing the models themselves
            \citet[p.~20]{Firesmith2015}; using the models to test the object
            under test seems to be called ``driver-based testing'' (p.~33).
      \item ``System testing'' is listed as a subtype of ``system testing'' by
            \citet[p.~23]{Firesmith2015}.
      \item ``Hardware-'' and ``human-in-the-loop testing'' have the same
            acronym: ``HIL''\footnote{``HiL'' is used for the former by
                  \citet[p.~2]{PreußeEtAl2012}.} \citep[p.~23]{Firesmith2015}.
      \item The same is true for ``customer'' and ``contract(ual) acceptance
            testing'' (``CAT'') \citep[p.~30]{Firesmith2015}.
      \item The acronym ``SoS'' is used but not defined by
            \citet[p.~23]{Firesmith2015}.
      \item It is ambiguous whether ``tool/environment testing'' refers to
            testing the tools/environment \emph{themselves} or \emph{using}
            them to test the object under test; the latter is implied, but the
            wording of its subtypes \citep[p.~25]{Firesmith2015} seems to imply
            the former.
      \item \citeauthor{ISO_IEC2023a} say that performance and security testing
            are subtypes of reliability testing \citeyearpar{ISO_IEC2023a}, but
            these are all listed separately by \citet[p.~53]{Firesmith2015}.
            \ifnotpaper
      \item The terms ``acceleration'' and ``acoustic tolerance testing'' seem
            to only refer to software testing in \citep[p.~56]{Firesmith2015};
            elsewhere, they seem to refer to testing the acoustic tolerance of
            rats \citep{HolleyEtAl1996} or the acceleration tolerance of
            \accelTolTest{}, which don't exactly seem relevant\dots
            \fi
      \item The distinctions between development testing \citep[p.~136]{IEEE2017},
            developmental testing \citep[p.~30]{Firesmith2015}, and developer
            testing (\citealp[p.~39]{Firesmith2015}; \citealp[p.~11]{Gerrard2000a})
            are unclear and seem miniscule.
\end{enumerate}

\subsubsection{In Textbooks \citep{Patton2006, PetersAndPedrycz2000,
            vanVliet2000}}

\begin{enumerate}[resume]
      \item ``Load testing'' is defined as using loads ``usually between
            anticipated conditions of low, typical, and peak usage''
            \citep[p.~5]{IEEE2022}, while \citeauthor{Patton2006} says the
            loads should as large as possible \citeyearpar[p.~86]{Patton2006}.
      \item ``Installability testing'' is given as a test type
            (\citealp[p.~22]{IEEE2022}; \citeyear[p.~38]{IEEE2021})
            but is sometimes called a test level as
            ``installation testing'' \citep[p.~445]{PetersAndPedrycz2000}.
\end{enumerate}

\subsubsection{In Other Sources}

\begin{enumerate}[resume]
      \item \citeauthor{Bas2024} lists ``three [backup] location categories:
            local, offsite and cloud based [sic]'' \citeyearpar[p.~16]{Bas2024},
            but does not define or discuss ``offsite backups'' (pp.~16-17).
      \item \citeauthor{DoğanEtAl2014} claim that \citep{SakamotoEtAl2013}
            defines ``prime path coverage'' \citeyearpar[p.~184]{DoğanEtAl2014},
            but it doesn't.
      \item ``State-based'' is misspelled by \citeauthor{Kam2008} as
            ``state-base'' \citeyearpar[pp.~13,~15]{Kam2008} and
            ``stated-base'' (Tab.~1).
      \item Although ad hoc testing is sometimes classified as a ``technique''
            \citep[p.~5-14]{SWEBOK2024}, it is one in which ``no recognized test
            design technique is used'' \citep[p.~42]{Kam2008}.
      \item \citeauthor{Kam2008}'s definition of ``boundary value testing''
            says ``See \emph{boundary value analysis},'' but this definition is
            not present \citeyearpar{Kam2008}.
      \item ``Conformance testing'' is implied to be a synonym of ``compliance
            testing'' by \citeauthor{Kam2008}, which only makes sense because
            of the vague definition of ``compliance testing'': ``testing to
            determine the compliance of the component or system''
            \citeyearpar[p.~43]{Kam2008}.
      \item \citeauthor{Kam2008} seems to imply that ``mutation testing'' is a
            synonym of ``back-to-back testing'' \citeyearpar[p.~46]{Kam2008},
            but these are two quite distinct techniques.
      \item \citeauthor{Kam2008} also says that the goal of negative testing is
            ``showing that a component or system does not work''
            \citeyearpar[p.~46]{Kam2008}\todo{OG Beizer}, which is not true;
            if robustness is an important quality for the system, then testing
            the system ``in a way for which it was not intended to be used''
            \citepISTQB{} (i.e., negative testing) is one way to help
            test this!
      \item ``Program testing'' is given as a synonym of ``component testing''
            \citep[p.~46]{Kam2008}, although it probably should be a synonym of
            ``system testing'' instead.
      \item \citeauthor{SneedAndGöschl2000} give ``white-'', ``grey-'', and
            ``black-box testing'' as synonyms for ``module'', ``integration'',
            and ``system testing'', respectively
            \citeyearpar[p.~18]{SneedAndGöschl2000}\todo{OG Hetzel88}, but
            this mapping is incorrect; black-box testing can be performed on a
            module, for example\todo{find source}. This makes the claim that
            ``red-box testing'' is a synonym for ``acceptance testing'' (p.~18)
            lose credibility.
      \item Availability testing isn't assigned to a test priority
            \citep[Tab.~2]{Gerrard2000a}, despite the claim that ``the test
            types\gerrardDistinctIEEE{type} have been allocated a slot against
            the four test priorities'' (p.~13); I think usability and/or
            performance would have made sense.
      \item ``Visual browser validation'' is described as both static \emph{and}
            dynamic in the same table \citep[Tab.~2]{Gerrard2000a}, even though
            they are implied to be orthogonal classifications: ``test
            types can be static \emph{or} dynamic'' (p.~12,~emphasis added).
      \item \citeauthor{Gerrard2000a} makes a distinction between ``transaction
            verification'' and ``transaction testing''
            \citeyearpar[Tab.~2]{Gerrard2000a} and also uses the phrase
            ``transaction flows'' (Fig.~5) but doesn't explain them.
      \item The phrase ``continuous automated testing''  \citep[p.~11]{Gerrard2000a}
            is redundant since continuous testing is a sub-category of automated
            testing (\citealp[p.~35]{IEEE2022}, \citealpISTQB{}).
      \item End-to-end functionality testing is \emph{not} indicated to be
            functionality testing \citep[Tab.~2]{Gerrard2000a}.
      \item \citeauthor{Gerrard2000b}'s definition for ``security audits''
            seems too specific, only applying to ``the products installed on a
            site'' and ``the known vulnerabilities for those products''
            \citeyearpar[p.~28]{Gerrard2000b}.
\end{enumerate}

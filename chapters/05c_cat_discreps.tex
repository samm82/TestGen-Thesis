\begin{enumerate}
    \item % Discrep count: {IEEE2022}
          \ifnotpaper \citeauthor{IEEE2022} \else ISO/IEC and IEEE \fi
          categorize experience-based testing as both a test design
          technique and a test practice on the same page---twice
          \ifnotpaper \citeyearpar[Fig.~2,~p.~34]{IEEE2022}\else
              \cite[Fig.~2,~p.~34]{IEEE2022}\fi!
          \ifnotpaper
              \begin{itemize}
                  \item % Discrep count: {IEEE2021} | {IEEE2022} {IEEE2021} {SWEBOK2024} ISTQB
                        These authors previously say ``experience-based testing
                        practices like exploratory testing \dots\ are not
                        \dots\ techniques for designing test cases'', although
                        they ``can use \dots\ test techniques''
                        \citeyearpar[p.~viii]{IEEE2021}. This implies that
                        ``experience-based test design techniques'' are
                        techniques used by the \emph{practice} of experience-based
                        testing, not that experience-based testing is
                        \emph{itself} a test technique. If this is the case, it
                        is not always clearly articulated
                        \ifnotpaper
                            (\citealp[pp.~4,~22]{IEEE2022}; \citeyear[p.~4]{IEEE2021};
                            \citealp[p.~5-13]{SWEBOK2024}; \citealpISTQB{})
                        \else
                            \cite[pp.~4,~22]{IEEE2022}, \cite[p.~4]{IEEE2021},
                            \cite[p.~5-13]{SWEBOK2024}, \cite{ISTQB}
                        \fi
                        and is
                        sometimes contradicted \citep[p.~46]{Firesmith2015}.
                        However, this conflates the distinction between
                        ``practice'' and ``technique'', making these terms less
                        useful, so this may just be a mistake\seeThesisIssuePar{64}.

                        % Furthermore, if a ``class of \dots\ techniques''
                        % is a practice, then other ``techniques'', such as combinatorial testing
                        % (\citealp[pp.~3,~22]{IEEE2022}; \citeyear[p.~2]{IEEE2021};
                        % \citealp[p.~5-11]{SWEBOK2024}; \citealpISTQB{}), data flow testing
                        % (\citealp[p.~22]{IEEE2022}; \citeyear[p.~3]{IEEE2021};
                        % \citealp[p.~5-13]{SWEBOK2024}; \citealp[p.~43]{Kam2008}), performance(-related)
                        % testing (\citealp[p.~38]{IEEE2021}; \citealp[p.~1187]{Moghadam2019}), and
                        % security testing \citep[p.~40]{IEEE2021} may \emph{also}
                        % actually be practices, since they are also described as classes or families of
                        % techniques. The same could be said of the more
                        % general specification- and structure-based testing, especially since these,
                        % plus experience-based testing, are described as ``complementary'' \citetext{p.~8,~Fig.~2}.
                        % % \citeyearpar[p.~8, Fig.~2]{IEEE2021}

                  \item % Discrep count: {IEEE2022}
                        % Discrep count: {IEEE2022} {IEEE2021} | {SWEBOK2024}
                        This also causes confusion about its children, such as
                        error guessing and exploratory testing; again, on the
                        same page,
                        \ifnotpaper
                            \citeauthor{IEEE2022} say
                        \else
                            \cite[p.~34]{IEEE2022} says
                        \fi error guessing is
                        an ``experience-based test design technique'' and
                        ``experience-based test practices include \dots\
                        exploratory testing, tours, attacks, and
                        checklist-based testing''%
                        \ifnotpaper{ \citeyearpar[p.~34]{IEEE2022}.}
                        \else{.}
                        \fi
                        Other sources also do not agree whether error guessing
                        is a technique
                        \ifnotpaper
                            (pp.~20,~22; \citeyear[p.~viii]{IEEE2021})
                        \else
                            \cite[pp.~20,~22]{IEEE2022}, \cite[p.~viii]{IEEE2021}
                        \fi
                        or a practice \citep[p.~5-14]{SWEBOK2024}.
              \end{itemize}
          \fi
    \item The following test approaches are categorized as test
          techniques by \citep[p.~38]{IEEE2021} and as test types by the
          sources provided:
          \begin{enumerate}
              \item % Discrep count: {IEEE2021} | {IEEE2022} {IEEE2013} implied by {ISO_IEC2023a} {IEEE2021} {Firesmith2015}
                    Capacity testing
                    \ifnotpaper
                        (\citealp[p.~22]{IEEE2022};
                        \citeyear[p.~2]{IEEE2013}; implied by its quality
                        (\citealp{ISO_IEC2023a}; \citealp[Tab.~A.1]{IEEE2021})
                        and by \citep[p.~53]{Firesmith2015})%
                    \else
                        \cite[p.~22]{IEEE2022}, \cite[p.~2]{IEEE2013}%
                    \fi,
              \item % Discrep count: {IEEE2021} | {IEEE2013} implied by {Firesmith2015}
                    Endurance testing
                    \ifnotpaper
                        (\citealp[p.~2]{IEEE2013};
                        implied by \citep[p.~55]{Firesmith2015})%
                    \else
                        \cite[p.~2]{IEEE2013}%
                    \fi,
              \item % Discrep count: {IEEE2021} | {IEEE2022} {IEEE2017} {IEEE2013} ISTQB implied by {Firesmith2015}
                    Load testing
                    \ifnotpaper
                        (\citealp[pp.~5,~20,~22]{IEEE2022};
                        \citeyear[p.~253]{IEEE2017}\todo{OG IEEE 2013};
                        \citealpISTQB{}; implied by \citep[p.~54]{Firesmith2015})%
                    \else
                        \cite[p.~253]{IEEE2017}\todo{OG IEEE 2013},
                        \cite{ISTQB}, \cite[pp.~5,~20,~22]{IEEE2022}%
                    \fi,
              \item % Discrep count: {IEEE2021} | {IEEE2022} {IEEE2021} implied by {Firesmith2015}
                    Performance testing
                    \ifnotpaper
                        (\citealp[pp.~7,~22,~26-27]{IEEE2022};
                        \citeyear[p.~7]{IEEE2021}; implied by
                        \citep[p.~53]{Firesmith2015})%
                    \else
                        \cite[pp.~7,~22,~26-27]{IEEE2022}, \cite[p.~7]{IEEE2021}%
                    \fi, and
              \item % Discrep count: {IEEE2021} | {IEEE2022} {IEEE2017} implied by {Firesmith2015}
                    Stress testing
                    \ifnotpaper
                        (\citealp[pp.~9,~22]{IEEE2022};
                        \citeyear[p.~442]{IEEE2017}; implied by
                        \citep[p.~54]{Firesmith2015})%
                    \else
                        \cite[p.~442]{IEEE2017}, \cite[pp.~9,~22]{IEEE2022}%
                    \fi.
          \end{enumerate}
    \item % Discrep count: {IEEE2022} {IEEE2021} | {PetersAndPedrycz2000}
          ``Installability testing'' is given as a test type
          \ifnotpaper
              (\citealp[p.~22]{IEEE2022}; \citeyear[p.~38]{IEEE2021})
          \else
              \cite[p.~22]{IEEE2022}, \cite[p.~38]{IEEE2021}
          \fi but is sometimes called a test level as
          ``installation testing'' \citep[p.~445]{PetersAndPedrycz2000}.
    \item % Discrep count: {IEEE2022} {IEEE2021} | {Kam2008} implied by {IEEE2021} {IEEE2017}
          Model-based testing is categorized as both a test practice
          \ifnotpaper
              (\citealp[p.~22]{IEEE2022}; \citeyear[p.~viii]{IEEE2021})
          \else
              \cite[p.~22]{IEEE2022}, \cite[p.~viii]{IEEE2021}
          \fi and a test technique
          \ifnotpaper
              (\citealp[p.~4]{Kam2008}; implied by
              \citealp[p.~7]{IEEE2021}; \citeyear[p.~469]{IEEE2017})%
          \else
              \cite[p.~4]{Kam2008}%
          \fi.
    \item % Discrep count: {IEEE2022} | {Kam2008}
          Data-driven testing is categorized as both a test practice
          \citep[p.~22]{IEEE2022} and a test technique
          \citep[p.~43]{Kam2008}\todo{OG Fewster and Graham}.
          \ifnotpaper
    \item % Discrep count: {IEEE2022} | {BarbosaEtAl2006}
          Retesting and regression testing seem to be separated from the rest
          of the testing approaches \citep[p.~23]{IEEE2022}, but it is not
          clearly detailed why; \citet[p.~3]{BarbosaEtAl2006} \ifnotpaper
              consider \else considers \fi regression testing to be a test level.
    \item % Discrep count: {SWEBOK2024} | {Kam2008}
          Although ad hoc testing is sometimes classified as a ``technique''
          \citep[p.~5-14]{SWEBOK2024}, it is one in which ``no recognized test
          design technique is used'' \citep[p.~42]{Kam2008}.
    \item % Discrep count: {Gerrard2000a}
          ``Visual browser validation'' is described as both static \emph{and}
          dynamic in the same table \citep[Tab.~2]{Gerrard2000a}, even though
          they are implied to be orthogonal classifications: ``test types can
          be static \emph{or} dynamic'' \citetext{p.~12,~emphasis added}.
          \fi
\end{enumerate}

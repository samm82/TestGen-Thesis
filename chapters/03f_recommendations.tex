\section{Recommendations}

The following sections provide different recommendations for resolving various
discrepancies\seeSectionParAlways{discrep}. This was done with the goal of
organizing them more logically and making them:
\begin{enumerate}
      \item Atomic (e.g., disaster/recovery testing seems to have two
            disjoint definitions)
      \item Straightforward (e.g., backup and recovery testing's definition
            implies the idea of performance, but its name does not;
            failover/recovery testing, failover and recovery testing,
            and failover testing are all given separately)
      \item Consistent (e.g., backup/recovery testing and failover/recovery
            testing explicitly exclude an aspect included in its parent
            disaster/recovery testing)
\end{enumerate}

\subsection{Recovery Testing}
The following terms should be used in place of the current terminology to
more clearly distinguish between different recovery-related test approaches:

\begin{itemize}
      \item \textbf{Recoverability Testing:} ``Testing \dots\ aimed at
            verifying software restart capabilities after a system crash or
            other disaster'' \citep[p.~5-9]{SWEBOK2024} including ``recover[ing]
            the data directly affected and re-establish[ing] the desired state
            of the system'' (\citealp{ISO_IEC2023a}; similar in
            \citealp[p.~7-10]{SWEBOK2024}) so that the system ``can perform
            required functions'' \citep[p.~370]{IEEE2017}. ``Recovery testing''
            will be a synonym, as in \citep[p.~47]{Kam2008} and since it is the
            more prevalent term throughout various sources, although
            ``recoverability testing'' is preferred to indicate that this
            explicitly focuses on the \emph{ability} to
            recover, not the \emph{performance} of recovering.
      \item \textbf{Failover Testing:} Testing that ``validates the SUT's
            ability to manage heavy loads or unexpected failure to continue
            typical operations'' \cite[p.~5-9]{SWEBOK2024} by entering a
            ``backup operational mode in which [these responsibilities] \dots\
            are assumed by a secondary system'' \citepISTQB{}. This will
            replace ``failover/recovery testing'', since it is more clear, and
            since this is one way that a system can recover from failure, it
            will be a subset of ``recovery testing''.
      \item \textbf{Transfer Recovery Testing:} Testing to evaluate if,
            in the case of a failure, ``operation of the test item can be
            transferred to a different operating site and \dots\ be transferred
            back again once the failure has been resolved''
            \citeyearpar[p.~37]{IEEE2021}. This replaces the second definition
            of ``disaster/recovery testing'', since the first is just a
            description of ``recovery testing'', and could potentially be
            considered as a kind of failover testing. This may not be
            intrinsic to the hardware/software (e.g., may be the responsibility
            of humans/processes).
      \item \textbf{Backup Recovery Testing:} Testing that determines the
            ability ``to restor[e] from back-up memory in the event of failure''
            \citep[p.~37]{IEEE2021}. The qualification that this occurs
            ``without transfer[ing] to a different operating site or back-up
            system'' (p.~37) \emph{could} be made explicit, but this is implied
            since it is separate from transfer recovery testing and failover
            testing, respectively.
      \item \textbf{Recovery Performance Testing:} Testing ``how well a system or
            software can recover \dots\ [from] an interruption or failure''
            (\citealp[p.~7-10]{SWEBOK2024}; similar in \citealp{ISO_IEC2023a})
            ``within specified parameters of time, cost, completeness, and
            accuracy'' \citep[p.~2]{IEEE2013}. The distinction between the
            performance-related elements of recovery testing seemed to be
            meaningful (see \thesisissueref{40}), but was not captured
            consistently by the literature. This will be a subset of
            ``performance-related testing''\seeSectionPar{perf-test-ambiguity}
            as ``recovery testing'' is in \citep[p.~22]{IEEE2022}. This could
            also be extended into testing the performance of specific elements
            of recovery (e.g., failover performance testing), but this be too
            fine-grained and may better be captured as an
            \hyperref[chap:testing:sec:orthogonal-tests]
            {orthogonally derived test approach}.
\end{itemize}

\begin{figure}[hbtp!]
      \centering
      \begin{minipage}{.575\linewidth}
            \centering
            \recoveryGraphCurrent{}
            \caption{The current state of the\\
                  relations between ``recovery\\
                  testing'' terms.}
            \label{fig:recovery-graph-current}
      \end{minipage}%
      \begin{minipage}{.425\linewidth}
            \centering
            \recoveryGraphProposed{}
            \caption{The proposed\\
                  relations between ``recovery testing'' terms.}
            \label{fig:recovery-graph-proposed}
      \end{minipage}
\end{figure}

\subsection{Scalability Testing}
A distinction between the discrepancies mentioned is made in the literature;
it just needs to be used universally and consistently!

\ifnotpaper \citeauthor{IEEE2021} give
\else \cite[p.~39]{IEEE2021} gives \fi ``scalability testing'' as a synonym
of ``capacity testing'', defined
as the testing of a system's ability to ``perform under conditions that may
need to be supported in the future'' which ``may include assessing what level
of additional resources (e.g. memory, disk capacity, network bandwidth) will
be required to support anticipated future loads''%
\ifnotpaper \ \citeyearpar[p.~39]{IEEE2021}%
\fi. This focus on ``the future'' is supported by
\ifnotpaper \citeauthor{ISTQB_author}, who define
\else \cite{ISTQB}, which defines
\fi ``scalability'' as ``the degree to which a component or system can be
adjusted for changing capacity''%
\ifnotpaper \ \citeyearpar{ISTQB}; the original source
      they reference agrees, defining it as ``the measure of a system's ability to be
      upgraded to accommodate increased loads'' \citep[p.~381]{GerrardAndThompson2002}%
\fi. In contrast, capacity testing focuses on the system's present state, evaluating
the ``capability of a product to meet requirements for the maximum limits of a
product parameter'', such as the number of concurrent users, transaction
throughput, or database size \citep{ISO_IEC2023a}. Because of this nuance, it
makes more sense to consider these terms separate and \emph{not} synonyms, as
done by
\ifnotpaper \citet[p.~53]{Firesmith2015} and \citet[pp.~22-23]{Bas2024}%
\else \cite[p.~53]{Firesmith2015} and \cite[pp.~22-23]{Bas2024}%
\fi.

Unfortunately, only focusing on future capacity requirements still leaves room
for ambiguity. While the above definition of ``scalability testing'' includes the
external modification of the system,
\ifnotpaper \citeauthor{ISO_IEC2023a} \else \cite{ISO_IEC2023a} \fi describes it as
testing the ``capability of a product to handle growing or shrinking
workloads or to adapt its capacity to handle variability''%
\ifnotpaper\ \citeyearpar{ISO_IEC2023a}%
\fi, implying that this is done by the system itself.
The potential reason for this is implied by SWEBOK V4's claim that one
objective of elasticity testing is ``to evaluate scalability''
\citep[p.~5-9]{SWEBOK2024}:
\ifnotpaper \citeauthor{ISO_IEC2023a}%
\else \cite{ISO_IEC2023a}%
\fi's notion of ``scalability''
likely refers more accurately to ``elasticity''! This also makes sense in the
context of other definitions provided by SWEBOK V4 \citep{SWEBOK2024}:
\begin{itemize}
      \item \textbf{Scalability:} ``the software's ability to increase and
            scale up on its nonfunctional requirements, such as load, number of
            transactions, and volume of data'' \citetext{p.~5-5}. Based on this
            definition, scalability testing is then a subtype of load testing
            and volume testing, as well as potentially transaction flow testing.
      \item \textbf{Elasticity Testing\footnote{While this definition seems
                        correct, it \swebokElasRef{}}:} testing that ``assesses
            the ability of the \acs{sut} \dots\ to rapidly expand or shrink
            compute, memory, and storage resources without compromising the
            capacity to meet peak utilization'' \citetext{p.~5-9}. Based on this
            definition, elasticity testing is then a subtype of memory
            management testing (with both being a subtype of resource
            utilization testing) and stress testing.
\end{itemize}
This distinction is also consistent with how the terms are used in industry:
\ifnotpaper \citeauthor{Pandey2023} \else \cite{Pandey2023} \fi says
that scalability is the ability to ``increase
\dots\ performance or efficiency as demand increases over time'', while
elasticity allows a system to ``tackle changes in the workload [that] occur for
a short period'' (%
\ifnotpaper \citeyear{Pandey2023};
\fi see \thesisissueref{35}).

To make things even more confusing, SWEBOK V4 says ``scalability
testing evaluates the capability to use and learn the system and the user
documentation'' and ``focuses on the system's effectiveness in supporting user
tasks and the ability to recover from user errors'' \citep[p.~5-9]{SWEBOK2024}.
\swebokScalDef{}, which is completely separate from the definitions of
``scalability'', ``capacity'', and ``elasticity testing''!
\section{Methodology}
\label{method}

\subsection{Source Order}
To gain the most accurate picture of how software testing terminology is used,
sources from multiple authors were selected.
By looking at a variety of sources, discrepancies both within and between them
can be uncovered, and the various classifications of terminology can be
analyzed. Other sources can then be found by ``snowballing'' to other sources
to further investigate specific test approaches.

\begin{enumerate}
      \setcounter{enumi}{-1}
      \item Textbooks trusted at McMaster
            \citep{Patton2006, PetersAndPedrycz2000, vanVliet2000}
            \begin{itemize}
                  \item Ad hoc and arbitrary; not systematic
                  \item Colored \textcolor{Maroon}{maroon}
            \end{itemize}
      \item Established standards
            \ifnotpaper
                  (\citealp{IEEE2022, SWEBOK2024, SWEBOK2014, IEEE2017, IEEE2013,
                        ISO_IEC2023b, IEEE2012, ISO_IEC2023a}; \citealpISTQB{};
                  \citealp{IEEE2021})
            \else
                  (such as IEEE, ISO/IEC, and SWEBOK)
                  \citep{IEEE2022, SWEBOK2024, SWEBOK2014, IEEE2017, IEEE2013,
                        ISO_IEC2023b, IEEE2012, ISO_IEC2023a, ISTQB, IEEE2021}
            \fi
            \begin{itemize}
                  \item Standards organizations colored \textcolor{green}{green}
                  \item ``Meta-level'' commentaries or collections of
                        terminology (often based on these standards), such as
                        \citep{Firesmith2015}, colored \textcolor{blue}{blue}
            \end{itemize}
      \item Other resources: less-formal classifications of terminology
            \ifnotpaper
                  \citep[e.g.,][]{KuļešovsEtAl2013}%
            \else
                  (such as \citep{KuļešovsEtAl2013})%
            \fi%
            , sources investigated to
            ``fill in'' missing definitions\seeSectionParAlways{undef-terms},
            and testing-related resources that emerged for unrelated reasons
            \begin{itemize}
                  \item Colored black, along with any ``surface-level''
                        analysis that followed trivially
            \end{itemize}
\end{enumerate}

\subsection{Procedure}

Except for some sources in \nameref{undef-terms}, all sources were looked
through in their entirely to ``extract'' as much terminology as possible.
Heuristics were used to guide this process, by investigating\dots

\begin{itemize}
      \item \dots glossaries and lists of terms
      \item \dots testing-related terms\\
            e.g., terms that included ``test(ing)'', ``validation'',
            ``verification'', ``review(s)'', or ``audit(s)''
      \item \dots terms that had emerged as part of already-discovered
            testing approaches, \emph{especially} those that were ambiguous
            or prompted further discussion\\
            e.g., terms that included ``performance'', ``recovery'',
            ``component'', ``bottom-up'', ``boundary'', or ``configuration''
      \item \dots terms that implied testing approaches\footnote{
                  Since these methods for deriving test approaches only arose
                  as research progressed, some examples would have been missed
                  during the first pass(es) of resources investigated earlier
                  in the process. While reiterating over them would be ideal,
                  this may not be possible due to time constraints.
            }%
            \seeSectionParAlways{derived-tests}
\end{itemize}

If a term's definition had already been recorded, either the ``new'' one
replaced it if the ``old'' one wasn't as clear/concise or parts of both were
merged to paint a more complete picture. If any discrepancies or ambiguities
arose, they were investigated to a reasonable extent and documented. If a
testing approach was mentioned but not defined, it was still added to the
glossary to indicate it should be investigated further%
\seeSectionParAlways{undef-terms}. A similar methodology
was used for tracking software qualities, albeit in a separate
document\seeSectionParAlways{derived-tests}.

During the first pass of data collection, all software-testing-focused terms
were included. Some of them are less applicable to test case automation
(such as \nameref{static-test}\ifnotpaper{, \thesisissueref{39}}\fi) or too
broad (such as \nameref{attacks}\ifnotpaper{, \thesisissueref{55}}\fi), so they
will be omitted over the course of analysis.

\ifnotpaper
      During this investigation, some terms came up that seemed to be relevant to
      testing but were so vague, they didn't provide any new information. These were
      decided to be not worth tracking\seeThesisIssuePar{39}[, \thesisissueref{44},
            \thesisissueref{28}] and are listed below:

      \begin{itemize}
            \item \textbf{Evaluation:} the ``systematic determination of the extent
                  to which an entity meets its specified criteria''
                  \citep[p.~167]{IEEE2017}
            \item \textbf{Product Analysis:} the ``process of evaluating a product by
                  manual or automated means to determine if the product has certain
                  characteristics'' \citep[p.~343]{IEEE2017}
            \item \textbf{Quality Audit:} ``a structured, independent process to
                  determine if project activities comply with organizational and
                  project policies, processes, and procedures'' \citep[p.~361]{IEEE2017}
                  \todo{OG PMBOK}
            \item \textbf{Software Product Evaluation:} a ``technical operation that
                  consists of producing an assessment of one or more characteristics
                  of a software product according to a specified procedure''
                  \citep[p.~424]{IEEE2017}
      \end{itemize}
\fi

\subsection{Undefined Terms}
\label{undef-terms}

% Define values to be easily reused and used in calculation!

\newcount\TotalBefore
\newcount\TotalAfter
\newcount\UndefBefore
\newcount\UndefAfter

\TotalBefore=432
\UndefBefore=153

\TotalAfter=515
\UndefAfter=171

This process also led to some testing approaches without definitions;
\citep{IEEE2022} and \citep{Firesmith2015} in particular introduced many.
Once more ``standard'' sources had been exhausted, a strategy was proposed to
look for sources that explicitly defined these terms, with the added benefit of
uncovering more terms to explore, potentially in different domains%
\seeThesisIssuePar{57}. This also uncovered some out-of-scope testing approaches,
including EMSEC testing, HTML testing, and aspects of loop testing and
orthogonal array testing\seeSectionPar{scope}; since these
are out of scope, relevant sources were not investigated fully.

The following terms (and their respective related terms)
were explored%
\ifnotpaper
      { in the following sources}%
\fi, bringing the number of testing
approaches from \the\TotalBefore~to \the\TotalAfter~and the number of
\emph{undefined} terms from \the\UndefBefore~to \the\UndefAfter~(the assumption
can be made that about \the\numexpr 100 - 100 * (\UndefAfter - \UndefBefore) /
(\TotalAfter - \TotalBefore)\relax\% of added terms also included a definition):

\input{build/undefTerms}
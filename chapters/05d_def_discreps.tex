\begin{enumerate}
    \item % Discrep count (DEFS): {IEEE2022} {IEEE2021}
          % TODO: NO DEFS?
          Integration testing, system testing, and system integration testing
          are all listed as ``common test levels'' \ifnotpaper
              \citetext{\citealp[p.~12]{IEEE2022}; \citeyear[p.~6]{IEEE2021}}%
          \else
              \cite[p.~12]{IEEE2022}, \cite[p.~6]{IEEE2021}%
          \fi, but no
          definitions are given for the latter two, making it unclear what
          ``system integration testing'' is; it is a combination of the two?
          somewhere on the spectrum between them? It is listed as a child
          % Discrep count (DEFS) (PARS): ISTQB | {Firesmith2015}
          of integration testing by \citetISTQB{}
          and of system testing by \citet[p.~23]{Firesmith2015}.
    \item % Discrep count (DEFS): {IEEE2017}
          % TODO: NO DEFS?
          Similarly, component testing, integration testing, and component
          integration testing are all listed in \citep{IEEE2017}, but ``component
          integration testing'' is only defined as ``testing of groups of
          related components'' \citep[p.~82]{IEEE2017}; it is a combination of
          the two? somewhere on the spectrum between them? As above, it is
          listed as a child of integration testing by \citetISTQB{}.
    \item % Discrep count (DEFS): {SWEBOK2024} | {IEEE2022}
          The \acs{swebok} V4 defines ``privacy testing'' as testing that
          ``assess[es] the security and privacy of users' personal data to
          prevent local attacks'' \citep[p.~5-10]{SWEBOK2024}; this seems to
          overlap (both in scope and name) with the definition of ``security
          testing'' in \citep[p.~7]{IEEE2022}: testing
          ``conducted to evaluate the degree to which a test item, and
          associated data and information, [sic] are protected so that'' only
          ``authorized persons or systems'' can use them as intended.
          \ifnotpaper
    \item % Discrep count (DEFS): {IEEE2021} | {IEEE2017}
          % TODO: SUPP DEFS?
          \citeauthor{IEEE2021} define an ``extended entry (decision) table''
          both as a decision table where the ``conditions consist of multiple
          values rather than simple Booleans'' \citeyearpar[p.~18]{IEEE2021}
          and one where ``the conditions and actions are generally described
          but are incomplete'' \citeyearpar[p.~175]{IEEE2017}\todo{OG ISO1984}.
    \item % Discrep count (DEFS): {IEEE2017}
          % TODO: SUPP DEFS?
          \citeauthor{IEEE2017} use the same definition for ``partial correctness''
          \citeyearpar[p.~314]{IEEE2017} and ``total correctness'' (p.~480).
    \item % Discrep count (DEFS): ISTQB
          While ergonomics testing is out of scope (as it tests hardware, not
          software), its definition of ``testing to determine whether a
          component or system and its input devices are being used properly
          with correct posture'' \citepISTQB{} seems to focus on how the
          system is \emph{used} as opposed to the system \emph{itself}.
    \item % Discrep count (DEFS): ISTQB
          % TODO: SRC? 
          % {SPICE2022} not included as part of this discrepancy since it
          % is used as the ground truth
          \citetISTQB{} describe the term ``software in the loop'' as a
          kind of testing, while the source \ifnotpaper they reference \else
              it references \fi seems to describe
          ``Software-in-the-Loop-Simulation'' as a ``simulation environment''
          that may support software integration testing
          \citep[p.~153]{SPICE2022}; is this a testing approach or a tool
          that supports testing?
    \item % Discrep count (DEFS): {Firesmith2015}
          While model testing is said to test the object under test,
          it seems to describe testing the models themselves
          \citep[p.~20]{Firesmith2015}; using the models to test the object
          under test seems to be called ``driver-based testing''
          \citetext{p.~33}.
    \item % Discrep count (DEFS): {Firesmith2015}
          Similarly, it is ambiguous whether ``tool/environment testing'' refers
          to testing the tools/environment \emph{themselves} or \emph{using}
          them to test the object under test; the latter is implied, but the
          wording of its subtypes \citep[p.~25]{Firesmith2015} seems to imply
          the former.\fi
    \item % Discrep count (DEFS): {IEEE2017} | {SWEBOK2024} | ISTQB
          Various sources say that alpha testing is performed by different
          people, including ``only by users within the organization
          developing the software'' \citep[p.~17]{IEEE2017}, by ``a small,
          selected group of potential users'' \citep[p.~5-8]{SWEBOK2024}, or
          ``in the developer's test environment by roles outside the
          development organization'' \citepISTQB{}.
    \item % Discrep count (DEFS): ISTQB
          % TODO: SUPP DEFS?
          \refHelper \citetISTQB{} \multAuthHelper{define}
          ``\acf{ml} model testing'' and ``\acs{ml} functional performance''
          in terms of ``\acs{ml} functional performance criteria'',
          which is defined in terms of ``\acs{ml} functional performance
          metrics'', which is defined as ``a set of measures that relate to the
          functional correctness of an \acs{ml} system''. The use
          of ``performance'' (or ``correctness'') in these definitions is at
          best ambiguous and at worst incorrect.
          \ifnotpaper
    \item % Discrep count (DEFS): ISTQB
          \phantomsection{} \label{specificISTQB}
          The definition of ``math testing'' given by \citetISTQB{} is
          too specific to be useful, likely taken from an example instead of
          a general definition: ``testing to determine the correctness of the
          pay table implementation, the random number generator results, and
          the return to player computations''.
    \item % Discrep count (DEFS): ISTQB
          A similar issue exists with multiplayer testing, where its
          definition specifies ``the casino game world'' \citepISTQB{}.\fi
    \item % Discrep count (DEFS): {IEEE2022} | ISTQB
          There is disagreement on the structure of tours: they can either be
          quite general \citep[p.~34]{IEEE2022} or ``organized around a
          special focus'' \citepISTQB{}.
          \ifnotpaper
    \item % Discrep count (DEFS): ISTQB
          % TODO: BAD DEFS?
          While correct, ISTQB's definition of ``specification-based testing''
          is not helpful: ``testing based on an analysis of the specification
          of the component or system'' \citepISTQB{}.
    \item % Discrep count (DEFS): {Firesmith2015}
          % TODO: NO DEFS?
          The acronym ``SoS'' is used but not defined by
          \citet[p.~23]{Firesmith2015}.\fi
    \item % Discrep count (DEFS): {IEEE2021} | {Patton2006}
          State testing requires that ``all states in the state model
          \dots\ [are] `visited'\,'' in \citep[p.~19]{IEEE2021} which
          is only one of its possible criteria in \citep[pp.~82-83]{Patton2006}.
    \item % Discrep count (DEFS): {IEEE2022} | {Patton2006}
          ``Load testing'' is performed either using loads that are ``between
          anticipated conditions of low, typical, and peak usage''
          \citep[p.~5]{IEEE2022} or that are as large as possible
          \citep[p.~86]{Patton2006}.
    \item % Discrep count (DEFS): implied by {Patton2006} | {vanVliet2000}
          \refHelper \citet[p.~92\ifnotpaper, emphasis added\fi]{Patton2006}
          says that reviews are ``\emph{the} process[es] under which static
          white-box testing is performed'' but correctness proofs are given
          as another example by \citet[pp.~418-419]{vanVliet2000}.
    \item % Discrep count (DEFS): {Kam2008} | ISTQB
          \refHelper \citet[p.~46]{Kam2008}\todo{OG Beizer} says that the goal
          of negative testing is ``showing that a component or system does not
          work'' which is not true; if robustness is an important quality for
          the system, then testing the system ``in a way for which it was not
          intended to be used'' \citepISTQB{} (i.e., negative testing) is one
          way to help test this! \ifnotpaper
    \item % Discrep count (DEFS): {Gerrard2000a}
          % TODO: NO DEFS?
          \refHelper \citet[Tab.~2]{Gerrard2000a} makes a distinction between
          ``transaction verification'' and ``transaction testing'' and
          uses the phrase ``transaction flows'' \citetext{Fig.~5} but doesn't
          explain them.
    \item % Discrep count (DEFS): {Bas2024}
          % TODO: NO DEFS?
          \refHelper \citet[p.~16]{Bas2024} lists ``three [backup] location
          categories: local, offsite and cloud based [sic]'' but does not
          define or discuss ``offsite backups'' \citetext{pp.~16-17}.
    \item % Discrep count (DEFS): {Gerrard2000a}
          % TODO: NO SUPP DEFS?
          Availability testing isn't assigned to a test priority
          \citep[Tab.~2]{Gerrard2000a}, despite the claim that ``the test
          types\gerrardDistinctIEEE{type} have been allocated a slot against
          the four test priorities'' \citetext{p.~13}; I think usability and/or
          performance would have made sense.\fi
    \item % Discrep count (DEFS): {Kam2008}
          % TODO: NO DEFS?
          \refHelper \citet[p.~42]{Kam2008} says ``See \emph{boundary value
              analysis},'' for the glossary entry of ``boundary value testing''
          but does not provide this definition.
\end{enumerate}

% TODO: re-investigate this after going through the rest of ISO/IEC/IEEE 29119
\ifnotpaper
    Also of note: \citep{IEEE2022, IEEE2021}, from the
    ISO/IEC/IEEE 29119 family of standards, mention the following 23 test
    approaches without defining them. This means that out of the 114 test
    approaches they mention, about 20\% have no associated definition!

    However, the previous version of this standard, \citeyearpar{IEEE2013},
    generally explained two, provided references for two, and explicitly defined
    one of these terms, for a total of five definitions that could (should) have
    been included in \citeyearpar{IEEE2022}! These terms have been
    \underline{underlined}\ifnotpaper%
        , \emph{italicized}, and \textbf{bolded}, respectively%
    \fi. Additionally, entries marked with an asterisk* were defined (at least
    partially) in \citeyearpar{IEEE2017}, which would have been available when
    creating this family of standards. These terms bring the total count of terms
    that could (should) have been defined to nine; almost 40\% of undefined test
    approaches could have been defined!

    \begin{itemize}
        \item \underline{Acceptance Testing*}
        \item Alpha Testing*
        \item Beta Testing*
        \item Capture-Replay Driven Testing
        \item Data-driven Testing
        \item Error-based Testing
        \item Factory Acceptance Testing
        \item Fault Injection Testing
        \item Functional Suitability Testing (also mentioned but not defined in
              \citep{IEEE2017})
        \item \underline{Integration Testing}*
        \item Model Verification
        \item Operational Acceptance Testing
        \item Orthogonal Array Testing
        \item Production Verification Testing
        \item Recovery Testing* (Failover/Recovery Testing, Back-up/Recovery
              Testing, \formatPaper{\textbf}{Backup and Recovery Testing*},
              Recovery*;\seeAlways{recov-discrep})
        \item Response-Time Testing
        \item \formatPaper{\emph}{Reviews} (ISO/IEC 20246) (Code Reviews*)
        \item Scalability Testing (defined as a synonym of ``capacity
              testing'';\seeAlways{scal-discrep})
        \item Statistical Testing
        \item System Integration Testing (System Integration*)
        \item System Testing* (also mentioned but not defined in \citep{IEEE2013})
        \item \formatPaper{\emph}{Unit Testing*}
              (IEEE Std 1008-1987, IEEE Standard for
              Software Unit Testing implicitly listed in the bibliography!)
        \item User Acceptance Testing
    \end{itemize}
\fi
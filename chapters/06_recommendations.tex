\section{Recommendations}
\label{recs}

We provide different recommendations for resolving various
discrepancies (see \Cref{discreps}). This was done with the goal of
organizing them more logically and making them:
\begin{enumerate}
    \item Atomic (e.g., disaster/recovery testing seems to have two
          disjoint definitions)
    \item Straightforward (e.g., backup and recovery testing's definition
          implies the idea of performance, but its name does not
          \ifnotpaper; failover/recovery testing, failover and recovery
          testing, and failover testing are all given separately\fi)
    \item Consistent (e.g., backup/recovery testing and failover/recovery
          testing explicitly exclude an aspect included in its parent
          disaster/recovery testing)
\end{enumerate}
We give recommendations for the areas of recovery testing (\Cref{rec-test-rec}),
scalability testing (\Cref{scal-test-rec}), and performance-related testing
(\Cref{perf-test-rec}). Graphical representations (described in
\Cref{\ifnotpaper graph-gen\else tools\fi}) of these subsets are given in
\recFigs{}, in which arrows representing relations between approaches are
coloured based on the source tier (see \Cref{sources}) that defines them.
Any added approaches or relations are colored \textcolor{orange}{orange}.
\ifnotpaper Note that
    inferred relations (colored \textcolor{gray}{grey}) are included for
    completeness\todo{Should this be the case?}, despite not coming from the
    literature (see \Cref{infers}).
\fi

\subsection{Recovery Testing}
\label{rec-test-rec}
The following terms should be used in place of the current terminology to
more clearly distinguish between different recovery-related test approaches.
The result of the proposed terminology, along with their relations, is
demonstrated in \Cref{fig:recovery-graph-proposed}.

\recoveryGraphs{}

\begin{itemize}
    \item \textbf{Recoverability Testing:} ``Testing \dots\ aimed at
          verifying software restart capabilities after a system crash or
          other disaster'' \citep[p.~5-9]{SWEBOK2024} including ``recover[ing]
          the data directly affected and re-establish[ing] the desired state
          of the system''
          \ifnotpaper
              (\citealp{ISO_IEC2023a}; similar in \citealp[p.~7-10]{SWEBOK2024})
          \else
              \cite{ISO_IEC2023a} (similar in \cite[p.~7-10]{SWEBOK2024})
          \fi so that the system ``can perform
          required functions'' \citep[p.~370]{IEEE2017}. ``Recovery testing''
          will be a synonym, as in \citep[p.~47]{Kam2008}, since it is the
          more prevalent term throughout various sources, although
          ``recoverability testing'' is preferred to indicate that this
          explicitly focuses on the \emph{ability} to
          recover, not the \emph{performance} of recovering.
    \item \textbf{Failover Testing:} Testing that ``validates the SUT's
          ability to manage heavy loads or unexpected failure to continue
          typical operations'' \cite[p.~5-9]{SWEBOK2024} by entering a
          ``backup operational mode in which [these responsibilities] \dots\
          are assumed by a secondary system'' \citepISTQB{}. This will
          replace ``failover/recovery testing'', since it is more clear, and
          since this is one way that a system can recover from failure, it
          will be a subset of ``recovery testing''.
    \item \textbf{Transfer Recovery Testing:} Testing to evaluate if,
          in the case of a failure, ``operation of the test item can be
          transferred to a different operating site and \dots\ be transferred
          back again once the failure has been resolved''
          \citeyearpar[p.~37]{IEEE2021}. This replaces the second definition
          of ``disaster/recovery testing'', since the first is just a
          description of ``recovery testing'', and could potentially be
          considered as a kind of failover testing. This may not be
          intrinsic to the hardware/software (e.g., may be the responsibility
          of humans/processes).
    \item \textbf{Backup Recovery Testing:} Testing that determines the
          ability ``to restor[e] from back-up memory in the event of failure''
          \citep[p.~37]{IEEE2021}. The qualification that this occurs
          ``without transfer[ing] to a different operating site or back-up
          system'' \citetext{p.~37} \emph{could} be made explicit, but this is
          implied since it is separate from transfer recovery testing and
          failover testing, respectively.
    \item \textbf{Recovery Performance Testing:} Testing ``how well a system or
          software can recover \dots\ [from] an interruption or failure''
          \ifnotpaper
              (\citealp[p.~7-10]{SWEBOK2024}; similar in \citealp{ISO_IEC2023a})
          \else
              \cite[p.~7-10]{SWEBOK2024} (similar in \cite{ISO_IEC2023a})
          \fi ``within specified parameters of time, cost, completeness, and
          accuracy'' \citep[p.~2]{IEEE2013}. The distinction between the
          performance-related elements of recovery testing seemed to be
          meaningful\thesisissueref{40}, but was not captured consistently
          by the literature. This will be a subset of ``performance-related
          testing'' \ifnotpaper (see \Cref{perf-test-rec}) \fi
          as ``recovery testing'' is in \citep[p.~22]{IEEE2022}. This could
          also be extended into testing the performance of specific elements
          of recovery (e.g., failover performance testing), but this be too
          fine-grained and may better be captured as an
          \hyperref[orth-test]{orthogonally derived test approach}.
\end{itemize}

\subsection{Scalability Testing}
\label{scal-test-rec}

The ambiguity around scalability testing found in the literature is resolved
and/or explained by other sources! \citet[p.~39]{IEEE2021} \multiAuthHelper{give}
``scalability testing'' as a synonym of ``capacity testing'', defined
as the testing of a system's ability to ``perform under conditions that may
need to be supported in the future'', which ``may include assessing what level
of additional resources (e.g. memory, disk capacity, network bandwidth) will
be required to support anticipated future loads''. This focus on ``the future''
is supported by \citetISTQB{}, \ifnotpaper who define \else which defines \fi
``scalability'' as ``the degree to which a component or system can be adjusted
for changing capacity''\ifnotpaper; the original source they reference agrees,
defining it as ``the measure of a system's ability to be upgraded to
accommodate increased loads'' \citep[p.~381]{GerrardAndThompson2002}\fi. In
contrast, capacity testing focuses on the system's present state, evaluating
the ``capability of a product to meet requirements for the maximum limits of a
product parameter'', such as the number of concurrent users, transaction
throughput, or database size \citep{ISO_IEC2023a}. Because of this nuance, it
makes more sense to consider these terms separate and \emph{not} synonyms, as
done by
\ifnotpaper \citet[p.~53]{Firesmith2015} and \citet[pp.~22-23]{Bas2024}%
\else \cite[p.~53]{Firesmith2015} and \cite[pp.~22-23]{Bas2024}%
\fi.

Unfortunately, only focusing on future capacity requirements still leaves room
for ambiguity. While the previous definition of ``scalability testing'' includes
the external modification of the system, \citet{ISO_IEC2023a}
\multiAuthHelper{describe} it as testing the ``capability of a product to
handle growing or shrinking workloads or to adapt its capacity to handle
variability'', implying that this is done by the system itself. The potential
reason for this is implied by \citet[p.~5-9]{SWEBOK2024}'s claim that one
objective of elasticity testing is ``to evaluate scalability'':
\citep{ISO_IEC2023a}'s notion of ``scalability''
likely refers more accurately to ``elasticity''! This also makes sense in the
context of other definitions provided by \citet{SWEBOK2024}:
\begin{itemize}
    \item \textbf{Scalability:} ``the software's ability to increase and
          scale up on its nonfunctional requirements, such as load, number of
          transactions, and volume of data'' \citetext{p.~5-5}.
          Based on this
          definition, scalability testing is then a subtype of load testing
          and volume testing, as well as potentially transaction flow testing.
    \item \textbf{Elasticity Testing\footnote{While this definition seems
                  correct, it \swebokElasRef{}}:} testing that ``assesses
          the ability of the \acs{sut} \dots\ to rapidly expand or shrink
          compute, memory, and storage resources without compromising the
          capacity to meet peak utilization'' \citetext{p.~5-9}. Based on this
          definition, elasticity testing is then a subtype of memory
          management testing (with both being a subtype of resource
          utilization testing) and stress testing.
\end{itemize}
This distinction is also consistent with how the terms are used in industry:
\citet{Pandey2023}\thesisissueref{35} says that scalability is the ability to
``increase \dots\ performance or efficiency as demand increases over time'',
while elasticity allows a system to ``tackle changes in the workload [that]
occur for a short period''.

\scalGraphs{}

\begin{paperFigure}
    \centering
    \performanceGraph{}
    \caption{Proposed relations between rationalized ``performance-related testing'' terms.}
    \label{fig:perf-graph}
\end{paperFigure}

To make things even more confusing, the \acs{swebok} V4 says ``scalability
testing evaluates the capability to use and learn the system and the user
documentation'' and ``focuses on the system's effectiveness in supporting user
tasks and the ability to recover from user errors'' \citep[p.~5-9]{SWEBOK2024}.
\swebokScalDef{}, which is completely separate from the definitions of
``scalability'', ``capacity'', and ``elasticity testing''! This definition
should simply be disregarded, since it is inconsistent with the rest of the
literature. The removal of the previous two synonym relations is demonstrated
in \Cref{fig:scal-graph-proposed}.

\subsection{Performance(-related) Testing}
\label{perf-test-rec}

``Performance testing'' is defined as testing ``conducted to evaluate the
degree to which a test item accomplishes its designated functions''
\ifnotpaper
    (\citealp[p.~7]{IEEE2022}; \citeyear[p.~320]{IEEE2017}; similar in
    \citeyear[pp.~38-39]{IEEE2021}; \citealp[p.~1187]{Moghadam2019})%
\else
    \cite[p.~320]{IEEE2017}, \cite[p.~7]{IEEE2022} (similar in
    \cite[pp.~38-39]{IEEE2021}, \cite[p.~1187]{Moghadam2019})%
\fi. It does this
by ``measuring the performance metrics''
\ifnotpaper
    (\citealp[p.~1187]{Moghadam2019}; similar in \citealpISTQB{})
\else
    \cite[p.~1187]{Moghadam2019} (similar in \cite{ISTQB})
\fi (such as the ``system's capacity for growth''
\citep[p.~23]{Gerrard2000b}), ``detecting the functional problems appearing
under certain execution conditions'' \citep[p.~1187]{Moghadam2019}, and
``detecting violations of non-functional requirements under expected and
stress conditions'' \ifnotpaper
    (\citealp[p.~1187]{Moghadam2019}; similar in \citealp[p.~5-9]{SWEBOK2024})%
\else
    \cite[p.~1187]{Moghadam2019} (similar in \cite[p.~5-9]{SWEBOK2024})%
\fi. It is performed either \dots\
\begin{enumerate}
    \item ``within given constraints of time and other resources''
          \ifnotpaper
              (\citealp[p.~7]{IEEE2022}; \citeyear[p.~320]{IEEE2017};
              similar in \citealp[p.~1187]{Moghadam2019})%
          \else
              \cite[p.~320]{IEEE2017}, \cite[p.~7]{IEEE2022} (similar
              in \cite[p.~1187]{Moghadam2019})%
          \fi, or
    \item ``under a `typical' load'' \citep[p.~39]{IEEE2021}.
\end{enumerate}

It is listed as a subset of performance-related testing, which is defined as
testing ``to determine whether a test item performs as required when it is
placed under various types and sizes of `load'\,'' \citeyearpar[p.~38]{IEEE2021},
along with other approaches like load and capacity testing
\citep[p.~22]{IEEE2022}. Note that ``performance, load and stress testing might
considerably overlap in many areas'' \citep[p.~1187]{Moghadam2019}.
In contrast, \citet[p.~5-9]{SWEBOK2024}
gives ``capacity and response time'' as examples of ``performance
characteristics'' that performance testing would seek to ``assess'', which
seems to imply that these are sub-approaches to performance testing instead.
This is consistent with how some sources treat ``performance testing'' and
``performance-related testing'' as synonyms \ifnotpaper
    (\citealp[p.~5-9]{SWEBOK2024}; \citealp[p.~1187]{Moghadam2019})%
\else \cite[p.~5-9]{SWEBOK2024}, \cite[p.~1187]{Moghadam2019}%
\fi, as noted in \Cref{syns}. This makes sense because of how general the
concept of ``performance'' is; most definitions of ``performance testing'' seem
to treat it as a category of tests.

However, it seems more consistent to infer
that the definition of ``performance-related testing'' is the more general one
often assigned to ``performance testing'' performed ``within given constraints
of time and other resources'' \ifnotpaper (\citealp[p.~7]{IEEE2022};
    \citeyear[p.~320]{IEEE2017}; similar in \citealp[p.~1187]{Moghadam2019})%
\else \cite[p.~320]{IEEE2017}, \cite[p.~7]{IEEE2022}
    (similar in \cite[p.~1187]{Moghadam2019})\fi, and
``performance testing'' is a sub-approach of this performed ``under a `typical'
load'' \citep[p.~39]{IEEE2021}. This has other implications for relations
between these types of testing; for example, ``load testing'' usually occurs
``between anticipated conditions of low, typical, and peak usage''
\ifnotpaper (\citealp[p.~5]{IEEE2022}; \citeyear[p.~39]{IEEE2021};
    \citeyear[p.~253]{IEEE2017}\todo{OG IEEE 2013}; \citealpISTQB{})%
\else \cite[p.~253]{IEEE2017}, \cite{ISTQB}, \cite[p.~5]{IEEE2022},
    \cite[p.~39]{IEEE2021}\fi, so it is a child of ``performance-related
testing'' and a parent of ``performance testing''.

After these changes, some finishing touches remain. The ``self-loops''
mentioned in \Cref{selfPars} provide no new information and can be removed.
Similarly, the term ``soak testing'' can be removed. Since it is given as a
synonym to both ``endurance testing'' \emph{and} ``reliability testing'' (see
\Cref{multiSyns}), it makes sense to just use these terms instead of one that
is potentially ambiguous. These changes (along with those from
\Cref{rec-test-rec,scal-test-rec} made implicitly) result in
the relations shown in \Cref{fig:perf-graph}.

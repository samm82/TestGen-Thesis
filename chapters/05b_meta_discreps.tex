\ifnotpaper\paragraph{High Severity}\fi
\begin{itemize}
      \item % Discrep count (OTHER): {SWEBOK2024} | {IEEE2022}
            The \acs{swebok} V4 defines ``privacy testing'' as testing that
            ``assess[es] the security and privacy of users' personal data to
            prevent local attacks'' \citep[p.~5-10]{SWEBOK2024}; this seems to
            overlap (both in scope and name) with
            the definition of ``security testing'' in \citep{IEEE2022}: testing
            ``conducted to evaluate the degree to which a test item, [sic] and
            associated data and information, are protected so that'' only
            ``authorized persons or systems'' can use them as intended.
            \ifnotpaper
      \item % Discrep count (OTHER): ISTQB
            While ergonomics testing is out of scope (as it tests hardware, not
            software), its definition of ``testing to determine whether a
            component or system and its input devices are being used properly
            with correct posture'' \citepISTQB{} seems to focus on how the
            system is \emph{used} as opposed to the system \emph{itself}.
      \item % Discrep count (OTHER): ISTQB | {SPICE2022}
            \citetISTQB{} describe the term ``software in the loop'' as a
            kind of testing, while the source it references seems to describe
            ``Software-in-the-Loop-Simulation'' as a ``simulation environment''
            that may support software integration testing
            \citep[p.~153]{SPICE2022}; is this a testing approach or a tool
            that supports testing?
      \item % Discrep count (OTHER): {Firesmith2015}
            While model testing is said to test the object under test,
            it seems to describe testing the models themselves
            \citet[p.~20]{Firesmith2015}; using the models to test the object
            under test seems to be called ``driver-based testing'' (p.~33).\fi
      \item % Discrep count (OTHER): {Firesmith2015}
            It is ambiguous whether ``tool/environment testing'' refers to
            testing the tools/environment \emph{themselves} or \emph{using}
            them to test the object under test; the latter is implied, but the
            wording of its subtypes \citep[p.~25]{Firesmith2015} seems to imply
            the former.
            \ifnotpaper
      \item % Discrep count (OTHER): {Firesmith2015}
            The terms ``acceleration'' and ``acoustic tolerance testing'' seem
            to only refer to software testing in \citep[p.~56]{Firesmith2015};
            elsewhere, they seem to refer to testing the acoustic tolerance of
            rats \citep{HolleyEtAl1996} or the acceleration tolerance of
            \accelTolTest{}, which don't exactly seem relevant\dots\fi
      \item % Discrep count (OTHER): {Firesmith2015} | {IEEE2017} {Gerrard2000a}
            % Discrep count (OTHER): {IEEE2017} | {Gerrard2000a}
            The distinctions between development testing \citep[p.~136]{IEEE2017},
            developmental testing \citep[p.~30]{Firesmith2015}, and developer
            testing
            % Severity: High (Paper)
            \ifnotpaper
                  (\citealp[p.~39]{Firesmith2015}; \citealp[p.~11]{Gerrard2000a})
            \else
                  \cite[p.~39]{Firesmith2015}, \cite[p.~11]{Gerrard2000a}
            \fi are unclear and seem miniscule.
            \ifnotpaper
\end{itemize}

\paragraph{Medium Severity}
\begin{itemize}\fi
      \item % Discrep count (OTHER): {IEEE2017} | {SWEBOK2024} ISTQB
            % Discrep count (OTHER): {SWEBOK2024} | ISTQB
            Various sources say that alpha testing is performed by different
            people, including ``only by users within the organization
            developing the software'' \citep[p.~17]{IEEE2017}, by ``a small,
            selected group of potential users'' \citep[p.~5-8]{SWEBOK2024}, or
            ``in the developer's test environment by roles outside the
            development organization'' \citepISTQB{}.
      \item % Discrep count (OTHER): ISTQB
            ``\acf{ml} model testing'' and ``\acs{ml} functional performance''
            are defined in terms of ``\acs{ml} functional performance criteria'',
            which is defined in terms of ``\acs{ml} functional performance
            metrics'', which is defined as ``a set of measures that relate to the
            functional correctness of an \acs{ml} system'' \citepISTQB{}. The use
            of ``performance'' (or ``correctness'') in these definitions is at
            best ambiguous and at worst incorrect.
            \ifnotpaper
      \item % Discrep count (OTHER): ISTQB
            The definition of ``math testing'' given by \citetISTQB{} is
            too specific to be useful, likely taken from an example instead of
            a general definition: ``testing to determine the correctness of the
            pay table implementation, the random number generator results, and
            the return to player computations''.
      \item % Discrep count (OTHER): ISTQB
            A similar issue exists with multiplayer testing, where its
            definition specifies ``the casino game world'' \citepISTQB{}.
      \item % Discrep count (OTHER): ISTQB
            % {Bluejay2024} not included as part of this discrepancy since it
            % is used as the ground truth
            Thirdly, ``par sheet testing'' from \citepISTQB{} seems to
            refer to this specific example and does not seem more widely
            applicable, since a ``PAR sheet'' is ``a list of all the symbols
            on each reel of a slot machine'' \citep{Bluejay2024}.\fi
      \item % Discrep count (OTHER): {IEEE2022} | ISTQB
            There is disagreement on the structure of tours; they can either be
            quite general \citep[p.~34]{IEEE2022} or ``organized around a
            special focus'' \citepISTQB{}.
      \item % Discrep count (PARS): {ISO_IEC2023a} | {Firesmith2015}
            Performance and security testing are given as subtypes of
            reliability testing by \citep{ISO_IEC2023a} but
            these are all listed separately by \citet[p.~53]{Firesmith2015}.
\end{itemize}

\ifnotpaper
      \paragraph{Low Severity}
      \begin{itemize}
            \item % Discrep count (OTHER): ISTQB
                  While correct, ISTQB's definition of ``specification-based testing''
                  is not helpful: ``testing based on an analysis of the specification
                  of the component or system'' \citepISTQB{}.
            \item % Discrep count (OTHER): ISTQB
                  The source cited for the definition of ``test type'' from
                  \citepISTQB{} does not seem to provide a definition itself.
            \item % Discrep count (OTHER): ISTQB
                  The same is true for ``visual testing'' \citepISTQB{}.
            \item % Discrep count (OTHER): ISTQB
                  The same is true for ``security attack'' \citepISTQB{}.
            \item % Discrep count (OTHER): {Firesmith2015}
                  % Discrep count (OTHER): {Firesmith2015} | {PreußeEtAl2012}
                  ``Hardware-'' and ``human-in-the-loop testing'' have the same
                  acronym: ``HIL''\footnote{``HiL'' is used for the former by
                        \citet[p.~2]{PreußeEtAl2012}.} \citep[p.~23]{Firesmith2015}.
            \item % Discrep count (OTHER): {Firesmith2015}
                  The same is true for ``customer'' and ``contract(ual) acceptance
                  testing'' (``CAT'') \citep[p.~30]{Firesmith2015}.
            \item % Discrep count (OTHER): {Firesmith2015}
                  The acronym ``SoS'' is used but not defined by
                  \citet[p.~23]{Firesmith2015}.
      \end{itemize}\fi
\begin{enumerate}
    \item % Discrep count (CATS, CONTRA): {IEEE2022}
          \ifnotpaper \citeauthor{IEEE2022} \else ISO/IEC and IEEE \fi
          categorize experience-based testing as both a test design
          technique and a test practice on the same page---twice
          \ifnotpaper \citeyearpar[Fig.~2,~p.~34]{IEEE2022}\else
              \cite[Fig.~2,~p.~34]{IEEE2022}\fi!
          \ifnotpaper
              \begin{itemize}
                  \item These authors previously say ``experience-based testing
                        % Discrep count (CATS, AMBI): {IEEE2021} | {IEEE2022} {IEEE2021} {SWEBOK2024} ISTQB
                        practices like exploratory testing \dots\ are not
                        \dots\ techniques for designing test cases'', although
                        they ``can use \dots\ test techniques''
                        \citeyearpar[p.~viii]{IEEE2021}. This implies that
                        ``experience-based test design techniques'' are
                        techniques used by the \emph{practice} of experience-based
                        testing, not that experience-based testing is
                        \emph{itself} a test technique. If this is the case, it
                        is not always clearly articulated
                        \ifnotpaper
                            (\citealp[pp.~4,~22]{IEEE2022}; \citeyear[p.~4]{IEEE2021};
                            \citealp[p.~5-13]{SWEBOK2024}; \citealpISTQB{})
                        \else
                            \cite[pp.~4,~22]{IEEE2022}, \cite[p.~4]{IEEE2021},
                            \cite[p.~5-13]{SWEBOK2024}, \cite{ISTQB}
                        \fi and is sometimes contradicted \citep[p.~46]{Firesmith2015}.
                        % Discrep count (CATS, CONTRA): {IEEE2021} | {Firesmith2015}
                        However, this conflates the distinction between
                        ``practice'' and ``technique'', making these terms less
                        useful, so this may just be a mistake\thesisissueref{64}.

                        % Furthermore, if a ``class of \dots\ techniques''
                        % is a practice, then other ``techniques'', such as combinatorial testing
                        % (\citealp[pp.~3,~22]{IEEE2022}; \citeyear[p.~2]{IEEE2021};
                        % \citealp[p.~5-11]{SWEBOK2024}; \citealpISTQB{}), data flow testing
                        % (\citealp[p.~22]{IEEE2022}; \citeyear[p.~3]{IEEE2021};
                        % \citealp[p.~5-13]{SWEBOK2024}; \citealp[p.~43]{Kam2008}), performance(-related)
                        % testing (\citealp[p.~38]{IEEE2021}; \citealp[p.~1187]{Moghadam2019}), and
                        % security testing \citep[p.~40]{IEEE2021} may \emph{also}
                        % actually be practices, since they are also described as classes or families of
                        % techniques. The same could be said of the more
                        % general specification- and structure-based testing, especially since these,
                        % plus experience-based testing, are described as ``complementary'' \citetext{p.~8,~Fig.~2}.
                        % % \citeyearpar[p.~8, Fig.~2]{IEEE2021}

                  \item This also causes confusion about its children, such as
                        % Discrep count (CATS, CONTRA): {IEEE2022}
                        % Discrep count (CATS, CONTRA): {IEEE2022} {IEEE2021} | {SWEBOK2024}
                        error guessing and exploratory testing. Also on the
                        same page, \citet[p.~34]{IEEE2022} \multAuthHelper{say}
                        error guessing is an ``experience-based test design
                        technique'' and ``experience-based test practices
                        include \dots\ exploratory testing, tours, attacks, and
                        checklist-based testing.'' Other sources also do not
                        agree whether error guessing is a technique
                        \ifnotpaper
                            \citetext{pp.~20,~22,~34; \citeyear[p.~viii]{IEEE2021}}
                        \else
                            \cite[pp.~20,~22,~34]{IEEE2022}, \cite[p.~viii]{IEEE2021}
                        \fi or a practice \citep[p.~5-14]{SWEBOK2024}.
              \end{itemize}
          \fi
    \item % Discrep count (CATS, CONTRA):
          The following test approaches are categorized as test
          techniques by \citep[p.~38]{IEEE2021} and as test types by the
          sources provided:
          \begin{enumerate}
              \item Capacity testing
                    \ifnotpaper
                        (\citealp[p.~22]{IEEE2022};
                        \citeyear[p.~2]{IEEE2013}; implied by its quality
                        (\citealp{ISO_IEC2023a}; \citealp[Tab.~A.1]{IEEE2021});
                        \citealp[p.~53]{Firesmith2015})%
                    \else
                        \cite[p.~22]{IEEE2022}, \cite[p.~2]{IEEE2013}%
                    \fi,
                    % Discrep count (CATS, CONTRA): {IEEE2021} | {IEEE2022} {IEEE2013} implied by {ISO_IEC2023a} {IEEE2021} {Firesmith2015}
              \item Endurance testing
                    \ifnotpaper
                        (\citealp[p.~2]{IEEE2013};
                        implied by \citealp[p.~55]{Firesmith2015})%
                    \else
                        \cite[p.~2]{IEEE2013}%
                    \fi,
                    % Discrep count (CATS, CONTRA): {IEEE2021} | {IEEE2013} implied by {Firesmith2015}
              \item Load testing
                    \ifnotpaper
                        (\citealp[pp.~5,~20,~22]{IEEE2022};
                        \citeyear[p.~253]{IEEE2017}\todo{OG IEEE 2013};
                        \citealpISTQB{}; implied by \citealp[p.~54]{Firesmith2015})%
                    \else
                        \cite[p.~253]{IEEE2017}\todo{OG IEEE 2013},
                        \cite{ISTQB}, \cite[pp.~5,~20,~22]{IEEE2022}%
                    \fi,
                    % Discrep count (CATS, CONTRA): {IEEE2021} | {IEEE2022} {IEEE2017} {IEEE2013} ISTQB implied by {Firesmith2015}
              \item Performance testing
                    \ifnotpaper
                        (\citealp[pp.~7,~22,~26-27]{IEEE2022};
                        \citeyear[p.~7]{IEEE2021}; implied by
                        \citealp[p.~53]{Firesmith2015})%
                    \else
                        \cite[pp.~7,~22,~26-27]{IEEE2022}, \cite[p.~7]{IEEE2021}%
                    \fi, and
                    % Discrep count (CATS, CONTRA): {IEEE2021} | {IEEE2022} {IEEE2021} implied by {Firesmith2015}
              \item Stress testing
                    \ifnotpaper
                        (\citealp[pp.~9,~22]{IEEE2022};
                        \citeyear[p.~442]{IEEE2017}; implied by
                        \citealp[p.~54]{Firesmith2015})%
                    \else
                        \cite[p.~442]{IEEE2017}, \cite[pp.~9,~22]{IEEE2022}%
                    \fi.
                    % Discrep count (CATS, CONTRA): {IEEE2021} | {IEEE2022} {IEEE2017} implied by {Firesmith2015}
          \end{enumerate}
    \item % Discrep count (DEFS, MISS): {IEEE2022} {IEEE2021}
          Integration testing, system testing, and system integration testing
          are all listed as ``common test levels'' \ifnotpaper
              \citetext{\citealp[p.~12]{IEEE2022}; \citeyear[p.~6]{IEEE2021}}%
          \else
              \cite[p.~12]{IEEE2022}, \cite[p.~6]{IEEE2021}%
          \fi, but no
          definitions are given for the latter two, making it unclear what
          ``system integration testing'' is; it is a combination of the two?
          somewhere on the spectrum between them? It is listed as a child
          % Discrep count (PARS, CONTRA): ISTQB | {Firesmith2015}
          of integration testing by \citetISTQB{}
          and of system testing by \citet[p.~23]{Firesmith2015}.
    \item % Discrep count (DEFS, MISS): {IEEE2017}
          Similarly, component testing, integration testing, and component
          integration testing are all listed in \citep{IEEE2017}, but ``component
          integration testing'' is only defined as ``testing of groups of
          related components'' \citep[p.~82]{IEEE2017}; it is a combination of
          the two? somewhere on the spectrum between them? As above, it is
          listed as a child of integration testing by \citetISTQB{}.
    \item % Discrep count (SYNS, CONTRA): {IEEE2017} ISTQB | {IEEE2017} {BaresiAndPezzè2006}
          A component is an ``entity with discrete structure \dots\ within a
          system considered at a particular level of analysis''
          \citep{ISO_IEC2023b} and ``the terms module, component, and unit
              [sic] are often used interchangeably or defined to be subelements
          of one another in different ways depending upon the context'' with
          no standardized relationship \citep[p.~82]{IEEE2017}. For example,
          \citetISTQB{} \multAuthHelper{define} them as synonyms while
          \citet[p.~107]{BaresiAndPezzè2006} \multAuthHelper{say} ``components
          differ from classical modules for being re-used in different contexts
          independently of their development''. Additionally, since components
          are structurally, functionally, or logically discrete
          \citep[p.~419]{IEEE2017} and ``can be tested in isolation''
          \citepISTQB{}, ``unit/component/module testing'' could refer to the
          testing of both a module \emph{and} a specific function in a module%
          \thesisissueref{14}, introducing a further level of ambiguity.
          % Discrep count (SYNS, AMBI): {IEEE2017} ISTQB
    \item % Discrep count (CATS, WRONG): {IEEE2016}
          Since keyword-driven testing can be used for automated \emph{or}
          manual testing \citep[pp.~4,~6]{IEEE2016}, the claim that ``test
          cases can be either manual test cases or keyword test cases''
          \citetext{p.~6} is incorrect.

          % STD | META
    \item % Discrep count (PARS, CONTRA): {ISO_IEC2023a} | {Firesmith2015}
          Performance testing and security testing are given as subtypes of
          reliability testing by \citep{ISO_IEC2023a}, but these are all listed
          separately by \citep[p.~53]{Firesmith2015}.
    \item % Discrep count (DEFS, OVER): {SWEBOK2024} | {IEEE2022}
          The \acs{swebok} V4 defines ``privacy testing'' as testing that
          ``assess[es] the security and privacy of users' personal data to
          prevent local attacks'' \citep[p.~5-10]{SWEBOK2024}; this seems to
          overlap (both in scope and name) with the definition of ``security
          testing'' in \citep[p.~7]{IEEE2022}: testing
          ``conducted to evaluate the degree to which a test item, and
          associated data and information, [sic] are protected so that'' only
          ``authorized persons or systems'' can use them as intended.
    \item % Discrep count (DEFS, CONTRA): {IEEE2022} | ISTQB
          \tourDiscrep{}
    \item % Discrep count (DEFS, CONTRA): {IEEE2017} | {SWEBOK2024} | ISTQB
          \alphaDiscrep{}
    \item % Discrep count (TERMS, AMBI): {IEEE2017} | {Firesmith2015} | {Firesmith2015} {Gerrard2000a}
          The distinctions between development testing \citep[p.~136]{IEEE2017},
          developmental testing \citep[p.~30]{Firesmith2015}, and developer
          testing
          \ifnotpaper
              (\citealp[p.~39]{Firesmith2015}; \citealp[p.~11]{Gerrard2000a})
          \else
              \cite[p.~39]{Firesmith2015}, \cite[p.~11]{Gerrard2000a}
          \fi are unclear and seem miniscule.\todo{Is this a def discrep?}
    \item % Discrep count (DEFS, AMBI): ISTQB
          % TODO: SUPP DEFS?
          \refHelper \citetISTQB{} \multAuthHelper{define}
          ``\acf{ml} model testing'' and ``\acs{ml} functional performance''
          in terms of ``\acs{ml} functional performance criteria'',
          which is defined in terms of ``\acs{ml} functional performance
          metrics'', which is defined as ``a set of measures that relate to the
          functional correctness of an \acs{ml} system''. The use
          of ``performance'' (or ``correctness'') in these definitions is at
          best ambiguous and at worst incorrect.
    \item % Discrep count (TERMS, WRONG): {Firesmith2015}
          % Other sources not included as part of this discrepancy since they
          % are used as the ground truth
          The terms ``acceleration tolerance testing'' and ``acoustic tolerance
          testing'' seem to only refer to software testing in
          \citep[p.~56]{Firesmith2015};
          elsewhere, they seem to refer to testing the acoustic tolerance of
          rats \citep{HolleyEtAl1996} or the acceleration tolerance of
          \accelTolTest{}, which don't exactly seem relevant\dots%
          \todo{Is this a scope discrep?}
    \item % Discrep count (TERMS, OVER): {SWEBOK2024} implied by {Valcheva2013} {YuEtAl2011} | {Firesmith2015}
          ``Orthogonal array testing'' \ifnotpaper \citetext{%
                  \citealp[pp.~5-1,~5-11]{SWEBOK2024};
                  implied by \citealp[pp.~467,~473]{Valcheva2013};
                  \citealp[pp.~1573-1577,~1580]{YuEtAl2011}} \else
              \cite[pp.~5-1,~5-11]{SWEBOK2024} \fi and ``operational
          acceptance testing'' \citep[p.~30]{Firesmith2015} have the same
          acronym (``OAT'').

          % META | TEXT
    \item % Discrep count (CATS, CONTRA): {IEEE2022} {IEEE2021} | {PetersAndPedrycz2000}
          ``Installability testing'' is given as a test type
          \ifnotpaper
              (\citealp[p.~22]{IEEE2022}; \citeyear[p.~38]{IEEE2021})
          \else
              \cite[p.~22]{IEEE2022}, \cite[p.~38]{IEEE2021}
          \fi but is sometimes called a test level as
          ``installation testing'' \citep[p.~445]{PetersAndPedrycz2000}.
    \item % Discrep count (DEFS, CONTRA): {IEEE2022} | {Patton2006}
          \loadDiscrep{}
    \item % Discrep count (DEFS, CONTRA): {IEEE2021} | {Patton2006}
          State testing requires that ``all states in the state model
          \dots\ [are] `visited'\,'' in \citep[p.~19]{IEEE2021} which
          is only one of its possible criteria in \citep[pp.~82-83]{Patton2006}.
    \item % Discrep count (DEFS, CONTRA): {IEEE2017} {PetersAndPedrycz2000} {vanVliet2000} | {Patton2006}
          \refHelper \citet[p.~456]{IEEE2017} \multAuthHelper{say} system
          testing is ``conducted on a complete, integrated system'' (which
          \citet[Tab.~12.3]{PetersAndPedrycz2000} and
          \citet[p.~439]{vanVliet2000} agree with), while
          \citet[p.~109]{Patton2006} says it can also be done on ``at least a
          major portion'' of the product.
    \item % Discrep count (SYNS, AMBI): ISTQB | {Patton2006}
          % TODO: NEAR SYNS?
          \refHelper \citetISTQB{} \multAuthHelper{claim} that code inspections
          are related to peer reviews but \citet[pp.~94-95]{Patton2006} makes
          them quite distinct.
    \item % Discrep count (SYNS, CONTRA): ISTQB | {PetersAndPedrycz2000}
          \phantomsection{} \label{walkthrough-syns}
          ``Walkthroughs'' and ``structured walkthroughs'' are given
          as synonyms by \citetISTQB{} but \citet[p.~484]{PetersAndPedrycz2000}
          \ifnotpaper imply \else implies \fi that they are different, saying a
          more structured walkthrough may have specific roles.
    \item % Discrep count (SYNS, WRONG): {PetersAndPedrycz2000}
          \refHelper \citeauthor{PetersAndPedrycz2000} \multAuthHelper{claim}
          that ``structural testing
          subsumes white box testing'' but they seem to describe the same thing:
          \ifnotpaper they say \else it says \fi ``structure tests are aimed at
          exercising the internal logic of a software system'' and ``in white box
          testing \dots, using detailed knowledge of code, one creates a battery of
          tests in such a way that they exercise all components of the code
          (say, statements, branches, paths)'' on the same page
          \citeyearpar[p.~447]{PetersAndPedrycz2000}!
    \item % Discrep count (DEFS, CONTRA): {vanVliet2000} | implied by {Patton2006}
          \refHelper \citet[p.~92\ifnotpaper, emphasis added\fi]{Patton2006}
          says that reviews are ``\emph{the} process[es] under which static
          white-box testing is performed'' but correctness proofs are given
          as another example by \citet[pp.~418-419]{vanVliet2000}.

          % TEXT | OTHER
    \item % Discrep count (CATS, CONTRA): {IEEE2022} {IEEE2021} | {Kam2008} implied by {IEEE2021} {IEEE2017}
          Model-based testing is categorized as both a test practice
          \ifnotpaper
              (\citealp[p.~22]{IEEE2022}; \citeyear[p.~viii]{IEEE2021})
          \else
              \cite[p.~22]{IEEE2022}, \cite[p.~viii]{IEEE2021}
          \fi and a test technique
          \ifnotpaper
              (\citealp[p.~4]{Kam2008}; implied by
              \citealp[p.~7]{IEEE2021}; \citeyear[p.~469]{IEEE2017})%
          \else
              \cite[p.~4]{Kam2008}%
          \fi.
    \item % Discrep count (CATS, CONTRA): {IEEE2022} | {Kam2008}
          Data-driven testing is categorized as both a test practice
          \citep[p.~22]{IEEE2022} and a test technique
          \citep[p.~43]{Kam2008}\todo{OG Fewster and Graham}.
    \item % Discrep count (DEFS, WRONG): {Kam2008} | ISTQB
          \refHelper \citet[p.~46]{Kam2008}\todo{OG Beizer} says that the goal
          of negative testing is ``showing that a component or system does not
          work'' which is not true; if robustness is an important quality for
          the system, then testing the system ``in a way for which it was not
          intended to be used'' \citepISTQB{} (i.e., negative testing) is one
          way to help test this!
    \item % Discrep count (DEFS, MISS): {Kam2008}
          \refHelper \citet[p.~42]{Kam2008} says ``See \emph{boundary value
              analysis},'' for the glossary entry of ``boundary value testing''
          but does not provide this definition.
\end{enumerate}
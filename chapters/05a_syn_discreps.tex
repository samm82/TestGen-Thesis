\begin{enumerate}
    \item % Discrep count: {ISO_IEC2023b} {IEEE2017} | implied by {BaresiAndPezzè2006}
          A component is an ``entity with discrete structure \dots\ within a
          system considered at a particular level of analysis''
          \citep{ISO_IEC2023b} and ``the terms module, component, and unit
              [sic] are often used interchangeably or defined to be subelements
          of one another in different ways depending upon the context'' with
          no standardized relationship \citep[p.~82]{IEEE2017}. This means
          unit/component/module testing can refer to the testing of both a
          module and a specific function in a module\seeThesisIssuePar{14}.
          However, ``component'' is sometimes defined differently than
          ``module'': ``components differ from classical modules for being
          re-used in different contexts independently of their development''
          \citep[p.~107]{BaresiAndPezzè2006}, so distinguishing the two
          may be necessary.
          \ifnotpaper
    \item % Discrep count: {IEEE2017} {IEEE2013} | {IEEE2022} {IEEE2017} {Perry2006}
          \citeauthor*{IEEE2017} say that ``test level'' and ``test phase''
          are synonyms, both meaning a ``specific instantiation of [a] test
          sub-process'' (\citeyear[pp.~469,~470]{IEEE2017}; \citeyear[p.~9]{IEEE2013}),
          but there are also alternative definitions for them.
          \procLevel{\citeyearpar}, while ``test phase'' \phaseDef{}
          \fi
    \item % Discrep count: {PetersAndPedrycz2000}
          \refHelper \citeauthor{PetersAndPedrycz2000} \multAuthHelper{claim}
          that ``structural testing
          subsumes white box testing'' but they seem to describe the same thing:
          \ifnotpaper they say \else it says \fi ``structure tests are aimed at
          exercising the internal logic of a software system'' and ``in white box
          testing \dots, using detailed knowledge of code, one creates a battery of
          tests in such a way that they exercise all components of the code
          (say, statements, branches, paths)'' on the same page
          \citeyearpar[p.~447]{PetersAndPedrycz2000}!
    \item % Discrep count: ISTQB | {Patton2006}
          % TODO: NEAR SYNS?
          \refHelper \citetISTQB{} \multAuthHelper{claim} that code inspections
          are related to peer reviews but \citet[pp.~94-95]{Patton2006} makes
          them quite distinct.
    \item % Discrep count: ISTQB | {PetersAndPedrycz2000}
          \phantomsection{} \label{walkthrough-syns}
          Likewise, ``walkthroughs'' and ``structured walkthroughs'' are given
          as synonyms by \citetISTQB{} but \citet[p.~484]{PetersAndPedrycz2000}
          \ifnotpaper imply \else implies \fi that they are different, saying a
          more structured walkthrough may have specific roles.
          \ifnotpaper
    \item % Discrep count: {SneedAndGöschl2000}
          \refHelper \citet[p.~18]{SneedAndGöschl2000}\todo{OG Hetzel88}
          \multAuthHelper{give} ``white-'',
          ``grey-'', and ``black-box testing'' as synonyms for ``module'',
          ``integration'', and ``system testing'', respectively, but
          this mapping is incorrect; black-box testing can be performed on a
          module, for example\todo{find source}.
    \item % Discrep count: {SneedAndGöschl2000}
          The previous discrepancy makes the claim that
          ``red-box testing'' is a synonym for ``acceptance testing''
          \citep[p.~18]{SneedAndGöschl2000} lose credibility.
    \item % Discrep count: implied by {Kam2008}
          \refHelper \citet[p.~46]{Kam2008} seems to imply that ``mutation
          testing'' is a synonym of ``back-to-back testing''
          but these are two quite distinct techniques.
    \item % Discrep count: implied by {Kam2008}
          ``Conformance testing'' is implied to be a synonym of ``compliance
          testing'' by \citet[p.~43]{Kam2008} which only makes sense because
          of the vague definition of ``compliance testing'': ``testing to
          determine the compliance of the component or system''. \fi
\end{enumerate}
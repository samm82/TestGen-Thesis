\ifnotpaper\paragraph{High Severity}\fi
\begin{itemize}
      \item % Discrep count (DEFS): {IEEE2022} | {IEEE2017} ISTQB {ISO_IEC2023a}
            ``Compatibility testing'' is defined as ``testing that measures the
            degree to which a test item can function satisfactorily alongside
            other independent products in a shared environment (co-existence),
            and where necessary, exchanges information with other systems or
            components (interoperability)'' \citep[p.~3]{IEEE2022}. This
            definition is nonatomic as it combines the ideas of ``co-existence''
            and ``interoperability''. The term ``interoperability testing'' is
            not defined, but is used three times \citep[pp.~22,~43]{IEEE2022}
            % Discrep count (TERMS): {IEEE2022}
            (although the third usage seems like it should be ``portability
            testing''). This implies that ``co-existence testing'' and
            ``interoperability testing'' should be defined as their own terms,
            which is supported by definitions of ``co-existence'' and
            ``interoperability'' often being separate
            \ifnotpaper
                  (\citealpISTQB{}; \citealp[pp.~73,~237]{IEEE2017})%
            \else
                  \cite[pp.~73,~237]{IEEE2017}, \cite{ISTQB}%
            \fi, the definition of
            ``interoperability testing'' from \citet[p.~238]{IEEE2017},
            and the decomposition of ``compatibility'' into ``co-existence''
            and ``interoperability'' by \citet{ISO_IEC2023a}!
            \begin{itemize}
                  \item % Discrep count (SYNS): implied by {IEEE2021} | {IEEE2022}
                        The ``interoperability'' element of ``compatibility
                        testing'' is explicitly excluded by
                        \citet[p.~37]{IEEE2021}, (incorrectly) implying that
                        ``compatibility testing'' and ``co-existence testing''
                        are synonyms.
                  \item % Discrep count (SYNS): {Kam2008} | {IEEE2022}
                        The definition of ``compatibility testing'' in
                        \citep[p.~43]{Kam2008} unhelpfully says ``See
                        \emph{interoperability testing}'', adding another
                        layer of confusion to the direction of their
                        relationship.
            \end{itemize}
            \ifnotpaper
\end{itemize}

\paragraph{Medium Severity}
\begin{itemize}
      \item % Discrep count (DEFS): {IEEE2021} | {IEEE2017}
            % TODO: SUPP DEFS?
            \citeauthor{IEEE2021} define an ``extended entry (decision) table''
            both as a decision table where the ``conditions consist of multiple
            values rather than simple Booleans'' \citeyearpar[p.~18]{IEEE2021}
            and one where ``the conditions and actions are generally described
            but are incomplete'' \citeyearpar[p.~175]{IEEE2017}\todo{OG ISO1984}.
      \item % Discrep count (DEFS): {IEEE2017}
            \citeauthor{IEEE2017} use the same definition for ``partial correctness''
            \citeyearpar[p.~314]{IEEE2017} and ``total correctness'' (p.~480).
            \fi
\end{itemize}

\ifnotpaper
      \paragraph{Low Severity}
      \begin{itemize}
            \item % Discrep count (DEFS): {IEEE2022} {IEEE2021}
                  % TODO: NO DEFS?
                  Integration, system, and system integration testing are all listed
                  as ``common test levels'' (\citealp[p.~12]{IEEE2022};
                  \citeyear[p.~6]{IEEE2021}), but no
                  definitions are given for the latter two, making it unclear what
                  ``system integration testing'' is; it is a combination of the two?
                  somewhere on the spectrum between them? It is listed as a child
                  % Discrep count (PARS): ISTQB | {Firesmith2015}
                  of integration testing by \citetISTQB{} and of system testing
                  by \citet[p.~23]{Firesmith2015}.
            \item % Discrep count (DEFS): {IEEE2017}
                  % TODO: NO DEFS?
                  Similarly, component, integration, and component integration
                  testing are all listed in \citep{IEEE2017}, but ``component
                  integration testing'' is only defined as ``testing of groups of
                  related components'' \citep[p.~82]{IEEE2017}; it is a combination of
                  the two? somewhere on the spectrum between them? As above, it is
                  listed as a child of integration testing by \citetISTQB{}.
            \item % Discrep count (TERMS): {IEEE2021}
                  A typo in \citep[Fig.~2]{IEEE2021} means that ``specification-based
                  techniques'' is listed twice, when the latter should be
                  ``structure-based techniques''.
      \end{itemize}\fi
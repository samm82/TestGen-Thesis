\ifnotpaper\paragraph{High Severity}\fi
\begin{itemize}
      \item % Discrep count (DEFS): {IEEE2022} | {IEEE2017} ISTQB {ISO_IEC2023a}
            ``Compatibility testing'' is defined as ``testing that measures the
            degree to which a test item can function satisfactorily alongside
            other independent products in a shared environment (co-existence),
            and where necessary, exchanges information with other systems or
            components (interoperability)'' \citep[p.~3]{IEEE2022}. This
            definition is nonatomic as it combines the ideas of ``co-existence''
            and ``interoperability''. The term ``interoperability testing'' is
            not defined, but is used three times \citep[pp.~22,~43]{IEEE2022}
            % Discrep count (TERMS): {IEEE2022}
            (although the third usage seems like it should be ``portability
            testing''). This implies that ``co-existence testing'' and
            ``interoperability testing'' should be defined as their own terms,
            which is supported by definitions of ``co-existence'' and
            ``interoperability'' often being separate
            \ifnotpaper
                  (\citealpISTQB{}; \citealp[pp.~73,~237]{IEEE2017})%
            \else
                  \cite[pp.~73,~237]{IEEE2017}, \cite{ISTQB}%
            \fi, the definition of
            ``interoperability testing'' from \citet[p.~238]{IEEE2017},
            and the decomposition of ``compatibility'' into ``co-existence''
            and ``interoperability'' by \citet{ISO_IEC2023a}!
            \begin{itemize}
                  \item % Discrep count (SYNS): implied by {IEEE2021} | {IEEE2022}
                        The ``interoperability'' element of ``compatibility
                        testing'' is explicitly excluded by
                        \citet[p.~37]{IEEE2021}, (incorrectly) implying that
                        ``compatibility testing'' and ``co-existence testing''
                        are synonyms.
                  \item % Discrep count (SYNS): {Kam2008} | {IEEE2022}
                        The definition of ``compatibility testing'' in
                        \citep[p.~43]{Kam2008} unhelpfully says ``See
                        \emph{interoperability testing}'', adding another
                        layer of confusion to the direction of their
                        relationship.
            \end{itemize}
\end{itemize}
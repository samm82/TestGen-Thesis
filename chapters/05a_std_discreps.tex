\ifnotpaper\paragraph{High Severity}\fi
\begin{itemize}
      \item % Discrep count (OTHER): {IEEE2022} {IEEE2021} {Kam2008} | {IEEE2017} ISTQB {ISO_IEC2023a}
            ``Compatibility testing'' is defined as ``testing that measures the
            degree to which a test item can function satisfactorily alongside
            other independent products in a shared environment (co-existence),
            and where necessary, exchanges information with other systems or
            components (interoperability)'' \citep[p.~3]{IEEE2022}. This
            definition is nonatomic as it combines the ideas of ``co-existence''
            and ``interoperability''. The term ``interoperability testing'' is
            not defined, but is used three times \citep[pp.~22,~43]{IEEE2022}
            % Severity: High (Std)
            (although the third usage seems like it should be ``portability
            testing''). This implies that ``co-existence testing'' and
            ``interoperability testing'' should be defined as their own terms,
            which is supported by definitions of ``co-existence'' and
            ``interoperability'' often being separate
            \ifnotpaper
                  (\citealpISTQB{}; \citealp[pp.~73,~237]{IEEE2017})%
            \else
                  \cite[pp.~73,~237]{IEEE2017}, \cite{ISTQB}%
            \fi, the definition of
            ``interoperability testing'' from \citet[p.~238]{IEEE2017},
            and the decomposition of ``compatibility'' into ``co-existence''
            and ``interoperability'' by \citet{ISO_IEC2023a}!
            \begin{itemize}
                  \item % Discrep count (SYNS): {IEEE2021}
                        The ``interoperability'' element of ``compatibility
                        testing'' is explicitly excluded by
                        \citet[p.~37]{IEEE2021}, (incorrectly) implying that
                        ``compatibility testing'' and ``co-existence testing''
                        are synonyms.
                  \item % Severity: Low (Other)
                        The definition of ``compatibility testing'' in
                        \citep[p.~43]{Kam2008} unhelpfully says ``See
                        \emph{interoperability testing}'', adding another
                        layer of confusion to the direction of their
                        relationship.
            \end{itemize}
      \item % Discrep count (OTHER): {ISO_IEC2023b} {IEEE2017} | {BaresiAndPezzè2006}
            A component is an ``entity with discrete structure \dots\ within a
            system considered at a particular level of analysis''
            \citep{ISO_IEC2023b} and ``the terms module, component, and unit
                  [sic] are often used interchangeably or defined to be subelements
            of one another in different ways depending upon the context'' with
            no standardized relationship \citep[p.~82]{IEEE2017}. This means
            unit/component/module testing can refer to the testing of both a
            module and a specific function in a module\seeThesisIssuePar{14}.
            However, ``component'' is sometimes defined differently than
            ``module'': ``components differ from classical modules for being
            re-used in different contexts independently of their development''
            \citep[p.~107]{BaresiAndPezzè2006}, so distinguishing the two
            may be necessary.
            \ifnotpaper
\end{itemize}

\paragraph{Medium Severity}
\begin{itemize}\fi
      \item % Discrep count (OTHER): {IEEE2022} | {BarbosaEtAl2006}
            Retesting and regression testing seem to be separated from the rest
            of the testing approaches \citep[p.~23]{IEEE2022}, but it is not
            clearly detailed why; \citet[p.~3]{BarbosaEtAl2006} \ifnotpaper
                  consider \else considers \fi regression testing to be a test level.
            \ifnotpaper
      \item % Discrep count (OTHER): {IEEE2021} | {IEEE2017}
            \citeauthor{IEEE2021} define an ``extended entry (decision) table''
            both as a decision table where the ``conditions consist of multiple
            values rather than simple Booleans'' \citeyearpar[p.~18]{IEEE2021}
            and one where ``the conditions and actions are generally described
            but are incomplete'' \citeyearpar[p.~175]{IEEE2017}\todo{OG ISO1984}.
      \item % Discrep count (OTHER): {IEEE2017} {IEEE2013} | {IEEE2022} {IEEE2017} {Perry2006}
            \citeauthor*{IEEE2017} say that ``test level'' and ``test phase''
            are synonyms\footnote{Although this is a discrepancy based on a
                  synonym relation, the ``synonyms'' are supporting terms and
                  not test approaches, which is why this is not included in
                  \Cref{syns} as a synonym discrepancy.}, both meaning a
            ``specific instantiation of [a] test sub-process''
            (\citeyear[pp.~469,~470]{IEEE2017}; \citeyear[p.~9]{IEEE2013}), but
            there are also alternative definitions for them.
            \procLevel{\citeyearpar}, while ``test phase'' \phaseDef{}
      \item \citeauthor{IEEE2017} use the same definition for ``partial correctness''
            \citeyearpar[p.~314]{IEEE2017} and ``total correctness'' (p.~480).
            \fi
\end{itemize}

\ifnotpaper
      \paragraph{Low Severity}
      \begin{itemize}
            \item % Discrep count (OTHER): {IEEE2022} {IEEE2021}
                  Integration, system, and system integration testing are all listed
                  as ``common test levels'' (\citealp[p.~12]{IEEE2022};
                  \citeyear[p.~6]{IEEE2021}), but no
                  definitions are given for the latter two, making it unclear what
                  ``system integration testing'' is; it is a combination of the two?
                  somewhere on the spectrum between them? It is listed as a child
                  % TODO: count this for pie charts as well?
                  % Severity: Med (Meta)
                  of integration testing by \citetISTQB{} and of system testing
                  by \citet[p.~23]{Firesmith2015}.
            \item % Discrep count (OTHER): {IEEE2017}
                  Similarly, component, integration, and component integration
                  testing are all listed in \citep{IEEE2017}, but ``component
                  integration testing'' is only defined as ``testing of groups of
                  related components'' \citep[p.~82]{IEEE2017}; it is a combination of
                  the two? somewhere on the spectrum between them? As above, it is
                  listed as a child of integration testing by \citetISTQB{}.
            \item % Discrep count (OTHER): {IEEE2021}
                  A typo in \citep[Fig.~2]{IEEE2021} means that ``specification-based
                  techniques'' is listed twice, when the latter should be
                  ``structure-based techniques''.
            \item % Discrep count (OTHER): {IEEE2017}
                  \citeauthor*{IEEE2017} provide a definition for ``inspections and
                  audits'' \citeyearpar[p.~228]{IEEE2017}, despite also giving
                  definitions for ``inspection'' (p.~227) and ``audit'' (p.~36);
                  while the first term \emph{could} be considered a superset of the
                  latter two, this distinction doesn't seem useful.
      \end{itemize}\fi
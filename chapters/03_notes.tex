\chapter{Notes}
\label{chap:notes}

\section{A Survey of Metaprogramming Languages}
\label{chap:notes:sec:metalang-survey}

\begin{itemize}
      \item Often done with Abstract Syntax Trees (ADTs),
            \todo{investigate more: Steele 1990?} although other bases are
            used:
            \begin{itemize}
                  \item Skeleton Syntax Tree (SST), used by Dylan
                        \cite[p.~113:6]{lilis_survey_2019}
            \end{itemize}
      \item Allows for improvements in:
            \begin{itemize}
                  \item "performance by generating efficient specialized
                        programs based on specifications instead of using
                        generic but inefficient programs"
                        \cite[p.~113:2]{lilis_survey_2019}
                  \item reasoning about object programs through "analyzing
                        and discovering object-program characteristics that
                        enable applying further optimizations as well as
                        inspecting and validating the behavior of the object
                        program" \cite[p.~113:2]{lilis_survey_2019}
                  \item code reuse through capturing "code patterns that cannot
                        be abstracted" \cite[p.~113:2]{lilis_survey_2019}
            \end{itemize}
\end{itemize}

\subsection{Definitions}

"\emph{Metaprogramming} is the process of writing computer programs, called
\emph{metaprograms}, that [can] \dots generate new programs or modify existing
ones" \cite[p.~113:1]{lilis_survey_2019}.

\begin{itemize}
      \item Metalanguage: "the language in which the metaprogram is written"
            \cite[p.~113:1]{lilis_survey_2019}
      \item Object language: "the language in which the generated or
            transformed program is written" \cite[p.~113:1]{lilis_survey_2019}
      \item Homogeneous metaprogramming: when "the object language and the
            metalanguage are the same" \cite[p.~113:1]{lilis_survey_2019}
      \item Heterogeneous metaprogramming: when "the object language and the
            metalanguage are \dots different" \cite[p.~113:1]{lilis_survey_2019}
\end{itemize}

\subsection{Metaprogramming Models}
\subsubsection{Macro Systems \cite[p.~113:3-7]{lilis_survey_2019}}
\begin{itemize}
      \item Map specified input sequences in a source file to corresponding
            output sequences ("macro expansion") until no input sequences
            remain \cite[p.~113:3]{lilis_survey_2019}; this process can be:
            \begin{enumerate}
                  \item procedural (involving algorithms), or
                  \item pattern-based (only using pattern matching)
                        \cite[p.~113:4]{lilis_survey_2019}
            \end{enumerate}
      \item Must avoid variable capture (unintended name conflicts) by being
            "hygienic" \cite[p.~113:4]{lilis_survey_2019}; this may be
            overridden to allow for "intentional variable capture", such as
            Scheme's \emph{syntax-case} macro \cite[p.~113:5]{lilis_survey_2019}
\end{itemize}

\paragraph{Lexical Macros}
\begin{itemize}
      \item Language agnostic \cite[p.~113:3]{lilis_survey_2019}
      \item Usually only sufficient for basic metaprogramming since changes to
            the code without considering its meaning "may cause unintended side
            effects or name clashes and may introduce difficult-to-solve bugs"
            \cite[p.~113:5]{lilis_survey_2019}
      \item Marco was the first safe, language-independent macro system that
            "enforce[s] specific rules that can be checked by special oracles"
            for given languages (as long as the languages "produce descriptive
            error messages") \cite[p.~113:6]{lilis_survey_2019}
\end{itemize}

\paragraph{Syntactic Macros}
\begin{itemize}
      \item "Aware of the language syntax and semantics"
            \cite[p.~113:3]{lilis_survey_2019}
      \item MS\textsuperscript{2} "was the first programmable syntactic macro
            system for syntactically rich languages", including by using "a
            type system to ensure that all generated code fragments are
            syntactically correct" \cite[p.~113:5]{lilis_survey_2019}
\end{itemize}

\subsubsection{Reflection Systems \cite[p.~113:7-9]{lilis_survey_2019}}
\begin{itemize}
      \item "Perform computations on [themselves] in the same way as for the
            target application, enabling one to adjust the system behavior
            based on the needs of its execution"
            \cite[p.~113:7]{lilis_survey_2019}
      \item Requires that the system can examine ("introspection") and modify
            ("intercession") how it is represented
            \cite[p.~113:7]{lilis_survey_2019}
            \begin{itemize}
                  \item The representation of a system can either be structural
                        or behavioural (e.g., variable assignment)
                        \cite[p.~113:7]{lilis_survey_2019}
            \end{itemize}
      \item "Runtime code generation based on source text can be impractical,
            inefficient, and unsafe, so alternatives have been explored based
            on ASTs and quasi-quote operators, offering a structured approach
            that is subject to typing for expressing and combining code at
            runtime" \cite[p.~113:8]{lilis_survey_2019}
      \item "Not limited to runtime systems", as some "compile-time systems
            \dots rely on some form of structural introspection to perform code
            generation" \cite[p.~113:9]{lilis_survey_2019}
\end{itemize}

\section{Software Metrics}
\label{chap:notes:sec:software-metrics}

\begin{itemize}
      \item The following branches of testing started as parts of quality
            testing:
            \begin{itemize}
                  \item Reliability testing \cite[p.~18, ch.~10]{fenton_software_1997}
                  \item Performance testing \cite[p.~18, ch.~7]{fenton_software_1997}
            \end{itemize}
      \item Reliability and maintainability can start to be tested even without
            code by ``measur[ing] structural attributes of representations of the
            software'' \cite[p.~18]{fenton_software_1997}
      \item The US Software Engineering Institute has a checklist for determining
            which types of lines of code are included when counting
            \cite[pp.~30-31]{fenton_software_1997}
      \item Measurements should include an entity to be measured, a specific
            attribute to measure, and the actual measure (i.e., units, starting
            state, ending state, what to include) \cite[p.~36]{fenton_software_1997}
            \begin{itemize}
                  \item These attributes must be defined before they can be
                        measured \cite[p.~38]{fenton_software_1997}
            \end{itemize}
\end{itemize}

\section{Software Testing}
\label{chap:notes:sec:software-testing}

\subsection{General Testing Notes}

\begin{itemize}
      \item Simple, normal test cases (test-to-pass) should always be developed
            and run before more complicated, unusual test cases (test-to-fail)
            \cite[p.~66]{patton_software_2006}
\end{itemize}

\subsection{Types of Testing}

\subsubsection{Static Black-Box (Specification) Testing
      \cite[p.~56-62]{patton_software_2006}}

Most of this section is irrelevant to generating test cases, as they require
human involvement (e.g., Pretend to Be the Customer
\cite[p.~57-58]{patton_software_2006}, Research Existing Standards and
Guidelines \cite[p.~58-59]{patton_software_2006}). However, it provides a
"Specification Terminology Checklist" \cite[p.~61]{patton_software_2006} that
includes some keywords that, if found, could trigger an applicable warning to
the user (similar to the idea behind the correctness/consistency checks
project):

\begin{itemize}
      \item \textbf{Potentially unrealistic:} always, every, all, none, every,
            certainly, therefore, clearly, obviously, evidently
      \item \textbf{Potentially vague:} some, sometimes, often, usually,
            ordinarily, customarily, most, mostly, good, high-quality, fast,
            quickly, cheap, inexpensive, efficient, small, stable
      \item \textbf{Potentially incomplete:} etc., and so forth, and so on,
            such as, handled, processed, rejected, skipped, eliminated,
            if \dots then \dots (without "else" or "otherwise")
\end{itemize}

\subsubsection{Dynamic Black-Box (Behavioural) Testing
      \cite[p.~64-65]{patton_software_2006}}

This is the process of "entering inputs, receiving outputs, and checking the
results" \cite[p.~64]{patton_software_2006}. Note that while black-box testing
is usually done at a higher (e.g., system) level, unit testing can also be
black-box \cite[p.~1]{jacob_comparative_2016}.

\paragraph{Requirements}
\begin{itemize}
      \item Requirements documentation (definition of what the software does)
            \cite[p.~64]{patton_software_2006}; relevant information could be:
            \begin{itemize}
                  \item Requirements: Input-Values and Output-Values
                  \item Input/output data constraints
            \end{itemize}
\end{itemize}

\subsubsection{Exploratory Testing \cite[p.~65]{patton_software_2006}}

An alternative to dynamic black-box testing when a specification is not
available \cite[p.~65]{patton_software_2006}. The software is explored to
determine its features, and these features are then tested
\cite[p.~65]{patton_software_2006}. Finding any bugs using this method is a
positive thing \cite[p.~65]{patton_software_2006}, since despite not knowing
what the software \emph{should} do, you were able to determine that something
is wrong.

This is not applicable to Drasil, because not only does it already generate a
specification, making this type of testing unnecessary, there is also a lot of
human-based trial and error required for this kind of testing
\cite{june_11_meeting}.

\subsubsection{Equivalence Partitioning/Classing \cite[p.~67-69]{patton_software_2006}}

The process of dividing the infinite set of test cases into a finite set that is
just as effective (i.e., by revealing the same bugs) \cite[p.~67]{patton_software_2006}.

\paragraph{Requirements}
\begin{itemize}
      \item Ranges of possible values \cite[p.~67]{patton_software_2006};
            could be obtained through:
            \begin{itemize}
                  \item Input/output data constraints
                  \item Case statements
            \end{itemize}
\end{itemize}

\subsubsection{Data Testing \cite[p.~70-79]{patton_software_2006}}

The process of "checking that information the user inputs [and] results",
both final and intermediate, "are handled correctly" \cite[p.~70]{patton_software_2006}.

\paragraph{Boundary Conditions \cite[p.~70-74]{patton_software_2006}}

"[S]ituations at the edge of the planned operational limits of the software"
\cite[p.~72]{patton_software_2006}. Often affects types of data (e.g., numeric,
speed, character, location, position, size, quantity
\cite[p.~72]{patton_software_2006}) each with its own set of (e.g., first/last,
min/max, start/finish, over/under, empty/full, shortest/longest,
slowest/fastest, soonest/latest, largest/smallest, highest/lowest,
next-to/farthest-from \cite[p.~72-73]{patton_software_2006}). Data at these
boundaries should be included in an equivalence partition, but so should
data in between them \cite[p.~73]{patton_software_2006}. Boundary conditions
should be tested using "the valid data just inside the boundary,
... the last possible valid data, and ... the invalid data just outside the
boundary" \cite[p.~73]{patton_software_2006}.

\subparagraph{Requirements}
\begin{itemize}
      \item Ranges of possible values \cite[p.~67, 73]{patton_software_2006};
            could be obtained through:
            \begin{itemize}
                  \item Case statements
                  \item Input/output data constraints (e.g., inputs that
                        would lead to a boundary output)
            \end{itemize}
\end{itemize}

\subparagraph{Buffer Overruns \cite[p.~201-205]{patton_software_2006}}

\emph{Buffer overruns} are "the number one cause of software security issues"
\cite[p.~75]{patton_software_2006}. They occur when the size of the destination
for some data is smaller than the data itself, causing existing data (including
code) to be overwritten and malicious code to potentially be injected
\cite[p.~202, 204-205]{patton_software_2006}. They often arise from bad
programming practices in "languages [sic] such as C and C++, that lack safe
string handling functions" \cite[p.~201]{patton_software_2006}. Any unsafe
versions of these functions that are used should be replaced with the
corresponding safe versions \cite[p.~203-204]{patton_software_2006}.

\paragraph{Sub-Boundary Conditions \cite[p.~75-77]{patton_software_2006}}

Boundary conditions "that are internal to the software [but] aren't necessarily
apparent to an end user" \cite[p.~75]{patton_software_2006}. These include
powers of two \cite[p.~75-76]{patton_software_2006} and ASCII and Unicode tables
\cite[p.~76-77]{patton_software_2006}.

While this is of interest to the domain of scientific computing, this is too
involved for Drasil right now, and the existing software constraints limit much
of the potential errors from over/underflow \cite{june_11_meeting}. Additionally,
strings are not really used as inputs to Drasil and only occur in output with
predefined values, so testing these values are unlikely to be fruitful.

\subparagraph{Requirements}
\begin{itemize}
      \item Increased knowledge of data type structures (e.g., monoids, rings,
            etc. \cite{june_11_meeting}); this would capture these sub-boundaries,
            as well as other information like relevant tests cases, along with
            our notion of these data types (\texttt{Space})
\end{itemize}

\paragraph{Default, Empty, Blank, Null, Zero, and None
      \cite[p.~77-78]{patton_software_2006}}

These should be their own equivalence class, since "the software usually
handles them differently" than "the valid cases or ... invalid cases"
\cite[p.~78]{patton_software_2006}.

Since these values may not always be applicable to a given scenario (e.g., a
test case for zero doesn't make sense if there is a constraint that the value
in question cannot be zero), the user should likely be able to select
categories of tests to generate instead of Drasil just generating all possible
test cases based on the inputs \cite{june_11_meeting}.

\subparagraph{Requirements}
\begin{itemize}
      \item Knowledge of an "empty" value for each \texttt{Space} (stored
            alongside each type in \texttt{Space}?)
      \item Knowledge of how input data could be omitted from an input
            (e.g., a missing command line argument, an empty line in a file);
            could be obtained from:
            \begin{itemize}
                  \item User responsibilities
            \end{itemize}
      \item Knowledge of how a programming language deals with \texttt{Null}
            values and how these can be passed as arguments
\end{itemize}

\paragraph{Invalid, Wrong, Incorrect, and Garbage Data
      \cite[p.~78-79]{patton_software_2006}}

This is testing-to-fail \cite[p.~77]{patton_software_2006}.

\subparagraph{Requirements}
This seems to be the most open-ended category of testing.
\begin{itemize}
      \item Specification of correct inputs that can be ignored;
            could be obtained through:
            \begin{itemize}
                  \item Input/output data constraints (e.g., inputs that would
                        lead to a violated output constraint)
                  \item Type information for each input (e.g., passing a string
                        instead of a number)
            \end{itemize}
\end{itemize}

\subsubsection{State Testing \cite[p.~79-87]{patton_software_2006}}

The process of testing "a program's states and the transitions between them"
\cite[p.~79]{patton_software_2006}.

\paragraph{Logic Flow Testing \cite[p.~80-84]{patton_software_2006}}

This is done by creating a state transition diagram that includes:

\begin{itemize}
      \item Every possible unique state
      \item The condition(s) that take(s) the program between states
      \item The condition(s) and output(s) when a state is entered or exited
\end{itemize}

to map out the logic flow from the user's perspective
\cite[p.~81-82]{patton_software_2006}. Next, these states should be
partitioned using one (or more) of the following methods:

\begin{enumerate}
      \item Test each state once
      \item Test the most common state transitions
      \item Test the least common state transitions
      \item Test all error states and error return transitions
      \item Test random state transitions
            \cite[p.~82-83]{patton_software_2006}
\end{enumerate}

For all of these tests, the values of the state variables should be verified
\cite[p.~83]{patton_software_2006}.

\subparagraph{Requirements}
\begin{itemize}
      \item Knowledge of the different states of the program
            \cite[p.~82]{patton_software_2006}; could be obtained through:
            \begin{itemize}
                  \item The program's modules and/or functions
                  \item The program's exceptions
            \end{itemize}
      \item Knowledge about the different state transitions
            \cite[p.~82]{patton_software_2006}; could be obtained through:
            \begin{itemize}
                  \item Testing the state transitions near the beginning of a
                        workflow more?
            \end{itemize}
\end{itemize}

\paragraph{Testing States to Fail \cite[p.~84-87]{patton_software_2006}}

The goal here is to try and put the program in a fail state by doing things
that are out of the ordinary. These include:

\begin{itemize}
      \item Race Conditions and Bad Timing \cite[p.~85-86]{patton_software_2006}
            (Is this relevant to our examples?)
      \item Repetition Testing: "doing the same operation over and over",
            potentially up to "thousands of attempts"
            \cite[p.~86]{patton_software_2006}
      \item Stress Testing: "running the software under less-than-ideal conditions"
            \cite[p.~86]{patton_software_2006}
      \item Load testing: running the software with as large of a load as
            possible (e.g., large inputs, many peripherals)
            \cite[p.~86]{patton_software_2006}
\end{itemize}

\subparagraph{Requirements}
\begin{itemize}
      \item Repetition Testing: The types of operations that are likely to lead
            to errors when repeated (e.g., overwriting files?)
      \item Stress testing: can these be automated with pytest or are they
            outside our scope? % TODO: investigate
      \item Load testing: Knowledge about the types of inputs that could
            overload the system (e.g., upper bounds on values of certain types)
\end{itemize}

\subsubsection{Other Black-Box Testing \cite[p.~87-89]{patton_software_2006}}
\begin{itemize}
      \item Act like an inexperienced user (likely cannot be generated by Drasil)
      \item Look for bugs where they've already been found (keep track of
            previous failed test cases?)
      \item Think like a hacker (is this out of scope?)
      \item Follow experience (this will implicitly be done just by using Drasil)
\end{itemize}

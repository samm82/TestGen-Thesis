\subsection{Discrepancy Analysis}
\label{discrep-analysis}

In addition to analyzing specific discrepancies, an overview of their amounts,
sources, rigidities (see \Cref{rigidity}), classes, and categories is
also useful. Subsets of this task can be automated (\Cref{auto-discrep-analysis})
and the remaining manual portion can be augmented with automated
tools (\Cref{aug-discrep-analysis}).

To understand where discrepancies exist in the literature, they are
grouped based on the source tiers (as described in \Cref{sources})
responsible for them. Each discrepancy is then counted \emph{once} per source
category if it appears within it \emph{and/or} between it and a more
``trusted'' category. This avoids counting the same discrepancy twice for a
given category\thesisissueref{83}, which would result in the number of
\emph{occurrences} of all discrepancies, instead of the number of discrepancies
\emph{themselves}, which is more useful. An exception to this is
\Cref{fig:discrepSources}, which counts the following sources of discrepancies
separately:
\begin{enumerate}
    \item those within a single document,
    \item those between documents by the same author(s) or standards
          organization(s), and
    \item those within a source tier.
\end{enumerate}
As before, these are not double counted, meaning that the maximum number of
counted discrepancies possible within a \emph{single} source tier in
\Cref{fig:discrepSources} is three (one for each type). This only occurs if
there is an example of each discrepancy source that is \emph{not} ignored to
avoid double counting; for example, while a single discrepancy within a single
document would technically fulfill all three criteria, it would only be counted
once.

\phantomsection{}
\label{discrep-analysis-example}
As an example of this process, consider a discrepancy \emph{within} an IEEE
document (e.g., two different definitions are given for a term within the same
IEEE document) \emph{and} between another IEEE document, the \acs{istqb}
glossary \emph{and} two papers. This would add one to the following rows of
\Cref{tab:sntxDiscreps,tab:smntcDiscreps} in the relevant column:

\begin{itemize}
    \item \textbf{\stds{}}: this discrepancy occurs:
          \begin{enumerate}
              \item within one standard and
              \item between two standards.
          \end{enumerate}
          This increments the count by just one to avoid double counting and
          would do so even if only one of the above conditions was true. A more
          nuanced breakdown of discrepancies that identifies those within a
          singular document and those between documents by the same author is
          given in \Cref{fig:discrepSources} and explained in more detail in
          \Cref{aug-discrep-analysis}.
    \item \textbf{\metas{}}: this discrepancy occurs between a
          source in this category and a ``more trusted'' one
          (the IEEE standards).
    \item \textbf{\papers{}}: this discrepancy occurs between a
          source in this category and a ``more trusted'' one. Even though there
          are two sources in this category \emph{and} two ``more trusted''
          categories involved, this increments the count by just one to avoid
          double counting.
\end{itemize}

\subsubsection{Automated Discrepancy Analysis}
\label{auto-discrep-analysis}

As outlined in \Cref{graph-gen}, some types of discrepancies can be detected
automatically. While just counting the total number of these types of
discrepancies is trivial, tracking the source(s) of these discrepancies is more
involved. Since the appropriate citations for each piece of information is
tracked (see \Cref{tab:exampleGlossary,tab:synExampleGlossary} for examples of
how these citations are formatted in the glossaries), they can be used to find
the offending source tiers. This comes with the added benefit of these
citations being available to be formatted for use with \LaTeX{}'s citation
commands for inclusion in this document.

Comparing the authors and years of each source related to a given discrepancy
can determine if it manifests within a single document and/or between documents
by the same author(s) when creating \Cref{fig:discrepSources}. Then, the
relevant sources can be sorted into their categories based on their citations
\seeSrcCode{e709acf}{discrepCounter}{39}{56}.
% done by the function in \Cref{lst:getSrcCat}, since each source tier
% outlined in \Cref{sources} is comprised of a small number of authors (with the
% exception of papers and other documents; see \Cref{papers}).
This determines
the appropriate row of \Cref{tab:sntxDiscreps,tab:smntcDiscreps} and the appropriate
graph and slice in \Cref{fig:discrepSources}. These lists of sources can then
be distilled down to sets of categories which are compared against
each other to determine how many times a given discrepancy manifests between
source tiers. Examples of this process are described in more detail in
\Cref{aug-discrep-analysis}.

\phantomsection{}
\label{auto-discrep-analysis-rigidity}
Alongside this citation information are the keywords relevant for assessing a
piece of information's rigidity (see \Cref{rigidity}). This is useful when
counting discrepancies, since a discrepancy can be both explicit and implicit,
but should not be double counted as both\thesisissueref{83}! When counting
discrepancies in \Cref{tab:sntxDiscreps,tab:smntcDiscreps}, a given discrepancy is
counted only for its most ``rigid'' manifestation (i.e., it will only increment
a value in the ``Implicit'' column if it is \emph{not} also explicit).

\subsubsection{Augmented Discrepancy Analysis}
\label{aug-discrep-analysis}
While some subsets of discrepancies can be deduced automatically from analyzing
the testing approach glossary, other types of discrepancies need to be tracked
manually. This is done by adding
comments to the relevant \LaTeX{} files (generated or not) of the form
\begin{displayquote}
    \texttt{\% Discrep count (CAT, CLS): \{A1\} \{A2\} \dots{} | \{B1\}
        %\{B2\}
        \dots{} | \{C1\} %\{C2\}
        \dots}
\end{displayquote}
which can then be parsed to detemine where discrepancies occur. \texttt{CAT} is
a placeholder for the discrepancy's category (see \Cref{smntcDiscreps})
identifier and \texttt{CLS} is a placeholder for its class (see
\Cref{sntxDiscreps}) identifier. These designations are ommitted
from the following examples of these comments. Each group of
sources is separated with a pipe symbol to be compared with the others, so any
number of groups are permitted. If only one group is present, it is compared
with itself. For example, the first line below means that source X has a
discrepancy with itself, while the second line adds a discrepancy between X and Y.
\begin{displayquote}
    \texttt{\% Discrep count: \{X\}\\\% Discrep count: \{X\} | \{X\} \{Y\}}
\end{displayquote}
Discrepancies between groups are not double counted; this means the following
line adds discrepancies between X and Z \emph{and} between Y and Z, without
counting the discrepancy between X and Z twice.
\begin{displayquote}
    \texttt{\% Discrep count: \{X\} | \{X\} \{Y\} | \{Z\}}
\end{displayquote}
Each source is given using its BibTeX key wrapped in curly braces to mimic
\LaTeX{}'s citation commands for ease of parsing, with the exception of the
\acs{istqb} glossary, due to its use of custom commands via
\texttt{\textbackslash citealias}. For example, the line
\begin{displayquote}
    \texttt{\% Discrep count: \{IEEE2022\} | \{IEEE2022\} \{IEEE2017\}\\
        \displayNL ISTQB \{Kam2008\} \{Bas2024\}}
\end{displayquote}
would be parsed as the example given in \Cref{discrep-analysis-example}. Since
the IEEE documents are written by the same standards organizations
(\begin{NoHyper}\citeauthor{IEEE2022}\end{NoHyper}), they are counted as a
discrepancy between documents by the same author(s) in \Cref{fig:discrepSources}.

The rigidity (see \Cref{rigidity}) of discrepancies can also be manually
specified by inserting the phrase ``implied by'' after the sources of explicit
information and before those of implicit information. Parsing this information
follows the same rules as the automatic discrepancy analysis
(see \Cref{auto-discrep-analysis-rigidity}).
\begin{displayquote}
    \texttt{\% Discrep count: \{IEEE2022\} implied by \{Kam2008\} |\\
        \displayNL \{IEEE2017\} implied by \{IEEE2022\}}
\end{displayquote}
For example, the above line indicates that the discrepancies given below are
present. The second discrepancy only affects \Cref{fig:discrepSources} since it
is less ``rigid'' than the first discrepancy within standards (see \Cref{stds}).
The rest increment their corresponding count in
\Cref{fig:discrepSources,tab:sntxDiscreps,tab:smntcDiscreps} by only one:
\begin{itemize}
    \item an explicit discrepancy between documents by
          \begin{NoHyper}\citeauthor{IEEE2022}\end{NoHyper},
    \item an implicit discrepancy within a single document, and
    \item an implicit discrepancy between a paper and a standard.
\end{itemize}

Occasionally, a source from a lower tier is used as the ``ground truth'' for a
discrepancy. For example, \tolTestingDiscrep*{} This discrepancy is supported
by additional papers found via a miniature literature review (described in
\Cref{undef-terms}) from a lower source tier than \citep{Firesmith2015}
(which is a terminology collection; see \Cref{sources}). However, this
discrepancy is really based in \citep{Firesmith2015} and not in these
additional papers, but if these sources where included as detailed above,
this is the way it would be counted. Therefore, we document these ``ground
truth'' sources separately to track them for traceability without incorrectly
counting discrepancies. This is done for this specific example (and similarly
for other cases) as follows:
\begin{displayquote}
    \texttt{\% Discrep count (TERMS, WRONG): \{Firesmith2015\}\\
        \% Ground truth: \{LiuEtAl2023\} \{MorgunEtAl1999\} \{HolleyEtAl1996\}
        \displayNL \{HoweAndJohnson1995\}}
\end{displayquote}

\subsection{Macros}\label{macros}
To improve maintainability, traceability, and reproducibility, many values are
calculated by a Python script, saved to a file, and read in to be used as a
\LaTeX{} macro. This lets us use this macro in place of manually replacing each
value as it changes with our research. \Cref{tab:macrosCalc} lists these
macros, along with descriptions of what they define and their current values.

\begin{longtblr}[
    note{a} = {Calculated in \LaTeX{} from source tier lists; see \Cref{text-macros}.},
    note{b} = {Alias for \texttt{\textbackslash totalSmntcFlawBrkdwn\{13\}}; see \Cref{flawCounts}.},
    note{c} = {These macros are defined as counters to allow them to be used in
            calculations within \LaTeX{} (such as in \Cref{undef-terms,fig:undefPies}).},
    caption = {\LaTeX{} macros for calculated values.},
    label = {tab:macrosCalc}
    ]{
    colspec={|X[0.3,l,m]X[0.5,c,m]X[0.2,c,m]|},
    width = \linewidth, rowhead = 1
    }
    \hline
    \thead{Macro}                                   & \thead{What it Counts}        & \thead{Value}    \\
    \hline
    \macro{approachCount}                           & Identified test approaches    & \approachCount{} \\
    \macro{qualityCount}                            & Identified software qualities & \qualityCount{}  \\
    \macro{srcCount}\TblrNote{a}                    & Sources used in glossaries    & \srcCount{}      \\
    \macro{flawCount}\TblrNote{b}                   & Identified flaws              & \flawCount{}     \\
    \hline
    \macro{TotalBefore}\TblrNote{c}                 & Test approaches identified
    before process in \Cref{undef-terms}            & \the\TotalBefore{}                               \\
    \macro{UndefBefore}\TblrNote{c}                 & Undefined test approaches
    identified before process in \Cref{undef-terms} & \the\UndefBefore{}                               \\
    \macro{TotalAfter}\TblrNote{c}                  & Test approaches identified
    after process in \Cref{undef-terms}             & \the\TotalAfter{}                                \\
    \macro{UndefAfter}\TblrNote{c}                  & Undefined test approaches
    identified after process in \Cref{undef-terms}  & \the\UndefAfter{}                                \\
    \hline
    \macro{parSynCount}                             & Pairs of test approaches
    with a child-parent \emph{and} synonym relation & \parSynCount{}                                   \\
    \macro{selfCycleCount}                          & Test approaches that are
    a parent of themselves                          & \selfCycleCount{}                                \\
    \hline
\end{longtblr}


Additionally, discrepancies are counted based on their rigidity, source tier,
and whether they are syntactic or semantic (see
\Cref{rigidity,sources,auto-discrep-analysis,aug-discrep-analysis,discreps}).
These counts are saved to files, a syntactic and semantic version for each
source tier; for example, syntactic discrepancies in standards are saved to
\texttt{build/stdSntxDiscBrkdwn.tex} and semantic discrepancies in standards to
\texttt{build/stdSmntcDiscBrkdwn.tex}. These data are then read in to macros
(such as \macro{stdSmntcDiscBrkdwn}) that are then used to populate
\Cref{tab:sntxDiscreps,tab:smntcDiscreps}. For example,
\macro[1]{stdSntxDiscBrkdwn} and \macro[2]{stdSntxDiscBrkdwn} correspond to
explicit and implicit mistakes in standards documents, respectively.

\phantomsection{}\label{text-macros}
Strings of text can also be generated by Python scripts in a similar fashion.
The lists of sources in \Cref{sources} are built by extracting all sources
cited in our three glossaries, categorizing, sorting, and formattting them, and
saving them to a file to be saved as the macros \macro{stdSources},
\macro{metaSources}, \macro{textSources}, and \macro{paperSources}.%
\todo{Should I explicitly display these lists in a table here? That feels
    redundant and might make things unnecessarily cluttered.}
The number of sources in each tier is also saved, so these values can be summed
to calcuate \macro{srcCount} in \Cref{tab:macrosCalc}. When text will be
repeated in different places, it is also helpful to define it as a macro. This
helps keeps the text consistent, as it only needs to changed in one place. We
define macros for the discrepancies in \Cref{tab:macrosDiscrep} that get
repeated elsewhere.

\begin{longtblr}[
    caption = {\LaTeX{} macros for reused discrepancies.},
    label = {tab:macrosDiscrep}
    ]{
    colspec={|Q[l,m]Q[c,m]Q[c,m]|},
    width = \linewidth, rowhead = 1
    }
    \hline
    \thead{Macro}             & \thead{Relevant Discussion} & \thead{Reused in}           \\
    \hline
    \macro{bugPattonDiscrep}  & \discrepref{bug-patton}     & \Cref{intro}                \\
    \macro{tourDiscrep}       & \discrepref{tour-def}       & \Cref{intro,discreps}       \\
    \macro{alphaDiscrep}      & \discrepref{alpha-def}      & \Cref{intro}                \\
    \macro{loadDiscrep}       & \discrepref{load-def}       & \Cref{intro}                \\
    \macro{tolTestingDiscrep} & \discrepref{ground-truth}   & \Cref{aug-discrep-analysis} \\
    \macro{expBasedCatMain}   & \Cref{multiCats}            & \Cref{intro}                \\
    \macro{perfAsFamily}      & \Cref{classFamilyDiscrep}   & \Cref{method-family}        \\
    \hline
\end{longtblr}

